\documentclass[11pt,a4paper]{article}
 \usepackage{amsfonts}
\usepackage{hyperref}
 \newcounter{EssayNumber} \setlength{\parindent}{0 pt}
 \newcommand{\EssayTitle}[2]{\refstepcounter{EssayNumber}

  \addcontentsline{toc}{section}{\arabic{EssayNumber}. #1\dotfill}}
 \newcommand{\Header}[1]{\subsubsection*{#1}}

 \begin{document}

  
  
%%%%%%%%%%%%%%%%%%%%%%%%%%%%%%%%%%%%%%%%%%%%%%%%%%%%%%%%%%%%%%%%%%%%%%%%%%%%%%%
 % % % Start editting here. % % % 
  
%%%%%%%%%%%%%%%%%%%%%%%%%%%%%%%%%%%%%%%%%%%%%%%%%%%%%%%%%%%%%%%%%%%%%%%%%%%%%%%

 \EssayTitle{Interpretations and Synonymy}{Dr Forster}




``A logician is that kind of mathematician who thinks that a formula is 
a mathematical object'' my {\sl Doktorvater} once said.  Mathematical 
objects can be described in formal languages, and the descriptions can 
be studied as mathematical objects in their own right.  Interpretations 
between them can give rise to relative consistency results. There will
be a category of descriptions (theories) and interpretations between 
them.  It turns out that the notion of interpretation-between-theories 
is much more fine-grained than people used to assume (there is more 
than one notion of interpretation) and there is quite a lot of recent 
work on this set of ideas that has not been systematised.  Collating 
it will be a useful scholarly discipline.

Thinking about what kind of interpretability holds between two
mathematical theories (the two theories of partial order and of strict
partial order are in some (pretty obvious) sense the same; Boolean
rings and boolean algebras are demonstrably the same, and that's
slightly less banal (and less obvious) tho' unproblematic. But what
about the equivalence of bit strings and natural numbers?  Waves and
particles??---that's another thing altogether.  Thinking about these
equivalences will put your background mathematical knowledge to good
use, and challenge your understanding of it.

Bibliography:

Visser Oxford Slides


\url{https://www.cambridge.org/core/books/logic-in-tehran/categories-of-theories-and-interpretations/4BD83A2F040957076D0C7ABF52DF65A8}

\end{document}