 ``Shakespear, dispensing with the customary exordium, announces his subject at once in the infinitive, in which mood it is presently repeated after a short connecting passage in which, brief as it is, we recognize the alternative and negative forms on which so much of the significance of repetition depends. Here we reach a colon; and a pointed pository phrase, in which the accent falls decisively on the relative pronoun, brings us to the first full stop.''


Shaw cared (and i think this is part of why i love him so much) about
<i>language</i>.  <i>Pygmalion</i> is of course full of treasures, but
one particularly worth making a fuss about is the tea-party scene.
There are all sorts of social points being made of course, but Shaw is
also delighting in Liza's language.  Higgins is a phonetician, and is
interested only in the <i>sounds</i> that Eliza makes; he has done
nothing to work on her vocabulary and syntax.  (She would have been up
for it had her made her - Eliza is <i>bright</i>) savour  <p>
``Do I not!  Them she lived with would have killed her for a hatpin, let alone a hat''<p>
and savour particularly how Shaw has given her all those initial `h's to hiss forth.