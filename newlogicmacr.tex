\def\forces{\mathrel\mathchar"326B\mkern-9mu -}
\newcommand{\TZT}{\mbox{\mbox{{\rm T}}{\mbox{\rm{Z \kern-7.7pt Z}}}\mbox{{\rm T}}}}
\newcommand{\smallTZT}{\mbox{\footnotesize{T{\mbox{\rm{Z \kern-7.5pt Z}}}T}}}
\newcommand{\smallPp}{\mbox{\footnotesize{\Pp}}}
\renewcommand{\TZT}{\mbox{T$\Z$T}}
%\renewcommand{\TZT}{\mbox{{\rm T}}$\mathbb{Z}$\mbox{{\rm T}}}

\newcommand{\XOR}{\mbox{\, {\tt XOR} \, }}
\newcommand{\TCZ}{\mbox{TC$\Z$}}
\newcommand{\smallZZ}{\mbox{\footnotesize{{\mbox{\rm{Z \kern-7.5pt Z}}}}}}

\newcommand{\C}{\mathbb{C}} 
%\usepackage{graphicx,xcolor}
%\graphicspath{{/tmp/df224/}}
\newcommand{\Pig}{\includegraphics[width=0.7cm]{pig5}}
\newcommand{\2}{\mbox{2\hspace{-0.22cm}2}}
\newcommand{\commentblock}[1]{}
\newcommand{\iNF}{{\sl i}NF}

\newcommand{\smallpara}[1]{\begin{small}\begin{spacing}{0.8}\begin{quote} #1  \end{quote}\end{spacing}\end{small}}


%\renewcommand{\smallpara}[1]{\begin{small}\begingroup\setstretch{0.6}#1\par\endgroup\end{small}}

 % defs

\renewcommand{\tilde}{\symbol{'176}} 
 % nat deduct stuff
 \newcommand{\orint}[1]{\RightLabel{$\lor$-int}\UnaryInfC{#1}}
 \newcommand{\orelim}[1]{\RightLabel{$\lor$-elim}\TrinaryInfC{#1}}
 \newcommand{\orelimlabel}[2]{\RightLabel{$\lor$-elim ({#1})}\TrinaryInfC{#2}}
 \newcommand{\identityrule}[1]{\RightLabel{identity rule}\BinaryInfC{#1}}
 \newcommand{\identityruleU}[1]{\RightLabel{identity rule}\UnaryInfC{#1}}
 \newcommand{\epsilonint}[1]{\RightLabel{$\in$-int}\UnaryInfC{#1}}

 \newcommand{\inint}[1]{\RightLabel{$\in$-int}\UnaryInfC{#1}}
 \newcommand{\inelim}[1]{\RightLabel{$\in$-elim}\UnaryInfC{#1}}

 \def\apercu{{\sl aper\c{c}u}\ }

 \newcommand{\andelim}[1]{\RightLabel{$\land$-elim}\UnaryInfC{#1}}
 \newcommand{\andint}[1]{\RightLabel{$\land$-int}\BinaryInfC{#1}}

 \newcommand{\arrowelim}[1]{\RightLabel{$\rightarrow$-elim}\BinaryInfC{#1}}
 \newcommand{\arrowint}[1]{\RightLabel{$\rightarrow$-int}\UnaryInfC{#1}}
 \newcommand{\arrowintlabel}[2]{\RightLabel{$\rightarrow$-int ({#2})}\UnaryInfC{#1}}

   \newcommand{\classicalneg}[1]{\RightLabel{classical negation}\UnaryInfC{#1}}
 \renewcommand{\classicalneg}[2]{\RightLabel{classical negation (#2)}\UnaryInfC{#1}}

 \newcommand{\doubleneg}[1]{\RightLabel{double negation}\UnaryInfC{#1}}
 %\newcommand{\falseelim}[1]{\AxiomC{$\thefalse$}\RightLabel{ex falso sequitur quodlibet}\UnaryInfC{#1}}
 \newcommand{\falseelim}[1]{\RightLabel{ex falso sequitur quodlibet}\UnaryInfC{#1}}
 \newcommand{\rrrr}[1]{\RightLabel{Rule 4}\UnaryInfC{#1}}


 \newcommand{\necc}[1]{\RightLabel{neccessitation}\UnaryInfC{#1}}



 \newcommand{\affixing}[1]{\RightLabel{Affixing}\UnaryInfC{#1}}
 \newcommand{\mixedrule}[1]{\RightLabel{Mixed Rule}\UnaryInfC{#1}}

 \newcommand{\subst}[1]{\RightLabel{subst}\BinaryInfC{#1}}
 \newcommand{\forallint}[1]{\RightLabel{$\forall$-int}\UnaryInfC{#1}}
 \newcommand{\forallelim}[1]{\RightLabel{$\forall$ elim}\UnaryInfC{#1}}
 \newcommand{\existsint}[1]{\RightLabel{$\exists$-int}\UnaryInfC{#1}}
 \newcommand{\existselim}[2]{\RightLabel{$\exists$-elim({#2})}\BinaryInfC{#1}}


 \newcommand{\ambiguity}[1]{\RightLabel{Typical Ambiguity}\UnaryInfC{#1}}
 % aliases
 \newcommand{\modpon}[1]{\arrowelim{#1}}


 % sequent deduction stuff
 \newenvironment{sequent}{\begin{equation}\def\fCenter{\ \vdash\ }}{\DisplayProof \end{equation}}

 \newcommand{\negr}[1]{\RightLabel{$\neg$ R}\UnaryInf#1}
 \newcommand{\negl}[1]{\RightLabel{$\neg$ L}\UnaryInf#1}

 \newcommand{\inr}[1]{\RightLabel{$\in$ R}\UnaryInf#1}
 \newcommand{\inl}[1]{\RightLabel{$\in$ L}\UnaryInf#1}



 \newcommand{\arrowr}[1]{\RightLabel{$\rightarrow$ R}\UnaryInf#1}
 \newcommand{\arrowl}[1]{\RightLabel{$\rightarrow$ L}\BinaryInf#1}

 \newcommand{\orl}[1]{\RightLabel{$\lor$ L}\BinaryInf#1}
 \newcommand{\orr}[1]{\RightLabel{$\lor$ R}\UnaryInf#1}
 \newcommand{\orrone}[1]{\RightLabel{$\lor$ R $_1$}\UnaryInf#1}
 \newcommand{\orrtwo}[1]{\RightLabel{$\lor$ R $_2$}\UnaryInf#1}

 \newcommand{\andr}[1]{\RightLabel{$\land$ R}\BinaryInf#1}
 \newcommand{\andlone}[1]{\RightLabel{$\land$ L$_1$}\UnaryInf#1}
 \newcommand{\andltwo}[1]{\RightLabel{$\land$ L$_2$}\UnaryInf#1}
 \newcommand{\wedgeL}[1]{\RightLabel{$\land$ L}\UnaryInf#1}

 \newcommand{\weakr}[1]{\RightLabel{Weakening-R}\UnaryInf#1}
 \newcommand{\contrR}[1]{\RightLabel{contraction-R}\UnaryInf#1}
 \newcommand{\weakl}[1]{\RightLabel{Weakening-L}\UnaryInf#1}

 \newcommand{\existsl}[1]{\RightLabel{$\exists$ L}\UnaryInf#1}
 \newcommand{\existsr}[1]{\RightLabel{$\exists$ R}\UnaryInf#1}

 \newcommand{\forallr}[1]{\RightLabel{$\forall$ R}\UnaryInf#1}
 \newcommand{\foralll}[1]{\RightLabel{$\forall$ L}\UnaryInf#1}

 \newcommand{\cut}[1]{\RightLabel{Cut}\BinaryInf#1}
 \newcommand{\contractl}[1]{\RightLabel{Contract-L}\UnaryInf#1}

 % Make equations restart their numbering at sections

 %\usepackage{remreset}
 \makeatletter
 %\@addtoreset{equation}{section}
 %\@addtoreset{equation}{subsection}
 \makeatother



% \newcommand{\say}{Say something about }
\newcommand{\unit}{\mbox{1 \kern-7.2pt 1}}
%\renewcommand{\unit}{\mathbb 1}
 \def\Proof{\noindent{\sl Proof:\/ }}
 \newcommand{\bic}{\longleftrightarrow}
 \newcommand{\M}{{\mathfrak M}}
  \newcommand{\Rez}{{\cal R}}
  \newcommand{\N}{{\mathfrak N}}
 \newcommand{\proof}[2]{\frac{\displaystyle {#1}}{\displaystyle {#2}}}
 \def\endproof{\nobreak\hfill{\rule{2mm}{2mm}}\medskip}
 \newcommand{\argument}[2]{\proof{\mbox{\rm {#1}}}{\mbox{\rm {#2}}}}
 \newcommand{\scripts}{\kern-2.7pt \int \kern-2.7pt}

\def\KF{\mbox{\rm KF}}
\def\KFI{\mbox{\rm KFI}}
\def\TST{\mbox{\rm TST}}
\def\TSTI{\mbox{\rm TSTI}}
\def\NF{\mbox{\rm NF}}
\def\ZF{\mbox{\bf Z}}
\def\ZFC{\mbox{\rm ZFC}}
\def\Z{\mbox{\rm{Z \kern-7.5pt Z}}}
\def\AC{\mbox{\rm AC}}
\def\TCl{\mbox{\rm TCl}}
\def\TCo{\mbox{\rm TCo}}
\def\Mac{\mbox{\rm Mac}}      
\def\Con{\mbox{\rm Con}}      
\def\Amb{\mbox{\rm Amb}}      
\def\AxInf{\mbox{\rm AxInf}}      
\def\foundation{\mbox{\rm foundation}}      
\def\can{\mbox{\rm can}}
\def\stcan{\mbox{\rm stcan}}
\def\Ala{As long as}
\def\ala{as long as}
\def\apercu{{\sl aper\c{c}u}\ }
%various other new commands/definitions:
\def\Proof{\noindent{\sl Proof:\/ }}
\def\endproof{\nobreak\hfill\blob\medskip}
\def\noproof{\nobreak\hfill\blob}
\def\eqdef{=_{\rm def}}     % intended to be used in Maths mode
%\DeclareMathSymbol{\strictif}{\mathrel}{symbolsC}{74}
%\newtheorem{thm}{Theorem}
\newtheorem{thm}{\mbox{T\footnotesize{HEOREM\ }}}
\newtheorem{factoid}{\mbox{F\footnotesize{ACTOID\ }}}
\newtheorem{rem}{R\mbox{\footnotesize{EMARK\ }}}
\newtheorem{prop}{P\mbox{\footnotesize{ROPOSITION\ }}}
\newtheorem{lem}{L\mbox{\footnotesize{EMMA\ }}}
\newtheorem{dfn}{D\mbox{\footnotesize{EFINITION\ }}}
\newtheorem{sublem}{S\mbox{\footnotesize{UBLEMMA\ }}}
\newtheorem{coroll}{C\mbox{\footnotesize{OROLLARY\ }}}
\newtheorem{Exercise}{E\mbox{\footnotesize{XERCISE\ }}}
\newtheorem{conjecture}{C\mbox{\footnotesize{ONJECTURE\ }}}
\newtheorem{question}{O\mbox{\footnotesize{PEN QUESTION\ }}}
\newtheorem{examquestion}{\mbox{\footnotesize{QUESTION\ }}}
\newtheorem{challenge}{C\mbox{\footnotesize{HALLENGE\ }}}
\newcommand{\mem}{\mbox{\tiny${\cal MEMBER\ }$}}
\newcommand{\set}{\mbox{\tiny${\cal SET\ }$}}

\newcommand{\Let}[3]{{\rm let}\ { #1} \mbox{$\Leftarrow$} { #2}\ {\rm in}\ { #3}}


\newcommand{\Henkin}[4]{^{(\forall{#1})(\exists{#2})}_{(\forall{#3})(\exists{#4})}}

\newcommand{\nf}{\mbox{\sl NF}}
\newcommand{\NFO}{\mbox{\sl NFO}}
\newcommand{\NFC}{\mbox{\sl NFC}}
\newcommand{\TNT}{\mbox{\sl TNT}}
\newcommand{\tnt}{\mbox{\sl TNT}}
\newcommand{\On}{\mbox{\sl On}}
\newcommand{\tnti}{\mbox{\sl TNTI}}
\newcommand{\tntu}{\mbox{\sl TNTU}}
\newcommand{\tsti}{\mbox{\sl TSTI}}
\newcommand{\tstu}{\mbox{\sl TSTU}}
\newcommand{\tst}{\mbox{\sl TST}}
\newcommand{\nc}{\mbox{\sl NC}}
\newcommand{\NCI}{\mbox{\sl NCI}}
\newcommand{\NFU}{\mbox{\sl NFU}}
\newcommand{\NO}{\mbox{\sl NO}}
\newcommand{\zf}{\mbox{\sl ZF}}
\newcommand{\CUS}{\mbox{\sl CUS}}
\newcommand{\BF}{\mbox{\sl BF}}
\newcommand{\No}{\mbox{\sl No}}
\newcommand{\SM}{\mbox{\sl SM}}
\newcommand{\WF}{\mbox{\sl WF}}
\newcommand{\USC}{\mbox{\tt USC}}
\newcommand{\RUSC}{\mbox{\tt RUSC}}
\newcommand{\GC}{\mbox{\sl GC}}
\newcommand{\kf}{\mbox{\sl KF}}
\newcommand{\mac}{\mbox{\sl MAC}}
\newcommand{\WC}{\mbox{\sl WC}}
\newcommand{\WO}{\mbox{\sl WO}}
\newcommand{\say}{Say something about }
\newcommand{\Los}{ \L o\'s's theorem\ }
\newcommand{\Dz}{Dzierzgowski}
\newcommand{\zi}{\Nn_{\rm Zm}}
\newcommand{\vno}{\Nn_{\rm vN}}
\newcommand{\vns}{S_{\rm vN}}
\newcommand{\gnumber}{G\"odel number}
\newcommand{\LTTT}{{\cal L}(TTT)}
\renewcommand{\partial}{\rightharpoondown}

\newcommand{\dual}[1]{#1^\circlearrowleft}
\def\LTST{{\cal L}_{\rm TST}}  %only works in maths mode 
\def\Lang{{\cal L}}            %only works in maths mode 
\newcommand{\aiw}{as it were}

\def\joinrelm{\mathrel{\mkern-3mu}}
\def\relbar{\mathrel{\smash-}}
\def\tailpiece{\vrule height 1ex width 0.3ex depth -.1ex} 
\def\seqsym{\mathrel{\tailpiece\joinrelm\relbar}}
\def\sequent#1#2{#1 \seqsym #2}

%Example:
%$$\sequent{\Gamma}{\Delta}$$


%\newcommand{\tsub}{\pile{\lower0.5ex\hbox{.} \\ -}}
\newcommand{\equal}{\mbox{{\LARGE{$=$}}}}
\newcommand{\notequal}{\mbox{{\LARGE{$\not =$}}}}
\newcommand{\canon}{\subset_\infty}

%On 28 March 2010 20:38,  <T.Forster@dpmms.cam.ac.uk> wrote:
%> What is the favoured \LaTeX macro for the underlined cdot wot means bounded
%> subtraction?

%If you use the excellent web tool

%http://detexify.kirelabs.org/classify.html

%to try to find the answer to this question, you find, surprisingly,
%that there is no pre-packaged symbol for this. I needed the standard
%symbol for truncated subtraction for my Computation Theory lectures
%and Robin Fairbairn kindly supplied the following:

\newcommand{\tsub}% truncated subtraction
{\ensuremath{\mathbin{\hbox{      \ooalign{\hfil\raisebox{4pt}{$\cdot$}          \hfil\cr\hfil$-$\hfil}}}}}
% texery via Robin Fairbairns

%which is OK provided you don't want to sub- or super-script this symbol.

%Best wishes,

%Andy


\newcommand{\splat}{\tsub}

\newcommand{\dfunset}{\mapsto\kern-7.5pt\to}
\renewcommand{\Re}{\hbox{\rm I\negthinspace R}}
%\newcommand{\Rez}{\mbox{\cal R}}
\newcommand{\K}{\hbox{\rm I\negthinspace K}}

%Or without serif, \newcommand{\dfunset}{\to\kern-7.5pt\to}
\newcommand{\diverges}{ \kern -4.5pt \uparrow}
\newcommand{\converges}{ \kern -4.5pt \downarrow}
\newcommand{\cons}{\kern-2.7pt::\kern-2.7pt}

\newcommand{\blob}{\rule{2mm}{2mm}}
\newcommand{\tfae}{the following are equivalent}
\newcommand{\AxC}{\mbox{{AxCount$_{\leq}$\ }}}
\newcommand{\Fraisse}{Fra\"\i ss\'e}
\newcommand{\restric}{\kern-2pt\upharpoonright\kern-3pt}
\newcommand{\tuple}[1]{\mbox{$\langle #1 \rangle$}}
\newcommand{\thefalse}{\perp}
\newcommand{\Nn}{\hbox{\rm I\negthinspace N}} %Richard's modification
\newcommand{\Pp}{\hbox{\rm I\negthinspace P}}
\newcommand{\poss}{\mbox{{\Large{$\diamond$}}}}
\newcommand{\true}{\tt true}
\newcommand{\false}{\tt false}             
\newcommand{\onto}{\to\kern-7.5pt\to}
\newcommand{\ontoreverse}{\rightarrow\kern-7.5pt\rightarrow}

\newcommand{\ivo}{\mbox{in virtue of }}
\newcommand{\lp}{\mbox{\rm (}}
\newcommand{\rp}{\mbox{\rm )}}
\newcommand{\bool}{\hbox{\rm I\negthinspace B}} %Richard's modification
\newcommand{\smallNn}{{\rm I\!N}} %to be used in maths mode as subscript
\newcommand{\smallRe}{{\rm I\!R}} %to be used in maths mode as subscript
\newcommand{\smallbool}{{\rm I\!B}} %to be used in maths mode as subscript
%\newcommand{\ZZ}{\hbox{\rm Z\negthinspace Z}} %Richard's modification
%\renewcommand{\ZZ}{\mathbb{Z}} %Richard's modification
\newcommand{\ZZ}{\mbox{\rm{Z \kern-7.5pt Z}}}
\renewcommand{\ZZ}{\mathbb{Z}}
\renewcommand{\Z}{\mathbb{Z}}
\newcommand{\naive}{na\"{\i}ve}
\newcommand{\Naive}{Na\"{\i}ve}
\newcommand{\belongs}{\mbox{$\footnotesize{\ \cal B\ }$}}
%\renewcommand{\choose}[2]{(\small{\stackrel{\scriptstyle #1}{\scriptstyle #2}})}

\newcommand{\card} [1]{|{#1}|}

\newcommand{\tree}{\mbox{{\Large{$\tau$}}}}
\newcommand{\KAPPA}{\mbox{{\LARGE{$\kappa$}}}}
\newcommand{\pow}{{\cal P}(}
\newcommand{\powk} [1]{{\cal P}_{#1}(}
%\newcommand{\emptyset}{\mbox{\verb#0#}}
\newcommand{\subend}{\subseteq^{\cal P}_e}
\newcommand{\injend}{\hookrightarrow^{\cal P}_e}
\newcommand{\hded}{H_{\mbox{{\footnotesize \rm Dedfin}}}}
\newcommand{\SiP}{\Sigma^{\cal P}}
\newcommand{\PiP}{\Pi^{\cal P}}
\newcommand{\DeP}{\Delta^{\cal P}}
\newcommand{\RR}{{\bf{R}}}
\newcommand{\QQ}{{\bf{Q}}}
\renewcommand{\QQ}{\mathbb{Q}}
\newcommand{\1}{{\bf{1}}}
\newcommand{\Wlog}{Without loss of generality}
\newcommand{\wwlog}{without loss of generality} 
\newcommand{\hole}[1]{[{\sl HOLE \index{HOLE} {#1}}]}

\newcommand{\smallhole}[1]{\begin{quote}\begin{small}{\sl HOLE \index{HOLE} {#1}}\end{small}\end{quote}}
\newcommand{\comment}[1]{\begin{quote}{\sl {#1}}\end{quote}}
\newcommand{\bighole}[1]{\begin{quote}[{\sl HOLE \index{HOLE}{#1}}]\end{quote}}
\newcommand{\inn}{\in_{\footnotesize{\mbox{\sl new}}}}
\newcommand{\m} [1]{\mbox{$\langle \langle { #1} \rangle \rangle$}}
\newcommand{\cnvg}{\downarrow =}
\newcommand{\inj}{\hookrightarrow}
\newcommand{\E}{{\cal E}}

%\newcommand{\N}{{\cal N}}
\newcommand{\U}{{\cal U}}
\newcommand{\V}{{\cal V}}
\newcommand{\A}{{\cal A}}
\newcommand{\B}{{\cal B}}
\newcommand{\fst}{{\mbox{\tt fst}}}
\newcommand{\snd}{{\mbox{\tt snd}}}
\newcommand{\one}{{\mbox{\tt I}}}
\newcommand{\two}{{\mbox{\tt II}}}

\newcommand{\footI}{{\mbox{\footnotesize{\it I}}}}
\newcommand{\footII}{{\mbox{\footnotesize{\it II}}}}


\newcommand{\bbar}[1]{\overline{B{#1}}}
\newcommand{\bbarr}{\overline{B}``}

\newcommand{\pair}{{\mbox{\tt pair}}}

\newcommand{\JT} [1]{\mbox{${ #1}$-{ary}}}

\newcommand{\rhobeta}{\rho\negthinspace\beta}

%Miscellaneous stuff that has been deleted but might be undead

%\renewcommand{\proof}[2]{\frac{\displaystyle {#1}}{\displaystyle {#2}}}

%quine quotes
%\newlength\ququ
%\newbox\quinebox
%\def\lquinequote{\rule[\ququ]{.1ex}{.5ex}
%\addtolength{\ququ}{.4ex}
%\rule[\ququ]{.5ex}{.1ex}}

%\def\rquinequote{\rule[\ququ]{.5ex}{.1ex}
%\addtolength{\ququ}{-.4ex}
%\rule[\ququ]{.1ex}{.5ex}}

%\def\lquine#1\rquine{%
%\ifinner\setbox\quinebox=\hbox{$\displaystyle #1$}
%\else\setbox\quinebox=\hbox{$#1$}
%\fi
%\setlength{\ququ}{\ht\quinebox}
%\addtolength{\ququ}{-.2ex}
%\lquinequote #1\rquinequote}
%\renewcommand{\lquine}{\ulcorner}
%\newcommand{\rquine}{\urcorner}


 %\input diagrams
%\font\tenmsym=msbm10
%\font\sevenmsym=msbm7
%\font\fivemsym=msbm5
%\newfam\msymfam
%\textfont\msymfam=\tenmsym
%\scriptfont\msymfam=\sevenmsym
%\scriptscriptfont\msymfam=\fivemsym
%{\newcount\n \n=\msymfam \multiply\n"100\advance\n'151
%\global\mathchardef\beth\n}



%\newcommand{\smallTZT}{\mbox{\footnotesize{T\Z T}}}

\renewcommand{\b}{\mbox{\rotatebox[origin=c]{170}{\reflectbox{\cal P}}}}
\newcommand{\skull}{\includegraphics[width=4mm]{kindpng_1191141.png}}
\newcommand{\sav}{\includegraphics[width=4.8mm]{BF2.png}}
%\def\N{\mathbb{N}}


