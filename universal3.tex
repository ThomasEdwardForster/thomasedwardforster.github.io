\documentclass[a4,10pt]{book}
 \usepackage{amssymb}
 \usepackage{amsbsy}
 \usepackage{amsmath}
 \usepackage[english]{babel}
 \usepackage{bussproofs}
 \usepackage{boxedminipage}
 \usepackage{latexsym}
 \usepackage{graphicx}
 \usepackage{makeidx}
 \usepackage{lscape}
\usepackage{hyperref}
\usepackage{wasysym}
\usepackage{pifont}
\usepackage{latexsym}
\usepackage{graphicx}
\usepackage{pgf}
\usepackage{tikz}
\usetikzlibrary{arrows,automata}
\usepackage[latin1]{inputenc}
\usepackage{dingbat}
\usepackage[utf8]{inputenc}
\usepackage{pgf,tikz}
\usetikzlibrary{arrows}
\input newlogicmacr
%\renewcommand{\twoheadrightarrow}{\to\kern-7.5pt\to}
%\newcommand{\ZZ}{\hbox{\rm Z\kern-7.5pt Z}} 
\renewcommand{\N}{\mathfrak N} 
\title{Scrapbook on Set Theory with a Universal Set}
\author{Thomas Forster}
\begin{document}
\maketitle

\sloppy
\tableofcontents
\vfill



\chapter{Stuff to fit in somewhere}

\subsection*{Spectra (well, i've got to call them {\sl something}}

Let $\M$ be a model of TST, and $\phi$ a stratifiable expression of
the language of set theory.  Let the {\sl spectrum} of $\phi$ (in
$\M$) be the set of $n \in \Nn$ such that $\phi$ is true at level $n$.
In pursuit of Con(NF) we want models $\M$ such that every spectrum is
finite or cofinite.  It might be an idea to consider what sort of
subsets of \Nn\ can be spectra.  Why do i have the feeling that the
Thue-Siegel set cannot be a spectrum?  Is it possible to arrange that
every spectrum is almost-periodic (periodic except at finitely many
points)?   I'm guessing that it is, and that that is compatible with AC.




\subsection*{There should really be a chapter on the symmetric group on $V$}

Things to tie together


Nathan's result that there are so few normal subgroups

Nathan's poset and sups of subsets of it

Normal generating subsets for permutation models.  Practically every conjugacy class generates the whole group

All the group theory that crops up in the attempt to obtain a strongly extensional model of $\TZT$ by OT

\bigskip

If $\sigma$ is a permutation then $j(\sigma)$ has lots of fixed
points.  $\sigma$ partitions $V$ into cycles, each of which is
countable, so it partitions $V$ into countable pieces.  There can't be
too few of them.  Every union of (any subset of) these pieces is a
fixed point for $j(\sigma)$.  It might be an idea to do some actual
calculation.


Dunedin mon 17/vii/2017  I seem to be cracking up

The result of Nathan's and mine that every orbit of a point in a model
of \TZT\ is either a singleton or is infinite relies on the symmetric
group on an infinite set having no normal subgroups of finite index.
But in the finite case that doesn't hold: the alternating subgroup is
normal and of index 2.  So let us consider a model of TST with finite
bottom level, and find ourselves an element of level $n$ whose orbit
has two elements. This ought to be easy but i am getting confused.

\smallskip

tuesday morning.  Let's have another look.  We are looking for two
things $A$ and $B$ such that even permutations fix them and odd
permutations swap them.  This is actually the same as saying that they
are exchanged by each single transposition! Now {\sl that} sounds like
something one can get one's teeth into.

The hope is that we can prove that this can't happen.  Well, actually
it can, co's level 0 could have precisely two atoms $a$ and $b$ with
$\{a\}= \bigcup^n A$ and $\{b\}= \bigcup^n B$.


Now if $A$ and $B$ are such a pair so are $A \setminus B$ and
$B \setminus A$ so \wwlog\ $A \cap B = \emptyset$.

We need to think also about which atoms (things of level 0) are 
in $\bigcup^n A$ and $\bigcup^nB$. Suppose $x$ is in $\bigcup^nA$ 
but $y$ is not.  Consider the transposition $(x,y)$.  This sends 
$A$ to $B$ and therefore must send $\bigcup^n A$ to $\bigcup^n B$. 
So $\bigcup^n B$ must be $\bigcup^n A \setminus \{x\} \cup \{y\}$.
Clearly both $\bigcup^nA$ and $\bigcup^n B$ must contain all atoms.

Suppose $x\not=y$ and both are in $\bigcup^n A$.  Swap them, 
you get $\bigcup^n B$, so they're both in $\bigcup^n B$.  So. 
if $|\bigcup^n A| \geq 2$, then $\bigcup^n A = \bigcup^n B$.

\medskip

OK, so $\bigcup^n A$ is the whole of the bottom level.  Some light
dawns.  Consider the equivalence relation on total orders of level 0
that makes two total orders equivalent if you can take one to the
other by means of an even permutation, and think about the partition
into equivalence classes.  The partition is a definable set and is
therefore symmetric, which is to say that its singleton is an orbit of
the full symmetric group.  It is also a union of orbits of the
alternating group.  How many orbits?  That depends on whether or not
level 0 is finite.  If it is finite then it is a single orbit with two
elements.  Observe that neither of these two elements is
symmetric/definable.

You might think, Dear Reader, that we can cast this entirely in terms
of permutations instead of total orders since (in the finite case)
they are in 1-1 correspondence.  Perhaps the two-element orbit is the
set of cosets of the alternating group\ldots?  No, beco's the two
cosets are definable.  The bijection between the set of total orders
and the symmetric group is not natural.

The point about total orders in this context is that every total order
of a finite set is rigid, and that they are all pairwise isomorphic.


Nothing mysterious about any of that, really.

\bigskip

Suppose $|[x]_n| = \kappa$.
Then there is an action of $J_n$ on a set of size $\kappa$ and
therefore a group homomorphism to a subgroup of Symm$([x]_n)$.
$|$Symm$([x]_n)| \leq 2^{\kappa^2}$, so this subgroup is of size
$2^{\kappa^2}$ at most.  So $J_n$ has a normal subgroup of index at
most $2^{\kappa^2}$.


I think there is a unique maximal normal subgroup of $J_0$ and that is
the subgroup of small support, where a set is small as long as it
doesn't map onto $V$.  We need to know the index of this subgroup.
How many cosets are there?  Two permutations will belong to different
cosets---will be {\sl dissimilar}---if the set of arguments on which
they differ is not small.  How large can a set of pairwise dissimilar
permutations be?  Well, for each $x$, consider $\bbar{x}$; it's a
power set algebra.  Consider the complementation of the power set
algebra. These local complementations are all pairwise dissimilar and
there are at least $T^2|V|$ of them. 


But we can probably do better than that.  Let's build a family of
pairwise dissimilar permutations all of which are subsets of the
complementation permutation.  In fact, let's make them all disjoint.
Let's have a prime ideal, so we can think of each of these
permutations $\pi$ as an element $p$ of the ideal: it swaps members of
$p$ with their complements and fixes everything else.  Let the prime
ideal be $\pow (V \setminus\{\emptyset\})$.  We can split $V \setminus
\{\emptyset\}$ into $T|V|$-many things each of size $V$, so we can
certainly find $T|V|$-many pairwise dissimilar permutation.  But this
still gives us only $T^2|V|$ cosets.  As i say, we should be able to
do better.

Here's another.  Fix a permutation $\pi$ of small support.  We can
make $T|V|$ disjoint copies of $V$: $V \times \{v\}$ for each $v \in
V$.  $\pi$ acts on $V \times \{V\}$ by acting on the first components.
This gives us a set of $T|V|$ pairwise dissimilar permutations.  If we
can find a family ${\cal F}$ all uniformly of size $|V|$ of sets whose
pairwise intersection is small then we can have a set of pairwise
dissimilar permutations of size $|V|$.



\subsection*{A Question of Adam Lewicki's}

Adam Lewicki reminds me that there seems nowhere to be a proof that
$|V \to V| = |V|$.  Clearly all we need to do is inject $V$ into 
$V \to V$.  Send $X$ to 
\begin{center}
$\lambda x.$\verb# if #$x \in X$\verb# then #$x$\verb# else #$V \setminus x$
\end{center}

7/v/2017

I don't know why i hadn't thought of this earlier, but Adam Lewicki
has, and has made me think about it.  Using Quine pairs every set is a
set of ordered pairs, so the function that takes $x$ and $y$ and
returns $x``y$ is well-defined and total---and homogeneous!  What kind
of algebra do we get? The operation clearly has a left-unit, which is
just the identity relation, $\{\tuple{x,x}: x = x\}$.  What about $K$
and $S$?  We don't get $K$---we don't even get $Kx$ for any $x$ that
isn't a singleton \ldots and presumably not $S$ either.  What {\sl do}
we get?  Have you thought about this?

It's quite disgraceful that i have known about Quine pairs and about
lambda calculus for years and have never thought about this algebra.
How can i show my face in public?

What is $V``x$? Presumably it is $V$ unless $x$ is empty.  $x``V$?

There is the set $\{\tuple{\tuple{x,y}, (x \times y)}:x,y \in V\}$.
That does something nice.

\section*{Music minus one}
\commentblock{While invigilating in MR3, afternoon of 5/vi/2017} For
any formula $\phi$ in ${\cal L}(TST)$ we can cook up a formula
$\phi^*$ which says that $\phi$ holds in the model obtained by
removing a single element from level zero.  We fix some thing $a$ at
level 0; then we replace all occurrences of `$(\exists x_0) \ldots $'
by `$(\exists x_0)(x_0 \not= a \wedge \ldots$' and replace all
occurrences of `$(\forall x_0) \ldots $' by `$(\exists x_0)(x_0 \not=
a \to \ldots$' similarly at higher levels.  Then we bind `$a$' with a
quantifier.  Since it doesn't matter which thing we delete the
quantifier can be whichever of $\exists$ and $\forall$ we find more
convenient.  Now we need to think about the scheme $\phi \bic \phi^*$.
It certainly follows from the assertion that the bottom level is
Dedekind-infinite.

We should really show that * (or whatever we end up calling it) commutes with Booleans

\subsection*{$B(x)$ and foundation}

\begin{rem}
The existence of $B(x)$ contradicts foundatation.  
\end{rem}
This is obvious of you have $\bigcup$, beco's $\bigcup B(x) = V$
always.  I suspect that you need $\bigcup$ to get $V$, but one can
contradict foundation just with the principle that Allen Hazen calls
{\sl adjunction}. (Or was it {\sl insertion}?)

Let's set this up the right way round.
\begin{tabbing}
$a \in \{a\}$\ \ \ \ \ \ \ \ \ \ \ \ \ \ \ \ \ \  \ \ \ \ \ \ \ \ \= definition of singleton;\\
\\
$a \in (B(a) \cup \{a\})$\> monotonicity of $\subseteq$; exists by adjunction\\
\\
$(B(a) \cup \{a\}) \in B(a)$\> definition of $B(a)$;\\
\\
$(B(a) \cup \{a\}) \in (B(a) \cup \{a\})$\> monotonicity of $\subseteq$.\\
\end{tabbing}

\endproof

So the existence of $B(x)$ implies the existence of self-membered sets as long as we have adjunction.


\section*{A Salutory Tale about Stratification, Variables and Recursive Definitions}


Alice told me I should write this up.\marginpar{This has got garbled: sort it out}

\smallskip


We always have $x \subseteq {\cal P}(\bigcup x)$.  Indeed we have $x
\subseteq {\cal P}^n(\bigcup^n x)$ for every (concrete) $n$.  And
these assertions are stratifiable.  There is the thought that we might
obtain the union $\bigcup_{n \in \smallNn}{\cal P}^n(\bigcup^n x)$.
Let's call this object $F(x)$ and hope to prove that it always exists.
Values of $F$ look a bit like Zermelo cones, which is why they are
interesting.  $F(x)$ looks like a kind of natural environment for $x$.

Consider the function $$f(n,x) = {\cal P}^n(\bigcup^n x).$$ It looks
as if we should be able to define it in NF; after all, for each
concrete $n$, `$x = f(n,x)$' is stratified.  So we can, for every
concrete $n$, prove that $(\forall x)(f(n,x)$ exists$)$.  Indeed we
can even, for each concrete $n$, prove the sethood of the graph
$\{\tuple{x,y}: y = f(n,x)\}$.  We can even prove further that if $g$ is any
function that is a set, the function $x \mapsto {\cal P}(g(\bigcup
x))$ is also a set!  What we can't do is prove the same about $f$ with
`$n$' a variable.

This merits reflection.

So let us try to declare $f$ by recursion on \Nn.  Thus $f(0,x) =: x$;
$f(n+1, x) = {\cal P}(f(n,\bigcup x))$.  That is to say, $f$ is the
$\subseteq$-least set of triples extending $\{\tuple{0,x,x}: x \in
V\}$ and closed under the operation $\tuple{n,\bigcup x,y} \mapsto
\tuple{n+1, x, {\cal P}(y)}$.  Observe, however, that this operation we are
closing under is not stratified.

\begin{equation}\tag{PHI}
(\forall y)(a \in y \wedge (\forall x)(x \in y \to (\forall z)(\phi(x,z) \to z \in y)))
\end{equation}

and if we want this inductively defined collection to be a set then
PHI had better be stratified.  But of course it will be stratified
only if $\phi$ is homogeneous.   In the recursive declaration of $f$ 
above $\phi$ relates $\tuple{n,\bigcup x,y}$ to $\tuple{n+1,x,{\cal P}(y)}$.  

\smallskip

So we can't be sure that the graph of $f$ is a set.  Can we be sure
that it isn't?  Suppose it were. Then we would have the graph of the
function $x\mapsto \bigcup_{n \in \smallNn}f(n,x)$.  Let's call this
function $F$ as above. 

My guess is that the graph of $F$ cannot be a set. However I am having 
more trouble with this than i expected.  Randall says that if it is
consistent that every transitive set is either $V$ or is hereditarily
finite then the graph of $F$ might be a set.  That doesn't {\sl quite}
work as it stands beco's if $F$ is a set then $\{x: F(x) \not= V\}$
(which is definitely a set) looks as if it might be $V_\omega$\ldots but 
the point is well-made.

Consider now the function $G(x)= \bigcup\{y: F(y) \subset x \wedge
F(y) \not= x\}$.  The graph of $G$ is a set, too.  Check that $G$ is
$\subseteq$-monotone.  So by Tarski-Knaster it has a greatest fixed
point.

\smallskip

Thinking aloud\ldots

Suppose $Y$ is a fixed point.  Then $Y = \bigcup\{X: Y \not=F(X) \subset
Y\}$.  But $X \subseteq F(X)$ so this is $Y = \bigcup\{F(X): Y
\not=F(X) \subset Y\}$.   I don't seem to be reaching a contradiction.

\medskip

Of course the desired $F$ is a fixpoint for the operation that sends a
function $H$ to $\lambda x. {\cal P}(H(\bigcup x))$.  This is a
type-raising operation, and there is a theorem about fixed points for
type-raising operations. If we can find $x$ s.t. $x$ and op$(x)$ are
$n$-equivalent for some $n$, then in a permutation model we have a fixpoint.


\bigskip

\bigskip

\bigskip

Let $\M$ be a model of \TZT.  Pick out finitely many elements; we want
to find a substructure of $\M$ containing those elements, and we want
the substructure to be an isomorphic copy of the canonical model of
TST with empty bottom level. Key observation (thank you Arran
Fernandez!) is that whenever we have a set $A$ of sets, with a set $D
\subseteq \bigcup A$ of discriminators (which is to say that whenever
$a \not=a' \in A$ then $(a$\verb# XOR #$b\ \cap D) \not= \emptyset$)
then, for any $a \not\in A$, the set $A \cup \{a\}$ has a
discriminator obtained by adding at most one new element to $D$.  As
Fernandez says, this means that, since any two distinct sets can
always be distinguished by any one element of the symmetric
difference, we can prove by induction that $n\in \Nn$ distinct sets
can always be distinguished by $n-1$ suitably selected members of
their union.

This means that we can add new elements to our original stock of
chosen elements of $\M$, descending, and eventually we will be down to
a single discrminator, and then none.  So we have a substructure of
$\M$ which contains all our chosen elements. It's extensional, and
it's finite, but it is not yet (an isomorphic copy of) the canonical
model with empty bottom level.  And it's certainly not transitive!  We
now close under \ldots what exactly?  Any level is a boolean algebra
under $\subseteq$, $\emptyset$, $\vee$ etc so---working upwards from
the lowest level that our activities have populated---we (i) expand
each level-of-our-construction to a sub-boolean-algebra of that level
of $\M$.  (ii) We then populate the next level up with all subsets of
the level we have just processed, and (iii) we add to the level {\sl
  two} steps up, $B(x)$ for all $x$ that we have constructed.

The result is an (intransitive) copy of the canonical model, which is a
substructure of $\M$ closed under the boolean operations, $\iota$ and
$B$.  Being thus closed, it is a substructure elementary for more than just $\in$


\subsection*{A Conjecture about Permutation Models}  

Presumably the following is true: Whenever $\Sigma$ is a stratified
$n$-type that is realized by an $n$-tuple of wellfounded sets then
there is a permutation model in which $\Sigma$ is realized by an
$n$-tuple of illfounded sets.

%20/xii/2015

Let $\Sigma$ be the $n$-type realized by all $n$-tuples of wellfounded
sets.  That is to say, $\Sigma$ is the set of all the $\sigma(x_1
\ldots x_n)$ s.t. NF $\vdash (\forall \vec x)(WF(\vec x) \to
\sigma(\vec x))$.  Can we suppose that no $n$-tuple of illfounded sets
realizes it?  What does the type contain? `$(\forall y)(y \in x)$' for
one.  More generally $x \not= \{y: \phi(y)\}$ for most stratified
$\phi$ with one free variable. So what {\sl was} the correct question?


\bigskip

I've had this thought more than once \ldots copying this in from another file

\smallskip

Can we characterise sensible {\sl versus} silly illfounded sets?  A
Quine atom is illfounded for a silly reason, and for every $n$ there
is a wellfounded set that is $n$-similar to it.  That is nature's way
of trying to tell you that it ought to be wellfounded.  I think that
is the condition we want.  That is to say, you are a silly illfounded
set iff you are in the completion of the topology on the wellfounded
sets given by the symmetry classes. Let's spell this out a bit.  We
have a notion of $n$-equivalence, which can be either the standard NF
version using permutations or the (possibly subtly different) version
in Church.  Anyway, take the equivalence classes to be the basic
closed (we do mean closed, not open\ldots ?) sets of a topology.  We
then complete it, thereby adding lots of illfounded sets. These
illfounded sets are all silly, useless illfounded sets, not
inhabitants of the attic.



I think i am correct in saying that these are precisely the illfounded
sets that can be added to a model of ZF by (Rieger-Bernays) permutation 
methods.  I wonder if NF has a model in which all sets that are illfounded 
are properly illfounded.  I think this would be a consequence of the 
assertion that if $x$ is not wellfounded then, for some concrete $n$, 
$\bigcup^n x = V$.

Having $V$ in your transitive closure is a sufficient condition for
not being wellfounded.  It's a sufficient condition even for the
status of not being, for every $n$, $n$-equivalent to a wellfounded
set.

Can we find an omitting types model in which \ldots if for every $n$,
$x$ is $n$-similar to a wellfounded set then $x$ is actually
wellfounded?  Call this property $\infty\psi$wf. If all your members
are $\infty\psi$wf are you $\infty\psi$wf too?

We can certainly try to omit the $1$-type that says that, for each $n$, 
there is a welfounded set that is $n$-similar to $x$ while insisting 
that $TC(x \not= V$.

Or, again, by OTT we might perhaps obtain models in which, or all $x$,
if $x$ is, for each $n$, $n$-similar to a wellfouded set, then it is
itself wellfounded.


\bigskip

\bigskip


%24/v/2015

Let $\kappa$ be a cardinal of very high rank, like {\sl much} bigger
than $\aleph(2^{\aleph_0})$.  Consider its tree.  There is an
equivalence relation on cardinals which says $\alpha \sim \beta$ iff
$\tuple{\tuple{\alpha}}$ and $\tuple{\tuple{\beta}}$ are elementarily
equivalent.  This give us a quotient of $\tree{\kappa}$, in the sense
that the function sending a cardinal to its equivalence class is a
graph homomorphism.

What sort of things can happen?  We have a concept of {\sl layer} in
this tree.  If two cardinals from different layers are equivalent then
we get a model of TST + Amb$^n$ for some $n$, and this we like.  So
suppose we don't.  

Let's look closely at the quotient.  The equivalence relation is finer
than the equivalence relation ``belong to the same layer'' by
assumption.  Is the quotient a wellfounded tree?  If it isn't then we
have an infinite path through it, and that gives us a rather special
extension of \TZT, which is a second thing we should consider (might
be useful).

So suppose neither of those aces take a trick; what will we be left
with?  We have a wellfounded tree, but this time it's a tree of
theories, not a tree of cardinals, and it is of cardinality at most
$2^{\aleph_0}$.  Doesn't seem to do anything \ldots

\subsection*{Coequalisers}
 
The category of NF sets has coequalisers iff every partition injects
into $\iota``V$.  My guess is that this assertion is independent of NF
but is not strong.

Might there be any hope of proving it?  Who knows! After all i saw no
hope of proving that there are precisely as many pairs as singletons
(and nor did Specker!) until Nathan showed us how to do it.

For $\alpha$ a reasonably small cardinal $\alpha$ (as-a-set) must be
the same size as $\iota``V$.  ``Finite'' is certainly sufficient.  (It
follows from Nathan's work that $|FIN| \leq T|V|$). So if a partition
$\mathbb{P}$ has $|\mathbb{P}| \not\leq T|V|$ then it must have some
infinite pieces.  One might think there is some leverage in that the
larger the pieces in a partition the fewer there can be of them, but
it doeesn't do very much for us. Just how little it does is
illustrated by the following factoid: If $\Pi$ is a partition of $V$
then $\{V \times p: p \in \mathbb{P}\}$ is a partition of $V$ the same
size as $\mathbb{P}$ all of whose pieces are of size $|V|$.  So if
there is a bad partition there is a bad partition every one of whose
pieces is as big as can be!

Reflect that, in general, if $\mathbb{P}_1$ and $\mathbb{P}_2$ are
partitions of $V$ then $\{p_1 \times p_2: p_1 \in \mathbb{P}_1 \wedge
p_2 \in \mathbb{P}_2\}$ is also a partition of $V$, and there are
natural embeddings \ldots

Let us say that an equivalence relation on $V$ is {\sl of small index}
if the quotient injects into $\iota``V$.  Then an intersection of two
equivalence relations of small index is another equivalence relation
of small index.
 
\subsection*{A conversation with Zachiri, Gothenburg 15th april 2015}

We know that Zermelo can have models in which every set of
infinite-sets-all-of-different-sizes is finite, but all known such
models are models of AC.  This raises the question: does Zermelo +
$\neg AC$ prove that any set that contains infinite sets of all sizes
(every infinite set is the same size as a member of it) must be
infinite?

\medskip

Zachiri observes that $\AxC \vee$ AxCount$_\geq$ implies that every
cantorian natural is strongly cantorian.  This implies that the
cantorian naturals are an initial segment of \Nn, as well as being an
elementary substructure. The arithmetic of the cantorian naturals must
prove con(TSTI) but for general reasons it an elementary subthingie of
the arithmetic of NF: the inclusion embedding from the cantorian
naturals into the naturals is elementary for arithmetic.

So what matters is that the fixed points for $T$ should be an initial
segment of \Nn.

\bigskip

\bigskip


A tho'rt prompted by a question of Oren's\ldots does the $S$ hierarchy
in NF satisfy anything like condensation?  

$$(\forall \alpha)(\forall \M \prec_{str(\Sigma_1)})(\exists \beta \leq \alpha)(\M \simeq S_\beta)?$$

\subsection*{Union of a low set of low sets}

Must it be low?  Suppose $\{X_w: w \in W\}$ is a low set of low sets,
where the index set $W$ is wellfounded.  The obvious thing to do is
the following.

First off, observe that without loss of generality $W$ can be taken to
be a set of singletons beco's (at least if we are in NF) every
wellfounded set is the size of a set of singleton$^k$ for any concrete
$k$.

For each $w \in W$ pick a wellfounded set $Y_w$ which is in bijection
with $X_w$, and consider the cartesian product $Y_w \times w$.  This
is wellfounded, is a bijective copy of $X_w$ and these products are
all distinct.  So consider the union $\bigcup_{w \in W} Y_w \times w$. 
This maps onto $\bigcup\{X_w: w \in W\}$.  The we take power sets.

What have we used?  Annoyingly, quite a lot.


\subsection*{Tangled Types}

Any model of \TZT\ has a canonical expansion (well, it's not quite an
expansion but never mind) to a model of Tangled \TZT U.  Is there a
synonymy result lurking in here anywhere?

Suppose we have a model of Tangled \TZT\ in front of us.  Fix a level
$l$ for the moment.  Every level above $l$ thinks that it is ${\cal
  P}(l)$, so $l$ induces a partial bijection between any two $l', l''
> l$ as follows.  Each $x \in l'$ encodes a subset $y$ of $l$; if this
$y$ is encoded by an element of $l''$ then send $x$ to that element.

{\sl Partial} bijection?  Why only partial?  Beco's our power sets are
not honest, they are only first-order mimics.  So not every subset is
coded at every level.  But we do at least know that each level defines
a commuting system of partial maps.

Can we get permutations of a level?  Yes.  Let $l_1 > l_2 > l_3 > l_4$
be four levels.  We can define a permutation of $l_1$ as follows. 
(Need a picture!)  $l_3$ and $l_4$ both induce partial bijections
between $l_1$ and $l_2$.  We can compose these bijections to obtain a
permutation of $l_1$.  Do we get a group?  Isn't there a worrying
possibility that the composition of two of these partial bijections
might be the empty map, at which point all information is destroyed 
and we lose associativity?  Yes, almost certainly.  


\subsection*{An Epimorphism that doesn't split}

Adam Lewicki wants an example of an epimorphism that doesn't split.
My first thought was that there can't be a choice function on the
ordinals, but actually that's not obvious.  There certainly can't be a
choice function that picks wellorderings that are pairwise disjoint.

We could deduce a contradiction from DC if we could show the following:

\begin{quote}
{\sl Let $\tuple{X,R}$ be a wellordering.  Then there is a wellordering
$\tuple{Y,S}$ with $X\cap Y = \emptyset$ and $|Y| \geq |X|$.}\end{quote}

Now i think this is correct.  Let $X$ be a wellorderable set, and
consider the partition of $V$ into $X$ and $V \setminus X$.  We would
like $|V \setminus X|$ to be $|V|$, so suppose it isn't, but is
smaller.  But then, by Bernstein's lemma, $X$ and $V \setminus X$ both
map onto $V$.  But $X$ is wellorderable, so $V$, being a surjective
image of a wellorderable set, would be wellorderable too. But it
isn't.

\medskip

Randall sez: consider the function that sends every wellordering $W$
with a last element to \verb#butlast#$(W)$.

\smallskip

Obvious, really.


\subsection*{Synonymy}

Three questions about synonymy that came up over dinner at number 11
after Grant's talk [lent term 2014] , when Nathan asked me ``What next for NF?''

\begin{quote}
Is NF synonymous with any theory of wellfounded sets?

Is NFU synonymous with any theory of wellfounded sets?

Is NFU + ``there are very few atoms'' synonymous with NF?

Is NFU + ``the atoms are indiscernible'' synonymous with NF?
\end{quote}

\begin{thm}
Every model of TST can be expanded in a definable way to a tangled
model of TSTU.\end{thm}

  Indeed that's the correct way to present the consistency
proof for NFU.  Introduce tangled models of TST and use them to prove
con(NF). Then show that tangled models of TSTU exist.

Let's write  this out in some detail.

\Proof

Let $\M$ be a model of TST. (Or \TZT, that works too.)  It has levels
$V_i$ for every $i \in \Nn$.  Whenever $i < j$ we define the binary
relation $x \in_{i,j} y$ that holds between members $x$ of $V_i$ and
members $y$ of $V_j$ iff $y$ is a set of$^{i-j-1}$ singletons and
$\iota^{i-j-1}(x) \in y$.  The expansion of $\M$ obtained by decorating 
$\M$ with these extra predicates is a model of the Tangled Theory of 
Types with {\sl urelemente}, also known as TTSU (or T\TZT U if $\M$ 
was a model of \TZT).

Now let $T$ be TSTU plus finitely many ambiguity axioms. We use
Ramsey's theorem to extract a model of $T$ from the decorated
expansion of $\M$.

\bigskip

Suppose we have a saturated model $\M$ of \TZT + Amb.  It has lots of
tsaus, and thereby gives rise to lots of models $\M^\tau$. one wants to 
say that all the theories $Th(\M^\tau)$ are synonymous.
13/iv/2013

\bigskip

SPF$\in$I: stratified parameter-free $\in$-induction\\
$\exists NO$: there is a universal set of wellordernestings\\


We know
\begin{quote}
(i)  SPF$\in$I $\to$ $\neg\exists V$
\end{quote}
But we don't know the converse
\begin{quote}
(ii) $\neg\exists V$ $\to$ SPF$\in$I
\end{quote}
nor do we know
\begin{quote}
(iii)  SPF$\in$I $\to$ $\neg\exists NO$
\end{quote}
Can we have a universal set of wellordernestings without a universal set:
\begin{quote}
(iv) $\exists NO$ $\to$ $\exists V$?
\end{quote}
It may be that we can tie these two together by means of SPF$\in$I

\bigskip

I had at one point the idea that one could have an axiom that said
that the closure ordinal of a rectype whose closure conditions were
homogeneous should be strongly cantorian.  But that's not true.  The
set of wellorderings is such a rectype, and the closure ordinal isn't
cantorian.  Something to do with it not being of bounded character,
perhaps?

\bigskip

I am starting to worry about the fact that among all the operations on
ordinals (as in Doner-Tarski) it is only the first three (+, $\times$
and exp) that correspond to operations on wellorderings.  In particular 
i'm worrying about this in connection with $T$, and commutation-with-$T$.

Actually the Hartogs aleph function, thought of as an operation on
ordinals, corresponds to an operation on wellorderings.

All these operations preserve cantorian/strongly cantorian

\bigskip

Richard has a model $\M$ of KF in which every set belongs to a fat
set.  Is there any mileage in the idea of the {\bf patch} ${\mathfrak
  p}(x)$ of a set $x$ being the intersection of all fat supersets of
$x$?  The patches form a directed system (think about ${\mathfrak
  p}(\{x,y\})$) whose direct limit is $\M$.  Are the patches
transitive?

%17/ii/2012


\subsection*{tf writes}

On Mon, Feb 13, 2012 at 5:05 PM,  <T.Forster@dpmms.cam.ac.uk> wrote:

 Dear Zachiri, David and Damien,

\bigskip

 (cc Randall, Richard, Marcel, Adrian, Andrei, Ali)

\bigskip

 Further to our conversation chez moi last night....

\bigskip

 KF is: union, power set, pairing, extensionality, stratified
 $\Delta_0$ separation.

\bigskip

 IO is the assertion that every set is the same size as a set of singletons.\\
 TCo is the asertion that every set has a transitive superset.

\bigskip

 It seems that\\
 (i)  KF is what one needs for the ZFJ construction to get a model of NF;\\
 (ii) KF is what one needs for the standard construction of a model of\\
 NFU from an automorphism.

\bigskip

I was very struck by Zachiri pointing out that I$\Delta_0$ does not
prove that $n \mapsto 2^n$ is total. Recalling Specker's definition of
$n \mapsto 2^n$, observe that IO is precisely what one needs to show
in KF that $n \mapsto 2^n$ is total. (One is using the homogeneous
definition of exponentiation of course).  KF obviously can't prove IO
unless NF is inconsistent (KF is subset of NF and NF refutes IO)
but---more to the point---KF does not appear to show that the
wellfounded sets obey IO. This suggests a very close relationship
between KF and I$\Delta_0$---which i hadn't suspected before our
conversation this evening.  Zachiri says that perhaps this means that
Mac is bi-interpretable with I$\Delta_0$ + exp.

KF appears to be precisely the theory of wellfounded sets in NF or NFU
(it seems to make no difference). A lot of work has been put into
ascertaining whether or not NF proves the existence of an infinite
wellfounded set: the best we can do at the moment is a result of
Holmes' that NF does not prove the existence of any infinite transitive 
subset of $V_\omega$. {\sl Nota bene}: KF does not appear to prove 
transitive containment or infinity. In contrast NZF (= NF $\cap$ ZF) 
proves both infinity and transitive containment.  (KF is here thought 
of as the NF-theory of wellfounded sets)

IO has mathematical (as opposed to merely set-theoretic) meaning: it says
that you can make arbitrarily many disjoint copies of any desired structure.

Then there is this nice stuff of Kaye and Wong, showing that the
mutual interpretations between PA and ZF$^-$ + C are nicer than
between PA and ZF$^-$ {\sl tout court}. (ZF$^-$ is ZF minus
infinity). �ZF$^-$ proves IO.  (Thanks, Ali! i was being an idiot)
Perhaps ZF$^-$ + IO behaves nicely in the same way that ZF$^-$ + TCo.

 I want David to write a Knight's Prize essay on this...

\bigskip

 More soon, perhaps

\subsection*{Ali Enayat writes}
Hello Thomas and other Colleagues,

You have put quite a few interesting items on the table. For the time
being, I will only address three of them.

1. There is a fairly substantial body of work on the set theory
available in of I-?? + Exp (where Exp states $2^x$ exists for all $x$).

To my knowledge this was first done in a substantial paper of Gaifman
and Dimitracopoulos, which deals with various aspects of  I-?? + Exp.
Recently Pettigrew wrote a paper on this topic, which includes all the
relevant bibliographical materials:

MR2535581 (2010j:03034) Pettigrew, Richard On interpretations of
bonded arithmetic and bounded set theory. Notre Dame J. Form. Log. 50
(2009), no. 2, 141?151.

2. NFU (as defined by Jensen, where one simply weakens the
extensionality axiom, but does not add the axiom of infinity) is
equiconsistent with I-?? + Exp, this was first shown by Solovay around
2002, but remains unpublished.

3. You wrote:

By the way, the following paper explore some aspects of ZF$^-$ + the
negation of Infinity.

$\omega$-models of finite set theory [with James Schmerl and Albert
Visser],  to appear in Set theory, Arithmetic, and Foundations of
Mathematics: Theorems, Philosophies (edited by J. Kennedy and R.
Kossak), Cambridge University Press, 2011.

A preprint of it is available on my website at:
\url{http://academic2.american.edu/~enayat/ESV%20(May19,2009).pdf}

Best regards,

Ali

\subsection*{tf writes}

Ali,
 
 I am trying to see how IO fits into this. (It was only in talking to  
Zachiri and David last night that it occurred to me that there might be
anything cute one could say).

Have i got this right...:
\begin{quote}
     Mac corresponds (in some nice way) to I$\Delta_0$ + exp
\end{quote}
so\ldots
\begin{quote}
     KF corresponds (in the same nice way) to I$\Delta_0$ tout court
\end{quote}
\ldots the thinking being that IO corresponds to exp and that IO
is (presumably) not a theorem of KF.  If IO is a theorem of KF +
infinity then NF is inconsistent.  If IO is *not* a theorem of
KF (without infinity) and KF *does* correspond to I$\Delta_0$ in
some nice way then this might point the way to a model of KF +
$\neg$IO.  I for one would very much like to see such a model.






\section*{NFU}


Boise, 2001.  Holmes is proving that if one adds to NFU the following:
Choice, cantorian sets are stcan, every definable class of scordinals
is the intersection of a set of ordinals and the class of scordinals
then there is a coding of sets of scordinals as scordinals making the
class of scordinals into a model of ZFC + the class ordinal is weakly
compact.   In fact the two theories are equiconsistent.

Holmes reassures me that on the whole constructions like this can be
run in \nf\ as well.  The domain of the model can be taken to be
scordinals (as above) or one can do a relational type construction or
even use $H_{stcan}$.


\section*{Equivalents of \AxC}

Consider the relation $Tx < y$ on \Nn. Let's write it `$E$'. Now $E$
is wellfounded iff \AxC. Thus \AxC\ is equivalent to the assertion
that we can do induction for stratified expressions over $E$.

Now i proved somewhere that \AxC\ is equivalent to $\poss($the graph
of the comparative-rank quasiorder on $V_\omega$ is a set$)$.  So
perhaps one should be able to prove something directly in the
arithmetic of NF+\AxC.  Let $R$ be a relation on \Nn\ satisfying
$(\forall n,m \in \Nn)( (n = 0 \vee(\forall k\,E\,n)(\exists k'\,E
m)(k R k'))\to nRm)$ (That is as much as to say that $R$ is a
comparative-rank relation for $E$.)  What can we prove about $R$
using $E$-induction?  That it is a wellfounded quasiorder?

\bigskip

(i) Prove by $E$-induction that every subset of \Nn\ has an $R$-minimal member?

(ii) Prove by $E$-induction that $(\forall n, m \in \Nn)(nRm \vee mRn)$?

\bigskip

(i) looks OK: any set of natural numbers containing 0 has a
$R$-minimal member.  Now suppose $n$ to be such that, for each
$m\,E\,n$, any set of natural numbers containing $m$ has an
$R$-minimal member.  Suppose $n \in X \subseteq \Nn$. If $n$ is
$R$-minimal we are done.  If not, then (by non--$R$-minimality of $n$)
there is $m R n$ with $m \in X$ and $\neg(nRm)$.  This last condition
gives $(\exists k\,E\,n)(\forall k'\,E\, m)(\neg(k R k'))$

 err......


Write `$x\leq^T y$' for `$Tx \leq y$'.  \AxC\ implies not only that
that the strict part $<^T$ is well-founded, it implies that $\leq^T$
is a well-quasi-order.  It is transitive because if $Tn \leq m \wedge
Tm \leq k$ then $T^2n \leq k$ and $k \leq Tk$ so $T^2n \leq Tk$ and
$Tn \leq k$ as desired.

For the condition concerning $\omega$-sequences let $\tuple{x_i:i \in
  \Nn}$ be an $\omega$-sequence of distinct natural numbers.  (If
they're not all distinct we're home and hosed).  By \AxC\ it has a
$\leq^T$-minimal element, $n$, say. (*)\marginpar{We need the result
  at *} Let $X$ be the set of elements of $\tuple{x_i:i \in \Nn}$ that
occur later in the sequence than $n$ does.  Suppose there is no $x \in
X$ s.t. $n\leq^T x$. That is to say $(\forall x \in X)(\neg(Tn \leq
x))$ which is to say $(\forall x \in X)(x \leq Tn)$. So $X$ must have
been finite, so some number appears more than once.  \index{<^T}

[presumably there is an analogous result for $(\forall n \in \Nn)(n\geq Tn)$. Indeed an analogous result for any endomorphism]

Is it BQO?


Is there any way one can discuss the tree of bad sequences for this
WQO?  In ML somehow?  There is no reason for it to be a set, but if it
is, it is a wellfounded tree.   And if it is a proper class, then ML 
will think it has a rank.



\bigskip

\bigskip

\bigskip

\bigskip

This probably doesn't work either...

\begin{thm}
Every $\forall^*\exists^*$ sentence true in arbitrarily large finitely
generated model of TST is true in all infinite models of TST.\end{thm}

\Proof
The key is to show that every model of TST can be obtained as a direct
limit of finitely generated models of TST.  The hard part is to find
the correct embedding.

Let $\M$ be a model of TST.  We will be interested in finite
subthingies characterised as follows.  Pick finitely many elements
$x_1 \ldots x_k$ from level 0 of $\M$; they will be level 0 of the
finite subthingie. Then take a partition of level 1 of $\M$ for which
the $x_i$ form a selection set (a ``transversal'').  The pieces of
this partition are the atoms of a boolean algebra that is to be level
1 of the finite subthingie.  That gives us level 1 of the subthingie.
To obtain level 2 we find a partition of level 2 of $\M$ such that the
carrier set of the boolean algebra we have just constructed (which is
level 1 of the subthingie) is a selection set for it.  The pieces of
this partition are the atoms of a boolean algebra that is to be level
2 of the finite subthingie. Thereafter one obtains level $n+1$ as a
boolean alegbra whose atoms are the pieces of a partition of level
$n+1$ of $\M$ for which level $n$ of the subthingie is a transversal.

There is, at each stage, an opportunity to choose a partition, so this
process generates not {\sl one} subthingie from the finitely many
elements $x_1 \ldots x_k$ from level 0 of $\M$, but infinitely many.
This means that the family of subthingies has not only a partial order
structure but also a topology.  Choosing $n$ things from level 0 does
not determine a single finite subthingie, co's you have a degree of
freedom at each step (when you add a new level).  It's a kind of
product topology, where each finite initial segment (a model of
$TST_k$ with $n$ things at level 0) determines an open set: the set of
its upward extensions.

Is the obvious inclusion embedding an example of what Richard calls 
an almost-$\forall$ embedding?

The long-term aim is to take a direct limit, and we want this direct
limit to be $\M$ itself, so we must check that every element of $M$
can be inserted into a subthingie somehow.

Clearly any finite set of elements of level 0 of $\M$ can be put into
a finite subthingie, but what about higher levels?  We prove by
induction on $n$ that every finite collection of things of level $n$
can be found in some finite subthingie or other.

The induction step works as follows.  We have a subthingie $\M_1$ and
we want to expand it to a subthingie $\M_2$ that at level $n+1$
contains finitely many things $x_1 \ldots x_k$.  To do this we have to
refine the partition of $V_n$ that is the set of atoms that $\M_1$ has
at level $n+1$ so that every $x_i$ is a union of pieces of the refined
partition.  There are only finitely many $x_i$ so any refinement that
does the job has only finitely many pieces.  Identify such a
refinement, and pick a transversal for it that refines the set which
is level $n$ of $\M_1$.  This transversal is a finite set of things 
of level $n$, and we can appeal to the induction hypothesis.\endproof

\bigskip

Next we ask, suppose at each level from 2 onwards, instead of picking
a partition of level $n$ of $\M$ to be the set of atoms of the boolean
algebra at level $n$, we simply take $B``$level $n-2$ to be a set of
generators for the boolean algebra of level $n$?  We lose a degree of
freedom but we get better behaviour of the embedding, since this
ensures that it preserves $B$.  Can we still ensure that every element
of $\M$ appears in the direct product?

Unfortunately the answer to this can be easily shown to be `no' since, 
for the answer to be `yes', one would have to be able to express every 
element of level $n$ of $\M$---for $n$ as big as you please---as 
a $\{B, \cup, \cap, V, \setminus\}$-word in the finitely many 
elements chosen to be level 0 of the subthingie and the elements of the 
partition that are to be level 1.  That is clearly not going to happen.

\bigskip

This proof is essentially the correct general version of the proof in
the book where the same result is claimed only for countable models.
This proof is more general and easier to follow.  The converse problem
remains: can we show that every $\forall^*\exists^*$ sentence true in
even one model of \TZT\ is true in the term model for \TZT 0?  
%LA oct 2011



Consider total injective functions $f:\Nn \to \Nn$ s.t. all the functions 
in $\{\{f(n)\}: n \in \Nn\}$ are distinct two-valued total functions.
Any such $f$ gives us a structure for ${\cal L}(\in)$.  The set of partial 
injections gives a topology. 

Each function gives us a di Giorgi structure.

\bigskip

Or, again.  Consider the set of computable (partial?) functions
\Nn$\to\{0,1\}$, and the equivalence relation on this set of having
the same graph. Is there a computable total function \Nn$\to$\Nn\
whose range is a transversal for this set?


\bigskip


The latest wheeze is to show that $\Pi^{B,\iota}_3$ things
generalise downward to the term model for $\TZT 0$.  Suppose $(\forall
\vec z)(\exists \vec x)(\forall \vec y)\phi$ is true in some model
$\M$ of $\TZT 0$.  Then, however we instantiate the $\vec z$ to $\TZT 0$
terms, we hope that the resulting $\Sigma^{B,\iota}_2$ formula with
parameters holds in the term model.  So we hope that witnesses to the 
$\vec x$ variables can be found in the term model.

So consider $(\exists \vec x)(\forall \vec y)\phi$ where $\phi$ now
contains parameters (from the term model) Let's process $\phi$\ldots.
First we put it into CNF, and then we extract (pull to the front) all
the atomic subformul{\ae} that contain only $x$ variables.  These are
not bound by the $y$ quantifiers and do not need to be within their
scope.  $\phi$ now looks says ``either the $\vec x$ are related to each
other like this and they are related to the $y$s like thus-and-so, or
\ldots, and so on with finitely many mutually exclusive disjuncts.
Since $\M$ believes this to be the case, it believes precisely one of
these can hold.  So $\M$ says that there are these $\vec x$ and they
are related to each other like thus-and-so, and there are finitely
many clauses we have to satisfy, all of them looking like $(\forall
\vec y)D$ where $D$ is a disjunction of atomics and negatomics.  We
now have to select from the term model things suitable to be witnesses
to the $\vec x$.  The wriggle room comes from the fact that we don't
have to literally satisfy all the $(\forall \vec y)D$ clauses, but
only all the substitution instances obtained by instantiating the
$\vec y$ to $TZT 0$ terms.

If you try this you find (at least i found) that mostly i could make
do with NF$_2$ words but it's easy to see that that won't work in
general.


There is some further processing we can do to the
conjuncts/disjunctions inside $\phi$\ldots leave at the front of the
formula the universal quantifiers over the $x$s of lowest type and
import everything else.  Then (inside that formula) leave outside the
quantifiers over $x$ variables of next lowest type and import
everything else.  The result is that each conjunct/disjunction ends up 
looking like
$$(\forall \vec x_1)(D_1 \vee (\forall \vec x_2)(D_2 \vee (\forall \vec x_3)(D_3 \vee \ldots)))$$

\section{Building the stratified analogue of $L$ {\sl very very slowly}}

 \subsection*{A nugget by Nathan Bowler written up by Thomas Forster}

\subsubsection{Fast Food/Slow Food}

This arose during the regular saturday meeting of the reading group
on the literature on hereditarily symmetric sets and related topics, 
occasioned by the visit of Edoardo Rivello to the Cambridge NF-istes.

The background to this note is that if we construct the stratified
analogue of $L$ very slowly, collecting (``banking'')\footnote{The
  reference is to {\sl The Weakest Link} where contestants have to
  {\sl bank} their winnings every now and then. We have to do this
  too, every time we close under anything, since otherwise the closed
  set we have just obtained might not be a set of the model.}  very
often, then we might end up constructing the whole of $L$.  The point
is that every time you bank you are adding a set that is defined only
as the closure of a set under stratified operations, and such a
definition is typically not stratified---unless the operations are all
homogeneous. So {\sl banking adds unstratified information}.  Vu said:
if there is anything in $L$ that is never going to be constructed at
all, however slow we go, then there is probably such a object of very
small rank.  How about the von Neumann \Nn? I said.  Nathan took up
the challenge of constructing the von Neumann \Nn\ by going slow
\ldots

The first attempt is the stratified $\Delta_0$ function.

\begin{equation}
x,y\ \mapsto\  \left\{ \begin{array}{r}
 x \cup \{y\}\ {\rm if}\ |x| = |\iota``y|\\
\\
                         = \emptyset\ {\rm o/w}\\
\end{array}\right\}
\end{equation}



The idea is that closing $\{\emptyset\}$ under this will give us the
set of von Neumann naturals.  But this function isn't $\Delta_0$, so
it doesn't work.  But the idea is a good one.  The next adjustment 
is due to Vu Dang.

The idea now is to find a stratified $\Delta_0$ function such that
closure under it will churn out the bijections we need.  The following
function will spit out, for each $n$, a bijection between the set of
(von Neumann) naturals below $n$ and the set of singletons of (von
Neumann) naturals below $n$
\begin{center}
$f: x,y \ \mapsto\ \  x \cup \{\tuple{\pi_1``y,\{\{z: \{z\} \in \pi_2``y\}\}}\}$
\end{center}
The $\pi_i$s are the unpairing functions and the angle brackets are
Wiener-Kuratowski ordered pairs.  This time the definition is indeed
$\Delta_0$---and still stratified.  

Let $B$ be the closure of $\{\emptyset\}$ under $f$. For each
$n \in \Nn$ $B$ contains the bijection $\{\tuple{i,\{i\}}: i < n\}$ 
---plus a lot of other rubbish besides (which we don't need to
worry about).




\begin{equation}
g: x,y,z\ \mapsto\  \left\{ 
\begin{array}{r}
x \cup \{y\}\ {\rm if}\ z\ {\rm is\ a\ bijection}\ x \simeq \iota``y\ {\rm and\ neither}\\
                                    x\ {\rm nor}\ y\ {\rm contain\ any\  ordered\ pairs}\\
\\  
                       = \emptyset\ {\rm o/w}\\
\end{array}
\end{equation}



Then we close $B$ under $g$.  The effect is to add all the von Neumann
naturals.  Call the result $C$.  We want $(C \setminus B) \cup
\{\emptyset\}$.  We can do this as long as we have $B$ and $C$.  If
$B$ is to be a set in the model then presumably we {\sl bank} $B$ as
soon as we make it.  This means that the thing we obtain at the next
stage by closing under $g$ is not the set $C$ we have just described
but the closure of $B \cup \{B\}$.  We can get round this by modifying 
$g$ to

\begin{equation}
g: x,y,z\ \mapsto\  \left\{ 
\begin{array}{r}
x \cup \{y\}\ {\rm if}\ z\ {\rm is\ a\ bijection}\ x \simeq y\ {\rm and\ neither}\ x\ {\rm nor}\ y\\ 
{\rm contain}\ {\rm  any\ ordered}\
                                    {\rm pairs\ nor\ even}\\ {\rm anything}\ {\rm containing\ any}\                                       {\rm  ordered\ pairs}\\
\\
                         = \emptyset\ {\rm o/w}\\

\end{array}\end{equation}


When we close $B \cup \{B\}$ under $g$ we never pick up anything with
$B$ inside it because $B$ contains ordered pairs.  This means that the
closure $C$ of $B \cup\{B\}$ under $g$ contains $B$ and all members of
$B$ and all the von Neumann naturals.  The von Neumann \Nn\ is
accordingly obtained as $(C \setminus(B\cup\{B\}))\cup \{\emptyset\}$.
However, as Nathan observes, there's actually no need to modify $g$,
since there's no possibility of $B$ bijecting with anything in $B \cup
\{B\}$ except itself---it's too big.  The point however is well-made:
at some point in our construction we have a set $B$, say.  We close it
under $f_1$, then under $f_2$, then under $f_3$ and so on.  It does
make a difference whether or not we bank after each closure.  In the 
above case we wanted to take away everything in some set $C$ that was 
the closure of a stage under an operation.  So we needed $S$ to be 
a set.  It so happens that we could get $C$ as a set {\sl without} 
banking it but that was down to good luck not good management.


\section{NF$\Omega$}


Here might be a useful chain of theories\ldots.  Start with NF0.  Add
function symbols for its operations, and call the new language ${\cal
  L}^1$.  Now consider the theory whose axioms are extensionality +
$\Delta^{{\cal L}^1}_0$ comprehension.  This theory is obtained from
NF0 by adding, for each NF0 word $W$, an axiom giving us the existence 
of $\{x: W(x,\vec z) \in y\}$ where $\vec z$ and $y$ are parameters. 
This theory properly extends NF0, because one of its axioms is the 
existence, for all $y$, of $\{x: B(x) \in y\}$, and there are models 
of NF0 (e.g., the term model) where some values of this function are 
missing.  Other examples are $\{x: \{x\} \in y\}$, and 
$\{x: \{x\}\cup z \in y\}$. There will be infinitely many of them.

Let us call this theory `NF(1)' (at least for the moment---we've got
to call it {\sl something}!)  Now let ${\cal L}^2$ be the language
obtained from ${\cal L}^1$ by adding function letters for all the
operations that NF(1) says that the universe is closed under.

Now consider the theory whose axioms are extensionality +
$\Delta^{{\cal L}^2}_0$ comprehension.  This will of course be notated
`NF(2)'.  We can keep on doing this, obtaining languages ${\cal L}^n$
and theories NF$n$ (extensionality + $\Delta^{{\cal L}^n}_0$
comprehension) for each $n \in \Nn$. Let the union of the languages be
${\cal L}^\Omega$ and the corresponding theory `NF$\Omega$' be
extensionality + $\Delta^{{\cal L}^\Omega}_0$ comprehension.

Observe that, for each $n$, NF$(n)$ has a $\Pi^{{\cal L}^n}_2$
axiomatisation.




Let us define $\Omega$-formul{\ae} and $\Omega$-terms by a
simultaneous recursion.

An $\Omega$-{\bf term} is either a variable or an expression of the form
`$\{x:\psi(x,\vec t)\}$' where $\psi$ is a stratified $\Omega$-formula and the
$\vec t$ are $\Omega$-terms.  An $\Omega$-{\bf formula} is a boolean
combination of expressions $t = t'$, $t'' \in t'''$ where $t,t',t''$
and $t'''$ are $\Omega$-terms.


NF$\Omega$ is extensionality plus the existence of $\{x: \phi(x, \vec
z)\}$ where $\phi$ is a stratified $\Omega$-formula.

How well behaved are these theories?  It turns out that NF(1) is
consistent and that the term model for NF0 is a model of NF(1). It
turns out that NF(2) properly extends NF0 and the term model for NF0 is
not a model of NF(2).


First we show that the term model for NF0 is a model of NF(1). We must
show that every word of the form $\{x: W_1(x) \in W_2(x)\}$ or $\{x:
W_1(x) \in W_2(x)\}$ (where $W_1$ and $W_2$ are NF0 words) is equal to
an NF0 word.  The way to do this is to show that every such word
(equation or membership-statement) is equal to a boolean combination
of equations and membership-statements between shorter words.


Consider the set abstract $\{x: W_1(x) \in W_2(x)\}$. $W_2(x)$ will be
a boolean combination of NF0 words in the generator `$x$' with a
finite amount of modification by addition or deletion of singletons.
So $\{x: W_1(x) \in W_2(x)\}$ will be a boolean combination of things
of the form $\{x: W_3(x) \in W_1(x)\}$ and singletons of shorter
words, so clearly we have a recursion on our hands.  

What about $\{x: W_1(x) = W_2(x)\}$?  Both $W_1(x)$ and $W_2(x)$ are boolean
combinations of $B$ of shorter terms with a finite amount of
modification by addition or deletion of singletons.  As before, we can
only have $W = W'$ when the things that $W$ is a boolean combination
of are pointwise identical with the things that $W'$ is a boolean
combination of.  So, again, we have reduced it to a finite
combination of smaller problems.  

Eventually we will have reduced both $\{x: W_1(x) = W_2(x)\}$ and
$\{x: W_1(x) \in W_2(x)\}$ to boolean combination of terms of that
flavour---which cannot be reduced any further.  These bedrock terms
are things like $\{x: W_1(x) = W_2(x)\}$ and $\{x: W_1(x) \in
W_2(x)\}$ where at least one of $W_1(x)$ and $W_2(x)$ are atomic---and
these are taken care of by NF0 words.

But this tells us that the term model for NF0 is in fact a term model
for NF$(1)$.  Why?  Well, any NF$(1)$ word can be thought of as a syntactic
tree.  We look inside this tree for the lowest occurrences of NF$(1)$
constructors.  But---as we have just seen---any such subterm can be
replaced with an NF0 term.  Thus we can ratchet our way up the syntax
tree and eventually end up with an NF0 term.

However, we cannot extend this to NF(2). This because, altho' every
NF(1) word (without generators) is equal to (has the same denotation
in all models) as an NF0 word, nevertheless an NF(1) word with a
generator is not reliably equal to an NF0 word in that generator.
This is certainly the case---since $\{x: B(x) \in y\}$ is not an NF0
word in `$x$'---and it may matter. Of course $\{x: B(x) \in t\}$ is an
NF0 word whenever $t$ is an NF0 term. But that isn't enough.  The
killer is the NF(2) term $\{x: \{y: B(y) \in x\} \in \{z: \{z\} \in
x\}\}$ (also known as $\{x: \{B^{-1}``x \in x\}$).  It should be easy
to show that this cannot have the same denotation as any NF0 term.

Why might this be interesting?  I can think of two reasons.  One is
that NF is finitely axiomatisable.  One dispiriting consequence of
this is that any infinite hierarchy of subsystems of NF either reaches
NF in finitely many steps or never reaches it at all---usually the
former.  This system NF$\omega$ is either going to be equal to NF---in
which case one of the NF($n$) is already equal to NF---or it is
strictly weaker, and might offer a stepping stone---in the sense that
it might be possible to prove it consistent and also prove NF
consistent relative to it.  Finally it's a nice theory because in the
language with all the function symbols it has a $\forall^*\exists^*$
axiomatisation.

But observe that the theory NF$\exists$ (aka NF$\forall$) also has a
$\forall^*\exists^*$ axiomatisation.  Simply add a function letter for
each axiom and then lots of axioms to tell you what the operations
mean, such as $\forall x\forall y(x \in {\cal P}(y) \bic x \subseteq
y)$---and all such axioms are $\forall^*\exists^*$.


\bigskip



Nathan has made me see some things..

If $\M$ is a countable model of $TZT0$ think of it as a direct limit of
its finitely generated substructures, but consider only those finitely
generated substructures that are generated by things that cannot be
$TZT0$ words.  Hereafter a {\sl generator} is something that is not a
singleton, $B$ of anything, not the empty set not the universe, not a
boolean combination etc.

I think the idea is to show that every $\Pi_2$ sentence generalises 
downwards to any of these guys.

\bigskip

I'm still trying to prove that every $\forall^*\exists^*$ sentence
consistent with $\TZT0$ is true in the term model.  Here is something
that might work.  let $\M$ be a countable model of $\TZT0$.  Then it is
a direct limit of a suitable $\omega$-chain $\tuple{S_i: i \in \Nn}$
of finite substructures, with embeddings ${\tuple f_i: i \in \Nn}$
from $S_i$ into $S_{i+1}$.  But each $S_i$ is of course embeddable
into ${\mathfrak T}$ the term model of $\TZT0$.  So we flesh out each
$S_i$ to a copy of ${\mathfrak T}$ and expand somehow each $f_i$ to an
injection also called $f_i$ from ${\mathfrak T}$ into ${\mathfrak T}$.
Can we do this?  Yes, every countable binary structure (so, in
particular, ${\mathfrak T}$) can be embedded into ${\mathfrak
  T}$---and, indeed, into any cofinite subset of ${\mathfrak T}$.  The
hard part is to ensure that the new direct limit is the same as the
old.  To bring this about we have to do is ensure that every
$f$-thread eventually lands inside an $S_i$.  To do this we will have
to exploit the fact that $\M$ is a model of $\TZT0$, not just any
arbitrary countable structure---because the result we are trying to
prove isn't true for an arbitrary countable structure!  Also it has to
be an argument that exploits the fact that $\M$ is a term model of
$\TZT0$ rather than NF0, beco's of unstratification and the possibility
of Quine atoms.


Express $\TZT0$ in the language with function letters for the
operations. Then it is universal-existential, which may or may not
help.  Let $\M$ be a countable model of $\TZT0$ (countability might not
help, but it's not going to do any harm).  Express $\M$ as a direct
limit of an $\omega$-sequence $\tuple{\M_i: i \in \Nn}$ of some of its
finitely generated substructures.  Each $\M_i$ can be embedded somehow
into a copy ${\mathfrak T}_i$ of ${\mathfrak T}$ the term model of
$\TZT0$.  We want to do this in such a way that the direct limit of the
$\tuple{{\cal T}_i: i \in \Nn}$ is actually $\M$.



\bigskip

I think (check it!) that if a model $\M$ of $\TZT0$ is thought of as an
${\cal L}^{B,\iota}$ structure then it embeds into ${\mathfrak T}$
thought of as a ${\cal L}^{B,\iota}$ structure.  

\bigskip



\bigskip


All these things i want to connect\ldots

universal-existential sentences in $TZT$.  Also $\forall^*_\infty
\exists^*_\infty$ sentences. The way in which every countable
structure embeds in the term model of NF0 in continuum many ways;
countable categorical theories; something to do with random
structures.  See \verb#quantifiertalk.tex#.  Are there any models for
TZT that are random?  Is the term model for NF0 a random structure for
the theory of extensionality? (Doesn't the existence of a universal
set bugger things up?) What about the model companion of NF? Aren't
model companions something to do with random structures?

Are co-term models a distraction?

model companions\\
random structures\\
universal-existential\\
zero-one\\
nice embeddings\\

\bigskip

\section{Co-term models}


Term models are inductively defined sets: they are
manifestations/denotations of the (inductively defined) set of words
in a suitable language. There is of course also the {\sl
  co}-inductively defined set of (co-)words, which are of course
infinite \ldots streams.  What about these coinductively defined
analogues?  %14/x/2010 

Is it by omitting types that we prove the existence of such models?

If $T$ is an algebraic theory then it has term models. The interesting
cases from our point of view are theories $T$ that, in addition to having
axioms that say that the universe is closed under certain operations,
have annoying extra axioms such as extensionality whch might prevent
the family of $T$-terms from being a model of $T$.

Consider NF0.  A co-term model of NF0 is a model of NF0 in which every
object is a boolean combination of $B$ objects $B(x)$ and singletons
$\{y\}$ , where the $x$s and $y$s are themselves boolean combinations
of \ldots.  But this is first-order isn't it?  ``$(\forall x)$ there
is a finite set of sets and a string of connectives such that \ldots''
Or, if we don't want quantifiers over finite sets, we can do it by
omitting the $1$-type $\Sigma^c$ that says 

\begin{equation}\tag{$\Sigma^c$}
(\forall y)(x\not= B(y)), (\forall y)(x\not= V \setminus B(y)), (\forall y)( x\not= \{y\}), (\forall y)(
x\not= V \setminus \{y\}), \ldots\end{equation}


Then there is the type $\Sigma^i$ that one has to omit to get a term
model.  Anything that realizes $\Sigma^c$ will realize $\Sigma^i$, but
there is no reason to expect the converse.  So it might be that it is
easier to omit $\Sigma^c$ than it is to omit $\Sigma^i$.  


For some theories $T$ the theory of co-term models of $T$ is axiomatisable.  
If $T$ has two operations $f$ and $g$ then the theory of a co-term model of
$T$ is just $T$ + $(\forall x)(\exists y)(x = f(y) \vee x = g(y))$, so
it's first-order.  If $T$ has infinitely many operations $f_i$ then a
co-term-model of $T$ is one that omits the type 
\begin{equation}\tag{$\Sigma^c$}
\displaystyle{\Sigma_{i \in \smallNn} (\forall y)(x \not= f_i(y))}\end{equation}

So:\begin{itemize}

\item for $T$ to have a term model is for $T$ to have a model that
  omits the type $\Sigma^i$;

\item for $T$ to have a co-term model is for $T$ to have a model that
  omits the type $\Sigma^c$.
\end{itemize}

If $T$ locally omits $\Sigma^c$ then it certainly locally omits $\Sigma^i$.  

If $T$ has a co-term model then it must have a term model, since the
term model is a substructure of the co-term model that is closed under
everything under the sun.  If the co-term model is a model of $T$ (by
satisfying extensionality or whatever) then presumably the term model
is too.

$T$ might have a co-term model for trivial reasons. If $T$ has a pair
of operations $f$ and $g$ that are inverse then clearly everything is
the denotation of the stream $fgfgfg \ldots$. (NF is such a theory,
because of $\iota$ and $\bigcup$).  Is there a nontrivial notion of
co-term model to be had for such theories?


So are there theories $T$ with lots of operations that don't have to
have inverses, such that $T$ might not have a term model (perhaps
beco's of problems like those we have with extensionality in NF) but
where $T$ perhaps has a co-term model?

With the term model it is clear what equality is. Not so clear with 
the co-term model: any bisimulation on the family of streams will do.

What about NF0?  The set of NF0 words defines a unique model. This
model satisfies every $\forall^*\exists^*$ sentence consistent with
NF0.  Now consider the co-term model.  It's not clear that the set of
co-words defines a unique model, nor that that structure has a 
decidable theory.  There may be lots of different ways of turning the
set of co-terms into a model.


To get a feel for what is going on, consider a particular theory and a
particular co-term model: $\TZT0$ and its co-term model.  What might
equality be in this structure?  If we have a notation that does not
distinguish $x \cup y$ from $y \cup x$ then we have a strict identity
that is simply identity of strings.  But then there is also a maximal
bisimulation.  But are these two not exactly the same? So what is
$\in$ between these things?  There is available to us the same
recursion as in the term model case, and the freeness of the
constructors will ensure that it usually halts. When might it not?
Well, ask whether $B^\infty$ at level $n$ is a member of the
$B^\infty$ at level $n+1$.  That enquiry never halts.  That seems to
be about it.  The corresponding enquiry about $\iota^\infty$ gets the
prompt answer `yes'.


The problem with the co-term model is of course extensionality. I have
found myself wondering if the inclusion embedding from the term model
into the co-term model is elementary \ldots and indeed it is.  Suppose
the co-term model satisfies $(\exists y)\phi(\vec x), y)$ where the
$\vec x$ are from the term model. The parameters $\vec x$ are all
$k$-symmetric for $k$ sufficiently large, so think about some type at
least $k$ below all the variables in $\phi$.  Any permutation of this
type will fix all the parameters, so we want one that will move the
witness $y$ to a denotation of a $\TZT0$ term.  Now, because $y$ is in
the co-term model, it can be expressed as some complicated horrendous
word in $B$ and $\iota$ and the booleans over a lot of generators at
level $-k-1$.  There are only finitely many of these generators, and
there are infinitely many denotations-of-$\TZT0$-words to swap them
with.  Let $\pi$ be one such permutation.  It fixes all the $\vec x$s
and swaps $y$ with a denotation of a $\TZT0$ word.

(Do we need all permutations of finite support to be setlike in the
term model and co-term model?  Perhaps we do, but---fortunately---they are!)

What does this rely on?  It'll work for any extension of $\TZT0$ all of
whose constructors are type-raising.  The other thing we are
exploiting is the feature that, for any $x$ in the co-term-model and
any $k$, $x$ is $k$-equivalent to a denotation of a closed term.  This
reminds me of the condition that cropped up in the attempt Randall and
i made to prove the existence of a symmetric model of $TZT$: For every
$x$ and every $k$, $x$ is $k$-equivalent to a symmetric set.  So
presumably every model in which this is true is elementarily
equivalent to a term model\ldots?

Now what about the theory (NFP? \ldots NFI\ldots?) which becomes NF
when you add the axiom of sumsets? Is it axiomatisable exclusively
with extensionality plus axioms giving closure under type-raising
operations?  If so, does its typed version have a term model/co-term
model?  Presumably we can do the same trick to show that the term
model is an elementary substructure of the co-term model.


***************************************************************

I think the same argument will prove that the inclusion embedding
$V_\omega \inj \bigcup\{X: X \subseteq {\cal P}_{\aleph_0}(X)\}$ is
elementary for weakly stratified formul{\ae}. 

However the same construction will not prove that the inclusion
embedding $\bigcap\{X: {\cal P}_{\aleph_1}(X) \subseteq X\} \inj
\bigcup\{X: X \subseteq {\cal P}_{\aleph_1}(X)\}$ is elementary for
(weakly?) stratified formul{\ae}.  We could prove that the inclusion
embedding $V_\omega \inj \bigcup\{X: X \subseteq {\cal
  P}_{\aleph_0}(X)\}$ is elementary for stratified formul{\ae} because
everything in $V_\omega$ is symmetric.  Sadly not everything in
$\bigcap\{X: {\cal P}_{\aleph_1}(X) \subseteq X\}$ (aka $H_{\aleph_1}$
or $HC$) is symmetric.  Can we do anything similar to this for
$HS$\ldots?  We'd need a model in which, for every set $x$ and
infinitely many $n$, $x$ is $n$-similar to something in $HS$.

There does seem to be a general question here\ldots if $F: V \to V$ is
an operation on sets, for which class $\Gamma$ of formul{\ae} is the
inclusion embedding from the lfp for $F$ into the gfp for $F$
elementary?

***************************************************************

The pathologies that we know must disfigure any model of $TZT$ in
which every element is symmetric are not easily to be distinguished by the
naked eye from the pathologies we routinely find in Fraenkel-Mostowski
models.  This gives us hope that FM constructions will furnish one of the
keys to the problem of finding a symmetric model of $TZT$.


Consider the following structure $\M$ for ${\cal L}(TZT)$. Level $-n$
is the set of finite subsets of level $-n-1$.  At positive levels, we
stipulate that level $n+1$ be the set of almost-symmetric subsets of
level $n$.  What is ``almost-symmetric''?  A set $x$ at level $n$ is
almost symmetric iff there is a finite subset $H$ of level 0 s.t.
every permutation fixing $H$ pointwise will also fix $x$ when it acts
$n$ levels up.

Observe that this ensures not only that every set at level $n$ is
almost-$n$-symmetric in the old (FM) sense of almost $n$-symmetric (when
you were almost-$n$-symmetric iff your support $n$ levels down was 
finite), it ensures that every set at level $n$ is almost-$k$-symmetric 
(in the old sense of almost $k$-symmetric) for {\sl every} $k > n$!  
Suppose $x$ is a set at level $n$.  It is almost-$n$-symmetic, with 
support $H$, say.  $H$ is finite, and so is $\bigcup^{(k-n)}H$.  But 
then $x$ is almost-$k$-symmetric, with support  $\bigcup^{(k-n)}H$.

This is a key feature, since it was its lack in the earlier attempt by
Holmes and Forster that caused that attempt to fail.  This structure
$\M$ looks like a Fraenkel-Mostowski model of TST grafted onto a
$\omega^*$ root-stock where each level is the coinductive object
corresponding to $V_\omega$. (What is that object called??)

Now consider a shifting ultraproduct of this structure.  (To be
precise: for each $i \in \Nn$, let $\M^{(i)}$ be the result of
relabelling the types of $\M$ so that level $i$ of $\M$ is level 0 of
$\M^{(i)}$). Let $\cal U$ be a nonprincipal ultrafilter on $\Nn$.
Then the {\sl shifting ultraproduct} $\M^{\cal U}$ is simply an
ultraproduct of the $\M^{(i)}$.)  What happens in it?  If $x$ is a
element of level 0 of the shifting ultraproduct $\M^{\cal U}$ then it
is an object whose $i$th coordinate is an element of $\M$ that is
almost-$j$-symmetric for every $j \geq i$---and therefore
almost-$j$-symmetric for all sufficiently large $j$.  So (in $\M^{\cal
  U}$) $x$ {\sl ought} to be almost-$j$-symmetric for all sufficiently
large $j$.  Of course that inference is blocked because
``almost-$j$-symmetric for all sufficiently large $j$'' isn't
first-order. However we do get {\sl something}.



$\M^{\cal U}$ is part of the way to our goal of a model in which every
set is almost-$k$-symmetric for all sufficiently large $k$.  To obtain
such a model we have to omit (at each level) all the $1$-types:
\begin{center}
$\{$ ``$x$ is not almost-$k$-symmetric'': $k > i\}$
\end{center}
for every $i \in \Nn$.  The extended omitting types theorem tells us
that we can do this if we have a theory $T$ that locally omits all
these types.  The obvious candidate for such a theory is $Th(\M^{\cal
  U})$.  So what we need to establish is that $Th(\M^{\cal U})$
locally omits each of these types.\footnote{Parenthetical remark:
  ``$Th(\M)$ locally omits $\Sigma$'' is not obviously the same as
  ``$\M$ omits $\Sigma$''.  If``$Th(\M)$ locally omits $\Sigma$'' then
  whenever $Th(\M) \vdash \phi(x) \to \sigma(x)$ for all $\sigma \in
  \Sigma$ then $Th(\M) \vdash (\forall x)\neg \phi(x)$.  $\M$ might
  realise $\Sigma$, but whenever $\phi$ is a property that holds of an
  $x$ that realises $\Sigma$ then $\phi$ also holds of some $y$ that
  does {\sl not} realise $\Sigma$.}  Let us write `$T$' for
`$Th(\M^{\cal U})$'.

Suppose $\phi$ is such that $T\vdash (\forall x)(\phi(x) \to x$ is not
$k$-symmetric for any $k > i)$.  Then, for each $k \in \Nn$, $T\vdash
(\forall x)(\phi(x) \to x$ is not $k$-symmetric$)$. This is first-order, 
and so must be true in each factor, which is to say, in a large set of 
the $\M^{(i)}$.  But then nothing in $\M^{(i)}$ can be $\phi$.  So nothing in 
$\M^{\cal U}$ can be $\phi$ either.  But this says that $T$ locally omits 
the $1$-type: 

$\{$ ``$x$ is not almost-$k$-symmetric'': $k > i\}$

as desired.


Let's pause to draw breath \ldots and recycle some letters \ldots

Let $\M$ be a model of $TZT$ in which every set is almost-$k$-symmetric 
for all sufficiently large $k$.  We will show that the substructure of 
$\M$ consisting of the symmetric sets of $\M$ is elementary.

Suppose $\M$ satisfies $(\exists y)\phi(\vec x), y)$ where the
parameters $\vec x$ are symmetric sets. The parameters $\vec x$ are
all $k$-symmetric for some $k$ suitably large, so think about some
type at least $k$ levels below all the variables in $\phi$.  Any 
permutation of this type will fix all the parameters, so we want 
one that will move the witness $y$ to a symmetric set.  Now $y$ is
almost-$k$-symmetric, and it has finite support.  Just find a
permutation $\pi$ that moves everything in the support of $y$ to
something symmetric (hereditarily finite will do).  $\pi$ has now
moved $y$ to something $y'$ whose support consists of hereditarily
finite sets of rank $< j$ for some $j$. But now any permutation $k+j$
levels down fixes everything in the support of $y'$.  But this means
that $y'$ is $k+j$-symmetric.  And of course $y'$ is also a witness to
$(\exists y)\phi(\vec x), y)$---because the parameters (being $\leq
k$-symmetric) are fixed.

***************************************************************

How difficult is it to show that all single transpositions are
setlike?  [let us reserve `$\tau$' as a variable to range over single
  transpositions $(a,b)$.]

I think it is true in the term model for $\TZT0$ that all
transpositions are setlike.  Any transposition is certainly
$1$-setlike, because $\tau``x$ is either $x$, or $(x \cup \{a\})
\setminus \{b\}$, or $(x \cup \{b\}) \setminus \{a\}$, and all these
things exist.  There is no easy move to be made at the next level up,
but if the model we are working in is a term or co-term model then we
have other tricks up our sleeve.

To see this trick, start by thinking about what $j^2(\tau)$ does to
$B(x)$.  If $x$ is $a$ or $b$ it sends it to $B$ of the other one, and
o/w $B(x)$ is fixed.  So all these values exist.  In general, an
element of the term model or co-term model is a boolean combination of
singletons and principal ultrafilters.  For $n > 0$, $j^n(\tau)$
commutes with the boolean operations, so we will be OK if we know how
to define $j^n(\tau)$ on singletons and principal ultrafilters.

We conclude that in the term model or any co-term model every permutation
of finite support is setlike.

Again, what does this depend on?  Will not the same work for term and 
co-term models of any fragment of $\TZT$ that has type-raising operations only?

*******************************************************

The almost-symmetric sets of \TZT standard.tex have the property that
they are $n$-equivalent to symmetric sets, but only for some $n$, not
infinitely many, which is what the above arguments need.  They behave
a bit like the denotations of co-terms.  Perhaps what we want is a
model of $\TZT$ containing at each level all possible illfounded
hereditarily finite sets.  Then we say a set is almost-symmetric if it
has a finite family of these things as its support.  Observe that a set
that is almost-symmetric in this sense is indeed $n$-equivalent to a 
symmetric set for infinitely (cofinitely!) many $n$!!

*******************************************************

I think i can show that there is no relation $R \subseteq \iota``V
\times V$ which is extensional and ``skew--well-founded'': $(\forall X
\subseteq V)(\exists x \in X)(\forall y \in x)(\tuple{\neg(\{x\}, y}
\in R))$.

Suppose there were such an $R$.  We could then copy it onto a relation
$S$ on a moiety of a rather special kind.  We want $\bigcup dom(S)$ to
be disjoint from $rn(S)$ so that we can do a Rieger-Bernays model with
the permutation $\prod_{x \in \bigcup(dom(S))}(x,\bigcup R^{-1}``x)$
which will give us a wellfounded set the same size as a moiety---which
we know to be impossible by a theorem of Bowler which says that any 
wellfounded set is smaller than $\iota^k``V$ for every concrete $k$.

Details: First split $V$ into two moieties $A$ and $B$.  Further split
$A$ into $A_1$ and $A_2$. We must set up the copy of $R$ as a relation
between singletons of things in $A_1$ and things in $A_2$.

This matters because any CO model of NF has an (external) engendering
relation on it which is wellfounded and not too far from being 
extensional. Let me explain. There will be a relation ${\cal E}$ such 
that, for any $x$, $|\{y:{\cal E}^{-1}``\{y\}={\cal E}^{-1}``\{x\}\}|$ 
is small, being the size of the set of wands.  ${\cal E}$ thus isn't 
extensional, but extensionality doesn't fail {\sl badly}.

If $\E$ were extensional it would correspond to an injection from $V
\inj \iota``V$. This weaker condition says it corresponds to an injection 
$\inj \iota``V$ from not $V$ but a partition of $V$ into countable pieces.
Can we generalise Bowler's argument to exclude that?

\section{Fixed points for type-raising operations}

I proved a theorem about this that i need to review.  I think the
thought ran along the following lines.

\begin{lem}
Suppose $f$ is a definable $n$-stratified inhomogenous function that
raises types by 1.  Then $x \sim_n y \to f(x) \sim_{n+1} f(y)$.\end{lem}

\Proof
`$z = f(x)$' is stratified with $z$ one type higher than 
$x$.  Suppose further that $x \sim_n y$ beco's $(j^n(\sigma)(x) = y$.

 Then we reason:

$$z = f(x)$$ iff 

$$j^{n+1}(\sigma)(z) = f(j^n(\sigma)(x)).$$


which is to say, since $z = f(x)$,

$$j^{n+1}(\sigma)(f(x)) = f(j^n(\sigma)(x)).$$

and thence

$$j^{n+1}(\sigma)(f(x)) = f(y).$$


(since $(j^n(\sigma)(x) = y$).

But this is merely to say that 

$f(x) \sim_{n+1} f(y)$ in virtue of $\sigma$

Now 


$$z = f(j^n(\sigma^{-1}(y)) \bic j^{n+1}(\sigma)(z) = f(y).$$

Now this gives us a strategy for finding fixed points for $f$ 
in Rieger-Bernays permutation models.

Suppose i want  $V^\pi \models (\exists x)(x = f(x))$

This is just

$$(\exists x)(\pi_{n+1}(x) = f(\pi_n(x)))$$

relettering $\pi_n(x)$ as $y$

$$(\exists y)((j^n \pi)(y) = f(y))$$


\bigskip

So, if we want a permutation model containing a fixed point for an
operation $f$ that raises types by 1, it suffices to find a
permutation $\pi$ that sends some $y$ to $f(y)$.

\section{Parameter-free-NF}

Let's give it a name: NFpf.

Is it finitely axiomatisable?  Presumably not unless it is
inconsistent.  Observe that if it has any models at all then it has
only infinite models. (It proves the existence of every concrete
Zermelo natural and proves that they are all distinct.)  Does it prove
the axiom of infinity?

If NFpf has a term model then NF is consistent: any term model for NFpf
is a term model for NF.

Consider theories of the form: Extensionality + axioms saying that
certain (closed, parameter-free) set abstracts exist.  Some of these
theories are consistent, some are inconsistent.  As things stand, I
know of no inconsistent theory of this kind whose inconsistency needs
extensionality.  Further I don't think excluded middle has any r\^ole
to play in the paradoxes of \naive\ set theory. So I float the
conjecture:
\begin{quote} {\sl If $T$ is a constructive theory whose nonlogical
    axioms are all assertions that certain (closed, parameter-free)
    set abstracts exist, then Con(T) $\to$ Con(T + Ext + Excluded
    middle)}\end{quote}

Getting rid of extensionality would be good, because the rules for
extensionality are cut-absorbing.  If we know that any inconsistency
in a finite fragment of NFpf has a genuine cut-free proof we would
surely be able to do something with the stratification.

[nov 2014: Michael Rathjen tells me that there are models of constructive ZF in which the collection of regular sets is $\{\emptyset\}$]

Do we ever need extensionality to obtain a contradiction? Zachiri
sugested the paradoxical collection $\{x: (\exists y)(\forall z)(z \in
y \bic z \in x) \wedge y \not\in x)\}$ but you don't need
extensionality to obtain a contradiction.  But something like that
might work.


However we do sometimes need trivial axioms like subscission.  See
Forster-Libert \ldots

\medskip 

Does every finite fragment of NFpf have a model?

\medskip 

---Probably

\medskip 

Does every finite fragment of NFpf have a term model?

\medskip 

We have to be careful here.  We can't expect that every finite
fragment has a term model in which every term answers to a single set
existence axiom {\sl of that fragment}.  Consider the theory that is
extensionality + existence of the von Neumann ordinal 2.  This has a
term model, but the model contains $\emptyset$, $\{\emptyset\}$ and
$\{\emptyset, \{\emptyset\}\}$.  So we mean that the fragment should
have a model consisting of the denotation of closed terms possibly
additional to those mentioned in the axioms.

Once that is cleared up the answer is: quite possibly, but it isn't
much use, beco's there is no obvious way to stitch them together.
Here's why.  Let ${\cal X}$ be the set of closed set abstracts.  The
finite fragments of NFpf are indexed by $\powk{\aleph_0}{\cal X})$.
Consider the collection of $\subseteq$-closed subsets of
$\powk{\aleph_0}{\cal X})$.  This has the finite intersection property
so we can find a nonprincipal ultrafilter $\U$ on ${\cal X}$
containing all these $\supseteq$-closed subsets.  This gives us a
model of NFpf without extensionality as follows.  We rule that $s \in
t$ holds iff the set of finite fragments that believe $s \in t$
belongs to $\U$.  But now consider the empty set and the set of all
total orders of $V$.  Our model might believe these terms are
distinct, because a large set of factors believe it.  Each factor will
have a witness to their distinctness.  But there may be infinitely
many witnesses, so that no one of them is believed by a large set to
be a witness to their distinctness.

Is it plausible that there should be a finite fragment of NFpf that is 
consistent but has no finite models?


Observe that finite fragments with apparently quite sophisticated
axioms can have trivial models.  Consider the fragment that says that
the Frege \Nn\ exists.  There might not be any set other than $V$ that
contains 0 and is closed under $S$---in which case \Nn\ is the
intersection over the empty set and is $V$! Thus a model consisting of
a solitary Quine atom is a model of extensionality + the existence of
the Frege \Nn.  In fact such a trivial model is a model of that
fragment of NFpf that asserts the existence of any set defined as the
$\subseteq$-minimal set contains this and closed under that---as long
as the set abstract is stratified of course.

Is the set of axioms of NFpf closed under conjunction?  Is the fragment
that asserts the existence of terms $t_1 \ldots t_n$ is implied by the
single axiom asserting the existence of $\{t_1 \ldots t_n\}$? (Tho'
the converse implication clearly does not hold).  Probably not: as Randall 
says, the axiom asserting the existence of the unordered pair of the empty 
set and the Russell Class succeeds in asserting the existence only of the 
empty set.




In `The Quantifier Complexity of NF', Bulletin of the Belgian Mathematical Society Simon Stevin, ISSN 1370-1444, 3 (1996), pp 301-312. Kaye shows that 

    NF = NFpf +  NFO + existence of sumset    (*)

(This is theorem 2.3.) %which seems to me to be saying that if parameterless NF is consistent it is a very strange theory and actually not very nice (there is something bad in the model!)

%Talking of bad, I don't appear to need the B axiom of NFO to prove the result (*), so something very natural and ZF-like goes wrong in any possible model of parameterless NF.



Consider the set of those theorems of ZF(C) that are of the form
$(\exists x)(\forall y)(y \in x \bic \phi(y))$ where 
$FV(\phi) = \{$`$y$'$\}$.

This is a recursively axiomatisable theory.  Let's call it $T$.  My
guess is that Con (NF) $\to$ Con(NF $\cup$ T).  Does the axiom of
choice make any difference?
 


\section{More stuff to fit in}

\subsubsection{Statement of the Bleeding obvious}

\ldots except i missed it.  The collection of BFEXTs is the
welllfounded part of the collection of APGs under the obvious ``child
of star'' embedding relation.

\subsubsection{Recursive APGs}

How about getting a model of INF using recursive APGs

A recursive APG is an APG whose domain is the natural numbers and
whose graph is a recursive subset of $\Nn \times \Nn$.  A possible
world is a general recursive function.  Given two RAPGs $A$ and $B$ a
world $W$ believes $A = B$ iff (i) For every child $a$ of $A$ there is
a child $b$ of $B$ s.t. some value of $W$ is a function that maps $a$
to $b$.  and conversely (ii) For every child $b$ of $B$ there is a
child $a$ of $A$ s.t. some value of $W$ is a function that maps $b$ to
$a$. (The possibility of the value of $W$ that does the work not being
1-1 takes care of the contraction condition.)


\subsection{Almost-symmetric sets again}

If we can find, by hook or by crook, a model of \TZT\ wherein for every
$x$ there are infinitely many concrete $k$ such that there is a
(concrete) $n$ such that $x$ is almost-$k$-$n$-symmetric then we can
have a model of \TZT\ in which every set is symmetric.  On that much we
are agreed.  So can we find such a model?

Let $M$ be a model of \TZT.  Let $M_0$ be the FM model whose bottom
level (type 0) is the level 0 of $M$, built as a substructure of $M$,
so that its atoms are just the elements of $M$ of level 0.  

Next let $M_1$ be that substructure of $M_0$ obtained by retaining
only those atoms of $M_0$ that are finite-or-cofinite subsets of $M_0$
and then sticking on the bottom the level $-1$ of $M$ to obtain a
model of TST whose bottom type is labelled `$-1$'

Next let $M_2$ be that substructure of $M_1$ obtained by retaining
only those atoms of $M_1$ that are finite-or-cofinite subsets of $M_1$
and then sticking on the bottom the level $-2$ of $M$ to obtain a
model of TST whose bottom type is labelled `$-2$'

and so on. What happens? I think that if you are an element of level $k$ 
of $M_i$ then you are almost $k$-symmetric, almost $k+1$-symmentric 
\ldots almost $k+i$-symmetric.  Now take an ultraproduct of these $M_i$.
This ought to give us a model in which if you are almost $k$-symmetric you
are almost $m$-symmetric for all $m > k$.

\subsubsection{Perhaps we need to be more subtle}

If $\M$ is a model of $\TZT$ and $S$ a notion-of-symmetry, let us say a
set $x$ is almost-$n$-$k$-symmetric (in the new sense) iff there is a
$k$-sized subset $y$ of the universe $n$ levels down (aka $V_{-n}$),
all of whose members are symmmetric-in-the-sense-of-$S$, s.t. $x$ is
fixed by all permutation of $V_{-n} \setminus y$.  I have just
described a way of getting a new notion-of-symmetry from an old one.
We want a fixed point for this operation that gives an extensional
family of sets. Clearly the operation is monotone wrt $\subseteq$.
Ordinary old symmetry is a fixed point, but we don't know that the
symmetric sets are extensional.  There will be a greatest fixed point \ldots

\subsection{Notes on the seminar of the gang of four}

Zachiri asks: do we know of any structures that obey stratified
separation and choice but fail at least some of unstratified
separation?

In asking this he is lowering his sights slightly from the project to
find a model of KF + $\exists NO$!  Answer: yes, but infinity fails as
well.  Work in NFU + $\neg \AxC$ so there is $n \in \Nn$ with $n < Tn$.  
Then there is $k \in \Nn$ with $k > 2^{Tk}$.  If we now do the Ackermann 
permutation we get a set that looks like $V_\omega$, so it's a model of 
the stratified axioms of ZF but Cantor's theorem fails---since the 
diagonal set that would prove Cantor's theorem does not exist.

While we are on that subject it seems to be an open question in NF
whether the power set of NO is bigger than NO or small or
incomparable.  It clearly can't be same size.  If it can be smaller
than we do the following: work in NFU, in a model where P(NO) is
smaller than NO, and consider the proper class of hereditarily
wellordered sets.  It'll be a model for stratified replacement, power
set and choice but presumably not unstratified separation.  Well, you
did ask!


\subsection*{Inductively defined sets}

Thierry has a nice observation: let $P$ be a property which is
possessed by every transitive set.  Then the least fixed point for

\begin{quote}
$x \mapsto$ set of all $P$-flavoured subsets of $x$   \hfill (A)\end{quote}

is paradoxical.

The point about paradoxical least fixpoints for operations like this
(vary $P$ {\sl ad libitum}) is that they seem to be paradoxical iff
$P$ is is some sense *unbounded*.

I have been writing up a section on inductive definitions in NF for
the handbok article.  This suggests to me an axiom for NF:

        Let $P$ be a set that misses at least one transitive set.  Then the
        least fixed point for (A) above is a set.

Might this be consistent?


$$(\forall P)(\forall x)(\bigcup x \subseteq x \not\in P \to (\exists X)(\forall y)(y \in x \bic (\forall Y)((\forall z)(z \subseteq Y \wedge z \in P \to z \in Y) \to y \in Y))$$


$$(\forall P)(\forall x)(\bigcup x \subseteq x \not\in P \to (\exists X)(\forall y)(y \in x \bic (\forall Y)(({\cal P}_P(Y)\subseteq Y) \to y \in Y))$$

There are transitive sets ($V$) that are not wellordered, so this will
tell us that the set of hereditarily wellordered sets (lfp) is a set.
Similarly we get the set of all wellfounded hereditarily cantorian
sets, the set of all wellfounded hereditarily strongly cantorian sets



\subsection{Refuting duality}

The Lads said:

\begin{quote}
{\bf First}: Add a Quine atom by $\tau = (\emptyset,\{\emptyset\})$;\\
{\bf Second}: Kill off all Quine atoms by $\displaystyle{\tau = \prod_{x \in \iota^2``V}(x, V\setminus x)}$.\end{quote}

Now it should be possible to do it with a single permutation.  I think
the idea is to swap with their
complements-in-the-sense-of-$(\emptyset,\{\emptyset\})$, all those sets
that are double singletons in the sense of
$V^{(\emptyset,\{\emptyset\})}$.  That is to say---writing `$\sigma$'
for the transposition $(\emptyset,\{\emptyset\})$ and `$c$' for 
complementation to keep things readable:
$$\tau:=\displaystyle{\prod_{(x \in \iota^2``V)^\sigma}(x, \sigma c
  \sigma(x))}$$ is the one-stop permutation we want.  (The fact that
this definition is legitimate is nontrivial: it's a great help that
$\sigma c \sigma$ is an involution.  We also need the fact that if $x$
is a double-singleton-in-the-sense-of-$\sigma$ then its
complement-in-the-sense-of-$\sigma$ cannot be a
double-singleton-in-the-sense-of-$\sigma$. This ensures that all the
transpositions in the big product are disjoint.)

\begin{thm}\mbox{\negthinspace}\\
Duality fails in $V^\tau$ because it contains
a Quine antiatom but no Quine atom.\end{thm}

\Proof Clearly the collection $A:= \{x: ((\exists z)(x =\{\{z\}\}))^\sigma)\}$ 
is going to be of interest.  Let's process `$(x \in \iota^2``V)^\sigma$'.
$$(x \in \iota^2``V)^\sigma$$
is
$$(\exists z)(x = \{\{z\}\})^\sigma$$
which is $$(\exists z)(\sigma``(\sigma(x)) = \{\{z\}\})$$
which is $$(\exists z)(\sigma(x) = \sigma``\{\{z\}\})$$
which is $$(\exists z)(\sigma(x) = \{\sigma\{z\}\}).$$

Two things to notice\begin{enumerate}

\item Since every Quine atom is fixed by $\sigma$ every Quine atom
  belongs to $A$.  Everything that starts life as a Quine atom is
  moved.

\item \label{quineatomtau} Notice too that if $x = \emptyset$ then it
  belongs to $A$: $\sigma(\emptyset) = \{\emptyset\} =
  \{\sigma\{\emptyset\}\}$.\end{enumerate}

So what is the fate of $\emptyset$ in the new model $V^\tau$?  
(Let's call it `$a$' in order not to confuse ourselves!)

$$(x \in a)^\tau$$
iff $$x \in \tau(a)$$ Now $\tau(a)$ is the
complement-in-the-sense-of-$V^\sigma$ of $a$ which is $\sigma c \sigma
(a) = \sigma c \{\emptyset\} = \sigma(V \setminus \{\emptyset\}) = V
\setminus \{\emptyset\}$.

iff $$x \in (V\setminus \{\emptyset\})$$
iff $$ x \not\in  \{\emptyset\}$$
iff $$x \not= \emptyset$$
iff $$x \not= a$$
So $a$ is a Quine antiatom in the new model $V^\tau$.

Now let's check that there are no Quine atoms  in the new model $V^\tau$.

Suppose $x$ is a Quine atom in the sense of $V^\tau$.  If $x$ is fixed
by $\tau$ then it was a Quine atom in the model in which we started.
We observed earlier (item 2 p \pageref{quineatomtau}) that any object
that starts life as a Quine atom is moved by $\tau$.  So $x$ is moved.
So $(x$ is a Quine atom$)^\tau$ is

$$(\forall y)(y = x \bic y \in \sigma c \sigma x)$$

We need not consider the case where $x = \emptyset$, since we have
already dealt with that and seen that $x$ is a Quine antiatom.  If $x =
\{\emptyset\}$ then the RHS becomes $y \in \sigma c \sigma
\{\emptyset\} = V$ which is clearly not equivalent to the LHS; clearly
$\{\emptyset\}$ is not a Quine atom in $V^\tau$.  

There remain the cases where $x$ is fixed by $\sigma$. These give

$$(\forall y)(y = x \bic y \in \sigma(V \setminus x))$$

For $x$ to be a Quine atom in $V^\tau$, $\sigma(V \setminus x)$ will
have to be a singleton.  This can happen if $x = V$, for then $V
\setminus x$ is empty and $\sigma(V \setminus x)$ is $\{\emptyset\}$
so $x$ would have to be both $V$ and $\emptyset$, so the case $x = V$
does not give rise to a Quine atom.  The only other way for $\sigma(V
\setminus x)$ to be a singleton is for $V \setminus x$ to be a
singleton, say $\{z\}$ and for it to be fixed by $\sigma$.  In that
case the condition for $(x$ is a Quine atom$)^\tau$ becomes

$$(\forall y)(y = x \bic y \in V \setminus \{z\}))$$

which is clearly impossible.\endproof

In contrast, we have not yet found a permutation model that satisfies duality.




\section*{This can now be deleted i think}

Thierry,

 thank you for your clarification.  I think i now understand what the 
closure of the class of wellfounded sets is, and why.  Tell me if i have 
got this right.

If we think of wellfounded sets inductively (which is the only sensible 
way to think of them) then a set is wellfounded iff it belongs to every 
set that contains all its subsets. We call this WF* (becos WF is to be the 
class of all wellfounded sets)  So WF* is the intersection of all sets 
that extend their own power sets.   This ought to be a paradoxical 
object.  However, if you look closely, the proof of the contradiction 
relies on our ability to perform what Allen Hazen calls *subscission*

$(\forall x y)(\exists z)(z = x \setminus \{y\})$

And subscission fails in GPC.  Subscission would give us WF from WF*, and 
WF is paradoxical!

How do we know that WF* is unique?  Might there not be lots of sets $WF$* 
such that $WF^* \setminus \{WF*\} = WF$?  No there can't be.  Suppose there 
were, then we could take the intersection of all of them---which would be 
a set beco's an arbitrary intersection of closed sets is closed---and that 
intersection would be WF.

   This reminds me of something---and it may be pure coincidence.  If we 
look at this a bit more closely it shows not only that there can be at 
most one $WF^*$ such that $WF^* \setminus \{WF^*\} = WF$; it shows that i 
cannot have two sets $WF_1$ and $WF_2$ such that


$WF_1 \setminus \{WF_2\} = WF_2 \setminus \{WF_1\} = WF$ 

and so on for larger finite loops.  What this reminds me of is a 
conjecture in NF---the universal-existential conjecture:

There is a model of NF satisfying simultaneously every $\forall\exists$
sentence individually consistent with NF.  One thing that appears to be 
consistent is

$(\forall y_1 y_2)(y_1 \setminus \{y_1\} = y_2 \setminus \{y_2\} \to y_1 = y_2)$

and a similar version for loops

$(\forall y_1 y_2)(y_1 \setminus \{y_2\} = y_2 \setminus \{y_1\} \to y_1 = y_2)$

The nonexistence of Quine atoms is a special case.  One reason why 
counterexamples to these assertions are pathological is that they can 
violate $\in$-determinacy.

So three things seem to be connected (i) the failure of subscission needed 
to avoid Mirimanoff's paradox in GPC, (ii) the universal-existential 
conjecture for NF and (iii) $\in$-determinacy.......

\section*{Inductive definitions}

In ZF we cannot in general define inductively defined sets
``top-down'' as the intersection of a suitably closed family of sets.
This is because we cannot---on the whole---rely on there being a set
that contains the founders and is closed under the operations in
question.  (A good illustration of this is the difficulty we have in
proving that the collection of hereditarily countable sets is a set.)
We can do it only ``bottom-up'' by recursion over the ordinals.  It
doesn't much matter how we implement ordinals, and in principle any
sufficiently long wellordering will do.  There's the rub: how do we
know that there always {\sl is} a sufficiently long wellordering?
That's where Hartogs' theorem comes in.  It tells us that if a
recursive definition crashes, it won't be for shortage of ordinals.
In NF the existence of big sets restores the possibility of direct
top-down definitions of inductively defined sets: any inductively
defined set that can be defined at all can be given direct
``top-down'' definition.  (This is for the gratifyingly simple reason
that---whatever your founders and operations---the universal set
contains all founders and is closed under all operations, so when we
take the intersection of the set of all sets containing the founders
and closed under the operations we are not taking the intersection of
the empty set.)  Thus we obtain the effect of Hartogs' theorem without
actually having the theorem itself.

However, altho' such inductive constructions as can be executed at all
can be executed in the direct top-down fashion, it is still possible
to import ordinals into a description of this activity.  Suppose our
inductive construction starts from a set $X$ with a stratified
definition (so it is $\{x:\phi\}$ for some stratified formula $\phi$ with
one free variable) and we want to obtain the least superset of $X$
closed under some infinitary homogeneous operation.  Examples would
be: union of countable subsets; or $F(X) := \{y: (\exists f: y \onto
X)(f$ is countable-to-one$)\}$.  The collection of $F$-stages is the
least set containing $X$, and closed under $F$ and unions of chains.
It is of course a set, and it is---for the usual reasons---wellordered
by $\subseteq$.  Therefore one can associate an ordinal with every
$F$-stage.  (As usual there are several ways of doing it: (i) the set of
stages and the set of ordinals are alike wordered so there is a
canonical map between them; (ii) each stage bounds an initial segment
which has an ordinal for its length. (ii) is guaranteed to work even
tho' (i) isn't.)

Now we are in a position to find an echo of the ZF way of doing
things.  The closure ordinal is in a weak sense well-behaved.  It must
at least be cantorian.  Let $f$ be the map that sends the ordinal
$\alpha$ to the $\alpha$th stage in the construction.  $f$ has a
stratified definition without parameters, so the expression
$$f(\alpha) = f(\beta) \bic f(T\alpha) = f(T\beta)$$ is stratified
(fully stratified: it has no parameters) and can be proved by
induction on ordinals.  This means that if $\alpha$ is the closure
ordinal (that is to say, the least $\beta$ such that $f(\beta) =
f(\beta + 1)$) then so is $T\alpha$.

It would close the circle very nicely if we knew that every closure
ordinal of a stratified recursion were strongly cantorian, but i see
no proof.  Perhaps it's a very strong assumption.  It would follow 
from Henson's axiom CS (``Every wellordered cantorian set is strongly
cantorian'' and i think NF + CS is as strong as ZF).  Is that why Henson 
thought of it\ldots?


\subsection{Hereditarily Strongly Cantorian Sets}

Suppose $V_\omega$ exists.  Then it contains sets of all finite sizes.

If counting fails, then $V_\omega$ contains all its stcan subsets and 
is therefore a superset of Hstcan, but is not equal to it, and Hstcan 
would not be a set.

If Hstcan is a set, so is the set of natural numbers that are
cardinals of its members, so we can prove the axiom of counting.

If Hstcan is a set then it isn't stcan, so it isn't countable: it will
be quite large.

Randall sez $V_\omega$ might be Hstcan\ldots but in those circumstancs i think neither of them would be sets
 
 
\subsection{A brief thought about extracted models}
%Jan 2006
Randall:

Just had a thought.  It concerns permutation models in a general
context (not just NF).  I've used a few times the following trick.
Let $f$ be a (preferably stratified but not necessarily homogeneous)
function satisfying the pseudo-extensionality condition:
\begin{equation}(\forall x \forall y)(f(x) = f``y \to x = y)\end{equation}

The singleton function has this property.  Pseudo-extensionality of $f$ is
useful because then the permutation
$$\pi := \prod_{x \in A}(f(x),f``x)$$
is well-defined.  What is $A$ here?  Could be anything---might be $V$.

Anyway, take $f$ to be $\iota$.  What happens in $V^\pi$?  Not hard to
see that $\pi$ must swap $\emptyset$ and $\{\emptyset\}$ so it adds a
Quine atom.  I got quite excited for a while beco's if $X$ is a
transitive set then $\iota``X$ is a transitive set in $V^\pi$ but that
doesn't really matter.  Of much more importance---it seems to me---is
the following observation.

Remember that there is always the possibility of a ${\cal P}$-embedding 
from $V$ into $V^\sigma$ whenever $\sigma$ is a permutation.  There is 
an obvious recursion:

\begin{equation}i(x) := \sigma^{-1}(i``x)\end{equation}

and if $V$ is actually wellfounded this is a legitimate definition.
For our transposition $\pi$ above, it turns out to be easy to prove
that the injection $i$ is precisely the singleton function.
Presumably in general it is going to be precisely $f$. This struck me.
Does this remind you of anything?  It reminded me of extracted models
of the kind that produce atoms.  Define a new membership relation on
$V$ by saying $x \in_{new} y$ iff $y = \{z\}$ and $x \in z$.  Anything
not a singleton is an urelement.

We seem to be doing something very similar here, the difference being
that the things that aren't copies of old sets become illfounded sets
rather than urelemente.  We don't throw {\sl all} their structure away, 
just some.

It occurs to me to wonder if one can reconstrue in the same way the
extracted models that one uses to get models of NFU Jensen-Boffa
style.  We probably have to be quite careful how we do it, and we
should start with a simple case.  Another thing we have to do is
reconstrue type-theory as a one-sorted theory of sets with an
I-am-the-same-type-as-you relation definable in terms of $\in$.

In this setting applying the permutation $\pi$ above should correspond
somehow to extracting every second type.

\subsection*{Dodgy Characteristic Subgroups}

We can show that whenever $G$ is a subgroup of $H$ all the cosets of
$G$ are the same size even if they are not uniformly the same size.
So if $G \lhd H$ then $H$ is a union of $[H:G]$ things all of size
$|G|$.  So where we have three groups $G \lhd H \lhd I$ we have a
not-completely trivial relation between $[I:H]$, $[H:G]$ and $[I:G]$.
For example $[H:G] \leq^* [I:G]$ and $[I:H] \leq^* [I:G]$.   It should
be possible to show that all these groups are of size $|V|$ so we
shouldn't get toooo excited.

 {\bf Claim}



H I A T U S




This could yet be useful.  Let $S^*$ be the group generated by 
$\{\pi: |\{y: \pi(y) = y\}| \geq^* |V|\}$.  This group may be larger than $S$.

Let $\tau$ be any old permutation, let $A \sqcup B$ be a partition of
$V$ into two strictly smaller pieces.  Then the permutation $\pi$ we
have constructed is a bijection between a subset of $A$ and a subset
of $B$.  That is to say it is a permutation whose support is twice the
size of a subset of $A$, and is therefore smaller than $V$.  So the
complement of its support maps onto $V$, by Bernstein's lemma, making
it an $S^*$ permutation.  So $\tau$ was in $S^*$ too.

\subsection*{Yablo's paradox in NF}

I have just discovered a wonderful connection between Yablo's paradox and 
wellfounded sets and permutation models in NF.

Suppose the largest fixed point for $\lambda x.(\{\emptyset\} \cup
\iota``x)$ exists.  This is the collection of all those $x$ s.t. every
nonempty thing in $TC(x)$ is a singleton.  Let's call it $H$.  Now let
$$\pi := \prod_{x \in H}(x, V \setminus x)$$

(Actually you don't have to swap $x$ with $V \setminus x$: anything large and
distant will do.)  What happens in $V^\pi$?  Suppose $\tuple{x_n: n
\in \Nn}$ were a descending $\in$-sequence of
singletons-in-the-sense-of-$V^\pi$, so that $\pi(x_n) = \{x_{n+1}\}$
for all $n$.  We derive a contradiction from this assumption.

The contradiction we obtain is a version of Yablo's paradox: we ask 
whether or not each $x_i$ is fixed by $\pi$.  $\pi$ swaps with its
complement everything that is a singleton$^n$ for every $n$. Also, if
$x \in H$ then $\pi(x)$ is a singleton, and in this sequence $\pi(x_n)
= \{x_{n+1}\}$.

Suppose $x_k$ is moved. Then one of $x_k$ and $\pi(x_k)$ is a
singleton$^\infty$ and since $\pi(x_k)$ is known to be a singleton (it
is actually $\{\pi(x_{k+1})\}$), it must be $\pi(x_k)$ that is a
singleton$^\infty$.  But then $x_{k+1}$ is a singleton$^\infty$ and is
therefore moved, and moved to $\pi(x_{k+2})$ which is a singleton and
is the complement of $x_{k+2}$.  This is impossible: we cannot have
two singletons which are complements!  So $x_k$ wasn't moved; $k$ was
arbitrary, so they are all fixed.  But if they are all fixed, $x_1$ is
a singleton$^\infty$ and must be moved.

I think this means that in the new model the only things whose
transitive closure consists entirely of singletons are the Zermelo
naturals.  Of course it doesn't prove that the Zermelo naturals is a
set, but it's good for a laugh.  Feel free to make any use of it you
like.

*********************************************************************

Now let's think about how to generalise this.  Let $S$ be a
$1$-stratified property (like being finite, or a singleton or something
like that) with the feature that we can't have both $S(x)$ and $S(V
\setminus x)$.  Suppose further that the set $H :=\{x: (\forall y \in
TC(\{x\}): S(y)\}$ exists.


Let $\pi$ be the permutation $$\prod_{x \in H}(x,V \setminus x).$$  

Notice that every set that is moved is either a thing in $H$ or the
complement of a thing in $H$, and we can always tell which.  

I claim that, in $V^\pi$, every set of things that are $S$ must have
an $\in$-minimal element.

Suppose not, and let $X$ be a counterexample.  Since $S$ is
$1$-stratifiable, $V^\pi \models S(x)$ iff $S(\pi(x))$.  Let $x$ be an
arbitrary element of $X$. We ask: ``is $x$ moved?''.  Suppose it
were. We know that $S(\pi(x))$ so $\pi(x)$ cannot be the complement of
a thing in $H$ so it must be in $H$.  So any $x'$ believed by $V^\pi$
to be in $x$ is also in $H$, and is therefore moved by $\pi$. Moved to
what?  Moved to $\pi(x')$ which we know is $S$, beco's $S$ is $1$-strat.
But then $x'$ and $\pi(x')$ are complements and both are $S$.  This
isn't possible.

So $x$ is fixed.  Now $x$ was arbitrary, so everything in $X$ is
fixed.  What we want to do now is to argue that any given $x \in X$
must now be moved beco's everything in its transitive closure is
fixed.  But this doesn't work: all we know is that everything in $\{y
\in TC(\{x\}): S(y)\}$ is fixed, and that's not enuff to place $x$ in $H$.

*********************************************************************

So on reflection perhaps the Yablo angle is a red herring. Can't we kill off 
all singletons$^\infty$ by swapping every singleton$^2$ with its complement?

\subsubsection{Proving Con(NF) by eliminating cuts from SF}

Randall sez: think about proving inequations in SF.  He sez: prove $x
\not= y$ by exhibiting a set that contains one but not the other. I
say: things might be unequal while having the same stratified
properties.  He sez, this is not a problem beco's consider.  Suppose
we have concluded that $x \not= y$ beco's $x\in x$ and $y \not\in y$.
Then we do a case split: either \begin{enumerate}
\item{$x \in y$} in which case we conclude $x\not= y$ beco's 
$x \not\in x$ and $x \in y$ or
 
\item{$x \not\in y$} in which case we conclude $x\not= y$ beco's 
$x \not\in y$ and $y \in y$ \end{enumerate}

\ldots and we have made one of the two stratified.  But this only works for
weakly stratified formulae

\hole{\say\ how Quine's trick for defining the naturals without
quantifying over infinite sets doesn't do anything for us here.  $x$
is wellfounded iff every $y$ it belongs to that meets all its
nonempty members contains the empty set.  This isn't
constructive---for the same reason as before (but [prove it!] july 1998}


\section{A puzzle of Randall's}

Find a permutation model containing, for each strongly cantorian
cardinal $\alpha$, a set of Quine atoms of size $\alpha$.  Beco's of
the analogy with the sentence IO (that says that every set is the same
size as a set of singletons) and the fact that it's due to Holmes i
shall call it `HO', thus:

\begin{center}  Every strongly cantorian set is the size of a set of 
Quine atoms \hfill HO \end{center}


One thinks immediately of Henson's permutation
$$\prod_{\alpha \in On}(T\alpha \{\alpha\}).$$

This gives a permutation model in which every old strongly cantorian
ordinal has become a Quine atom, and in which every Quine atom arises
from a strongly cantorian ordinal.  The significance of the Henson
permutation in this context is that it gives us a model in which every
{\sl wellordered} strongly cantorian set is the same size as a set of
Quine atoms, whereas what we are after is the same assertion with the
`wellordered' dropped.  Perhaps a similar idea will give us the
stronger result we want \ldots ?


Think: $\iota``V$ is NO, $\{x\} \mapsto \{\iota``x\}$ is $T$.  So let
$\pi$ be

$$\prod_{\{x\} \in \iota``V}(\{\{x\}\}, \{\iota``x\})$$

The trouble is: this analogue of the $T$ function doesn't have enuff
fixed points.  As Randall says, this permutation turns any set of
Quine atoms into a Quine atom.  What one really wants is a kind of $T$
operation on a set larger than any strongly cantorian set.  One can do
this to $BF$ or even the set of all set pictures.  However there are
deep reasons why one cannot do it to $V$.

One would need a set $X$ larger than any strongly cantorian set,
together with a stratified but inhomogeneous injective function $f: X
\to X$ (That is to say, the graph of $f \cdot \iota$ is a set) such
that $f$ has a lot of fixed points.


The {\bf Henson permutation} for $D_n$ is the product of all
transpositions $(\{X\}, T(X))$ for $X \in D_n$.  The question now is:
is there an $n$ such that every strongly cantorian set is size of a
set of $T$-fixed members of $D_n$?

There is a surjection from $D_{n+1}$ to $D_n$ and this surjection
commutes with $T$, so it sends $T$-fixed things to $T$-fixed things.
Thus the chances of a successful search improve as $n$ gets bigger.

Holmes and i both feel that this is the only hope of finding a
permutation that makes his proposition true.

Further observations.\begin{itemize} 
\item If AxCount fails then the Henson permutation makes Holmes' formula true; 
\item There doesn't seem to be any obvious objection to the assertion
  that there is a function defined on $V$ which, to every stcan set
  $x$, assigns a set of singletons the same size as $x$. 
\end{itemize}
Of course, the natural thing to consider is not HO but $\poss$HO.

3/vii/06


\subsection{Part IV Set Theory}

Spend a lot of time explaining stratification and explaining how to
compute sizes of noncantorian sets.

Then talk about cantorian and strongly cantorian sets and subversion
of stratification.  Tell them to read relaxing.tex

**********************************************************************
 
It is important not to think of this as a pathology of NF, and
accordingly as a good reason for eschewing NF.  The correct point to
take away from this is that we have here an important fact about the
nature of syntax and the type distinctions that arise from it.  There
is a moral here for typing systems everywhere.

The hard part is to fully understand stratification.  There is an easy
rule of thumb with formul{\ae} that are in primitive notation, for one
can just ask oneself whether the formula could become a wff of type
theory by adding type indices.  It's harder when one has formul{\ae}
no longer in primitive notation, and the reader encounters these
difficulties very early on, since the ordered pair is not a
set-theoretic notion.  How does one determine whether or not a formula
is stratified when it contains subformul{\ae} like $f(x) = y$?  The
technical/notational difficulty here lands on top of---as so often---a
conceptual difficulty.  The answer is that of course one has to fix an
implementation of ordered pair and stick to it.  Does that mean
that---for formul{\ae} involving ordered pairs---whether or not the
given formula is stratified depends on how one implements ordered
pairs?  The answer is `yes' but the situation is not as grave as this
suggests, and this is for a logically deep reason that I want you to
understand.  Let us consider again the formula $x = f(y)$.  This is of
course a molecular formula, and how we stratify it will depend on what
formula it turns out to be in primitive notation once we have settled
on an imp[lementation of ordered pairs.  If we use Wiener-Kuratowski
ordered pairs then the formula we abbreviate to $x = f(y)$ is
stratified with $x$ and $y$ having the same type, and that type is
three types lower than the type of $f$.  If we use Quine ordered pairs
then the formula we abbreviate to $x = f(y)$ is stratified with $x$
and $y$ having the same type, and that type is one type lower than the
type of $f$.  There are yet other implementations of ordered pair
under which the formula we abbreviate to $x = f(y)$ is stratified with
$x$ and $y$ having the same type, and that type is two or possibly
more types lower than the type of $f$.

The point is that our choice among the possible implementations will
affect the difference in level between $x$ (and $y$) and $f$ but will
not change the formula from a stratified one to an unstratified one.
This is subject to two important provisos:\begin{enumerate}
\item we must restrict ourselves to ordered pair implementations that 
ensure that in $x = \tuple{y,z}$ $y$ and $z$ are given the same type.
\item We do not admit self-application: ($f(f)$).
\end{enumerate}

Thes two provisos are of course related.  The second will seem
reasonable to anyone who thinks that mathematics is strongly typed.
(The typing system in NF interacts quite well with the endogenous
strong typing system of mathematics.)  If we consider expressions like
$x = \tuple{x,y}$ we see that their truth-value depends on how we
implement ordered pairs.  There is a noncontroversial sense (entirely
transparent in the theoretical CS tradition) in which expressions of
this kind are not part of mathematics---in contrast to expressions
like $x = f(y)$ which are.  The only formul{\ae} whose stratification
status are implementation sensitive in this way are formul{\ae} that are not
in this sense part of mathematics.

The second one is a bit harder to understand: why should we not have
an implementation that compels $y$ and $z$ to be given different types
in a stratification of $x = \tuple{y,z}$---or even make the whole
formula unstratified?  

H I A T U S


If we make $x = \tuple{y,z}$ into something unstratified then we
cannot be sure that $X \times Y$ exists, nor that compositions of
relations (that are sets) are sets; converses of relations might fail
to exist; and we will not really be able to do any mathematics. After
all, $X \times Y$ is $\{z: (\exists x \in X)(\exists y \in Y)(z =
\tuple{x,y})\}$ and if $z = \tuple{x,y}$ is not stratified then the
set abstraction expression might not denote a set.

However, even if we muck things up only to the extent of allowing $x =
\tuple{y,z}$ to be stratified with $y$ and $z$ of different types then
we will find not only that some compositions of relations (that are
sets) are not sets but also that for some big sets $X$ (such as $X:=
V$) that the identity function $1_X$ is not a set.  Let's look into
this last point a bit more closely.  Suppse ``$x = \tuple{y,z}$'' is
stratified but with $y$ and $z$ being given different types.  Then $X
\times Y$ is $\{z: (\exists x \in X)(\exists y \in Y)(z =
\tuple{x,y})\}$ which this time is stratified, so $X \times Y$ is a
set.  However if $R \subseteq X \times Y$ and $S \subseteq Y \times
Z$ then $R \circ S$ is $\{z: (\exists x \in X)(\exists y \in
Y)(\exists z \in Z)(\tuple{x,y} \in R \wedge \tuple{y,z} \in S \wedge
z = \tuple{x,z})\}$

This is not stratified.  If the difference between the types of the
two components of an ordered pair is $n$, then $x$ and $y$ have types
differing by $n$, and $y$ and $z$ too have types differing by $n$, and
$x$ and $z$ have types differing by $n$!

The problem with $1_X$ arises because $(\exists x \in X)(y =
\tuple{x,x})$ is not stratified, so its extension is not certain to be
a set.  By the same token no permutation of a set can be relied upon
to be a set.  The (graph of the) relation of equipollence might fail
to be reflexive, or symmetrical, or transitive.

The conclusion is that if we want our implementation of mathematical
concepts into set theory to be tractable from the NF point of view,
then we want a pairing/unpairing function that interprets $x =
\tuple{y,z}$ as a stratified formula with $y$ and $z$ having the same
type.  One such ordered pair is the Wiener-Kuratowski ordered pair
that we all know and love.  In fact in NF we usually use the Quine
ordered pair which i will now explain.

Does the difference between Quine pairs and W-K pairs matter?  Much
less than you might think.  In some deep sense it doesn't matter at
all.  Let me explain.

[discussion of Cantor's theorem here: the problem is caused by the fact
that the argument and the values of the surjection are of different
types.  That cannot be cured by changing from W-K to Quine or Quine to
W-K.]


If your mathematics is strongly typed, and all your mathematical operations are implemented by stratified operations on sets, then everything is OK.  


START REWRITING HERE


There are various standard definitions of ordered pair, and
they are all legitimate in NF, and all satisfactory in the sense that
they are ``level'' or {\sl homogeneous}. All of them make the formula
``$\tuple{x,y} = z$'' stratified and give the variables $x$ and $y$
the same type; $z$ takes a higher type in most cases (never lower). {\sl How} 
much higher depends on the version of ordered pair being used, but there are 
very few formul{\ae} that come out stratified on one version of ordered pair 
but unstratified on another, and they are all pathological in ways reminscent
of the paradoxes.  The best way to illustrate this is by considering ordinals 
(= isomorphism classes of wellorderings) in NF. For any ordinal $\alpha$
the order type of the set (and it is a set) of the ordinals below $\alpha$ 
is wellordered.  In ZF one can prove that the wellordering of the ordinals 
below $\alpha$ is of length $\alpha$.  In  NF one cannot prove this equation 
for arbitrary $\alpha$ since the formula in the set abstract whose extension is
the graph of the isomorphism is not stratified for any implementation of 
ordered pair. Now any wellordering $R$ of a set $A$ to length $\alpha$ gives 
rise to a wellordering of $\{\{\{a\}\}: a \in A\}$, and if instead one tries
to prove (in NF) that the ordinals below $\alpha$ are isomorphic to
the wellordering of length $\alpha$ decorated with curly brackets, one
finds that the very assertion that there is an isomorphism between
these two wellorderings comes out stratified or unstratified depending
on one's choice of implementation of ordered pair!  This is because,
in some sense, the applications of the pairing function are {\sl two}
deep in wellordering of the ordinals below $\alpha$, but only {\sl
one} deep in the wellordering of the set of double singletons.  If we
use Quine ordered pairs, the assertion is stratified---and provable.
If one uses Wiener-Kuratowski ordered pairs then the assertion is
unstratified and refutable. However if one uses Wiener-Kuratowki
ordered pairs there is instead the assertion that the ordinals below
$\alpha$ are isomorphic to the obvious wellordering of
$\{\{\{\{\{a\}\}\}\}: a \in A\}$, which comes out stratified (and
provable).  In general for each implementation of ordered pair there
is a depth of nesting of curly brackets which will make a version of
this equality come out stratified and true.  This does not work with
deviant implementations of ordered pair under which ``$\tuple{x,y} =
z$'' is unstratified or even with those which are stratified but give
the variables $x$ and $y$ different types.  Use of such
implementations of ordered pairs result in certain sets not being the
same size as themelves!

Perhaps a concrete example would help.  Let us try to prove Cantor's
theorem.  The key step in showing there is no surjection $f:X \onto
{\mathcal P}(X)$ by {\sl reductio ad absurdum} is the construction of
the diagonal set $\{x \in X: x \not\in f(x)\}$.  The proof relies on
this object being a set, which it will be a set if ``$x \in X \wedge x
\not\in f(x) \wedge f:X \to {\mathcal P}(X)$'' is stratified.  This in
turn depends on ``$(\exists y)(y \in {\mathcal P}(X) \wedge
\tuple{y,x} \in f \wedge f:X \to {\mathcal P}(X))$'' being stratified.
And it {\sl isn't} stratified, because ``$\tuple{y,x} \in f$'' compels
`$x$' and `$y$' to be given the same type, while ``$f:X \to {\mathcal
P}(X)$'' will compel `$y$' to be given a type one higher than `$x$'.
This is because we have subformul{\ae} `$x \in X$' and `$y \subseteq
x$'.  Notice that we can draw this melancholy conclusion without
knowing whether the type of `$f$' is one higher than that type of its
argument, or two, or three \ldots. We cannot prove Cantor's theorem.

However if we try instead to prove that $\{\{x\}: x\in X\}$ is not the
same size as ${\mathcal P}(X)$ we find that the diagonal set is
defined by a stratified condition and exists, so the proof succeeds.
This tells us that we cannot prove that $|X| = |\{\{x\}: x \in X\}|$
for arbitrary $X$: graphs of restrictions of the singleton function
tend not to exist.  (If they did, we would be able to prove Cantor's
theorem in full generality.)  This gives rise to an endomorphism $T$ on
cardinals, where $T|X| := |\{\{x\}: x \in X\}|$.  $T$ misbehaves in
connection with the sets that in NF studies we call {\bf big} (as
opposed to {\sl large}, as in {\sl large cardinals} (in ZF)).  These
are the collections like the universal set, and the set of all
cardinals and the set of all ordinals: collections denoted by
expressions which in ZF-like theories will pick out proper classes.
If $|X| = |\{\{x\}: x \in X\}|$ we say that $X$ is {\bf cantorian}.
If the singleton function restricted to $X$ exists, we say that $X$ is
{\bf strongly cantorian}.  Sets whose sizes are concrete natural
numbers are strongly cantorian.  $\Nn$ (the set of Frege natural
numbers) is cantorian, but the assertion that it is strongly cantorian
implies the consistency of NF.













\section*{Weakly stratified}

To explain weakly stratified we have to think of stratifications as
defined on {\sl occurrences} of variables not on variables.
Something is weakly stratified if there is a
stratification that gives all occurrences of each bound variable the
same type.  Two occurrences of a free variable may be given two
different types.  If a variable has only one occurrence then it can
never be responsible for the failure of a stratification: each
occurrence can be connected to only one other occurrence of one other
variable.  So what happens if we have three-placed predicates???


If we write an $\in$-restricted-to-small-sets-is-wellfounded condition
into the definition of small we find that $\iota''V$ is not small:
$\iota``V \in \{\iota``V\} \in \iota``V$.  Perhaps the correct notion
of smallness is being the size of a set of singleton$^n$ for every $n$


partiii2006: get straight the definition of extracted model: use
Barnaby's trick: 

We start by thinking of the old $\in$-relation as a
single one-sorted global relation in terms of which one can define the types.

Have an axiom to say that the relation \verb#sametype#$(y_1, y_2)$
defined by $(\exists x)(y_1 \in x \wedge y_2 \in x)$ is an equivalence
relation.  (This is universal-existential, for what it's worth.)  Then
there is a relation $S(x,y)$ which says that $x$ is one type lower
than $y$: $(\exists z_1, z_2)(x \in z_1 \in z_2 \wedge y \in z_2)$


Extensionality now says

$(\forall x_1 x_2)(\verb#sametype#(x_1, x_2) \to (x_1 = x_2 \bic (\forall y)(y \in x_1 \bic y \in x_2)))$

We need the \verb#sametype# clause in lest we make empty sets at different 
types identical.  We could use the other version of extensionality

$$(\forall x_1 x_2)(x_1 = x_2 \bic (\forall y)(x_1 \in y \bic x_2 \in y))$$

but this might upset some purists since it relies on the existence of 
singletons.


We can now set up an axiom scheme of comprehension. Let $\phi$ be a
stratified formula with $k$ variables to wit: $n$ bound variables $z_1
\ldots z_n$, one free variable $x_n$ and remaining free variables
$y_{n +1} \ldots y_k$.  Suppose further that the variable with
subscript $j$ has type $\sigma(j)$ in $\phi$.  Then the following is an axiom

$(\forall x_1 \ldots x_n)(A(x_1 \ldots x_n) \to$

(Where $A$ is the conjunction of all the true assertions about the type 
relations between the various $x$, assertible using $S$)

$(\exists y)(\forall z)(S(z,y) \to (z \in y \bic \ldots$

and now comes the hard bit: we have to restrict the variables to their types, 
in order to make sense when we assert existence axioms like complement etc.


Think of the new $\in$-relation as one-sorted: global.  $x
\in_{extract} y$ iff $\iota^k(x) \in y$ where $y$ is $k+1$ types
higher than $x$

\subsubsection{Cooking up a nontrivial congruence relation for $\in$}


Ain't none.

If $x \sim x'$ then $x \in \{x\} \to x' \in \{x\}$ so $x = x'$

But that's the wrong definition.  What we can sensibly ask for is a
relation $\sim$ such that if $x \sim x'$ and $x \in y$ then there is
$y' \sim y$ with $x' \in y'$.  And the answer to this---on quite weak
assumptions---is `yes'.  Cycles of $\in$-automorphisms are equivalence
classes for equivalence relations like this.

\subsubsection{Modal equivalence classes}


Let us say that $\phi$ and $\psi$ are $\Box$-equivalent if $\Box \phi
\bic \Box \psi$ and $\poss$-equivalent if $\poss \psi \bic \poss
\phi$.  It's not clear to me that these two equivalence relations are
the same, tho' they look as if they should be.

Can we prove any theorems like: Let $\Gamma$ be a class of formul{\ae}
(a quantifer class or something like that) then Every $\poss$-equivalence 
class contains a member of $\Gamma$..?



\subsection{K\"orner Functions}

%jan 2005

A {\bf K\"orner function} is a function $f: X \to X$, $X$ an initial
segment of the ordinals, such that $(\forall x \in X)(x \leq f(Tx))$.
Friederike K\"orner and I realised independently at about the same
time that these were the gadget needed to refine Boffa permutations to
obtain models of $NF$ in which $\in$ restricted to finite sets is
wellfounded.  Friederike used Henson-style Ehrenfeucht-Mostowski
models to show that for any consistent stratified extension $S$ of
$NF$, if $S$ has models at all, then it has models with K\"orner
functions for \Nn.  That's why I call them ``K\"orner functions''.

Friederike's original model had a special kind of natural number,
which i call a {\bf K\"orner number}, which is a natural number $k$
such that for all $k' > k$, $k' < Tk'$.  This gives a K\"orner
function immediately (``add $k$''!)  and this K\"orner function is
inflationary and monotone, but sadly it does not commute with $T$.

First we check that
\begin{rem}
If there is a K\"orner function $f:X \to X$ then there is one ($f^*$
say) that is inflationary and monotone increasing.
\end{rem}

Moreover, if $f$ commutes with $T$ we can take $f^*$ similarly to
commute with $T$.

And, with some very weak, sensible, conditions on $X$:

\begin{rem}
If there is a K\"orner function on $X$ {\rm that commutes with} $T$
then \AxC\ holds\end{rem}

\Proof

Let $f$ be a K\"orner function $\Nn \to \Nn$ that commutes with $T$.
Using the remark, we can safely assume that $f$ is monotone and inflationary.
Define $g(n) = f^n(0).$ We want $g$ to commute with $T$.  Are we to
prove this by induction?  True for $n = 0$.  Now suppose
$$g(Tn + 1) = f(g(Tn)) = f(T(g(n)) = T(f(g(n)) = T(g(n+1)).$$
But is the induction stratified?  We have to give $g$ (and therefore $f$) two
different types, so it isn't stratified, but it is weakly stratified
which should be enough.

Now suppose we have $Tn < n$ for some $n$.  Consider $g(Tn)$ and $g(n)$.
We have $$g(n) \leq f(T(g(n))) = f(g(Tn)) = g(Tn +1)$$

But in these circumstances $Tn + 1 < n$ so $g$ is not increasing!  So
in particular $f$ is not increasing.  But we could have made $f$
increasing by ``rounding-up'', since rounding-up doesn't destroy 
commuting-with-$T$.  (This might involve  some work!)  So we conclude:

\begin{rem}
  If there is a K\"orner function $\Nn \to \Nn$ that commutes with
  $T$, then \AxC\ holds.
\end{rem}


Suppose $f$ is a K\"orner function $NO \to NO$.  Does this proof work?  Is
$g$ defined for all naturals?

 
If $f$ is a function $NO \to NO$ the rounded-up function $f^*$ is defined by

$$f^*(\alpha) := max(f(\alpha), \sum\limits_{\beta < \alpha}^{} f^*(\beta))$$

Two things to check.  If $f$ is a K\"orner function then so is $f^*$ and 
if $f$ commutes with $T$ so does $f^*$.

Notice that the existence of a K\"orner function on $NO$ doesn't
obviously imply the existence of a K\"orner function on the naturals:
if $f$ is a K\"orner function on the ordinals its restriction to the
naturals might not be a K\"orner function on the naturals!

There can be no K\"orner function $f: NO \to NO$ that commutes with
$T$.  Suppose there were, and reason, {\`a la Henson}, about
$\phi(\alpha, f)$.  We argue that the least $\alpha$ such that
$\phi(\alpha, f)$ is finite must be cantorian and we then find that
$|\phi(\alpha, f)|$ is both odd and even. \endproof


(i) The existence of a K\"orner function: $f: \Nn \to \Nn$ s.t. $n
\leq f(Tn)$ fits in nicely here.  I think we will need to consider its
extension to the ordinals.  What about a function $f:On \to On$ st
$\alpha \leq f(T\alpha)$?  It's not obviously impossible.  It clearly
implies that $cf(\Omega)$ is cantorian (and so $\Omega$ is not
regular, contradicting AC$_2$).  If there is such an $f$, set
$g(\alpha) := \sum\limits_{\beta < \alpha}^{}f(\beta) + 1$.  Then $g$
has the same nice property and is both cts and inflationary.  Of
course there can't be such a function which commutes with $T$, since
presumably the least thing moved would be a disaster, or the least
thing that cannot have $g$ applied to it infinitely often and so on.


We might have to consider the extension of K\"orner functions to
BF.  It seems to me that this should have modal consequences.  I mean:
what happens in the model given by the Ackermann permutation?


\section*{$\in$-games}


There is this paradox, that Isaac calls `Forster's paradox', to the
effect that \one\ and \two\ cannot be sets.  In what sense can
$\in$-determinacy hold in a model of NF?  There is a result in the
book that sez that $V$ can be the disjoint union of $X$ and $Y$ where $X
= \pow Y)$ and $Y = b(X)$ but that's a red herring, beco's by the
preceding result, in those circumstances, they can't be \one\ and
\two!

But can there be a global nondeterministic winning strategy for \one?
This would be a relation $E$, say, such that $\{\tuple{\{x\},y}: x E
y\}$ is a set, and $x E y \to x \in y$, the idea being that $x E y$ if
$x \in y$ and \one\ has no winning strategy in $G_y$ or $x$ is a
member of $y \cap \two$ of minimal rank if there is such a thing.
That way all \one\ has to do, on being confronted with $y$, is to reach
for any $x$ s.t. $x E y$.  $E$ must not only be wellfounded but must
satisfy the extra condition that for any $x$, either every descending
$E$-chain starting at $x$ is even, or every descending $E$-chain
starting at $x$ is odd.  This condition, being ``even'' or ``odd'' is
not stratified, so there is no obvious lapse into paradox.  Noteworthy
that the existence of $E$ as a set does not allow us to define a rank
relation, any more than the existence of $H_\kappa$ implies the
existence of $\Pi_\kappa$.

So is there a permutation model containing such a set of ordered pairs??

``There is a set $E \subseteq \{\tuple{\{x\},y}: x \in y\}$ s.t.,
$E^{-1}``\{x\}$ isn't empty unless $x$ is, and for every $x$ either
every descending $E$-chain starting at $x$ is even or every descending
$E$-chain starting at $x$ is odd.''

\section{Stuff to fit in}

$J_0$ is paradoxical (in Wagon's sense).  Any two countable sets are
$J_0$-equidecomposable with one piece, or---simpler---$J_0$
equivalent.  So we can find disjoint sets $A$ and $B$, and $\sigma,
\tau \in J_0$ such that $A = \sigma``A$ and $A \cup B = \tau``A$.  All
we need was a couple of disjoint countable sets.  By the same token,
all we need to show $J_n$ paradoxical is a couple of disjoint set of
singletons$^n$.





 It still isn't clear to me whether or not \AxC implies the
analogue for ctbl ordinals.  What is clear to me is that if i wish to
get to the bottom of this i will have to *really* understand the
theory of ordinal notations.  If you are interested in this you may
wish to take up my suggestion that the way in is to consider why \AxC
implies that $\alpha \leq T\alpha$ for all ordinals below $\epsilon_0$,
for example.  It's beco's we have a system of notation for the
ordinals below $\epsilon_0$ that makes each such ordinal a finite
object---in the sense that there is a bijection between them and \Nn\
that commutes with $T$.  In the standard treatment in the literature
this is just the condition that the bijection be definable.  Now there
is a theorem of Diana Schmidt that says that for each ctbl alpha there
is a `nice' system of notation for the ordinals below alpha.  If the
proof is `nice' enuff then presumably one can recover a proof that the
notation system respects T.  But i think this is going to be hard.
Worth getting to the bottom of tho'....



Find a model for INF in the recursive functions.



\section{Chores and Open problems}

reflect on the following banality.  If $T$ is a
constructive set theory which admits cut elimination,
then $T$ does not refute the existence of $V$.

In particular, no cut-elimination for Zermelo!!
(which we probably knew anyway)

    This argument doesn't use the fact that ${\mathcal P}^n X$
can be a subset of $x$.  The point is this: once you've
got to the line $\vdash t_x \not\in y$, this can only have
arise from $\neg$-R, so take it over to the left.  Then
$t_x \in x \vdash$ doesn't match the output of any sequent
rule.  (You need cut-elimination and constructivity to be
able to reason that any proof must look like that....)

I've looked at constr weaker things that say $V$ isn't a set
and the same argument works.

So at the very least, cut-elimination for KF proves
con (INF)  - and that's without using the bounding
lemma style stuff.


\nf O $\subseteq$ \nf$_3$. 
\nf O $\subseteq$ \nf P $\subseteq$ \nf I. 

Holmes sez \nf$_3$ + \nf P = \nf, beco's \nf$_3$ has unions (which is
what we have to add to \nf P to get \nf) and \nf P has a type-level
ordered pair (which is what we have to add to \nf$_3$ to get \nf).
Note \nf$_3 \not\subseteq \nf I$ beco's $\bigcup x$ is in \nf$_3$.

\nf$\forall \subseteq \nf_3$.

Holmes' axiom of small ordinals: for any property $\phi$ of ordinals
whatever, there is a {\sl set} $X$ s.t. the class of all $\alpha$ such that 
$(\phi(\alpha) \wedge
\alpha = T\alpha) = X \cap$ the class of all $\alpha$ such that $\phi(\alpha)$.

(Explain how this is like P = NP)

In conjunction with large ordinals we can show that there is a
canonical set that will do.  let $\phi$ be any property. It is shadowed
by a set. $C$.  $C$ and $T^{-1}``C$ both shadow it.  If they are the
same, we're done. If not, conside the smallest element of $C \Delta
T^{-1}``C$.  It is above $T^n`\Omega$ for some $n$, so grab
$T^{-n}``(C \Delta T^{-1}``C)$.  Gulp. This is closed under $T$ and
$T^{-1}$.


Randall next sez: call sets which ``commute with $T$'' {\sl natural
sets}.  Then postulate that every property of natural sets (of
ordinals) is coded by a set.



Randall is also concerned about what he calls the ``downward
cofinality'' of the noncantorian ordinals.  How long can a descending
class of noncantorian ordinals be?  A natural axiom to consider is one
that sez that, if you are a noncantorian ordinal, then, for some $n$,
$T^n\Omega$ is below you.  This is something one can approach with
omitting types....


$\Omega$

\begin{enumerate}                                                             

%\item Fra\"{\i}ss\'e formul{ae} for typical ambiguity and duality. Also       compare and contrast the  Fra\"{\i}ss\'e formul{ae} and Barwise approximants to Henkin formul{\ae} announcing isomorphisms.  Done!

\item Randall asks: how about a pairing function that raises types by
one in \nf$_3$?  Does it give \nf? (You can't use his clever pair
beco's---since it looks inside the components---it uses too many
types) Add a primitive pairing relation.

\item Is there a $\forall^* \exists^*$ version of the axiom of
  infinity?  (see Parlamento and Policriti JSL {\bf 56} dec 91 pp
  1230--1235; see also Marko Djordjevic JSL {\bf 68} (2004) pp
  329--339)

\item $\nf \vdash \poss \neg\exists V_\omega$?  If we express AxInf in
  Zermelo in the form ``there is an infinite set" then we cannot prove
  the existence of $V_\omega$ or indeed any particular infinite set.

\item $\nf \vdash Con(TSTI_\omega)$? $\nf \vdash Con(TSTI_{\omega^*})$? Do
      either of these follow from \AxC?

\item Is $\nf_3$ as strong as $TST$? Holmes thinks so.  He adds that
$\nf_3I$ is much weaker than $TSTI$? Perhaps Pabion's result is
relevant here: $\nf_3I$ = second-order arithmetic.

\item Once you've solved the universal-existential question for $\TZT$ do it
      for $\TZT\lambda$.\footnote{The obvious comprehension axiom for $\TZT\lambda$
      is $$(\bigwedge \alpha)(\bigwedge \beta)((\forall x_\alpha)(\exists!y_\beta)(\Phi(x_\alpha,y_\beta))\to (\exists f_{\alpha \to \beta})((\forall x_\alpha)\Phi(x_\alpha,f`x_\alpha)))$$ \ldots with parameters of course!}

\item Are the $f \in \Nn^{\Nn}$ that commute with $T$ cofinal in the partial 
      order under dominance?

\item \AxC$\to (\forall \alpha < \omega_1)(\alpha \leq T\alpha)$?

\item If $\Phi$ is a sentence in arithmetic-with-$T$ that is true of the identity but
      not provable in arithmetic-with-$T$ is there an Ehrenfeucht-Mostowski model in
      which it fails?

\item Andr\'e's question.  $(\exists n \in \Nn)(n \not = Tn \wedge (\forall m < n)(m \leq Tm))$

\item What holds in the constructible model of $KF$?

\item Understand Orey's proof well enough to know whether or not
      \AxC suffices to prove Con(\nf).

\item Takahashi's proof that every $\SiP_n$ formulae is in 
      $\Sigma^{Levy}_{n+1}$.   Does it really need foundation?

\item\label{dominateTinverse} Can there be $f: NO \to NO$ with $(\forall \alpha)(\alpha \leq f`T\alpha)$?

Well, if there is, then AC$_{wo}$ fails beco's $cf(\Omega)$ must be cantorian.
If there is such an $f$ then there must be one that is cts and inflationary i think.  Set $g(\alpha) := \sum\limits_{\beta < \alpha}^{} f(\beta) + 1.$

\item is there a bijection $V \bic V^V$ that enables us to interpret the $\lambda$-calculus in \nf?

\item  Aczel's point about $V\sim V \to V$ being possible constructively.....

\item Is the theory of wellfounded sets in \nf\ invariant?

  Holmes has this permutation that kills off infinite transitive
  wellfounded sets, whatever model you start in.  This means that if
  \AxC, for example, then the theory of wellfounded sets is not
  invariant, since possibly there are infinite transitive wellfounded sets and
  possibly there aren't.


\item Is there always a permutation model of Forti-Honsell Antifoundation?

It says:

$$(\forall X)(\forall g: X \to \pow X))(\exists ! Y,f)(f``X = Y \wedge f = (j`f) \circ g)$$

To find out what $\poss$ of this is, reletter the failures of stratification.

$$(\forall X, Z)(\forall g: X \to \pow Z))(\exists ! Y,f,h)(f``X = Y \wedge h = (j`f) \circ g)$$


$$(\forall X, Z)(\forall g: \pi_{n+1}`X \to \pow \pi_n`Z))(\exists ! Y,f,h)(f``X = Y \wedge \pi_{n+1}h = (j`(\pi_{n+1}f)) \circ \pi_{n+2}`g)$$

$$(\forall X)(\forall g: j^{n+1}`X \to \pow X))(\exists ! Y,f)(f``X = Y \wedge f^{j^n`\pi} = (j`f) \circ g)$$



\end{enumerate}

\subsection{Multisets}



Dear Wayne\footnote{Blizard},

I hope you remember me!  I discovered something yesterday that made me
think of you. to the extent of looking up the ASL membership list in
the hope of getting your email address---which it didn't have---but it
did at least give me a snailmail that looks as if it might reach you.
Here's hoping.


You get dumped on at this point beco's you are the only person i know
who knows anything about multisets!  What i noticed yesterday was the
following.

Ackermann discovered a cute relation on \Nn\ which make it into a copy
of $V_\omega$: set $n E m$ iff the $n$th bit of $m$ is 1.  We can
think of this as giving us a model of ZF minus infinity but with
foundation.  I had the idea that if the antifoundations axioms with
which we are regaled are as natural as we are invited to believe, then
there should be a similarly natural model of ZF minus infinity plus
antifoundation.  Maurice Boffa suggested (and i quote) ``countable
ordinals''.  


I had a look, and it didn't seem to give models of
antifoundation, but it did seem to give models of a multiset version
of ZF minus infinity but with foundation.  It works as follows: Think
about the ordinals below $\epsilon_0$, the smallest $\alpha$ such that
$\alpha = \omega^\alpha$.  By the Cantor normal form theorem every
ordinal $\alpha$ below $\epsilon_0$ is a finite sum $\omega^{\alpha_1}
+ \omega^{\alpha_2} + \omega^{\alpha_3} \ldots$ where the exponents
are nonincreasing.  That is to say, $\alpha$ codes the multiset
$[\alpha_1, \alpha_2 \ldots]$.  Every ordinal below $\epsilon_0$ codes
a unique multset of other---smaller!---ordinals below
$\epsilon_0$. (So it's wellfounded!)

Now I cannot be the first person to have noticed this!  Do you know of
any literature on models of ZF-with-multisets-without-infinity that
arise in this way?  Specifically what happens with larger ctbl
ordinals?

(Do you still write poetry?  I have always treasured the line ``One eye
on woman, the other on death''.  From a one-eyed poet! One wonders:
which eye is which??)

\ \ \ \ \ very best wishes


\smallskip

\ \ \ \ \ \ \ Thomas

\bigskip

\bigskip


Each ordinal $\alpha$ has a unique representation in the form
$2^{\alpha_1} + 2^{\alpha_2} + \ldots $ with that $\alpha_i$ strictly
decreasing.  Consider $\alpha$ as the set $\{\alpha_1,\alpha_2
\ldots\}$. Then $\omega=\{\omega\}$ is non-well-founded, but we only get
non-well-founded sets of particular form. If we only consider ordinals
less than $\epsilon_0$ then we only get one autosingleton. Which part
of Aczel's AFA holds in this case?
                                                 
                                                                    Maurice.


Holmes sez: Marcel has a proof that NFU can be interpreted in the
theory of stratified comprehension.  Define $eq$ to be the relation of
having the same extension.  A thing is a set if it is a union of
$eq$-equivalence classes, otherwise it is an urelement.

\subsubsection{}

Proofs by {\sl reductio} where the {\sl absurdus} is an allegation that
all ordinals can be embedded in the propositum. (!) Specker (and Conway's
generalisation).




\subsection*{}

Diag$(y,x)$ sez $y$ is a formula with one free variable and $x$ is the
result of substituting the gnumber of $y$ in $y$.

We want things like: diag$(y,x) \wedge \phi(x)$.  This is a formula
that sez of itself that it is $\phi$.  Now i want a three-place relation.

$S(A,B,C)$: $A$ is the result of substituting for the free variable in
$B$ the numeral of the gnumber of $C$.

(why don't i know whether or not to insert the words ``numeral of''
before ``gnumber''??) 

So once we start thinking ``permutation models'' we get


$S(A,B,C)^\pi$: $A$ is the result of substituting for the free
variable in $B$ the numeral of $\pi$ of the gnumber of $C$.

So there is an operation \verb#splat# such that

$(\forall ABC)(S(A,B,C)^\pi \bic S((\verb#splat#`\pi)`A,B,C))$.

to be continued


\subsection{Transitive sets}

5/xi/97

The class of all transitive sets is the set of all prefixed points for
the increasing (but nohow cts) function ${\cal P}: V \to V$.  The following
cute facts may be helpful.  $\in\restric$ the set of transitive sets is
transitive.  This is standard.  It's also antisymmetrical: $x \in y
\to x \subseteq y$ and $y \in x \to y \subseteq x$ so $x = y$!  
Transitive sets form a wellfounded CPO under $\subset$.  The funny
thing is: they also form a wellfounded CPO under (the irreflexive part
of) $\in$!  (Actually i'm not sure that they form a wellfounded CPO
but we do know that every set has a GLB, namely its (settheoretic)
intersection $\bigcap$).  We knew this equivalence of $\subset$ and
$\in$ with Von Neumann ordinals but i for one hadn't expected to see
it in this more general context.

The other night i think i had persuaded myself that $\in$ restricted
to the GFP set of hereditarily transitive sets was connected but i
can't now remember why and i now think i was mistaken.



\begin{quote}
Is there anything to be said for adopting an axiom scheme that says
that for any set $x$ and any finite family of stratified (but possibly
inhomogeneous) $\DeP_0$ operations $\vec f$ the $f$-closure of $x$ is
a set?  What we've just shown is that \AxC\ is equivalent to the
special case where $x$ is $\{\emptyset\}$.

We would use this, starting with $\{V\}$ and setting $\vec f$ to be
the stratrud operations, to get lots of models of \nf. Let us write
$Sr(x)$ for the stratrud closure of $x$.  (It might be an idea to pause
and check that $Sr(\{V\})$ does not contain a Quine atom, or
$H_{\aleph_0}$ by showing that if $a$ is a Quine atom then $V\setminus \{a\}$
contains $V$ and is stratrud closed.  Ditto $V\setminus \{H_{\aleph_0}\}$. It
might also be an idea to check that $Sr(\{V\}) \not\in Sr(\{V\})$.)
Another question: if $x = Sr(x)$ does $x$ contain all constructible sets?

Then we consider the inductively defined class containing $\{V\}$ and
closed under $\lambda x. Sr(x \cup \{x\})$.  Consider the wellfounded
part of its sumclass.  That is $L$.

     The attraction of this is that it draws our attention to a new
kind of submodel.  Submodels which preserve complementation are not
transitive. Another way to put it: any model has a universal set. Does
this universal set have to be the same as the domain of the model?
No, of course not.  Another detail to check: does respecting
complementation ensure that inclusion is a 1-embedding? One needs $B$
as one of the operations but relativised $B$ is stratrud \ldots

\end{quote}
(18.viii.97)

Actually that isn't {\sl quite} what we want.  We want the
intersection of all stratrud-closed sets containing $V$ {\sl that also
contain wellfounded sets of arbitrarily high rank.}---because there
doesn't seem to be any reason to believe that there can't be a
countable set with the first condition and we want something that has
the effect achieved in the ZF case by requiring the sets concerned to
contain all Von Neumann ordinals. We haven't got a rank function that
is a set, but it doen't matter, because (see section ~\ref{rank}
remark ~\ref{rank1}) all the various rank relations that we might have
all agree on wellfounded sets.  Perhaps we could replace ``{\sl that
also contain wellfounded sets of arbitrarily high rank.}'' with {\sl
that meets every ordinal containing a wellordering of a wellfounded
set.}''  Are these perhaps equivalent? They are both attempts at
saying ``contains all Von Neumann ordinals'' which is emphatically
{\sl not} what we {\sl really} mean beco's they might stop at
$\omega$.

\subsubsection{}


Can we define $L$ as the intersection of all rud-closed sets $X$ such
that $(\forall \alpha)(X \cap V_\alpha \in X)$?


\subsection*{}

$\poss \exists V_\omega$ {\sl ought} to be equivalent to an assertion
in arithmetic-with-$T$ \ldots but which?  The search from obscure bits
of unstratified arithmetic reminds me of the (at times) acrimonious
exchange between Richard K and me about the unstratified version of
Paris-Harrington. 

A subset of \Nn\ is relatively large (or `0-large` for short) if its
size is bigger than its smallest member.  Thereafter $x$ is
$n+1$-large iff $x$ minus its bottom element is $n$-large.  Now let
$f$ be a slowly increasing function $\Nn\ \to \Nn$.  We say $x$ is
$f$-large iff $x$ is $f(|x|)$-large.  Does this give a version of P-H?

\subsubsection*{}

In my 1980 Namur article i considered the possibility of a form of
strong extensionality where if $x$ and $y$ are $n$-similar for every
$n$ then they are identical.  Now in \TZT\ we can easily arrange for
this since \TZT\ locally omits the type $\{x \sim_1 y, x \sim_2 y,
\ldots x \sim_n y \ldots x \not= y\}$.  This is probably worth
spelling out.

   Suppose $\phi(x,y)$ implies $x \sim_1 y, x \sim_2 y,
\ldots x \sim_n y \ldots x \not= y$.  Let the two variables be of type $n$ in $\phi(x,y)$\ldots



\subsubsection{surjections}

If $\M$ and $\N$ are two natural models of simple type theory, with
$f$ a surjection from the bottom type of $\M$ onto the bottom type of
$\N$, then we can lift $f$ successively to surjections from the $n$th
level of $\M$ onto the $n$th level of $\N$ by the obvious recursion:
$f(x) =: f``x$.  This shows that $\N$ is a homomorphic image of $\M$.
This implies ambiguity for positive formul{\ae}.  Let us say a formula
is {\sl stable} if it is preserved both ways.  Then $x = \emptyset$ is
stable. Let us say a term $t$ is stable if $x = t$ is stable.  Then
$\{t_1 \ldots t_n\}$ is stable if the $t_i$ are stable.



Observe that every model of TST is dual, and the dual of a positive
formula is a special kind of negative formula, where we have $\not\in$
and never $\in$, but = and never $\not=$.  So $f$ will preserve any
conjunction of disjunctions of positive formul{\ae} and duals of
positive formul{\ae}.


Now consider a ``basic'' $\forall^*\exists^*$ sentence in the bigger
model.  It says that, for all $\vec x$ if the things in the tuple are
related in a certain way [conj of atomics and negatomics] , then we
can add a lot of stuff to obtain a larger tuple related in some way.
It has an antecedent and a consequent.   

We want to see how much of such a basic AE sentence we can recover by
using only stable AE basic sentences.  Such sentences either have
negative antecedents and posive consequents or positive antecedents
and negative consequents

Any antecedent is a conjunction of a purely positive antecedent and a
purely negative one.  These two conjuncts can be thought of as the
antecedents of a purely positive basic AE fmla and a purely negative
one.  Then we look up all the stable basic AE sentences with those
consequences, and infer the consequences.  Unfortunately that doesn't
do very much for us.


\subsection{An axiom for H?}

For any property $\phi$, let $H_\phi = \bigcap\{x:\powk{\phi}x)
\subseteq x\}$.  Easy to show that $H_\phi \not\in H_\phi$ beco's
$\neg\phi(H_\phi)$.  So $H_\phi$ can be taken as a generic example of
something that is not $\phi$?  Tasty!  Let's see what goes wrong.  If
$\phi$ is self-identity then we get WF, which cannot be a set, so this
is only going to work if there are some things that aren't $\phi$.  It
won't work if $\phi$ is transitivity beco's that way we get the von
Neumann ordinals.  So we chuck out unstratified properties as well.
But then we have things like being hereditarily not equal to $V$ (and
in ZF we'd have a problem with `wellordered') or we express it in
terms of sets.  So how about:

$WF \not\subseteq x \to H_x$ is a set?

This implies for example that if there are any infinite wellfounded
sets then $V_\omega$ exists.  Unlikely to be a theorem of \nf\ but not
obviously terribly strong.   Is it related to assertions of the kind
$WF \prec_{\Gamma} V$?

I once had an axiom that said for each $\Phi$ either $H_\Phi$ is a set
or it is WF.  This doesn't work beco's $H_{trans}$ is paradoxical but
not equal to WF.  Presumably we have to restrict it to $\Phi$ that are
downward-closed.


\subsection{A message from Holmes on reflection}

The idea is to redefine $x \in y$ as $Tx \in y$ (where the older $\in$ is
the natural relation on isomorphism classes of digraphs).  But this
does not work out exactly as one would wish.  The definition which
works is to define $x \in y$ (new sense) as $Tx \in y$ and for all $z \in
y, T^{-1}z$ exists.  This gives a fine interpretation of NFU!

To get an interpretation of NF, you need a class of isomorphism types
such that all ``elements" are images under T and which has adequate
comprehension properties.  Even in NF, I haven't been able to define
such a class; in fact, there is no reason to expect that one could, since
an interpretation of NFU constructed in this way will generally
satisfy the Axiom of Endomorphism, which is false in NF.

                                        --Randall

But a suitable version of NFU will reflect itself exactly in this
way!  --Randall
\subsubsection*{}

Even if $H_{\aleph_0}$ exists there is no guarantee that we can define
functions on it by $\in$-recursion.  However we can try the following.
Start with the branching quantifier formula that says that there is a
function of the sort you want, and then look at the approximants.  

A good place to start would be with the formula that says that $f`x =
$ sup $T2^{f`y}$ for $y \in x$, or perhaps the formula that says there is a homomorphism from
$\tuple{FIN,\in}$ to $\tuple{\Nn,<^T}$


%\newcommand{\Henkin}[4]{^{\forall{#1}\exists{#2}}_{\forall{#3}\exists{#4}}}

This is $$A:\ \ \ \  \Henkin{y_1\in FIN}{n_1}{y_2\in FIN}{x_2}(y_1 \in y_2 \to Tn_1 < n_2\  \wedge\ y_1 = y_2 \to n_1 = n_2)$$

One also immediately thinks of branching-quantifier formul{\ae} saying
that $<^T$ is wellfounded. (or rather, that there is a homomorphism from
$\tuple{\Nn, <^T}$ to $\tuple{\Nn, <}$.) This is
$$\Henkin{m_1}{n_1}{m_2}{n_2}(Tm_1 < m_2 \to n_1 < n_2 \wedge m_1 = m_2 \to n_1 = n_2)$$
But even the {\sl first} approximant implies \AxC.

There is also the formula stating that there is a homomorphism in the opposite direction:
$$\Henkin{m_1}{n_1}{m_2}{n_2}(m_1< m_2 \to Tn_1 < n_2 \wedge m_1 = m_2 \to n_1 = n_2)$$
\ldots which is presumably true.
But what is the difference between $A$ and

$$A': \Henkin{y_1 \in FIN}{n_1\in \smallNn}{y_2\in FIN}{x_2\in \smallNn}(y_1 \in y_2 \to n_1 < n_2 \wedge y_1 = y_2 \to n_1 = n_2)$$


Isn't this going to show something quite general? Namely that assuming
that a structure is wellfounded is no stronger than assuming that it lacks loops.




\section{A message from Isaac}


[A] Big Sur is real - it is on the cliffs overlooking the Pacific
Ocean, about 200 miles south of San Francisco. Its unique character is
something like this: Rapidly changing conditions and views, but it
(almost) always looks like something out of a Chinese landscape
painting. Big Sur is associated with Henry Miller and Robinson
Jeffers, who both lived there.  Also the Esalen Institute, a
cutting-edge humanistic/peak-experience academically-oriented
psychological institute there. Steep cliffs, deep forests, difficult
access, unspoiled.

[B] I looked up 'burble' in the OED: It is a verb which means to confuse,
confound (to *paradox* ?). I wonder how to use it?


I have a lot of thoughts relating to your paradox and game, here are my current
rough ideas (I am forwarding these thoughts to you as-is because I suspect you
can think through some of them very rapidly, whereas it might take me several
months; also I surmise that some feedback may be useful to you before I got to
Big Sur. When I return, I will attempt to write up some thoughts in a more
thorough fashion)

[C] Regarding time limits in your game: Let's suppose that for all x,
Player I has a winning strategy. Then for all x, we can associate a
"rank" for x. E.g., 0 has the lowest rank, all well-founded sets have
the standard rank, the rank of {a,b} is one higher than the rank of
either member, and so on.

The ``rank" is a measure of how fast Player I can win - that's what I
meant about time limit.

[D] Is the following true?

     (NF is consistent) $\to$  (NF + Player-I-always-wins) is consistent


[E] (Straight off, I think that Player-I-always-wins is a truth about sets)


In Malitz set theory, Player I always wins. In NF, this is not the case (e.g.
suppose the game begins with V,  and V minus its own singeleton.)

On account of this, you could say that in Malitz set theory, all sets are
semi-well-founded.


[G] Items I am thinking about:

     Hypothesis [D] above.

     Is there some variation of the Malitz Game for which it is consistent
     that Player I always wins in NF?

     The Malitz Game is nice because it leads to a characterization
     of all sets as being semi-well-founded, it provides simple ways
     to build models. Is there a variation of the Malitz Game or
     Forster's game that allows similar stuff for NF?!?


%From ardm@bianya.crm.es Wed May  8 20:09:48 1996
%Received: by emu.pmms.cam.ac.uk (UK-Smail 3.1.25.1/1); Wed, 8 May 96 20:09 BST
%Received: by bianya.crm.es (AIX 3.2/UCB 5.64/4.03)
%          id AA18518; Wed, 8 May 1996 21:13:28 -0500
%Date: Wed, 8 May 1996 21:13:28 -0500
%From: ardm@bianya.crm.es (Adrian Mathias)
%Message-Id: <9605090213.AA18518@bianya.crm.es>
%To: T.Forster@pmms.cam.ac.uk
%Subject: your conjecture proved
%Cc: aki@math.bu.edu
%Status: RO

%\nopagenumbers
\section{A message from Adrian}

Let $M$ be the model obtained as follows. 
Put $t = \{0, \{0\}, \{\{0\}\}, ...\}$. 
Notice that $t$ is transitive. Set $M(0) = t$, $M(n+1) = Power(M(n))$ and 
let $M = \bigcup_{n < \omega} M(n)$. Then $\omega$ is not a member of $M$: 
that follows from our first 
\smallskip
\noindent {\bf Lemma }  $x \cap \omega = n$ implies $Power(x) \cap \omega = n+1$
\smallskip
\noindent which is readily proved by induction on $n$ and since 
$t \cap \omega = 2$ has the  
\smallskip 
\noindent {\bf Corollary } $Power^k(t) \cap \omega = k + 2$
\smallskip
$M$ is a model of the rest of Zermelo (the set of $M(n)$'s is fruitful in 
the sense of my paper except for 1.0.1), and  
$t$ is a member of $M$, and is Dedekind infinite under the map
$z \mapsto \{z\}$. 
\smallskip
{\sl [amusing question: does this model contain a relation on $t$ which well-orders 
it in order type $\omega$ ? actually it does, but can you prove  
in our weakened Zermelo, that there is a dedekind infinite set which is 
well-orderable ?
If you're desperate, start from the assumption that there is 
a set $z$ such that $0 \in z$ and whenever $y \in z$ then $\{y\} \in z$.  

Bonus marks if you do NOT USE the power set operation.]}
\medskip

On the other hand, if you start from $N(0) = \omega$ and then iterate 
the power set operation $\omega$ times, 
you get a model of Zermelo containing $\omega$ 
but not containing $t$. Call it $N$. 
\medskip

\noindent {\bf Theorem} $N \cap M = HF$.  
\smallskip

Define $z(0) = 0$, $z(n+1) = \{z(n)\}$, so $t = \{z(n)\mid n \in \omega\}$. 

Define $s(n) = \{z(m) \mid m < n\}$. 
\smallskip

\noindent {\bf Lemma } $0 =s(0)$, $1 = s(1)$, $2 = s(2)$. 
%\smallskip

there it stops, baby. 
\smallskip

\noindent {\bf Lemma } $\omega \cap t = 2 = s(2)$. 
\smallskip
\noindent {\bf Lemma } $x \cap t = s(n)$ implies $Power(x) \cap t = s(n+1)$. 
\smallskip
\noindent {\bf Corollary }  $Power^k(\omega) \cap t = s(k+2)$. 
\medskip
\noindent {\sl Proof of the theorem: }Suppose $x \in Power^k(t) \cap Power^m(\omega)$. 
We show that $x \in HF$. 
\smallskip
Case 1: $k \geq m$: then $\bigcup^m x \subseteq Power^{k-m}(t) \cap 
\omega = k - m + 2$

Case 2: $k < m$: then $\bigcup^k x  \subseteq t \cap Power^{m - k}(\omega) 
= s(m - k + 2)$. 
\smallskip
In either case, $\bigcup^j x $ for some finite $j$ is a subset of a 
hereditarily finite set, and therefore $x$ is hereditarily finite. $\dashv$

%\end



\section{Does \nf + \AxC prove $Con(\nf)$?}

Since \nf +\AxC proves $Con($Zermelo$)$ and various people have conjectured
that \nf\ is no stronger than Zermelo, we would expect that \nf +\AxC
proves $Con(\nf)$.

Actually \nf +\AxC proves $Con($Zermelo$)$ by a pretty roundabout route
(You have to prove that there is a wellfounded extensional relation of rank
$\omega_\omega$ with no holes, and you infer this from the existence of
sets of size $\aleph_\omega$) so we shouldn't be too discouraged by the
apparent difficulty of proving that \nf +\AxC proves $Con(\nf)$. See Roland:
\nf\ et l'axiome d'universalit\'e: jaune $n$)

Anyway, if we are to show that \nf +\AxC\ proves $Con(\nf)$ the obvious
thing to do is to try to recreate in \nf + \AxC Orey's demonstration that
$\nf +$ Axiom of counting$\vdash Con(\nf)$.

OK, let's have an Orey model with four types.  That is to say $T_0 =
\iota^{3}``V$; $T_1 = \iota^{2}``V$; $T_2 = \iota``V$ and $T_3 = V$. Also
$\in_2$ ($\in$ between types 2 and 3) is $\subseteq$, $\in_1$ ($\in$
between types 1 and 2) is $RUSC(\subseteq)$, $\in_0$ ($\in$ between
types 0 and 1) is $RUSC^2(\subseteq)$. Let the variables of bottom
type be $a$ with subscripts. Then $b$ for type 1 and so on.

We might be interested in assignment functions $f$ that commute with $T$ in the
sense that \begin{quote}
$(\forall n)(\forall x)(f`\ulcorner d_n \urcorner = x \to (f`\ulcorner c_{Tn}
\urcorner = \{x\})) \wedge (\forall n)(\forall x)(f`\ulcorner c_n \urcorner = x
\to (f`\ulcorner b_{Tn} \urcorner = \{x\})) \wedge (\forall n)(\forall
x)(f`\ulcorner b_n \urcorner = x \to (f`\ulcorner a_{Tn} \urcorner =
\{x\}))$\end{quote}
but this condition is clearly unstratified.  The right thing to do is to look for 
some relation on which to do induction that is wellfounded only if \AxC\ holds.

There is a pretty obvious tro on assignment functions.  If $f$ sends
`$a_n$' to $x$, $f^*$ must send `$b_{T^{-1}n}$' to $\iota^{-1}`x$; if $f$ sends
`$b_n$' to $x$, $f^*$ must send `$c_{T^{-1}n}$' to $\iota^{-1}`x$; if $f$ sends
`$c_n$' to $x$, $f^*$ must send `$d_{T^{-1}n}$' to $\iota^{-1}`x$.

Slight worry about this: $f^*$ contains less information than $f$
beco's it says nothing about what happens to $a$ variables.



To recap.  Type 0 is $\iota^{3}``V \times \{0\}$; type 1 is $\iota^{2}``V \times \{1\}$;
 type 2 is $\iota``V \times \{2\}$; type 3 is $V \times \{3\}$.   The tro $\tau$ is
$\lambda x. \tuple{\iota^{-1}\fst(x), \snd(x) + 1}$.

Notice that the $+$ operation on formul{\ae} must commute with $T$ if we are to 
stay sane, but it will commute if the gnumbering is natural and recursive

If $f$ is an assignment function defined on variables of type 0, 1 and
2, then $caf(f)$ (Orey's notation) is the function that, on being
given a variable of types 1 2 or 3, with gnumber $n$, shunts it down
one type (remember + and its inverse are homogeneous operations!), 
applies $T$ to it (presumably it doesn't matter in which order it does these two 
things since $+$ commutes with $T$) applies $f$ to the
resulting variable to obtain $\tuple{x,k}$ (where $k$ is 0, 1 or 2)
and returns $\tau(\tuple{x,k})$ which is to say $\tuple{\iota^{-1}(x),k+1}$

$$caf(f) = \lambda n. \tau(f(T(n^-)))$$

The idea is that $caf$ is a bijection between the assignment functions
for types 0, 1 and 2 and the assignment functions for types 1,2 and 3.
A P\'etry diagram will show that $caf(g) =f$ is stratified but
inhomogeneous: `$g$' is one type higher than `$f$'.

Is it obvious that $f$ satisfies $Tn$ iff $caf(f)$ satisfies $n^+$?
Is this immediate or something very hard that we have to prove?
(``yes, gentlemen, it is obvious'') and we prove it by structural
induction on formul{\ae}

I think the hard thing to prove is that that $f$ satisfies $n$ iff it
satisfies $Tn$.  (Remember that $Tn$ and $n$ talk about the same
types).  So perhaps what we should be trying to prove is that $n$ is
true (satisfied by all assignment functions) iff $Tn$ is true.  Any
chance of proving this by induction on the funny wellfounded relation
in \Nn?  It's not looking hopeful: There doesn't seem to be any reason
why \AxC\ should be any more useful than AxCount$_\geq$.




This may be the place to think about Andr\'e's axiom scheme.  Also
Friderieke's axiom about fast-growing functions.

Something worth bearing in mind is that \AxC\ is strong only when there are
big sets around: Mac is equiconsistent with \kf. So we must make use of big
sets.


\subsubsection{Maybe the time is ripe to look again at this}

 Randall,

  this isn't profound, but it might amuse you.

 Think: TTT, Tangled type theory with, say, 18 levels. Let $\phi$ be a
 formula with three levels. Then, by Ramsey's theorem, there is a
 selection of four levels giving us a model of $TT_4$ that satisfies
 ambiguity for $\phi$. Since this holds for any model of TTT with 18
 levels, this must be a theorem of TTT with 18 levels. (TTT is a
 first-order theory after all, so we can appeal to the completeness
 theorem). So TTT with 18 levels proves a huge disjunction $D$ with
 $18\choose 4$ disjuncts, saying that there is an extracted structure
 that is a model of $TT_4$ that satisfies ambiguity for $\phi$.


My question is this: do we have any bound at all on the length of a
proof of $D$ in TTT-with-18-levels?  I can't see any.  This reminds me
a bit of the phenomenon of the proof of the pigeonhole principle (for
$n$ pigeons and $m$ pigeonholes) in propositional logic.  There must be 
one of course but the proofs are of length exponential in $n$ and $m$.

Let us now consider TTT with infinitely many levels, and a formula
$\psi$ in the language of TST. No model of this theory can fail to
have an extracted model of height, say, 16 in which ambiguity holds
for $\psi$.  That is to say, TTT with infinitely many levels omits the
type $\Sigma$ that says that no extracted model of height 16 satisfies
ambiguity for $\psi$. Indeed {\sl every model of} TTT with infinitely
many levels omits $\Sigma$.

Thus TTT with infinitely many levels proves a disjunction without
proving any of the disjuncts. Nothing wrong with that\ldots except
that the set of the disjuncts is an orbit of the automorphism group of
the theory!  So if it proves one it ought to be able to prove all of
them.  It's a bit worse than proving $p \vee \neg p$ while not proving
either of $p$ and $\neg p$.


\chapter[Boise Diary August 2014]
{Boise Diary August 2014: Notes on Conversations with Randall}

The intermediate goal is to construct cardinal trees of infinite rank.
What does a cardinal tree of infinite rank look like?  It has a top
element, and branches downwards, and all paths are finite.  What would
be nice would be to find some pre\"existing trees of this kind, and
perhaps use these trees---with their structure---as scaffolding on
which to build a cardinal tree of infinite rank.

The obvious source for trees of this kind is ordinals.  Fix an ordinal
$\lambda$ and consider finite sets of ordinals below $\lambda$.  For
two such finite sets $s$ and $t$ we say $s < t$ if $t \subseteq s$ and
every member of $s \setminus t$ is below every element of $t$. Let us 
reserve the symbol `$<$' for the order relation of this tree.    

So, let us consider that done: $\lambda$ is given, and we are going to
construct an FM model containing a cardinal whose tree is isomorphic to 
the tree of finite sets of ordinals below $\lambda$.  The idea is to 
define a cardinal-valued function $\tau$ from the tree.  $\tau$ must of 
course satisfy the condition $\tau(t \setminus\{$min$(t)\}) = 2^{\tau(t)}$
for all finite sets $t$ of ordinals below $\lambda$.

Coming up with such a function $\tau$ is not completely
straightforward, as the reader can surely believe.  We know that the
existence of a cardinal of infinite rank contradicts choice so we are
going to have to use FM models [in the first instance at least] and
that means {\sl atoms}.  To each finite set $t$ of ordinals below
$\lambda$ w are going to associate a {\bf parent set} and a {\bf
  clan}.  The parent set is just that, a set.  Each member of the
parent set points to a {\bf litter}, and a litter is a set of atoms.
All litters are the same size seen from outside, and that size is a
fixed aleph, always called $\kappa$.  (This $\kappa$ has nothing to do
with $\lambda$ by the way.)  The parent sets (in contrast to the
litters) are emphatically {\sl not} all the same size.  (The litters
are not all the same size from the point of view of the FM model).
The {\bf clan} associated with a finite set $t$ of ordinals is the
union of all the litters pointed to by the parent set associated with
$t$.  In the FM model the litters will be $\kappa$-amorphous, so the
clans will be unions of $\kappa$-many $\kappa$-amorphous sets.  This
will mean that the clans do not have terribly many subsets [in the FM
  model] so the power set of a clan will be a fairly impoverished
object.  Double power sets of clans will, as we shall see, contain a
lot of infomation.

So far, in the endeavour to construct our function $\tau$ defined on
finite sets of ordinals below $\lambda$ we have [so far!] {\sl two}
auxilliary functions: \verb#parent-set# and \verb#clan#.  I haven't
yet told you what the group or the normal filter are, i know; be
patient.

I haven't yet said anything about how the functions \verb#clan# and
\verb#parent-set# are to be defined, but there is clearly going to
have to be some sort of recursion going on.  The first thing to note 
is that we are going to insist that

\begin{dfn}For every $t$, the \verb#parent-set# associated with 
$t$ must extend [a copy of] the \verb#clan# associated with $t 
\setminus\{$min$(t)\}$.\end{dfn}

In fact we are going to be doing a lot of deletion-of-minimum-elements, 
so let us write $t \setminus\{$min$(t)\}$ as $t'$, and $t_2$ is the 
result of deleting the {\sl two} bottom elements\ldots and so on.  The 
quoted text is of course not yet a definition, but it is a constraint 
that our definition will have to meet.  Observe that the \verb#clan# 
associated with a $t$ maps onto the \verb#parent-set# associated with 
$t$ (and may indeed be a lot bigger than it) so this looks as if, as
you walk own the tree, the \verb#parent-set# associated with each node
get bigger and bigger.

We reflected earlier that $\tau$ must satisfy the condition $\tau(t
\setminus\{$min$(t)\}) = 2^{\tau(t)}$ for all finite sets $t$ of
ordinals below $\lambda$.  The tricky part is of course that a finite
set $s$ can be $t'$ for more than one $t$!   We are going to need some tricks.

Here is a very useful elementary observation.  Suppose $A \subset
\verb#parent-set#(t)$.  Consider an $a \in A$, and the set $\cal A$ of
all the subsets of \verb#clan# that are the same size as $a$ in the
sense of the FM model (whose parameters we have not yet specified!!)
Observe that $\cal A$ is a set of the FM model (it is definable, after
all) and $a$ was an arbitrary element of \verb#parent-set#$(t)$.  So
what we have just described is an injection from ${\cal P}(A)$ into
${\cal P}^2(\verb#clan#(t))$.   Let us record this fact

\begin{rem} \label{injectivity} ``Injectivity''

There is an injection from ${\cal P}(\verb#parent-set#(t))$ into
${\cal P}^2(\verb#clan#(t))$.
\end{rem}

The effect of the definition and the remark is that the iterated power
sets of the \verb#clan#s associated with finite sets $t$ will have
concealed within them copies of the \verb#clan#s associated with the
various truncations $t_n$ of $t$, and of course $\tau$ associates
larger cardinals to those truncations than it does to $t$.

\medskip


Now we are in a position to start defining $\tau$.  We start with what
Randall calls ``base clans''. These are the clans corresponding to
finite sets $t$ that have $0$ as a member.  Clearly $\tau$ of such a
finite set is going to be an endpoint of the cardinal tree we are
trying to build.   We stipulate

\begin{dfn}
When $0 \in t$ we stipulate $\tau(t) = 2^{2^{|{\tt clan}(t)|}}$.\end{dfn}

Now comes the bit i don't yet understand.   We intend to ensure that

$\tau(t_i) = |{\cal P}^{i+2}(\verb#clan(t)#)|$ for every $t$ with $0 \in t$.

To this end we will want \verb#parent-set#$(t)$ to be something the same
size as 
\begin{center}
\verb#clan#$(t') \cup \displaystyle{\bigcup_{0\in s < t}{\cal P}^{|s| - |t|+1}}($\verb#clan#$(s))$
\end{center}
and this looks as if it could be [part of] a recursive declaration of
\verb#clan# [the recusion seems to be on $<$ but we have to tweak
  things so that $t'$ is arlier han $t$] but there is actually some
circularity involved.

Let us consider a simple case.  Suppose $t$ is $\{2\}$.   Then the $s$s 
over which we have to take the union are $\{0,2\}$  and $\{0,1,2\}$. 
That is to say, \verb#clan#$(\{2\})$ must be the same size as

\begin{center}\verb#clan#$(\emptyset) \cup {\cal P}^3(\verb#clan#(\{0,1,2\}))
\cup {\cal P}^2(\verb#clan#(\{0,2\}))$.\end{center}

We haven't yet defined \verb#clan#$(\emptyset)$ (it is actually going
to be a union of $\kappa$-many $\kappa$-amorphous sets of atoms) but
that's not where the problem lies.

Observe that, by injectivity, ${\cal P}^2($\verb#clan#$(\{0,2\}))$ has
a subset the same size as ${\cal P}($\verb#parent-set#$(\{2\}))$.  So
it would seem that we have to have access to
\verb#parent-set#$(\{2\})$ and we are back where we started.

Don't understand this yet \ldots !

%\end{document}

\chapter{Leftovers from the Boffa festschrift paper}


There are various loose ends to be tidied up.\begin{itemize}

\item There is the game $G^*_X$ played like $G_X$ only player \one\ 
wins if it ever comes to an end (as opposed to being the last player!).  
There is a dual version in which \two\ is trying to get it to end.

\item Some miscellaneous facts about $\canon$.

We know that $\canon$ is a strict partial order.  Is it also a
complete lattice?  The (easy) answer is: no.  Consider the two
sequences of $a_n$ and $b_n$ as above.

$$a_0 := \emptyset;\ \ a_{n+1} := \{b_n\};\ \ \ b_0 := V;\ \ b_{n+1} := \mathop{-}\{a_n\}$$

If we were to have $a_\infty := \displaystyle{\bigvee_{i \in \smallNn} a_i}$ and
$b_\infty := \displaystyle{\bigwedge_{i \in \smallNn} b_i}$ we would have
$a_\infty = \{b_\infty\}$ and $b_\infty = \mathop{-}\{a_\infty\}$.  This is 
independent of (for example) \nf. (See Forster [1995] proposition 3.1.5.)

\end{itemize}



Antimorphisms not monotonic on the $\canon$. For suppose they were.
Then let $\sigma$ be an antimorphism. Then $$\sigma `x \ < \ \sigma
`y$$ iff $$-\sigma ``x \ < \ V\setminus\sigma ``y$$ iff $$\sigma ``y \ <
\ \sigma ``x$$ iff (several cases! such as) $$(\exists z \in
\sigma``(x\setminus y))(\forall w \in \sigma ``(y\setminus x))(z <
w)$$ Now reletter $$(\exists z\in (x\setminus y))(\forall w\in
(y\setminus x))((\sigma^{-1}`z < \sigma^{-1}`w)$$ and invoke
monotonicity $$(\exists z \in (x\setminus y))(\forall w \in
(y\setminus x))(z < w)$$ which is $$y < x$$ so $\sigma$ would
have to be antimonotonic.

Note that $(\forall \sigma)(j^n`\sigma$ is an automorphism of
$\tuple{V,\subseteq_n})$.  So the class of automorphisms of the
canonical p.o. is closed under $j$.
 
  Now consider the CPO $V \times V$ ordered by pointwise set inclusion.
Let $S$ be the map $\lambda X. \tuple{\pow\snd X), \b(\fst X)}$ which is an
increasing map $V \times V \to V \times V$.  $V \times V$ is clearly chain
complete (directed-complete, indeed), and so has a fixed point for $S$.
The displayed formula tells us that the least such fixed point is the pair
$\tuple{\two, \one}$.  We will need this slightly cumbersome formulation in
the proof of the following theorem which ties together the $\in$-game and
fixed points for $P$.

\begin{thm}
$$(\forall x \in \two)(\forall y \in \one)(x \canon y)$$
\end{thm}

\Proof

There is a simple proof by induction on pseudorank. If $y \in \one$ and $x
\in \two$ then there is $z \in y \cap \two$.  This $z$ cannot be in $x$,
because $x \subseteq \one$ and by induction hypothesis it precedes
everything in $x$.  So $x \canon y$. \endproof

However, some readers might prefer something a bit more general and robust.
        
\Proof

Suppose $P(R) \subseteq R$.  Suppose $A \cap B = \emptyset$ and $\tuple{A,B}$
satisfies $(\forall x \in A)(\forall y \in B)(xRy)$.  Then so does
$\tuple{\pow B), \b(A)}$. $\pow B) \cap \b(A)= \emptyset$ is easy.  Suppose $x \in
\pow B)$, $y \in \b(A)$.  Notice that $y \setminus  x$ is nonempty because 
$y$ meets $A$ and $x \subseteq B$.  Everything in $x \setminus  y$ is in 
$B$, and there must be something in $y \setminus  x$ that is in $A$, so 
$\tuple{x,y} \in P(R)$ whence $\tuple{x,y} \in R$.

  Now consider the CPO ${\cal P} = \tuple{P, \leq_P}$ where $P$ is the
set of of pairs $\tuple{A,B}$ where $(\forall x \in A)(\forall y \in
B)(xRy)$, and $\leq_P$ is pointwise set inclusion.  Let $S$ be the map
$\lambda X. \tuple{\pow\snd X), \b(\fst X}$ which is an increasing map
${\cal P} \to {\cal P}$.  ${\cal P}$ is clearly chain complete (closed
under directed unions), and so has a fixed point for $S$.  But this
fixed point for $S$ must be above the least fixed point for $S$ in the
CPO $V \times V$, so by induction we infer that the least fixed point
for $S$, namely $\tuple{\two, \one}$ satisfies $(\forall x \in
\two)(\forall y \in \one)(x R y)$. \endproof



Andy Pitts suggested to me that $x$ and $y$ are Forster/Malitz bisimilar
iff there is a bisimulation between the transitive closures $TC(x)$ and
$TC(y)$.  This isn't quite true.  The left-to-right implication is good: if
$X \sim_{min} Y$ then \equal\ has a strategy to stay alive in the game
$G_{X=Y}$ for ever.  The union of any number of nondeterministic strategies
to do this is another nondeterministic strategy, so think about the union
of all of them. It's a bisimulation.  But the converse direction is not
good.  Consider $V$ and $\mathop{-} \emptyset$.  These have the same
transitive closure but \notequal\ Wins the Malitz game by picking
$\emptyset$.  To state the version of this \apercu\ that {\sl is} true we
need the notion of a {\bf layered bisimulation}.

A layered bisimulation between $X$ and $Y$ is a family of binary relations
$\simeq_n$ $\subseteq$ $\bigcup^n X \times \bigcup^n Y$ such that $\simeq_{n+1}^+ = \simeq_n$.
Then

\begin{rem}
$X \sim_{min} Y$ iff there is a layered bisimulation between $X$ and $Y$.
\end{rem}

\Proof Obvious.

\section{Lifts}

%october 2000

I'n beginning to understand this better.  Lifts defined using a
leading existential quantifier will preserve irreflexivity and are to
be used on strict partial orders; lifts defined using universal
quantifiers preserve reflexivity and are to be used on quasiorders.
Partial orders are a red herring.

\subsection{Lifts for strict partial orders}

Let's look at some lifts defined using existential quantifiers, and apply them
to strict partial orders.

First there is the `obvious' one:

\begin{quote}
$A P(>) B$ iff 
$(\exists x \in A)(\forall y \in B)(x < y)$
\end{quote}

Clearly if $<$ is irreflexive then $P(<)$ is irreflexive, and if $<$
is transitive then $P(<)$ is transitive, so it carries strict partial
orders to strict partial orders.  It actually---quite
separately---preserves asymmetry but (for the moment) we don't care.

Only trouble is, $P(<)$ is an incredibly strong relation.  Let's
redefine $P$ so as to get a lift that might be more useful.

\begin{quote}
$A P(>) B$ iff 
$(\exists x \in A \setminus B)(\forall y \in B \setminus A)(x < y)$
\end{quote}


Evidently $P(<)$ is always irreflexive. It preserves asymmetry.

Sadly it does not preserve
transitivity, as the following example shows.

Define $<$ on the domain $\{a,b,c,d\}$ by $a < b$ and $c < d$.  Then
$\{a,c\} P(<) \{a,d\}$ and $\{a,d\} P(<) \{b,d\}$ but not $\{a,c\}$
below $\{b,d\}$.\footnote{Is this yet another example of the bad behaviour
of the set some combinatorists call `\smallNn'?---beco's its graph looks
like the letter `N'.  See Rival, Contemp Maths {\bf 65} pp. 263-285.
Actually this thing is not an N but we could add one arm and get an N}

Despite this we have the following small factoid which may be useful one day:

\begin{small}\begin{quote}
Let $<$ be a strict total order, then $P(<)$ is transitive.\end{quote}
\end{small}

\Proof 


Let $A$, $B$ and $C$ be three sets such that $A\ P(>)\ B$ and $B\
P(>)\ C$.  That is to say, there is $a \in A\setminus B$ which $<$
everything in $B \setminus A$, and $b \in B \setminus C$ which $<$
everything in $C\setminus B$. We seek an $x \in A \setminus C$ which
$<$ everything in $C\setminus A$.  In fact it will turn out that this
$x$ can always be taken to be $a$ or $b$. Since $a$ may be in
$A\setminus C$ or in $A \cap C$, and $b$ may be in $B\setminus A$ or
$B\cap A$ there are four cases to consider.

$a \in A\setminus C\ \wedge\ b \in B\setminus A$

\begin{quote}
Then $a < b$, so $a <$ everything in $C\setminus B$ and we need only
check that $a <$ everything in $(B \cap C) \setminus A$. But $a <$
everything in $B \setminus A$.  So set $x$ to be $a$.\end{quote}

$a \in A\cap C\ \wedge\ b \in B \setminus A$

\begin{quote}
This case is impossible because $b \in (B\setminus A)$ implies $a < b$
and $a \in A \cap C$ implies $a \in (C\setminus B)$ whence $b <
a$.\end{quote}

$a \in A \setminus C\ \wedge\ b \in B \cap A$

\begin{quote} 
Both $a$ and $b$ are in $A\setminus C$ in this case so both are
candidates for $x$.  $a <$ everything in $(B \setminus A)$ and $b <$
everything in $(C\setminus B)$. Since $<$ is a total order one of them
is smaller, and that smaller one is $<$ everything in $(B \setminus A)
\cup (C\setminus B)$ which is certainly a superset of $C\setminus
A$.\end{quote}

$a \in A\cap C\ \wedge\ b \in B \cap A$

\begin{quote} 

$b <$ everything in $C\setminus B$ so in particular $b < a$.  But $a
<$ everything in $B\setminus A$ so $b <$ everything in $((C\setminus
B) \cup (B\setminus A))$ which is certainly a superset of $C\setminus
A$ as before, and $b \in A\setminus C$ so we can take $x$ to be
$b$.\end{quote}

\endproof

Sadly this really needs the input to be a strict {\sl total} order.

It might be worth ascertaining what properties $P$ preserves

%\end{small}

This suggests that we should use instead the following definition.

\begin{dfn}
$x$ $P(>)$ $y$ if there is a finite antichain $a \subseteq (x \setminus y)$ 
such that $(\forall y' \in y \setminus x)(\exists x' \in a)(y' > x')$.
\end{dfn}


Why an antichain?  Well, if it is just a subset then $P$ of a strict
partial order might not be irreflexive.  And why finite?  This is to ensure
that $P$ is monotone.  That is to say, if $\leq'$ is stronger than $\leq$
then $P(\leq')$ is stronger than $P(\leq')$.  If we do not require
antichains to be finite we might find that $X\ P(\leq')\ Y$ in virtue of
some antichain $\subseteq Y \setminus X$ and we can add ordered pairs to
$\leq$ to get a relation according to which the antichain is a chain with
no least element.  If the antichain is required to be finite this cannot
happen.

\begin{lem}\label{lem:trans}
$P$ is a monotone function from the CPO (chain-complete poset) of 
all strict partial orders of the universe (partially ordered by set 
inclusion) into itself.
\end{lem}

\Proof This new $P$ evidently preserves irreflexivity as before.
The only hard part is to show that it takes transitive relations to
transitive relations.

Let $>$ be a transitive relation and let $A$, $B$ and $C$ be three subsets
of Dom$(>)$ such that $A\ P(>)\ B$ and $B\ P(>)\ C$.  That is to say, there
are antichains $a \subseteq A\setminus B$ such that everything in $(B
\setminus A) >$ something in $a$,  and $b \subseteq B\setminus C$
such that everything in $(C \setminus B) >$ something in $b$.

We will show that the antichain included in $A \setminus C$ that we need as
a witness to $A\ P(>)\ C$ can be taken to be $(a \setminus C) \cup (b \cap
A)$.  Or rather, it can be taken to be that antichain obtained from $(a
\setminus C) \cup (b \cap A)$ by discarding nonminimal elements.

We'd better start by showing that $(a \setminus C) \cup (b \cap A)$ cannot
be empty.  Suppose it were and $x \in b$.  Then $x$ is in $B \setminus A$
and is bigger than something in $a$, $y$, say. Then $y \in C \setminus B$
and is bigger than something in $b$ contradicting the fact that $b$ is an
antichain.  This argument will be recycled twice in what follows.

Let $w$ be an arbitrary element of $C \setminus A$.  We will show that $w$
is above something in $(a \setminus C) \cup (b \cap A)$. There are two cases
to consider.

(i) $w \in C \cap B$.  Then it is bigger than something in $a$.  If it is
bigger than something in $(a \setminus C)$ we can stop, so suppose it
isn't.  Then it is bigger than something, $x$ say, that is in $a \cap C$.
Things in $a \cap C$ are in $C \setminus B$ and so must be bigger than
something in $b$.  If $x$ is bigger than something in $b \cap A$ we can
stop (since this implies that $w$ is bigger than something in $b \cap A$),
so suppose $x$ is bigger than something in $b \setminus A$.  Things in
$b \setminus A$ are in $B \setminus A$ and therefore are bigger than
something in $a$, so $x$ is bigger than something in $a$.  But this is
impossible because $x \in a$.

(ii) $w \in (C \setminus B)$.  Then it is bigger than something in $b$.  If
it is bigger than something in $(b \cap A)$ we can stop, so suppose it
isn't.  Then it is bigger than something, $x$ say, that is in $b \setminus
A$.  Things in $b \setminus A$ are in $B \setminus A$ and are bigger than
something in $a$.  If $x$ is bigger than something in $a \setminus C$ we
can stop (since this implies that $w$ is bigger than something in $a
\setminus C$) so suppose $x$ is bigger than something in $a \cap C$.
Things in $a \cap C$ are in $C \setminus B$ and so are bigger than
something in $b$, so $x$ is bigger than something in $b$. But this is
impossible because $x \in b$.

\endproof

This assures us that we can safely conclude that there is a least fixed
point for $P$ and that it is indeed a strict partial order.  (Notice that
the collection of strict partial orders of an arbitrary set is merely a CPO
under $\subseteq$ {\sl not} a complete lattice---unlike the collection of
quasi-orders of an arbitrary set---so there is no presumption that there
will be a unique greatest fixed point.

Let's just check that the same works for $P$ defined the ``right'' way round.

\begin{dfn}
$x$ $P(>)$ $y$ if there is a finite antichain $a \subseteq (x \setminus y)$ 
such that $(\forall y' \in y \setminus x)(\exists x' \in a)(y' < x')$.
\end{dfn}

Only the last occurrence of `$<$' has been changed.

equivalently 

$y P(<) x$ if there is a finite antichain $a \subseteq (x \setminus y)$ 
such that $(\forall y' \in y \setminus x)(\exists x' \in a)(y' < x')$.

\subsection{Lifts of quasiorders}

The structure of this section should echo that of section ref{},
the the obvious $\forall\exists$ lift is well understood, so we 
procede immediately to

$$X P(\leq) Y \bic (\forall x \in X \setminus Y)(\exists y \in Y\setminus X)(x \leq y)$$


$P(\leq)$ is vacuously reflexive: no problem there.  Trouble is,
it isn't transitive.

Consider the carrier set $\{a,b,c\}$,  with $c \leq a$, $b \leq a \leq b$.
Set $Z := \{a\};\ Y := \{b,c\};\ X := \{a,c\}$.  Then $X P(\leq) Y$ and $Y P(\leq) Z$ but not $X P(\leq) Z$.

It is not yet clear to me whether or not this feature relies on this
$\leq$ being a quasi order and not a partial order.

I think i now have a slightly clearer idea why this finite antichain
is a good idea, to the extent that it is. I think the point is that if
$\tuple{Q,\leq}$ is a WQO, then $\tuple{\pow Q),P(\leq)}$ is one too.  
When comparing two subsets of $Q$ all we have to look at is the two 
(finite!) sets of minimal elements of them.   To complete this explanation
i need to establish that  if $\tuple{Q,\leq}$ is a WQO, then the set of
antichains in $Q$ is WQO by ``everything in me $\leq$ something in you''.

This ought to be easy!
%march 1998



Notice that this operation $P$ is obviously monotone but not obviously
increasing, in the sense that we do not expect (the graph of) $P(<)$ to be
a superset of the graph of $<$. For example if $x = \{y\}$ and $y = \{x\}$
and we add the ordered pair $\tuple{x,y}$ to a relation $R$ over a domain
containing $x$ and $y$ we find that $P(R)$ contains $\tuple{y,x}$.

\begin{center}
\begin{tabular}{c| c | c|}

          &  antisymmetrical & not antisymmetrical \\ \hline
          &                  &      \\
reflexive &  partial order   & quasi-order \\ \hline 
          &                   & \\
irreflexive & strict partial order & ?  \\ \hline  
\end{tabular}
\end{center}
The question mark is my way of reminding myself that there isn't a nice
(read ``horn") property that looks like transitivity with strictness
(irreflexivity) and nontrivial failure of antisymmetry. This is because
$R(x,y)$ and $R(y,x)$ give $R(x,x)$ by transitivity, contradicting
irreflexivity.   We would need to assert that $R(x,y) \wedge R(y,z)$ implies
$R(x,z)$ only if $x \not= z$.





\marginpar{Perhaps this next bit belongs in TZTstuff.tex}

 No model of \TZT\ can contain all copies of the set \two.  (That is to
say, it cannot have \two\ at all types). (This is proved very similarly
to the way that we prove the non-obvious fact that WF cannot be a set
at any level of any model of \TZT.) Suppose it does.  Think about \one\
at level $n$.  This set is a win for player \two\ and has rank
$\alpha$, say.  Its rank is the sup of the ranks of its members beco's
\one\ can choose how long he wants to live.  Now think about \one\ two
levels up.  \one\ is going to lose this game of course, but he can play
$\{\two\}$, forcing \two\ to pick the set \two\ at level $n$ so the
rank of \two\ at level $n +2$ must be greater.  This gives us a
descending sequence of ordinals.

Notice now that if \two\ is present at any level it is present at all
later levels, which is impossible, so there are no levels containing
\two. 

In fact this doesn't depend on the model being $\in$-determinate.

Isn't the point that if \one\ or \two\ exist at any type then they
exist at all types, and that is impossible, rather in the way that
$WF$ if it exists at one level exists at all levels.  I think this is
correct: if we have \one\ and \two\ at a given type we can recover
\one\ and \two\ one type down beco's $b$ and ${\cal P}$ are injective.

%Boffa's birthday meeting in march 2000

Can we obtain models of strong extensionality by omitting types?


\subsection{Totally ordering term models}
%\newcommand{\bbar}{\overline{B}`}


\nf$_2$ is the set theory whose axioms are extensionality,
existence of $\{x\}$, $V\setminus x$ and $x \cup y$.  \nf O is the set theory whose
axioms are extensionality and comprehension for stratified quantifier-free
formul{\ae}.  This is actually the same as adding to \nf$_2$ an axiom
$(\forall x)(\exists y)(\forall z)(z \in y \bic x \in z)$. The operation
involved here is notated ``$B`x$".  $\bbar{x}$ is $\mathop{-}B`x$.  We need
a notion of {\bf rank} of \nf O terms.

Rank of $\emptyset$ is 0; rank of $\mathop{-} t$ := the rank of $t$; 

rank of $t_1 \cup t_2$ := $max($rank of $t_1$, rank of $t_2)$; 

rank of $\{t\}$ := (rank of $t$) + 1.

Those were the \nf$_2$ operations. They increase rank only by a finite
amount.  Finally we have the characteristic \nf O operation.

rank of $B`t$ := the first limit
ordinal $>$ the rank of $t$.

Another fact we will need is that 
\begin{rem}\label{rem:twiddle}
$X \subset_\alpha Y \bic (\mathop{-} Y
\subset_\alpha \mathop{-} X)$. 
\end{rem}

We now prove by induction on rank that 
\begin{thm}
$\subset_{\omega + \alpha}$ (strictly) totally orders
\nf O terms of rank at most $\alpha$. 
\end{thm}

\Proof 


We will actually prove something a bit stronger, since the lift we will be
working with here gives a weaker strict order than the $P$ we considered
earlier.  We will use the lexicographic lift: 
$$X P(\leq) Y {\mbox{\rm\  iff\ }} (\exists y \in Y\setminus X)(\forall x \in X \setminus Y)(y \leq x).\footnote{The quantifiers could be in either order and so could the inequality.  Four possibilities!}$$  
The reasons for our abandoning it originally---namely that it does not
always output transitive relations---do not cause problems in this
special context.

We start with a discussion of terms of finite rank.  Consider the two
sequences $a_0 := \emptyset;\ \ a_{n+1} := \{b_n\}$ and $b_0 := V;\ \ b_{n+1}
:= \mathop{-}\{a_n\}$.  It is simple to prove by induction on $n$ that the
$\{a_i: i< n\}$ are the first $n$ things and $\{b_i: i < n\}$ the last $n$
things in the poset of $\nf_2$ terms ordered by $\subset_\omega$.  (The
$b_n$ don't matter, but we will need to make use of the fact that the
collection of $a_n$ is wellordered by $\subset_\omega$.)

Now we can consider terms of finite rank.  The case $\alpha = 0$ is
just $\emptyset$ and $V$.  The remaining cases where $\alpha$ is
finite are those with \nf$_2$ constructors only.  Suppose we are
trying to compare two sets $X$ and $Y$ denoted by terms of rank at
most $\alpha$. In $\nf_2$ every term denotes either a finite object or
a cofinite object.  If $X$ and $Y$ are both finite we can compare the
least member of $X \setminus Y$ with the least member of $Y \setminus
X$ by induction hypothesis; if $X$ and $Y$ are cofinite then
$\mathop{-} X$ and $\mathop{-} Y$ are finite and we can use remark
~\ref{rem:twiddle} to reduce this case to the preceding one.  The same
trick reduces the final case (one of $X$ and $Y$ finite, the other
cofinite) without loss of generality to comparing a cofinite object
with a finite object.

Now we appeal to the fact that the $a_n$ with $n \in \Nn$ form an initial
segment of $V$ under $\subset_\omega$.  Any finite object can contain only
finitely many of them and any cofinite object must contain all but finitely
many of them.  If the finite object contains none of the $a_n$ then it is
later than the cofinite object in the sense of $\subset_\omega$. Otherwise
compare the bottom $a_n$ in the cofinite object with the bottom $a_n$ in
the finite object.
 
Now for terms of transfinite rank.  Assume true for $\beta < \alpha$.  A
directed union of strict total orders is a total order and $P$ of a strict
total order is a total order so irrespective of whether $\alpha$ is
successor or limit $\subset_\alpha$ (restricted to terms of rank no more
than $\alpha$) is at least transitive.  We already know that it is
irreflexive so all that has to be proved is trichotomy.

Consider a couple of $\nf O$ terms of rank at most $\alpha$:
$\displaystyle{\bigvee_{i \in I}\bigwedge_{j \in J} t_{i,j}}$ and
$\displaystyle{\bigvee_{k \in K}\bigwedge_{l \in L}s_{k,l}}$ where each $s$
and $t$ is $B`r$ or $\bbar r$ for $r$s of lower rank.
 
If $$\bigvee_{i \in I}\bigwedge_{j \in J} t_{i,j} \subset_{\alpha}
\bigvee_{k \in K}\bigwedge_{l \in L} s_{k,l}$$ is to be true there is an
antichain $\subseteq$ the set on the right (minus the set on the left) that
is below everything in the set on the left (minus the set on the right) in
the sense of $\subset_{\beta}$ (with $\beta < \alpha$)\footnote{Readers who
feel that the subscript should be $\omega + \alpha$ should remember that if
$\alpha \geq \omega$ these two ordinals are the same}.  In fact we will
even be able to show that the antichain has only one element, because we
are simultaneously proving by induction that the order is total! Now both
the set on the left and the set on the right have finitely many
$\subset_\beta$ minimal elements. This is because they are a union of
finitely many things each of which is an intersection of things of the form
$B`x$ and $\bbar y$, and any such intersection has a unique
$\subseteq$-minimal member which will also be the unique $\subset_{\beta}$-minimal
member.

 So if there is a thing in the set on the right (minus the set on the left)
that is is below everything in the set on the left (minus the set on the
right) in the sense of $\subset_{\beta}$ then it must be one of those
minimal elements, and it is enough to check that it is less than the
minimal elements of the set on the left (minus the set on the right).  Now
these minimal elements are just finite sets of things of lower rank.  By
induction hypothesis all terms of lower rank are ordered by some
$\subset_{\beta}$ (with $\beta < \alpha$) and so certainly finite sets of
them are too.  So really all we have to do is compare the minimal elements
of the set on the left (minus the set on the right) with the minimal
elements of the set on the right (minus the set on the left).  There is
only a finite set of them and it is totally ordered, so there is a least
one (in the sense of $\subset_{\beta}$).

The alert reader will have noticed that this is not the most general form of
an $\nf O$ word.  There should be addition and deletion of singletons.  But
this makes no difference to the fact that we only need consider a finite
basis, which is the bit that does the work!\endproof


As it happens \nf O has a model in which every element is the denotation of
a closed term, a {\bf term model}.  This model is unique.

\begin{coroll}
The term model for \nf O is totally ordered by the least fixed point for $P$
\end{coroll}

Of course term models can always be totally ordered in canonical ways, but
one does not routinely expect to be able to describe such a total ordering
within the language for which the structure is a model.  For some light
relief, I shall write out this formula in fairly primitive notation.

\nf O is too weak to manipulate ordered pairs so we will have to represent
strict partial orders as the set of their initial segments. This motivates
the following definitions.

Let Prec$(R,x,y)$ (``$x$ precedes $y$ according to $R$") abbreviate

$$(\forall z \in R)(y \in z \to x \in z) \wedge x \not= y.$$

Let Refines$(R,S)$ (``$R$ refines $S$") abbreviate \hfill

$(\forall x y)($Prec$(S,x,y) \to $Prec$(R,x,y)).$

Let Prec$(R^+,x,y)$ abbreviate \hfill

$$(\exists x' \in y \setminus x)(\forall y' \in x \setminus y)(\mbox{\rm Prec}(R,x',y'))).$$  Then finally

$x \canon y$ is $(\forall R)($Refines$(R,R^+) \to$ Prec$(R,x,y))$

Then in the term model it is true that $\canon$ is a strict total order.

\bigskip

It would be nice to know whether or not this result extends to theories stronger than \nf O.

What can one say about other fixed points for $P$?  We can invoke a
fixed-point theorem for CPO's to argue that $P$ must have lots of
fixed points---a CPO of them in fact.  One can then invoke Zorn's
lemma to conclude that there are maximal fixed points.  By reasoning
in the manner of the standard proof of the order extension principle
from Zorn's lemma one can deduce that any maximal fixed point must be
a total order.  We now reach a point at which the \naive\ set theory
in which we have been operating will no longer work.  Let us assume DC
for the moment, and let $\tuple{X, \leq}$ be a total order that is not
wellfounded. Take $X' \subseteq X$ with no $\leq$-least element.  Use
DC to pick two descending sequences $\tuple{a_n: n \in \Nn}$ and
$\tuple{b_n: n\in \Nn}$ with $b_{n+1} < a_n$ and $a_{n+1} < b_n$.  The
domains of these two sequences are a pair of subsets of $X$ which are
incomparable under $P(\leq)$.  In other words, $P$ of a strict total order
$R$ is a strict total order only if $R$ is a wellorder, and even then $P(R)$
will not be wellfounded.  So if DC holds, no fixed point for $P$ can
be a total order.  But any maximal fixed point must be a total order,
and Zorn's lemma tells us that there are some. Therefore the axiom of
choice is false.

   The message seems to be that this is the point at which we should start
treating these ideas axiomatically. That should be the scope of another
article.


\section{Lifting quasi-orders: fixed points and more games}


The obvious order on partitions of a set is simply the lift
of the identity relation on the set.

If $X$ is a set that meets $\pow X)$, its power set, and $\sim$ is an
equivalence relation on $X$, and if $\sim^+$ agrees with $\sim$ on $X \cap
\pow X)$ we say that $\sim$ is a {\bf bisimulation}.  (Hinnion called them
{\bf contractions} but this usage doesn't seem to have caught on.)
Typically we will be interested in this only when $X \subseteq \pow X)$,
which is to say when $X$ is {\bf transitive}.

If $\leq$ is a transitive relation on a domain $D$ define $\leq^+$ on 
$\pow D)$ by 
$X\ \leq^+\ Y$ by $(\exists y \in Y)(\forall x \in X)(y \leq x)$.\label{plus}

This operation preserves transitivity but apparently not much else. 


It is simple to check that the collection
of quasi-orders on the universe is a complete lattice and that + is a
continuous increasing function from this complete lattice into itself.
Thus by the Tarski-Knaster theorem there will be a complete lattice of
fixed points.  The following is the Aczel-Hintikka game for these
fixed points.


 HOLE


Now we are in a position to 
show that the least bisimulation is indeed the intersection of a quasi-order and its converse.

\begin{thm}
$(\forall x)(\forall y)(x \sim_{min} y \bic (x<_o y \wedge y <_o x))$
\end{thm}


\Proof   
L $\to$ R

Clearly if $x \sim_{min} y$ then \equal\ has a strategy to win $G_{x=y}$ in
finitely many moves.  Arthur can use \equal's Winning strategy to play in
both $G_{x \leq y}$ and $G_{y \leq x}$.  Since \equal's strategy wins in
$G_{x=y}$ in finitely many moves, Arthur must win $G_{x \leq y}$ and $G_{y
\leq x}$ in finitely many moves.

R $\to$ L

Now suppose $x <_o y$ and $y <_o x$.  That is to say that Arthur has
winning strategies $\sigma$ and $\tau$ in the open games $G_{x \leq y}$ and
$G_{y \leq x}$.  Player \equal\ can use these in $G_{x=y}$ as follows.
Whatever \notequal\ plays in $x$ (or $y$), \equal\ can reply in $y$ (or
$x$) using $\tau$ (or $\sigma$). Since she is never at a loss for a reply,
she Wins the closed game $G_{x=y}$. \endproof


We note without proof that an analogous result holds for the greatest fixed
points.  That is to say, if we define $x \sim_{max} y$ to hold iff \equal\
Wins the {\sl open} game $G_{x = y}$ and $x <_c y$ as above then
$(\forall x)(\forall y)(x \sim_{max} y \bic (x<_c y \wedge y <_c
x))$. \marginpar{Might be an idea to check this}












If $R$ is a binary relation,
let $R^+$ be $\{\tuple{X,Y}: (\forall x \in X)(\exists y \in Y)(R(x,y))\}$.

I think this `+' notation is due to Hinnion.  It takes quasiorders to
quasiorders and the set of all quasiorders is a complete lattice under
$\subseteq$ and has lots of fixed points.  The least fixed point
corresponds to the game where Arthur wins all infinite plays and the
greatest fixed point corresponds to the game where Bertha wins all
infinite plays.

Say $x <_o y$ if Bertha has a
Winning strategy for the open game and $x <_c y$ if Bertha has a
Winning strategy for the closed game.

I shall use the molecular letter
`$\rhobeta$' (``{\underline{r}}anked {\underline{b}}elow'') to range
over fixed points and prefixed points and postfixed points.

The first point to notice is that if $R$ is reflexive then $R^+$ is a
superset of $\subseteq$.  The operation is increasing in the sense
that $R\subseteq S \to R^+ \subseteq S^+$.  Suppose $R \subseteq S$ and $x
R^+y$.  Then for every $z \in x$ there is $w \in y$ $R(z,w)$ whence
$S(z,w)$ whence $R^+ \subseteq S^+$.

Now for limits. Suppose $R_\infty = \bigcup_{i \in I}R_i$.  Clearly,
for all $i \in I$, ${R_i}^+ \subseteq {R_\infty}^+$ so 
${\bigcup_{i \in I}R_i}^+ \subseteq {R_\infty}^+$. For the converse

$x {R_\infty}^+ y$ iff $(\forall z \in x)(\exists w \in y)(z R_\infty
w)$ iff $(\forall z \in x)(\exists w \in y)(\exists i)(z R_i w)$ so it
is not cts at limits.  (Presumably this is for the same reason that
${\cal P}$ is not continuous.)

\begin{rem}
$\in\ \ \subseteq$ the GFP
\end{rem}

\Proof If $x \in y$ then $(\forall z \in x)(\exists w \in y)(z \in w)$
\ldots and the $w$ is of course $x$ itself.  That is to say $\in 
\subseteq \in^+$: $\in$ is a postfixed point

Obvious questions: does $\rhobeta$ extend $\in$?  Is it connected?  Is
it wellfounded? Is $\rhobeta$ restricted to wellfounded sets
wellfounded? Is it a WQO or a BQO?

There are other way of deriving a rank relation.  We could consider
sets containing $\emptyset$ and closed under ${\cal P}$ and (i) unions
or (ii) directed unions or (iii) unions of chains.  Then if $X$ is
such a set we say $x \rhobeta y$ if $(\forall Y \in X)(y \in Y \to x
\in Y)$.  For each of these three we can prove by induction that the 
least fixed point consists (for any $X \supseteq \pow X)$), entirely of
sets in $X$.  We should also prove that if $X$ is a prefixed point
under the heading (i) (ii) or (iii) then every wellfounded set is in a
member of $X$.

We need to check that the LFP and the GFP are nontrivial.  The
identity is a postfixed point and the universal relation is a prefixed
point.  (Incidentally this shows that the GFP is reflexive) But LFP
$\subseteq$ GFP?  It is if there is a fixed point.



\begin{rem}

The GFP is transitive
\end{rem}

\Proof  First we show that $\rhobeta^+ \subseteq \rhobeta \wedge \rhobeta'^+ \subseteq \rhobeta' \to 
(\rhobeta \circ \rhobeta')^+ \subseteq \rhobeta \circ \rhobeta'$.  Suppose
$\tuple{X,Z} \in (\rhobeta \circ \rhobeta')^+$.  That is to say, $(\forall x
\in X)(\exists z \in Z)(\tuple{x,z} \in \rhobeta \circ \rhobeta')$.  This is
$(\forall x \in X)(\exists z
\in Z)(\exists y)(\tuple{x,y} \in \rhobeta \wedge \tuple{y,z} \in \rhobeta)$.  
or $(\forall x \in X)(\exists y)(\tuple{x,y} \in \rhobeta \wedge (\exists z
\in Z)(\tuple{y,z} \in \rhobeta))$.   Then for this $y$ we have 
$\tuple{X, \{y\}} \in \rhobeta^+$ and thence  $\tuple{X, \{y\}} \in \rhobeta$ and 
$\tuple{\{y\},Z} \in \rhobeta'^+$ and thence  $\tuple{\{y\},Z} \in \rhobeta'$ which 
is to say $\tuple{X,Z} \in \rhobeta \circ \rhobeta'$.

Similarly the set of post-fixed points is closed under composition,
which means that the GFP is transitive.

We can prove by $\in$-induction that any fixed point is reflexive on
wellfounded sets.

\begin{rem}\label{rank1}
Any two fixed points agree on wellfounded sets.  
\end{rem}

\Proof Let $\rhobeta$ and $\rhobeta'$ be fixed points.  We will show that 
for all  wellfounded $x$ and for all $y$, $\tuple{x,y} \in \rhobeta$ iff 
$\tuple{x,y} \in \rhobeta'$.

We need to show that $\pow\{x: (\forall
y)(\tuple{x,y} \in \rhobeta \bic \tuple{x,y} \in  \rhobeta')\}) \subseteq \{x: (\forall
y)(\tuple{x,y} \in \rhobeta \bic \tuple{x,y} \in  \rhobeta')\}$. 

Let $X$ be a subset of $\{x: (\forall y)(\tuple{x,y} \in \rhobeta \bic
\tuple{x,y} \in  \rhobeta')\}$. Then for all $Y$ 

$\tuple{X,Y} \in \rhobeta$ iff

$(\forall x \in X)(\exists y \in Y)(\tuple{x, y} \in \rhobeta)$ which by induction hypothesis is the same as

$(\forall x \in X)(\exists y \in Y)(\tuple{x, y} \in  \rhobeta')$ which is 

$\tuple{X,Y} \in  \rhobeta'$

 We will also need to show that for all wellfounded $y$ and for all $x$, 
$\tuple{x,y} \in \rhobeta$ iff $\tuple{x,y} \in  \rhobeta'$.

We need to show that $\pow\{y: (\forall
x)(\tuple{x,y} \in \rhobeta \bic \tuple{x,y} \in  \rhobeta'\}) \subseteq \{y: (\forall
x)(\tuple{x,y} \in \rhobeta \bic \tuple{x,y} \in  \rhobeta'\}$. 

Let $Y$ be a subset of $\{y: (\forall x)(\tuple{x,y} \in \rhobeta \bic
\tuple{x,y} \in  \rhobeta'\}$. Then for all $X$ 

$\tuple{X,Y} \in \rhobeta$ iff

$(\forall x \in X)(\exists y \in Y)(\tuple{x, y} \in \rhobeta)$ which by induction hypothesis is the same as

$(\forall x \in X)(\exists y \in Y)(\tuple{x, y} \in  \rhobeta')$ which is 

$\tuple{X,Y} \in  \rhobeta'$


\begin{rem}
If $\rhobeta^+ \subseteq \rhobeta$ then

$(\forall y \in WF)(\forall x)(\tuple{x,y} \in \rhobeta \vee \tuple{y,x} \in \rhobeta)$
\end{rem}
\Proof

We prove by $\in$-induction on `$y$' that $(\forall x)(\tuple{x,y}
\in \rhobeta \vee \tuple{y,x} \in \rhobeta)$. Suppose this is true for all 
members of $Y$, and let $X$ be an arbitrary set. Then either
everything in $Y$ is $\rhobeta$-related to something in $X$ (in which case
$\tuple{Y,X} \in \rhobeta^+$ and therefore also in $\rhobeta$) or there
is something in $Y$ not $\rhobeta$-related to anything in $X$, in which case,
by induction hypothesis, everything in $X$ is $\rhobeta$-related to it, and
$\tuple{X,Y} \in \rhobeta^+$ (and therefore in $\rhobeta$) follows.

\endproof

\begin{rem}

If $\rhobeta \subseteq \rhobeta^+$ and $\pow X) \subseteq X$ then  
$(\forall y \in WF)(\forall x)(\tuple{x,y} \in \rhobeta \to x \in
X)$. 

\end{rem}

If $\rhobeta \subseteq \rhobeta^+$ and $\pow X) \subseteq X$ we prove by
$\in$-induction on `$y$' that $(\forall x)(\tuple{x,y} \in \rhobeta \to x \in
X)$.  Suppose $(\forall y \in Y)(\forall x)(\tuple{x,y} \in \rhobeta \to x
\in X)$ and $\tuple{X',Y} \in \rhobeta$. $\tuple{X',Y} \in \rhobeta$ gives
$\tuple{X',Y} \in \rhobeta^+$ which is to say $(\forall x \in X')(\exists y
\in Y)(\tuple{x,y} \in \rhobeta)$.  By induction hypothesis this implies that
 $(\forall x \in X')(x \in X)$ which is $X' \in \pow X)$ but $\pow X)
\subseteq X$ whence $X' \in X$ as desired.\endproof

\begin{coroll}
If $\rhobeta \subseteq \rhobeta^+$, $y \in \WF$ and $x\ \rhobeta\ y$
then $x \in \WF$
\end{coroll}



One obvious conjecture is that if $\rhobeta$ is a fixed point then $x \in y
\to \tuple{x,y} \in \rhobeta$.

  There is an obvious proof by $\in$-induction on `$x$' that $(\forall
y)(x \in y \to \tuple{x,y} \in \rhobeta)$ but the assertion is unstratified
and so the inductive proof is obstructed, at least in \nf.

   Suppose $\rhobeta^+ \subseteq \rhobeta$ and $x$ is an illfounded
set such that $y\ \rhobeta\ x \to y \in \WF$.  Since $x$ is illfounded
it has a member $x'$ that is illfounded.  $\neg(x'\rhobeta\ x)$
because everything related to $x$ is wellfounded. Now suppose $y
\rhobeta x'$. Then $\{y\} \rhobeta^+ x$ and $\{y\} \rhobeta x$ (since
$\rhobeta^+ \subseteq \rhobeta$) and $\{y\}$ is wellfounded.  So $y$
is wellfounded as well, and $x'$ is similarly minimal.










Now suppose $x$ is such that $G \circ F(x) \subseteq x$.  Then
$F(x) \in x$.
$G \circ F(x\setminus \{Fx\}) \subseteq G \circ F(x) \subseteq x$
As before, we want `$x \setminus \{Fx\}$' on the RHS.  So we want

$z \in G \circ F(x\setminus \{Fx\}) \to z \not= Fx$ which is to say
$Fx \not\in G \circ F(x\setminus \{Fx\})$. But this follows by
monotonicity and injectivity of $F$ and the fact that $F(x\setminus
\{Fx\})$ is the largest element of $G \circ F(x\setminus \{Fx\})$.

So $G \circ F(x\setminus \{Fx\}) \subseteq (x\setminus \{Fx\})$ and
$x$ was not minimal.  \endproof




 
\subsection{Fremlin: A transitive ordering on the class of relations} 
 

 
I extract an idea from a lecture given by T.Forster, 27.9.00. 
 
 
{\bf Definition} Let $R$ and $S$ be relations and $X_0$ and $Y_0$ sets. 
Consider the following game $G(X_0,R,Y_0,S)$. 
 
Player $A$ chooses $y_0\in Y_0$. 
 
Player $B$ chooses $x_0\in X_0$. 
 
Player $A$ chooses $x_1$ such that $(x_1,x_0)\in R$. 
 
Player $B$ chooses $y_1$ such that $(y_1,y_0)\in S$. 
 
Player $A$ chooses $y_2$ such that $(y_2,y_1)\in S$ 
 
Player $B$ chooses $x_2$ such that $(x_2,x_1)\in R$ 
 
Player $A$ chooses $x_3$ such that $(x_3,x_2)\in R$
 
\noindent and so on.   Generally, at the $n$th move, for $n\ge 3$, 
 
if $n=4k$, Player $B$ chooses $y_{2k-1}$ such that 
$(y_{2k-1},y_{2k-2})\in S$, 
 
if $n=4k+1$, Player $A$ chooses $y_{2k}$ such that 
$(y_{2k},y_{2k-1})\in S$, 
 
if $n=4k+2$, Player $B$ chooses $x_{2k}$ such that 
$(x_{2k},x_{2k-1})\in R$, 
 
if $n=4k+3$, Player $A$ chooses $x_{2k+1}$ such that 
$(x_{2k+1},x_{2k})\in R$.
 
\noindent If a player cannot move, the other wins;  if the game 
continues for ever, $A$ wins. 
 
Now say that $(X_0,R)\preccurlyeq(Y_0,S)$ if $A$ has a winning strategyi 
in the game $G(X_0,R,Y_0,S)$.
 
Note that because the payoff set for $A$ is closed in 
$V^{\Nn}$, where $V$ is such that $X_0\cup Y_0\subseteq V$ and 
$R\cup S\subseteq V\times V$, and is given the discrete topology, the 
game is determined. 
 
\medskip 
 
{\bf Proposition} $\preccurlyeq$ is transitive. 
 
\medskip 
 
\Proof Suppose that $(X_0,R)\preccurlyeq(Y_0,S)$ and that 
$(Y_0,S)\preccurlyeq(Z_0,T)$.   Let $\sigma$ be a winning strategy for 
$A$ in $G(X_0,R,Y_0,S)$ and $\tau$ a winning strategy for $A$ in 
$G(Y_0,S,T,Z_0)$. 
 
Construct a strategy $\upsilon$ for $A$ in $G(X_0,R,T,Z_0)$ as follows. 
 
$A$ starts by playing $z_0\in Z_0$, the first move prescribed by 
the strategy $\tau$, and also by playing $y_0\in Y_0$, the first move 
prescribed by $\sigma$. 
 
$B$ replies with $x_0\in X_0$. 
 
$A$ plays $x_1$ prescribed by the rule $\sigma$ in the game starting 
$(y_0,x_0)$, and $y_1$ prescribed by the rule $\tau$ in the game 
starting $(z_0,y_0)$. 
 
$B$ plays $z_1$. 
 
$A$ plays $z_2$ prescribed by the rule $\tau$ in the game starting 
$(z_0,y_0,y_1,z_1)$, and $y_2$ prescribed by the rule $\sigma$ in the 
game starting $(y_0,x_0,x_1,y_1)$. 
 
$B$ plays $x_2$. 
 
$A$ plays $x_3$ prescribed by the rule $\sigma$ in the game starting 
$(y_0,x_0,x_1,y_1,y_2,x_2)$, and $y_3$ prescribed by the rule $\tau$ in 
the game starting $(z_0,y_0,y_1,z_1,z_2,y_2)$. 

 
\noindent Generally, 
 
$B$ plays $x_{2k}$, 
 
$A$ plays $x_{2k+1}$ following the rule $\sigma$ in the game starting 
$(y_0,x_0,x_1,\ldots,y_{2k},x_{2k})$, and $y_{2k+1}$ following the rule 
$\tau$ in the game starting $(z_0,y_0,y_1,\ldots,z_{2k},y_{2k})$, 
 
$B$ plays $z_{2k+1}$, 
 
$A$ plays $z_{2k+2}$ prescribed by the rule $\tau$ in the game starting 
$(z_0,y_0,\ldots,y_{2k+1},z_{2k+1})$, and $y_{2k+2}$ prescribed by the 
rule $\sigma$ in the game starting 
$(y_0,x_0,\ldots,x_{2k+1},y_{2k+1})$. 
 
\noindent Since (if $B$ has played legally) $A$ always has a move, $A$ 
wins.   So $(X_0,R)\preccurlyeq(Z_0,T)$. 
 
\medskip 
 
{\bf Problem} Find invariants of relations from 
which it is easy to decide whether $(X_0,R)\preccurlyeq(Y_0,S)$. 
 
If we define the game $G(X_0,R)$ as follows: 
 
\begin{quote}$A$ plays $x_0\in X_0$, 
 
$B$ plays $x_{2k+1}$ such that $(x_{2k+1},x_{2k})\in R$, 
 
$A$ plays $x_{2k+2}$ such that $(x_{2k+2}, x_{2k+1})\in R$,\end{quote} 
 
\noindent with $A$ winning if either $B$ cannot move or the game goes on 
for ever, then if $B$ wins $G(X_0,R)$ and $A$ wins $G(Y_0,S)$, 
$(X_0,R)\preccurlyeq(Y_0,S)$.   On the other hand, even if $A$ wins 
$G(X_0,R)$, it is still possible to have $(X_0,R)\preccurlyeq(Y_0,S)$ if 
$A$ can win $G(Y_0,S)$ sooner. 
 
\verb#From fremdh@essex.ac.uk Thu Sep 28 15:04:54 2000#

I extracted an idea from your talk and  wrote it up in my own preferred
language.

David Fremlin

\verb#From t.forster@dpmms.cam.ac.uk Fri Sep 29 15:38:56 2000#

Dear David, 

       Thanks for your note.  I think what is going on is that
simultaneous displays of open (or closed) games give rise to
quasiorders.  With your usual merciless acuteness you spotted
that the fact that this is a game played on $\in$ is completely
irrelevant (but this was supposed to be a meeting on sets and
games, after all) which i had been trying to conceal for that
reason.  I hadn't reflected on the fact you draw my attention 
to, namely that the binary relation in the two games need not be
related in any way at all.  What i find so intruiging about game
theory is that one never ever seems - or at least i never feel
that i manage - to reach the appropriate level of generality.
With those games of Martin, for example, it seems to me that he
is considering games where the two players pick elements from 
a set - as it might be $X$, and thereby build a play which is an
element of $[X]^\omega$.   The clever bit is using extra structure
on $X$ to put extra structure on the play, so it isn't just an
$\omega$-string.  Where will it all end?

       I found myself wondering to what extent this quasiorder
is the same, au fond, as the quasiorder of Conway Games.  I
don't think i can pursue that for the present, as i have to turn
this into something for the Boffa festschrift in a very small
number of weeks....

       Let's talk about this some more before too long.   I seem
to recall you have dining rights in Churchill - as do i, and very 
handy to the new building it is too.  Do you come here often?

          v best wishes

            Thomas


\section{The Equality Game}

This is familiar: just maximal and minimal bisimulations.

\subsection{Prologue on Aczel-Hintikka Games}

Aczel-Hintikka games are a very pretty way of presenting fixed points.  In
general they add nothing of substance to the material they enable one to
present, and this is presumably why Aczel never published the work he did
on them in the early `70's. However they are worth using in this context
because there are other games involved and this makes a game-theoretic
treatment of fixed points more sensible.

\subsubsection{Hintikka games}

    Hintikka games are games played with formul{\ae} amd models.  The
formul{\ae} are all built up from atomics and negatomics by means of
$\wedge$, $\vee$, $\forall$ and $\exists$ and the two restricted quantifiers.

I am assuming that the reader knows the usual rules for the Hintikka game 
$G_\phi$. Here we have two extra rules for the restricted quantifiers, which 
are as follows.  When the players are confronted with $(\forall x \in a)\Phi$
player {\bf False} picks an element $b$ of $a$ (if he can, and loses at
once if he can't) and they play $G_{\Phi[b/x]}$; when the players are
confronted with $(\exists x \in a)\Phi$ player {\bf True} picks an element
$b$ of $a$ (if she can, and loses at once if she can't) and they play
$G_{\Phi[b/x]}$.

\subsubsection{What Aczel did to Hintikka games}

   If $\phi$ belongs to any normal sensible language (i.e., to a language
that is a recursive datatype) the Hintikka game $G_\phi$ is of course a
game of finite length.  Interesting things happen, however, if $\phi$ is a
nasty formula of the kind that Aczel calls a {\sl syntactic fixed point}.

   We start as we mean to go on, with an example that will concern us later.
Suppose \verb/#/ is a formula with two free variables in it, such that when
we put `$X$' and `$Y$' in for the two free variables in \verb/#/ we obtain

$$(\forall x\in X)(\exists y)(y \in Y \wedge ???)) 
 \wedge (\forall y \in Y)(\exists x)(x \in X \wedge ???))$$
where the question marks identify a subformula which is the result of 
putting `$x$' and `$y$' in for the two free variables in \verb/#/ and
adding a prime to the two outermost variables bound by restricted
quantifiers.  It is clear that any formula satisfying this condition
must be infinite and---worse!---must have an illfounded subformula
relation.  Nevertheless formul{\ae} that are {\sl syntactic fixed
points} can have a perfectly intelligible semantics provided by means
of the corresponding Hintikka games.


    Let us consider the Hintikka game for this formula.  In a play of
this game, {\bf False} picks a member of $X$ or a member of $Y$, and
{\bf True} has to reply with a member of the other.  They continue
doing this until one of them is unable to play, and thereby loses.
This game was discovered independently by Malitz many years later, and
i do not at present know if he knew if this game could be seen as
arising in this way from Hintikka games.  For obvious reasons i prefer
to call the players ``\notequal" and ``\equal" instead of ``{\bf
False}" and ``{\bf True}".  Let us notate this, the {\bf Malitz game},
``$G_{X=Y}$".

   This is an illustration of a more general phenomenon.  If a
relation of interest comes to us as a fixed point for an operation, so
that $\psi(x,y) \bic \Gamma(x,y)$ where $\psi$ occurs as a subformula
of $\Gamma$, then $\psi(x,y)$ gives rise to a {\sl syntactic fixed
point,} a formula whose subformula relation is illfounded.  The
Hintikka game for this formula then gives us a game with the feature
that if \one\ (say) has a winning strategy for it then $\psi(x,y)$.

\begin{small}
   In Forster [1982] I published another set-theoretical game designed
to capture contractions and not surprisingly it turned out to be
equivalent.  This game is played as follows.  \equal\ announces a
binary relation which is a subset of $x \times y$ whose domain is $x$
and whose range is $y$.  \notequal\ then picks an ordered pair
$\tuple{x',y'}$ in this set and they play $G_{x' = y'}$.  The first
player to be unable to move loses.

This does not tell us who wins an infinite play.  Any bisimulation
corresponds to a valuation (a ``referee") awarding each infinite
(``disputable") play of $G_{x=y}$ to \equal\ or to \notequal. (There's no
need for a referee to decide who wins completed plays of finite length!)
The valuation that awards no disputable plays to \equal\ corresponds to the
least fixed point, and the valuation that awards all disputable plays to
\notequal\ corresponds to the greatest fixed point.  There will in fact be a
greatest fixed point because the collection of equivalence relations on a
set is always a complete lattice and $+$ is a strict monotone function.


\begin{rem}
The open (resp. closed) Forster game and the closed (resp. open) Malitz Game are equivalent.
\end{rem}

\Proof

The equivalence is the wrong way round because\  \equal\ \ moves first in the
Forster game but moves second in the Malitz game. This is a good reason for
not retaining Malitz's notation.

We sketch a way of turning strategies for \equal\ in one game into
strategies for \equal\ in the other.

Suppose \equal\ Wins the Forster game $G_{x=y}$.  Then she Wins the Malitz
game as follows.  Because she has a winning strategy in the Forster game
$G_{x=y}$, she has a binary relation $R$ which is a subset of $x \times y$
whose domain is $x$ and whose range is $y$. When \notequal\ plays $x' \in
x$ or $y' \in y$, she replies with an $R$-relative of $x'$ (or $y'$ {\sl
mutatis mutandis}).   (What's a bit of AC between friends?)

Conversely suppose \equal\ Wins the Malitz game $G_{x=y}$.  Then she Wins
the Forster Game as follows.  She has a strategy, and the strategy,
initially at least, is a map from $x$ to $y$ and a map from $y$ to $x$.
But this gives her a binary relation $R$ which is a subset of $x \times
y$ whose domain is $x$ and whose range is $y$, which is what she needs to
make her first move in the Forster Game.

\endproof

Since the Forster games and the Malitz games are equivalent we can
concentrate our treatment on only one of them.  Henceforth the game
$G_{x=y}$ will be the Malitz game, so that when we speak of the open game
$G_{x=y}$ we mean the game in which the player who goes first (namely
\notequal) wins, if at all, after finitely many moves.

\begin{dfn} 
Let us say $x$ and $y$ are Forster/Malitz bisimilar iff \equal\
Wins the closed game $G_{x=y}$.  Let us write this $x \sim_{min} y$.
\end{dfn}

Evidently $\sim_{min}$ is an equivalence relation. We note that

\begin{rem} $\sim_{min}$ is the least fixed point for +.  \end{rem}

\Proof

Notice that the least fixed point is not the equality, as one might think.
Strictly, it's not even an equivalence relation at all, but only a PER.  If
$x$ is a set that is not wellfounded, so that $\tuple{x_n: n < \omega}$ is
a descending $\in$-chain ($x_0 =x$ and $(\forall n)(x_{n+1} \in x_n)$),
then player \notequal\ can stave off defeat in $G_{x=x}$ indefinitely by
picking $x_n$ for his $n$th move.  Player \equal\ certainly cannot do any
better than to copy him.  That means that if $x$ is not wellfounded then it
is not bisimilar even to itself (according to the least fixed point).  In
fact the least fixed point is the identity relation restricted to
wellfounded sets.

Generally Malitz was interested only in the maximal fixed point for +,
corresponding to the open game in which \notequal\ has to win in finitely
many moves if at all.  This is because in all the usual models of the set
theory he was studying this maximal bisimulation is equality.  He points
out that \equal\ will win the open game $G_{V = V\setminus \{V\}}$.  For consider:
what can \notequal\ do?  He cannot pick something in $V\setminus \{V\}$ that isn't in
$V$ so his only hope is to pick something in $V$ that isn't in $V\setminus \{V\}$,
namely $V$.  But even if he does pick $V$, \equal\ need only pick $V\setminus \{V\}$
and they are back where they started. Anything else allows \equal\ to copy
his moves blindfold and, if not actually win in finitely many moves, at
least never lose in finitely many moves, which is enough to ensure that she
can Win the open game.  This means that the ordered pair $\tuple{V,-\{V\}}$
belongs to the {\sl greatest} fixed point for +.

A moment's reflection will reveal that this depends only on very
general properties of $V$ and $V\setminus \{V\}$, and that what this
reasoning proves is the following

\begin{rem} If $x \in x$ and $(x \setminus\{x\}) \in x$
then $x\sim (x \setminus \{x\})$ where $\sim$ is the greatest fixed point for +.
\end{rem}


A rather bizarre corollary of this now appears in Malitz's set theory.
Even tho' $V$ is a set, $V\setminus  \{V\}$ isn't!  If it existed it 
would have to be distinct from $V$.  However the maximal bisimulation 
is the identity, and $V$ is maximally-bisimilar to $V\setminus \{V\}$.

Malitz noticed that in consequence of this Quine's \nf\ cannot have a
model in which player \equal\ has a Winning strategy in $G_{x=y}$ iff
$x = y$.  This is an infelicity.  The revised version of Malitz'
identity game, with an eye on an axiom of strong extensionality that
{\sl is} compatible with Quine's \nf, is the following.

On being presented with $x$ and $y$, player \notequal\ has two further
possibilitiss in addition to the two possibilities of picking a member
of $x$ or a member of $y$. He now can pick something that is not in
$x$ or something that is not in $y$.  If he picks something in $V\setminus y$,
\equal\ must reply with something in $V\setminus x$.  In general \equal\ cannot 
distinguish (merely from observing \notequal's move) whether he has
picked something in $x$, or something in $y$, so she doesn't even know
what she is supposed to do next, let alone how to succeed in it.  So the
rules must specify that \notequal\ has to say ``I have picked a member of
$x$'' (or whatever).

Actually the same holds in the original game.  This time there is the
additional problem that \equal\ can't distinguish between \notequal\
picking something in $y$ and something in $V\setminus x$.

%April 1999

It becomes clearer what is going on if we go back to Aczel
formul{\ae}: again.  The equivalence relation we are interested in is
this one: $A \sim B$ iff $(\forall x)(\exists y)(x \sim y \wedge (x \in
A \bic y \in B))$.
or, more symmetrically in `$A$' and `$B$':
 $$(\forall x)(\exists y)(x \sim y \wedge (x \in
A \bic y \in B) \wedge (x \in B \bic y \in A)).$$

  It looks a bit like one of the Barwise approximants.

\end{small}
 
Sse $X = \{a,b,c,d\};\ Y = \{c,d,f,g\};\ Z = \{b,d,e,f\}$.

we desire $X \leq Y \leq Z$ but $X \not\leq Z$.

so we want

$a < g   \vee    a < f$

$b < g   \vee    b < f$

$g < e   \vee    g < b$

$c < e   \vee    c < b$

and $(c \not< e\wedge c\not< f \vee a \not< e \wedge a\not< f)$

So this is the DNF.  Each row is a conjunction.


$a < g   b < g   g < e   c < e   c \not< e c\not< f$

$a < g   b < g   g < e   c < b   c \not< e c\not< f$

$a < g   b < g   g < b   c < e   c \not< e c\not< f $             

$a < g   b < g   g < b   c < b   c \not< e c\not< f$

$a < g   b < f   g < e   c < e   c \not< e c\not< f$

$a < g   b < f   g < e   c < b   c \not< e c\not< f$

$a < g   b < f   g < b   c < e   c \not< e c\not< f$

$a < g   b < f   g < b   c < b   c \not< e c\not< f$

$a < f   b < g   g < e   c < e   c \not< e c\not< f$

$a < f   b < g   g < e   c < b   c \not< e c\not< f$

$a < f   b < g   g < b   c < e   c \not< e c\not< f$

$a < f   b < g   g < b   c < b   c \not< e c\not< f$

$a < f   b < f   g < e   c < e   c \not< e c\not< f$

$a < f   b < f   g < e   c < b   c \not< e c\not< f$

$a < f   b < f   g < b   c < e   c \not< e c\not< f$

$a < f   b < f   g < b   c < b   c \not< e c\not< f$

$a < g   b < g   g < e   c < e   a \not< e a\not< f$

$a < g   b < g   g < e   c < b   a \not< e a\not< f$

$a < g   b < g   g < b   c < e   a \not< e a\not< f $             

$a < g   b < g   g < b   c < b   a \not< e a\not< f$

$a < g   b < f   g < e   c < e   a \not< e a\not< f$

$a < g   b < f   g < e   c < b   a \not< e a\not< f$

$a < g   b < f   g < b   c < e   a \not< e a\not< f$

$a < g   b < f   g < b   c < b   a \not< e a\not< f$

$a < f   b < g   g < e   c < e   a \not< e a\not< f$

$a < f   b < g   g < e   c < b   a \not< e a\not< f$

$a < f   b < g   g < b   c < e   a \not< e a\not< f$

$a < f   b < g   g < b   c < b   a \not< e a\not< f$

$a < f   b < f   g < e   c < e   a \not< e a\not< f$

$a < f   b < f   g < e   c < b   a \not< e a\not< f$

$a < f   b < f   g < b   c < e   a \not< e a\not< f$

$a < f   b < f   g < b   c < b   a \not< e a\not< f$



Now to process them


$a < g   b < g   g < e   c < e   c \not< e c\not< f$      imposs ce

$a < g   b < g   g < e   c < b   c \not< e c\not< f$      imposs cbge

$a < g   b < g   g < b   c < e   c \not< e c\not< f$      imposs ce

$a < g   b < g   g < b   c < b   c \not< e c\not< f$      imposs bggb

$a < g   b < f   g < e   c < e   c \not< e c\not< f$      imposs ce

$a < g   b < f   g < e   c < b   c \not< e c\not< f$      imposs cbf

$a < g   b < f   g < b   c < e   c \not< e c\not< f$      imposs ce

$a < g   b < f   g < b   c < b   c \not< e c\not< f$      imposs cbf

$a < f   b < g   g < e   c < e   c \not< e c\not< f$      imposs ce

$a < f   b < g   g < e   c < b   c \not< e c\not< f$      imposs cbge

$a < f   b < g   g < b   c < e   c \not< e c\not< f$      imposs ce

$a < f   b < g   g < b   c < b   c \not< e c\not< f$      imposs bggb

$a < f   b < f   g < e   c < e   c \not< e c\not< f$      imposs ce

$a < f   b < f   g < e   c < b   c \not< e c\not< f$      imposs cbf

$a < f   b < f   g < b   c < e   c \not< e c\not< f$      imposs ce

$a < f   b < f   g < b   c < b   c \not< e c\not< f$      imposs cbf

$a < g   b < g   g < e   c < e   a \not< e a\not< f$      imposs age

$a < g   b < g   g < e   c < b   a \not< e a\not< f$      imposs age

$a < g   b < g   g < b   c < e   a \not< e a\not< f$      imposs bggb

$a < g   b < g   g < b   c < b   a \not< e a\not< f$      imposs bggb

$a < g   b < f   g < e   c < e   a \not< e a\not< f$      imposs age

$a < g   b < f   g < e   c < b   a \not< e a\not< f$      imposs age

$a < g   b < f   g < b   c < e   a \not< e a\not< f$      imposs agbf

$a < g   b < f   g < b   c < b   a \not< e a\not< f$      imposs agbf

$a < f   b < g   g < e   c < e   a \not< e a\not< f$      imposs af

$a < f   b < g   g < e   c < b   a \not< e a\not< f$      imposs af

$a < f   b < g   g < b   c < e   a \not< e a\not< f $     imposs af

$a < f   b < g   g < b   c < b   a \not< e a\not< f$      imposs af

$a < f   b < f   g < e   c < e   a \not< e a\not< f$      imposs af

$a < f   b < f   g < e   c < b   a \not< e a\not< f$      imposs af

$a < f   b < f   g < b   c < e   a \not< e a\not< f$      imposs af

$a < f   b < f   g < b   c < b   a \not< e a\not< f$      imposs af

the bggb lines are impossible only becos of antisymmetry. If we drop
antisymmetry, so that < is merely a quasiorder then these become
possible counterexamples. So perhaps transitivity of the lift holds if
the imput is antisymmetrical.  But does it preserve antisymmetry?  No,
consider two disjoint mutually cofinal sequences.

We haven't shown that it takes partial orders to quasiorders but even
if we did it wouldn't be useful to us beco's this shows that we can't
expect it to preserve antisymmetry.


\chapter{injections}


\section{Boolean injections}

Let $A \subseteq B$ be sets.  There is a surjection $\pow B) \onto
\pow A)$ defined by $x \mapsto x \cap A$.  And any surjection lifts in
the obvious way so we have an injection ${\cal P}^2(A) \inj {\cal
P}^2(B)$ by $i:X \mapsto \{y\subseteq B: y \cap A \in X\}$.

Things to check.\begin{enumerate}

\item it sends generators to generators.  $i(B(a)) = \{y\subseteq B: y
\cap A \in B(a)\}$ = $\{y\subseteq B: a \in y \cap A\}$ =
$\{y\subseteq B: a \in y \cap A\}$.  But, since $a \in A$, $a \in Y$
iff $a \in y \cap A$, so this is $\{y\subseteq B: a \in y\}$ which is
$B(a)$ in the sense of $B$.

\item It doesn't preserve singletons or sets of singletons so it
doesn't interact well with extraction of models.

\item It preserves $\in^2$.  Sse $a \in^2 i(X)$.  This is $a \in^2
   \{y\subseteq B: y \cap A \in X\}$.  So there is $y \subseteq B$
   with $ y \cap A \in X$ and $a \in y$. But again, since $a \in A$,
   $a \in y$ iff $a \in y \cap A$. So this is equivalent to $a \in^2
   X$.



\end{enumerate}

Now let's think about the surjection from $\pow B) \onto \pow A)$.  It
would be nice if we can cook up a right inverse.  For $x \subset A$,
what sort of things get sent to $x$?  Only supersets of $x$.  Only the
empty subset of $B$ gets sent to the empty subset of $A$, but (the
whole of) $B$ gets sent to the whole of $A$.  So if we want a right
inverse we have to find some extra stuff to add to $x$ to get what we
want.

Now let $f$ be any boolean homomorphism $\pow B) \to \pow B \setminus
A)$.  It will turn out that if the kernel of the homomorphism contains
all singletons then the injection we eventually build will preserve
singletons. But let's not make any assumptions just yet.

The map $x \subseteq A \mapsto x \cup f(x)$ is now a
right-inverse to the surjection $\pow B) \onto \pow A)$.


Let us now overload `$i$' to mean the identity on $A$, $x \mapsto x
\cup f(x)$ on $\pow A)$, and $i$ on ${\cal P}^2 (A)$.  Is $i$ an
$\in$-isomorphism?

Sse $x \in A$ and $y \subseteq A$.  Then $i(x) \in i(y)$ iff $x \in y \cup f(y)$ but $f(y) \cap A = \emptyset$ so this is just $x \in y$.

Now sse $x \subseteq A$ and $y \subseteq \pow A)$.  Then $i(x) \in i(y)$ iff
$x \cup f(x) \in \{z \subseteq B: z \cap A \in y\}$. This is 
$x \cup f(x) \subseteq B$ and $(x \cup f(x)) \cap A \in y$.  Now of course 
$(x \cup f(x)) \cap A = x$ so this reduces to $x \in y$ as desired.

Now can we lift $i$ on the second type to $i$ on the fourth type?

For this we want $i$ (at the third level) to be a right-inverse for
the surjection arising form the $x$-goes-to-$x \cup f(x)$-injection at
the second level.  Let's call this surjection $h$.  We want:

$h(i(X)) = X$.  Now 
$$h(i(X)) = \{x \subseteq A: i(x) \in i(X)\}$$

$$= \{x \subseteq A: x \cup f(x) \in i(X)\}$$

$$= \{x \subseteq A: x \cup f(x) \in \{y \subseteq B: y \cap A \in X\}\}$$

$$= \{x \subseteq A: (x \cup f(x)) \cap A  \in X\}$$

$$= \{x \subseteq A: x  \in X\}$$

$$= X$$

Does this respect $\in$??

Let us now write down what $i$ at the fourth level is. Actually i
suspect that before we can do this intelligibly we'd better generalise
al this to the case where $B$ is not a superset of $A$ but where there
is an injection from $A$ into $B$.



So let's start all over again.  We have two sets of atoms, $A$ and $B$,
with $i:A \inj B$.  We'll agree to start counting the types of our
variables so that $A$ and $B$ are of type $1$, and $i_n$ accepts
inputs of level $n$.

This injection induces a surjection $h:\pow B)
\onto \pow A)$. $h(x) := \{a \in A: i(a) \in x\}$.  This in turn
induces an injection $i:{\cal P}^2 (A) \inj {\cal P}^2 (B)$ by $i(x_3) =
\{y_2: h(y_2) \in x_3\}$ or, in other words, 
$i(x_3) = \{y_2: \{a \in A: i(a) \in y_2\} \in x_3\}$.

Now let $f:\pow A) \to \pow B\setminus A)$ be a boolean algebra homomorphism.
set $i(x_1) := i``x_1 \cup f(x_1)$

[at some point rerun the proof that this $i$ on the first three levels
is still an $\in$-isomorphism]


To get the dfn of $i_4$ just copy the dfn of $i_3$:

 $$i(x_4) = \{y_3: \{x \in \pow A): i_2(x) \in y_3\} \in x_4\}.$$


$i_2(x)$ is $i``x \cup f(x)$ so this is

 $$i(x_4) = \{y_3: \{x \in \pow A): (i``x \cup f(x)) \in y_3\} \in x_4\}.$$


Now we want to simplify $i(x_3) \in i(x_4)$

$$\{y_2: \{a \in A: i(a) \in y_2\} \in x_3\} \in  \{y_3: \{x \in \pow A): (i``x \cup f(x)) \in y_3\} \in x_4\}$$

$$\{w \in \pow A): (i``w \cup f(w)) \in \{y_2: \{a \in A: i(a) \in y_2\} \in x_3\}\} \in x_4$$

Now $ (i``w \cup f(w)) \in \{y_2: \{a \in A: i(a) \in y_2\} \in x_3\}$ is just

 $\{a \in A: i(a) \in  (i``w \cup f(w))\} \in x_3\}$

so we can simplify to

$$\{w \in \pow A):  \{a \in A: i(a) \in  (i``w \cup f(w))\} \in x_3\}  \in x_4$$

Now $ \{a \in A: i(a) \in  (i``w \cup f(w))\}$ is just $w$, so this becomes


$$\{w \in \pow A):  w \in x_3\}  \in x_4$$

and $\{w \in \pow A):  w \in x_3\}$ is obviously just $x_3$ so we get extensionality as desired.


So we've got *something*!!   (Not sure what!!)


$i_3$ doesn't send singletons to singletons.  $i_3$ sends $x$ to the
set of {\sl all} $y$ s.t. $i^{-1}``(y \cap i``A) \in x$ not just some
of them.  That's how $i_3$ sends $V$ to $V$.  We could have sent $x$
to $\{y \subseteq i``A: i^{-1}``y \in x\}$ but then it wouldn't send
$V$ to $V$.  So we want to send $x$ to some $z$ s.t $\{y \subseteq
i``A: i^{-1}``y \in x\} \subseteq z \subseteq \{y: i^{-1}``(y \cap
i``A) \in x\}$.  So we have to ``inflate'' $\{y \subseteq i``A:
i^{-1}``y \in x\}$ with some quantity that is the empty set for
singletons and is the whole of $V_3 \setminus i``V_2$ for $V$.
Clearly we need another boolean homomorphism killing all singletons!
But beware: once we have such a thing, can we be confident that the
revised version of $i_3$ will preserve $B$?


\subsection{Can there be a $\forall$-elementary embedding $M \to \ \chi`M$?} 
\begin{quote}
This section needs radical revision.  First we must establish that for
an embedding to be $\forall$-elementary it is necessary and sufficient
that it should also preserve $B$ and $\iota$.  We prove this by
considering a formula in prenex normal form, withe the matrix in CNF
so we can import the universal quantifier so it is applied to
disjunctions of atomics and negatomics: to wit, things like $(\forall
x)(u \in x \vee v \not\in x \vee x = y \vee x \in z)$ which of course
is $\{y\} \cup B`u \cup \bbar{v} = V$.  Then we have the sad duty of
showing that any $B$-and-singleton-preserving boolean homomorphism
will force there to be a nonprincipal prime ideal $\subseteq V_2$ which 
blows away any hope of showing that the ambiguity we might get from this
doesn't just drop out of the infinitude of the model of TST
  


\end{quote}

  A $\forall$-elementary map is one that preserves formul{\ae} of the
form ($\forall x)\Phi$ where $x$ is the sole bound variable. (these are
sometimes called ``1-embeddings" by model theorists). I here consider the
task of building a $\forall$-elementary map $h$ from a model $M$ of
simple type theory into $\chi`M$, (When $M$ is a model of simple type
theory $\chi`M$ is the result of truncating the bottom type and relabelling
the new bottom type---which had been 1---as 0). We will trade on the fact
that for an embedding $h: M \to \ \chi`M$ to be $\forall$-elementary is
sufficient (because in type theory we need consider only {\sl stratified}
$\forall$-formul{\ae}) that it should respect $B$, $\iota$ and the boolean
operations.  At the time of writing it is not known whether there can be
such an embedding or not. Any model with one must at least have infinitely
many elements of type 0.

Why should any NF-ist care? Two reasons. (i) it is a natural subcase of
full ambiguity. (ii) finding a method for constructing
$\forall$-elementary embedding $M \to \ \chi`M$ when $M$ has infinitely
many elements at type 0 would prove conjecture 2, that NF decides all
stratified $\forall_2$ sentences.
  
We can think of constructing a $\forall$-elementary embedding $M \to \
\chi`M$ as building a series of maps $h_i: M_i \to \ M_{i+1}$ where
$M_i$ is the i$^{th}$ type of $M$. We shall try to construct these
maps $h_i$ so that they can be coded inside $M$ in the usual way. The
precise nature of this coding is not important: what {\sl does} matter
is that the image of a set in the embedding will be a {\sl set} of the
model if $h_i$ is coded in the model. ($h$ {\sl of} an element of the
model must be an element of the model, but if $h$ is not coded in the
model there is no reason to suppose that {\sl the image of} $x$ {\sl
in} $h$ is an element of the model) in general, so we shall want $h$
to be {\sl setlike}. $h_0$ can be any old map $M_0 \to \ M_1$ that is
1-1. If the only thing $h_1$ had to do was respect $\in$, (that is, if
we were content merely to preserve quantifier-free sentences) we would
set $h_1`x =_{\small df}\ h_0``x$, and indeed the idea survives in
part in this more complicated context. As it is, $h_1$ must be a map
$M_1 \to \ M_2$ which also respects the boolean operations and the
singleton operator $\iota$, i.e., we must have $h_1`\{ x\} \ = \{
h_0`x\} $. The requirement that $h_1$ respect the boolean operations
means that in particular $h`V_1 = V_2$.

 We can construct $h_1$ if we have a nonprincipal prime ideal on the
boolean algebra $M_1$. If $x$ is in the ideal $h_1`x$ is to be $h_0``x$. If
not, then $V_1-x$ is in the ideal and we set $h_1`x =_{\small df}\
V_2-h_0``(V_1-x)$. The ideal must be nonprincipal because otherwise some
singletons might be ``large", would not get sent to singletons and thus
$\iota $ would not be respected.

 It is only when we reach $M_n$ with $n \geq 2$ that we have to
consider the remaining operation $B$.  For each $n$, $M_{n+2}$ is a
complete boolean algebra, and it is generated by the $B`x$, for $x$ in
$M_n$. On this important fact will turn the rest of the
construction. Thus every object in $M_{n+2}$ can be regarded as an (in
some cases infinitary) word in the generators $B`x_i$.  We may as well
fix now a notation which we will need later: $g_n$ of a word (at type
$n$) is simply the same word in generators $B`h`x$ instead of
$B`x$. $g_n$ thus preserves $B$ and the boolean operations, tho' not
necessarily $\iota$. For a lot of $x$, $g_n`x$ is what we want $h_n`x$
to be. For example if $x$ is a finite boolean combination of the
$B`x$, then $h_n`x$ {\sl must} be $g_n`x$ in order for $h_n$ to
respect $B$. However if $x$ is an infinitary word $h_n`x$ need not be
taken to be $g_n`x$, and indeed in some cases (when $x$ is a singleton
for example) {\sl cannot}, for $g_n`x$ will be infinite as we shall
see, and $h_n$ of a singleton must be a singleton, for $\iota$ must be
preserved. For singletons, and indeed finite sets $x$ in general
$h_n`x$ must be $h_{n-1}``x$.  The apparent conflict with the need to
preserve $B$ causes no problem as long as $M_n$ is infinite, for then
no singleton is a finitary word in the $B`x$, and it is only finitary
first-order properties we have to preserve. Finally the empty set
(universe) at each type must be sent to the empty set (universe) at
the next type. Thus $V_n$ gets sent neither to $h_{n-1}``{ V}_n$ nor
to $g_n`{ V}_n$, but to something bigger than either of these. $h_n`x$
must always extend $h_{n-1}``x$ in order for the family of $h_i$ to
respect $\in $. Small things $x$, like $\Lambda $, get sent to
$h_{n-1}``x$, but bigger things $x$ get sent to $h_{n-1}``x \cup \
something$, with the something depending on $x$. Let us call this
something the ``inflator" of $x$, since it is what we have to inflate
$h_{n-1}``x$ by to get $h_n`x$. To be explicit,
\begin{dfn}

$infl`x = h_n`x\setminus h_{n-1}``x$ 
\end{dfn}

 First we show that if we are to succeed in constructing $h_n$ 
at all then $infl$ must be a boolean algebra homomorphism. 

\begin{prop}
$x \subseteq y \to  infl`x \subseteq infl`y$  
\end{prop}
 
\Proof
 

Suppose {\sl per impossibile} that we could find $x$, $y$ such that 
$x \subseteq y \wedge infl`x \not\subset infl`y$. Then there is $z$ such that 

 $z \in infl`x \wedge z \not\in infl`y $

Now $z \not\in infl`y $ is $z \not\in (h_n`y \setminus h_{n-1}``y$ and similarly $x$, whence

$z \in \ h_n`x \wedge z \not\in \ h_{n-1}``x \wedge (z \in \ h_n`y \vee z \not\in \ h_{n-1}``y)$ 

 Now $z \in \ h_n`x$ so $z \in \ h_n`y$ since $h_n$ respects $\subseteq$.
So the first disjunct is impossible, and we conclude $z \in \ h_{n-1}``y$.
But since $z$ is in the range of $h_{n-1}$ it must be $h_{n-1}`w$ for some
$w$. But then $h_{n-1}`w \in h_n`x$ so $w \in \ x$ and $h_{n-1}`w \in \
h_{n-1}``x$ contradicting $z \not\in \ h_{n-1}``x$.
\endproof

\begin{prop} 
$infl(x \cap \ y) = infl`x \cap \ infl`y $ 
\end{prop}

\Proof

$ infl`(x\cap y) = $

$h_n`(x\cap y) \setminus h_{n-1}``(x\cap y) = $

$h_n`x \cap \ h_n`y \cap \setminus h_{n-1}``x \cap \setminus h_{n-1}``y  = $

$(h_n`x \cap \setminus h_{n-1}``x) \cap  (h_n`y \cap \setminus h_{n-1}``y)$ 

$= infl`x \cap  infl`y$

 \endproof

 \begin{prop}

$infl`-x$ and $infl`x$ are complements in $h_n`V_n \setminus h_{n-1}``V_n$. 
 \end{prop}
  \Proof  

They are disjoint since they are included in $h_n`-x$ and $h_n`x$ respectively which are 
disjoint by $\forall$-elementarity of $h_n$. $infl`x \cup \ infl`-x$ is 

 $$(h_n` -x \setminus h_{n-1}``-x) \cup (h_n`x \setminus h_{n-1}``x).$$ 

Now since $h_n`-x$ and $h_n`x$ are disjoint we can rearrange this to 

 $$(h_n`-x \cup h_n`x)\setminus (h_{n-1}``-x \cup \ h_{n-1}``x)$$ 
which is 

 $$V_{n+1}\setminus h_{n-1}``V_n$$ 
 \endproof

 Thus $infl$ is a boolean algebra homomorphism. Let $I$ be the kernel.
We will use the notation $[w]_I$ (the subscript $I$ usually omitted) to
mean that $w \in \ M_n$ and $[w]_I$ is the element of $M_n/I$ to
which $w$ belongs.

\begin{rem} 
There is in each element of $M_n/I$ at most one object $x$ such that $h_n`x = g_n`x$ 
\end{rem}

\Proof

Suppose we had $x$, $y$ such that 

 $g_n`x = h_n`x$, $g_n`y = h_n`y$, $x \Delta y \in \ I$

$h_n`(x \Delta \ y) = h_{n-1}``(x \Delta \ y)$ since $x \Delta \ y$ 
is small. But $h_n$ and $g_n$ both commute with boolean operations so 
 $h_n`(x \Delta y) = \ g_n`(x \Delta y)$. We conclude 

 $h_{n-1}``(x \Delta \ y) = g_n`(x \Delta \ y)$. 

We shall now show that these two objects are of impossibly different
sizes. The first object is bounded in size by $M_{n-1}$. To ascertain
the size of the second we think of $(x \Delta \ y)$ as a union of
singletons $z$.

 $g_n`(x \Delta \ y)$ as a union of $g_n`$singletons$z$ What is such a
$g_n`$ singleton $z?$ Well, $z$ is an intersection of things $B`u \cap
-B`v$ so g`z is $B`h_{n-2`}u \cap - B`h_{n-2`}v$ where the $u$
and the $v$ between them exhaust $M_{n-2}$.

 Thus each member of $g_n`z$ must have as members $h_{n-2}`u$ 

\ldots not have as members $h_{n-2}`v$.

 This was enough to determine the member of $z$ uniquely, as $u$ and
$v$ exhausted $M_{n-2}$ but there are now more generators in
$M_{n-1}$ ( $\card{M_{n-1}}$) of them in fact) and so $\card{M_n}$
possibilities for members of $g_n`z$.  Thus $h_{n-1}``(x \Delta \ y)$
and $g_n`(x \Delta y)$ are of impossibly different sizes as promised.

\endproof

From this we can conclude that each equivalence class in $M_n/I$
contains at most one $x$ such that $g_n`x = h_n`x$ and infer the
important

\begin{coroll} 
Distinct finitary words are sent to distinct members of $M_n/I$ 
\end{coroll}

 So far we have been trying to deduce information about $h$ from the fact
that it is $\forall$-elementary . If conversely we are using this
knowledge to build a $\forall$-embedding this shows that at the very
least we will need to find a quotient algebra $M_n/I$ of $M_n$. If we can
find an order-preserving set of representatives to get a subalgebra of
$M_n$, then, given $a \in M_n/I$ we compute $h_n`x$ for $x \in a$ by
$infl`x = _{\small df}(g_n`a_x)-(h_{n-1}``a_x$) where $a_x$ is the
representative from a.  If all we have is an $M_n/I$ without such a set of
representatives we know that all members of any a $\in M_n/I$ have the
same inflator, but we do not know what that inflator is, and therefore have
no obvious means of constructing $h_n$ for members of a.

  {\sl Thus to construct a $\forall$-elementary embedding by this method we
must find an ideal $I$ in $M_n$ which is non-principal (because singletons
must be preserved) and contains no finitary words in the generators $B`x$,
and such that $M_n/I$ has an order-preserving set of representatives}.
  
  Obvious questions are 
\begin{enumerate}
 \item [(i)] Can we ever do this? and 
\item [(ii)] Is there a converse? 
\end{enumerate}

Distinct generators must be sent to distinct members of $M_n/I$. So if an
element $a$ of $M_n/I$ contains a generator (or, {\sl a fortiori}) a finitary
word in those generators, then that generator (or word) must be the chosen
representative, and we know what $h_n$ does to members of $a$. This is
because $h$ must respect $B$ and finitary boolean algebra operations, so
for finitary words $w$ we know $h_n`w = g_n`w$. We have seen above that no
quotient class can contain more than one $x$ such that $g_n`x = h_n`x$, and
so can contain at most one finitary word.

  Now consider $b$, an element of $M_n/I$ which contains no finitary words.
What is $b_x$, the representative of $b$, to be? We have some guidance in
this from the consideration that the set of representatives is to be
order-preserving, and so if $b$ contains a word $W$ which is $\subseteq$
infinitely many finitary words $W_i$, then $b \leq [W_i]$ for each $i$, and
the chosen $b_x$ must $\subseteq$ the representatives of the $[W_i]$ which
will be $W_i$ of course.  Thus $b_x \subseteq W$. So if $b$ contains any
infinitary intersections of finitary words, $b_x$ must be (included in) the
intersection of all those infinitary intersections. Dually if $b$ contained
elements that were infinitary unions of finitary words.




\section{The direct limit construction}

There is an old idea that i have never written about.  Start with the
canonical model of TST with empty bottom type.  Define $f$ by
picking, for each $i$, an injection $f_i: T_i \inj T_{i+1}$ satisfying
$x \in y$ iff $f(x) \in f(y)$.  This gives a direct limit.  We define
$\in$ on the direct limit in the obvious way.  There is an obvious
profinite family of direct limits with an obvious topology.  There is
of course also a logical (``Stone'') topology as well.  This pair of
topologies reminds me of the pair of topologies on the family of all
permutation models.  These two topologies seem to take no notice of
each other in exactly the way the two topologies on the space of
permutation models take no notice of one another.




\chapter[Arithmetic and Wellfounded Sets]{Arithmetic-with-an-automorphism and wellfounded sets in stratified set theories}
\begin{dfn}

We will make frequent use of the following 
permutation:
$$\alpha = \prod_{n \in \smallNn}(Tn, \{m:m\, E\, n\})$$
where $m\,E\,n$ iff the $m$th bit of $n$ is 1. We will call it `$\alpha$' for
Ackermann.
\end{dfn}

It is a commonplace in stratified set theories that $\iota$, the singleton
function, is not necessarily a set, even locally, and we let $T|x| =
|\iota``x|$.  $x$ is finite iff $\iota``x$ is finite and in fact $T$ is an
automorphism of \Nn.

  Thus $x$ and $\iota``x$ do not automatically have the same cardinal,
even if $x$ is finite.  If there are finite $x$ such that $|x|
\not= |\iota``x|$ we have a nontrivial automorphism of \Nn,
usually written $T$.  Among assertions about this automorphism the
most obvious to adopt as an axiom is the assertion that it is the
identity, and this is the axiom of counting, identified as
important---and named---years ago by Rosser.  It turns out that a
weaker assertion, namely that $(\forall n \in Nn)(n \leq Tn)$ is
equivalent to assertions about the consistency of the existence of
particular countable inductively defined wellfounded sets.

  In ``Trois r\'esultats concernant les ensembles fortement cantoriens
dans les ``New Foundations" de Quine, {\sl Comptes Rendues
hebdomadaires des s\'eances de l'Acad\'emie des Sciences de Paris
s\'erie A} {\bf 279} (1974) pp.  41$-$4, Roland Hinnion proved that if
the Axiom of counting holds, then there are permutation models
containing severally $V_\omega$, the set of von Neumann naturals
(hereafter ``$\vno$'') and the Zermelo naturals (hereafter ``$\zi$'').
(Notice that the existence of these things is not an obvious
consequence of the comprehension scheme of \nf.) It is a reasonable
and natural question to ask if Hinnion's result is best possible: can
the hypothesis can be weakened to the extent that there are converses
to any of these results?  The idea is that the (``possible") existence
of things like $V_\omega$, $\vno$, $\zi$ may turn out to be equivalent
to assertions inside arithmetic-with-$T$.  It is claimed in Forster
[1992] that if there is a permutation model in which $V_\omega$ is a
countable set then \AxC\ holds.  Although this proof is erroneous and
the proposition almost certainly false, converses like this can be
proved, and it is the purpose of this note to prove one.  All the
necessary background is to be found in Forster [1995].

All the collections whose potential sethood in permutation models was
proved by Hinnion to follow from the axiom of counting are sets inductively
defined by unstratified inductions.  For example, the collection of Zermelo
integers is $\bigcap\{y: (\Lambda \in y) \wedge (\iota``y \subseteq y)\}$.
There are at least some inductively defined collections of this kind that
cannot be sets at all.  To take an example from \nf, if $\Omega$ is the
length of $\tuple{NO, \leq_{NO}}$ (the set of all ordinals wellordered in
the obvious way) then the collection $\{\Omega, T\Omega, T^2\Omega
\ldots\}$ cannot be a set. Suppose there were a set that was the
intersection of all sets containing $\Omega$ and closed under $T$.  It
clearly contains only ordinals, so look at the least ordinal in it,
$\kappa$, say.  It's closed under $T$, so $\kappa \leq T\kappa$ by
minimality.  $\kappa = T\kappa$ is not possible (o/w we could safely delete
$\kappa$) so $\kappa < T\kappa$.  But then $T^{-1}\kappa$ exists and is
less than $\kappa$, and is therefore not in our set.  But if $T^{-1}\kappa$
is not in our set, we can safely delete $\kappa$ from it too.

  This sharpens the problem of finding the correct statement of a
converse. This definition is not $\DeP_0$, and it will turn out that this
is a large part of the trouble.  We will prove the following:

\begin{thm}
The Axiom of Counting is equivalent to the assertion that there is a permutation
$\pi$ such that $V^{\pi} \models (\exists x)(\forall y)(y \in x \bic (\forall z)(\Lambda \in z
\wedge f``z \subseteq z \to y \in z))$ for all functions $f$ such that
`$y = f(\vec x)$' is in $\DeP_0$.
\end{thm}

\Proof

\begin{quote}{\bf Right-to-Left}\end{quote}

It is actually an old result of Henson's that any set of Von Neumann
ordinals is strongly cantorian, so if the Von Neumann $\omega$ is a set
there is an infinite strongly cantorian set, and this is one version of the
axiom of counting. However we want to deduce the axiom of counting from the
existence of the Von Neumann $\omega$ defined as that inductively defined
set constructed by closing the singleton of the empty set under the
operation $\lambda x.(x \cup \iota`x)$. We cannot use Henson's result
unless we know that everything in this set i have given the inductive
definition of is indeed a Von Neumann ordinal, and that it is infinite.

So suppose $$\bigcap\{X: \Lambda \in X \wedge (\forall y)(y \in X \to y \cup \{y\} \in X)\}$$
exists.

Let us write $\vns$ for von Neumann successor, and let $\vno$ be the
von Neumann $\omega$.  First we note that if $\pow x \subseteq x$ then
$x$ contains $\Lambda$ and is closed under von Neumann successor, so
that our set is wellfounded. Wellfoundedness of $\vno$ implies that
$\vns$ is 1-1, as follows: $x \cup \iota`x = y \cup \{y\} \to x = y
\vee (x \in y
\wedge y \in x)$. The second disjunct contradicts foundation and can be
discarded.

The strategy is to show that $\in$ and $\subseteq$ agree on $\vno$.\begin{itemize}

\item First we show $(\forall x y \in \vno)((x \subseteq y \wedge x \not= y)
\to x \in y)$.

Let $x$ be an arbitrary member of $\vno$. Consider $\{y \in \vno: x
\subseteq y \wedge x \not= y \wedge x \not\in y\}$.  This is a set because
the matrix is weakly stratified.  This set must have an $\in$-least member,
$z \cup \iota`z$.  So we know the following:

(i) $x \subseteq z \cup \iota`z$ 

(ii) $x \not= z \cup \iota`z$ 

(iii) $x \not\in z \cup \iota`z$

(iv) $x \subseteq z \wedge x \not= z \to x \in z$.

\ldots and we want to derive a contradiction from this.

By (i) $x \subseteq z$ unless possibly if $x = z$, but by (iii) that cannot
happen, so $x$ is a proper subset of $z$.  Therefore $x \in z$ by (iv) which
contradicts (iii).

\item Now for the converse. We want $(\forall x y \in \vno)(x \in y \to x \subseteq y)$.
As before let $x$ be an arbitrary von Neumann integer and $y$ an
$\in$-minimal object s.t. $x \in y \wedge x \not\subseteq y$. \Wlog\ $y = z \cup
\iota`z$.  As before this looks unstratified but isn't, so we have

(i) $x \in (z \cup \iota`z)$

(ii) $x \not\subseteq (z \cup \iota`z)$

(iii) $x \in z \to x \subseteq z$.

By (i) either $x \in z$ or $x = z$. If $x \in z$ then by (iii) we have $x
\subseteq z$, so either way $x \subseteq z$. This contradicts (ii).
Therefore, for $\vno$, $\in$ and $\subseteq$ are the same.  \end{itemize}

Next we check that distinct things in $\vno$ have distinct members in
$\vno$.  For suppose two chaps in $\vno$ are distinct.  \Wlog\ they can be
taken to be $x \cup \iota`x$ and $y \cup \{y\}$.  If these two chaps have
the same members we infer $y \in (x \cup \iota`x)$ and $x \in (y \cup
\{y\})$.  These two conditions are equivalent to $x \in y \vee x = y$ and
$ y \in x \vee y = x$ respectively.  By hypothesis we have to discard the
second disjunct, so we have $x \in y \in x$, contradicting wellfoundedness.

Now $\subseteq$ restricted to $\vno$ is a set, so $\in$ restricted to
$\vno$ is a set too.  But if $\in$ is a set restricted to $x$ then
$stcan(x)$ follows immediately because we can send $\iota`x$ to $\{y \in
\vno: y \subseteq x\}$ which is just $x$, by substitutivity of the
biconditional and extensionality.

$\iota`x \mapsto \{y \in \vno: y \subseteq x\}$ = 

$\{y \in \vno: y \in x\}$ =

$x \cap \vno \mapsto$ 

the unique $z \in \vno$ $z \cap \vno$

= $x \cap \vno$ = $x$. 

This tells us that $\vns$ is actually a set of ordered pairs.  We already
know it is 1-1, so $\vno$ is infinite.  So we can conclude that
$\exists\vno \bic \mbox{\rm The Axiom of Counting}$. But since the
antecedent is invariant, we have proved:$$\poss\exists\vno \bic \mbox{\rm
The Axiom of Counting}$$

\begin{quote}{\bf Left-to-right}\end{quote}

It is a simple matter to verify that if we start in a model of \nf C,
$\alpha$ gives us a permutation model containing $V_\omega$, and this
set is clearly strongly cantorian, so we have all the comprehension
that is known to hold for strongly cantorian sets.  This is certainly
enough to prove the existence of the Von Neumann $\omega$ and indeed
any other inductively defined subset of $V_\omega$

\endproof

Two brief points. (i) Of course if all one wants is a permutation
model in which the Von Neumann $\omega$ is a set then it is easier to
use Hinnion's permutation.  (ii) The same ideas will be used to prove
the corresponding direction of the next theorem, and there we have to
be more alert.


We will need the following lemma
\begin{lem}\label{lem:weakhenson}

If $f:\Nn \to \Nn$ is an increasing function that commutes with $T$ then
$$(\forall n \in \Nn)(n \leq f`Tn) \to \AxC.$$
\end{lem}
\Proof If there is an $n > Tn$ then consider the $Tn$th member of the
sequence $\{0, f`0, f^2`0, \ldots f^n`0 \ldots\}$.  This will be a
counterexample to the antecedent.

\begin{thm}

Let $\alpha$ be the Ackermann permutation. Then \AxC\ holds iff $V^{\pi}$ 
contains all sets inductively defined as the closure of $\{\Lambda\}$ under 
any finite number of finitary {\bf stratified} (but not necessarily homogeneous) $\DeP_0$ operations.\footnote{OUCH: do we need the result to be free?}
\end{thm}

\Proof Examples of sets defined in this way are $\zi$, the Zermelo
naturals and $V_{\omega}$ (the closure of $\{\Lambda\}$ under the operation
$\lambda xy.x \cup\{y\}$).   

\begin{quote}{\bf Right-to-Left}\end{quote}

This is in Forster [1995] but we recapitulate for the sake of completeness.
We deduce \AxC\ from the existence of the set of all finite $V_n$s.
Suppose the collection $$\bigcap\{y: (\Lambda \in y) \wedge (\pow` y
\subseteq y)\}$$ is a set.  We'd better have a name for it, \verb#X#, say.
We are going to deduce \AxC. 

First we show that \verb#X# is wellfounded. This is less than
blindingly obvious, because not every $x$ s.t.  $\pow x \subseteq x$ is
closed under ${\cal P}$. However the power set of any such is, so we can reason
as follows.  Suppose $z \in$ \verb#X#. $\pow x \subseteq x$. Then
$\pow`(\pow x) \subseteq \pow x$ and $\Lambda \in \pow x$ so $z \in \pow
x$. But $\pow x \subseteq x$ so $z \in x$ as desired.

   Next we show that \verb#X# is totally ordered by $\subseteq$. Let $x$ be
$\in$-minimal such that $(\exists y)(x \not\subseteq y \not\subseteq x)$,
and let $y$ be $\in$-minimal such that $x \not\subseteq y \not\subseteq x$.
In fact we can take these to be power sets $\pow x$ and $\pow y$ and so we
have $x$ and $y$ such that $x \subseteq y \vee y \subseteq x$ (by
$\in$-minimality) but $\pow x \not\subseteq \pow y \not\subseteq \pow x$
which is clearly impossible.

Since \verb#X# is totally ordered by $\subseteq$ we must have $(\forall
x)(x \subseteq \pow x \vee \pow x \subseteq x)$. The second disjunct
contradicts foundation so we must have $(\forall x \in \verb#X#)(x
\subseteq \pow x)$.

Next we prove by induction that each member of \verb#X# is finite (has
cardinal in \Nn).  Suppose not, and let $\pow x$ be a $\in$-minimal
infinite member of \verb#X#. But if $|\pow x)| \not\in \Nn$\  then
clearly $|x| \not\in \Nn$\ too.

Notice also that there can be no $\subseteq$-maximal member of \verb#X#,
for if $x$ were one we would have $\pow x \subseteq x$ and $x \in x$
contradicting foundation.

Therefore the sizes of elements of \verb#X# are unbounded in \Nn.  Now let
$n$ be an arbitrary member of \Nn.  By unboundedness we infer that for some
$x \in$ \verb#X# we have $|x| \leq n \leq |\pow x|)$ and therefore
$|x| \leq n \leq |\pow x)| \leq 2^{Tn}$. But $n$ was arbitrary, so
$(\forall n \in \Nn)(n \leq 2^{Tn})$. 

But by lemma ~\ref{lem:weakhenson} this implies \AxC.   


\begin{quote}{\bf Left-to-Right}\end{quote}

If $f$ is an operation of the kind we are interested in, there will be a
corresponding operation on natural numbers.  For example $\lambda x.\{x\}$
corresponds to $\lambda n.2^n$.  If $f$ is the operation we start with, let
us notate the correponding operation on natural numbers `$f^*$'. For
example, if $f$ is the singleton operation, $f^*$ is $\lambda n.2^n$.
Suppose now we have a number of such operations (one is easiest for
illustration!!)  and consider the result of closing $\{0\}$ under $f^*$.

It will turn out that in $V^{\pi}$ this is the smallest set containing
$\Lambda$ and closed under $f$.  Showing that it contains $\Lambda$ and is
closed under $f$ is easy.  We need \AxC\ to show that it is the least set
containing $\Lambda$ and closed under $f$.

For the moment, consider the following illustration, which just happens to be lying around.
(Later i'll write out a more general proof)

Let us write $n\,E^T\,m$ for $Tn\,E\,m$.  That is to say: $n\,E^T\,m$ iff the $Tn$th bit of $m$ is 1.

We will need to know that \AxC\ implies that $E^T$ is wellfounded.

Suppose it isn't, and $X\subseteq$\Nn\ has no $E^T$-minimal member.  Let
$n$ be the least member of $X$.  Since $n$ is not $E^T$-minimal, it
follows that there is $m \in X$, $m \geq n$ and $m\,E^T\,n$.  But then $Tn
\leq Tm < n$ contradicting \AxC.

\begin{small}The converse (that $E^T$ wellfounded implies \AxC) is also true but we
don't need it here. (This is in Forster [1995].)  If we have \AxC\ we know
that $E^T$ is wellfounded and we use this to prove by induction on it that
if $y$ is a set s.t.  $V^\pi \models \powk {\aleph_0} y) \subseteq y$ then
all naturals belong to $y$. So \verb#a# is minimal with this property, and
is indeed $V_\omega$ in $V^\pi$.\end{small}

We will show also that $\AxC \to 
(\forall y)(V^\pi \models (\powk{\aleph_0} y) \subseteq y) \to (\forall n \in \verb#a#)(n \in y))$

We prove this by UG on `$y$' and by induction (on $E^T$) over the
naturals.  Since $\AxC$ implies that $E^T$ is wellfounded, this task is
precisely that of proving $$(\forall y)(V^\pi \models (\powk{\aleph_0}
y) \subseteq y) \to (\forall n \in \verb#a#)(n \in y))$$ by $E^T$-induction.

Now 
$$V^\pi \models ((\powk{\aleph_0} y) \subseteq y) \to (\forall n \in \verb#a#)(n \in y))$$
is
$$((\powk{\aleph_0} \pi`y) \subseteq \pi``\pi`y) \to (\forall n \in \verb#a#)(n \in \pi`y))$$
(since \verb#a# is fixed by $\pi$) and we can reletter $\pi`y$ to get
$$((\powk{\aleph_0} y) \subseteq \pi``y) \to (\forall n \in \verb#a#)(n \in y))$$

Now let $y$ be an arbitrary object satisfying $(\powk{\aleph_0} y) \subseteq \pi``y)$.
Suppose $(\forall m)(m\,E^T\,n \to m \in y)$   Consider $\{m: m\,E^T\,n\}$.  This is a finite
set, so is in $\powk{\aleph_0} y)$ and therefore in $\pi``y$.  Therefore 
$\pi^{-1}`\{m: m\,E^T\, n\} \in y$.  But $\pi^{-1}`\{m: m\,E^T\,n\}$ is $n$.  This proves the
induction. \endproof


\begin{rem}
\verb#X# exists iff $V_\omega$ exists and a rank function on $V_\omega$ exists.
\end{rem}


\Proof
If \verb#X# exists then its sumset is $V_\omega$.  The rank of a set
in $V_\omega$ is the number of elements of \verb#X# to which it
doesn't belong.

Conversely, if $V_\omega$ exists and a rank function---$f$, say---on
$V_\omega$ exists, then \verb#X# is $\{f^{-1}``\{n\}:n \in \Nn\}$



\endproof



Suppose the inductively defined set $V_\omega$ exists. Can we even
show that it is countable?  There is no total order of $V_\omega$ definable
by a stratified formula.



If $V_\omega$ is countable, does \AxC\ follow?

We can show it is countable if there is a countable set $X$ equal to
the set of its finite subsets because then $V_\omega \subseteq X$.
There is a always a permutation model in which such a set exists (even
if $\neg$\AxC) so the idea is: show not that $V_\omega \subseteq X$
(which would be true in the permutation model), but rather that there
is an embedding from $V_\omega \inj$ the set that becomes $X$ in the
permutation model.  In other words, map $V_\omega$ recursively into
\Nn.  The obvious thing would be to define a map by recursion on 
$\in$ but this we cannot do!

\begin{dfn}
$\nu = |V_\omega|$ 
\end{dfn}

\begin{rem}

.

\begin{enumerate}
\item $\aleph_0 \leq \nu \to \nu = \nu^2$
\item $T\nu \leq \nu \leq 2^{T\nu}$
\item $\aleph_0 \leq_* \nu$
\item $\nu = T\nu \to \aleph_0 \leq \nu$
\item If $\alpha$ is a cardinal s.t. there is a set in $V_\omega$
      of size $\alpha$, then there is $\beta$ such that $2^\alpha \cdot \beta = \nu$
\item $T^2(\nu^2) \leq \nu$
\end{enumerate}
\end{rem}
\Proof

(1) By coding ordered pairs

(2) $V_\omega$ is transitive and contains all its singletons.

(3) By wellfoundedness $V_\omega$ cannot be finite (i.e. $\nu \not\in
\Nn$).  Therefore it has subsets (and consequently members) of all finite
(in \Nn) sizes, and a countable partition.

(4) follows from a lovely theorem of Tarski's that says (in \nf-speak) that
if there is a bijection between $\iota``X$ and $\powk {\kappa} X)$ then $X$
has a wellordered subset of size $\aleph(\kappa)$.  The proof is as
follows: There is a bijection $f:V_\omega \bic \iota``V_\omega$.  We define
a sequence $$g`0 = \Lambda$$ $$g`(n+1) = (g`n) \cup f`\{y \in g`n: y
\not\in f^{-1}`\{y\}\}$$ 

(5) Let $X$ be a member of $V_\omega$ of size $\alpha$.  Consider the equivalence
relation on members of $V_\omega$ defined by $$x \sim y \bic (x \cap
X) = (y \cap X)$$ For each equivalence class there is a subset $X'
\subseteq X$ such that all member of that equivalence class are of the
form $X \cup y$ where $y \in \powk {\aleph_0} (V_\omega - X)$.
Therefore all equivalence classes are the same size, namely
$|\powk {\aleph_0} (V_\omega - X)|$. Since there is also a
canonical representative for each equivalence class (each equivalence
class contains precisely one subset of $X$) we infer that $2^\alpha$
divides $\nu$.

(6) Follows from the availability of Wiener-Kuratowski ordered pairs
in $V_\omega$.  Similar results hold for higher exponents.

\endproof

A consequence of (5) would appear to be that for each $n \in \Nn$ there
is $\beta$ such that $\beta^{2^n} = \nu$.  This does not seem to be be about
to turn into a proof that $\nu = \aleph_0$.

There doesn't seem to be any proof that $\aleph_1 \not\leq \nu$, and i
can't see any reason why we should expect \nf\ to be abl to prove
things like that.

\subsubsection{Discussion}

let $f$, $g$ be bijections $\iota``\Nn \bic \powk{\aleph_0} \Nn)$ and think
about the structures $\tuple{\Nn, \{\tuple{x,y}: x \in f`\{y\}\}}$ and 
$\tuple{\Nn, \{\tuple{x,y}: x \in g`\{y\}\}}$. Call these $\tuple{\Nn,\in_f}$
and $\tuple{\Nn,\in_g}$.
Notice that all these structures
are or ought to be, end extensions of $\tuple{H_{\aleph_0}, \in}$

A morphism from $f$ to $g$ is an injection $\pi: \Nn \inj \Nn$ such that
\begin{enumerate}
\item $(\forall x, y \in \Nn)(x \in f`\{y\} \bic \pi`x \in g`\{\pi`y\})$
\item $(\forall x,y\in\Nn)((x\in_g`\{y\}\wedge(y\in\pi``\Nn))\to x\in\pi``\Nn)$
\end{enumerate}

(That is to say $\tuple{\Nn,\in_g}$ is an end-extension of
$\tuple{\Nn,\in_f}$ iff there is an arrow from $f$ to $g$.)

Now the assertion that this category has an initial object is stratified.
It therefore cannot imply \AxC.  It is a consequence of \AxC, though. We'd
better prove this.  The idea is that if \AxC, then we take $f`\{n\} = \{m:$
the $m$th bit of $n$ is 1$\}$ and construct an embedding by recursion of
$\in_f$ which we know is wellfounded.


\marginpar{probably snip from here \ldots}

What happens if $\neg$\AxC? Work in a model $\M$ of $\neg$\AxC\ and consider
$\M^\pi$. On the face of it there are three possibilities:
\begin{enumerate}
\item $V_\omega$ does not exist;
\item \verb#a# (the old \Nn) is the new $V_\omega$;
\item Some other set is the new $V_\omega$.
\end{enumerate}

First we show that case 3 is impossible. $V_\omega$ would be (in $\M$) a
subset of the old \Nn.  In fact it would have to be an initial segment.
Think of its size. This would have to be a number $n = 2^{Tn}$ (since a
finite set equal to the set of all its finite subsets is in fact a set
equal to its power set) and we know this is not possible.  A slightly
more elementary proof reasons that a finite $V_\omega$ would have to be
self-membered, contradicting wellfoundedness.

To deal with case 2 we note that if $n$ is a power of 2 and $2^{Tn} < n$
then the integers below $n$ form a set which---in $\M^\pi$---thinks it contains
all its finite subsets. (write this out)  But, as long as $\neg$\AxC, there
will be such $n$ and so the old \Nn\ cannot be the new $V_\omega$.

This leaves only 1. So we have proved

$$\neg\AxC \to \poss \neg\exists V_\omega$$
but {\sl not}
$$\neg\AxC \to \neg \poss \exists V_\omega$$
\marginpar{ \ldots to here}

\section{Sideshow: Hereditarily Dedekind-finite sets}

It might be an idea to think about the set of hereditarily
Dedekind-finite sets.  It isn't directly involved, but it lives next
door, and might illuminate the events at home. Later still we can
think about hereditarily countable sets, and other collections that
cannot be coded as subsets of $V_\omega$.  Perhaps the correct way to
deal with them is to think about $BF$ instead of \Nn.

\begin{quote}\begin{small} We can prove that a set with a finite 
partition into finite pieces is finite. (By induction on $n$, any
union of $n$ finite sets is finite).  We can also prove that a set
with a dedekind-finite partition into dedekind-finite pieces is
dedekind-finite. (If it weren't then we would have a dedekind-finite
partition of a countable set into dedekind-finite pieces, which we
can't have.)  Curious that these two proofs should be so
different!\end{small}\end{quote}

$\hded = \bigcap\{y: \powk {\mbox{\footnotesize dedekind-finite}} y)
\subseteq y\}$.  In \zf\ we can prove that this collection is
$V_\omega$ without any use of choice: $V_\omega$ exists, and because
it is countable it is a $y$ such that $\powk {\mbox{\footnotesize
dedekind-finite}} y \subseteq y$ so $\hded \subseteq V_\omega$.  In
\kf\ or \nf\ we know a lot less.

Do we even know that if $V_\omega$ and $\hded$ both exist then
they are identical?  We can prove

\begin{rem}
If $V_\omega$ exists and is countable then $\hded$ exists and is equal to
$V_\omega$.
\end{rem}
\Proof
If $V_\omega$ exists and is countable then it contains all its 
dedekind-finite subsets.  Therefore $\hded \subseteq V_\omega$. 
The inclusion in the other direction is easy. 

\endproof

\begin{dfn}
$\delta = |\hded|$
\end{dfn}

\begin{rem}

.

\begin{enumerate}
\item $T\delta \leq \delta \leq 2^{T\delta}$
\item $\aleph_0 \leq \nu$
\item $\delta = \delta^2$
\item If $\alpha$ is a cardinal s.t. there is a set in $\hded$ 
of size $\alpha$, then there is $\beta$ such that $2^\alpha \cdot \beta = \delta$

\end{enumerate}
\end{rem}
\Proof

(1) just as with $\nu$

(2)  First, since $\hded$ is the
intersection of all sets extending their set of finite subsets it must be
wellfounded. In particular it is not self membered so it cannot be
dedekind-finite. So it has a countable subset.  (If we knew
$can(\hded)$ we could derive this from
Tarski's theorem but we can do it anyway!)

(3) $\hded$ has a countable subset so we can fake Quine ordered pairs.


(4) Let $X$ be a member of $\hded$.  Consider the equivalence
relation on members of $\hded$ defined by $$x \sim y \bic (x \cap
X) = (y \cap X)$$ For each equivalence class there is a subset $X'
\subseteq X$ such that all member of that equivalence class are of the
form $X \cup y$ where $y \in \powk {\mbox{\footnotesize
dedekind-finite}} (V_\omega - X)$.  Therefore all equivalence classes
are the same size, namely $|\powk {\mbox{\footnotesize
dedekind-finite}} (V_\omega - X)|$. Since there is also a canonical
representative for each equivalence class (each equivalence class
contains precisely one subset of $X$) we infer that $2^\alpha$ divides
$\delta$.
(This is just like the corresponding proof for $V_\omega$)
\section{Discussion}
Can we show $\aleph_1 \not\leq \delta?$

All of this talk of small permutations involving \Nn\ can be done in
\kf\ too of course.  In this context it seems important to note that
\kf+\AxC\ is no stronger than \kf, even tho'
\nf+\AxC\ probably is stronger than \nf.
 
\subsubsection{Eight propositions about wellfounded sets: second version}
(A bottomless set is one with no $\in$-minimal element; $PFIN$ is the set of fine power sets)

\begin{enumerate}
\item  $\{V_n: n \in \Nn\}$ exists.
\item  $V_\omega$ exists.
\item There is an infinite transitive wellfounded set.
\item There is an infinite wellfounded set.
\item Every natural number contains a wellfounded set.
\item $WF$ has no finite superset.
\item (i) Every fat set is infinite; (ii) $\tuple{PFIN, \in}$ is wellfounded; (iii) every bottomless set of power sets consists entirely of infinite sets.
\item $\poss(\in\restric FIN$ is wellfounded $)$.
\end{enumerate}

Everything in this list implies everything below it.  All the propositions in item 7 are equivalent.  I do not know how to reverse any of the arrows.  If you strip the `$\poss$' off item 8 you get something that implies item 7

6 $\to$ 7.  If $WF$ has no finite superset then it certainly has no finite fat superset.  But every fat set is a superset of $WF$, so there are no finite fat sets.

Various forms of 7: if $x$ is fat then $\{{\cal P}(x)\}$ is a bottomless set of power sets.  

``There are infinitely many (wellfounded) hereditarily finite sets'' (aka: ``no finite set contains every hereditarily finite wellfounded set'')
  doesn't seem to fit into this linear sequence \ldots  Let's think about this last one.  It follows from (3), as follows.
Suppose $x$ is an infinite transitive wellfounded set, and $y$ is a
finite set containing all hereditarily finite sets.  Consider $x
\setminus y$.  This must have an $\in$-minimal member.  (It's worth
spelling out why this is the case, beco's ``I have an $\in$-minimal
member'' is not stratified and cannot be proved by $\in$-induction.
However we can prove that every member and every subset of a
wellfounded set is wellfounded, and certainly every wellfounded set is
regular: If $u$ is wellfounded every $v$ with $u \in v$ is disjoint
from one of its members). Clearly $x \setminus y$ is nonempty and all
its members are wellordered, so it has an $\in$-minimal member---$w$,
say. Now $w \in x$ so $w \subseteq x$ by transitivity of $x$.  By
$\in$-minimality we have $w \subseteq x \cap y$, so $w$ is finite
(co's $y$ is finite) so $w$ is a finite set of hereditarily finite
sets and so is hereditarily finite, contradicting assumption.
\ldots and it implies 6.

So what we should do now is either:
(i) show that if there is an
infinite wellfounded set then no finite set can contain all
hereditarily finite (wellfounded) sets; or\\

(ii) show that if no finite
set can contain all hereditarily finite (wellfounded) sets then there
is an infinite wellfounded set.

Another thing to look at is this.   

$F(X) := \{x \in FIN: x \cap FIN \subseteq X\}$\\  
${\cal F} :=$ least set containing $\emptyset$ and closed under $F$. Or the even stronger:



Where does the existence of ${\cal F}$ fit in all this?

\subsubsection{Eight propositions about wellfounded sets}

Consider the following assertions.

1 $\{V_n: n \in \Nn\}$ exists

2 $V_\omega$ exists

2$'$ There is an infinite transitive wellfounded set

3 There is an infinite wellfounded set

4 There is no finite bound on the size of wellfounded sets
 
5$'$ Every set of power sets with no $\in$-minimal member has only infinite members

5 $\pow x \subseteq x \to$ $x$ is infinite

6: There are infinitely many wellfounded sets

7: There are infinitely many (wellfounded) hereditarily finite sets

Randall has recently shown that `2$'$ is not a theorem of any
consistent invariant extension of NF.

I think i prove somewhere that the least fixed point for
Hinnion's + is wellfounded on the wellfounded sets.  To be clear about
it, there is a binary relation $x \leq y$ iff$_{df}$ $(\forall R)(R^+
\subseteq R \to \tuple{x,y} \in R)$, and this is wellfounded in the
sense that any set of wellfounded sets has an $R$-minimal element.
And anything $\leq$ a wellfounded set is wellfounded.  (I think all
this is true)

I then go on to say 


``So if there is an
infinite wellfounded set there is one of minimal rank''.  

But i think this is wrong.  The collection of infinite wellfounded
sets is not a set.  so i think i cannot therefore draw the conclusion i claimed:

``So if there is an infinite wellfounded set there must be infinitely
many hereditarily finite sets''

Obviously 3 implies 6.  Does 6 imply 7? It ought to, but we can't reason
about ranks here!  Certainly in $V^\alpha$ 6 implies 7.  Suppose 7 is
false.  Then \AxC\ fails.  But if \AxC\ fails, there is $n > 2^{Tn}$ and in
$V^\alpha$ this becomes a finite thing extending its own power set, so all
wellfounded sets are finite.

Similarly in $V^\alpha$ 5 implies 7.  Suppose 7 is false.  Then \AxC\
fails.  But if \AxC\ fails, there is $n > 2^{Tn}$ and in $V^\alpha$ this
becomes a finite thing extending its own power set.  In general one would
expect that 5 doesn't imply 7.

If one suspects these things are separate, then we will have to reason in
things other than $V^\alpha$ to prove it.

It would be nice to be able to prove that 2 implies 3.  Suppose 2 is
true but 3 is false, and $X$ is a finite set that contains all
hereditarily finite sets.  Then every infinite wellfounded set has an
infinite wellfounded member.  This is no use unless the class of
infinite wellfounded sets is a set!  If we had an axiom of transitive
closures we'd be ok....


(Should try to fit in ``Every finite wellfounded set has a transitive
closure". Come to think of it, can we even prove that the transitive
closure of a wellfounded set---if it exists---must be wellfounded? I don't
see how!)  

Obviously 1 $\to$ 2 $\to$ 3 $\to$ 4 $\to$ 5.  The point about 1
is that it is equivalent to the assertion that there is a rank function on
$V_\omega$.  If 5 looks out of place, remember that a wellfounded set is
simply something that is included in all $x$ such that $\pow x) \subseteq x$
so there cannot be any infinite wellfounded sets at all unless 5 is true.
At the moment there is the theoretical possibility that all the arrows
might be reversed, however improbable such an outcome may seem.  My guess
is that {\sl none} of them can be. We have seen that $\poss 1$ is an
equivalent of \AxC, but none of the others seem to imply \AxC, so there
remains the unexcluded possibility that $\nf \vdash \poss 2$. I don't
believe that either. In fact i don't believe even that $\nf \vdash \poss
5$, even tho' 5 is so weak that we have to hang a `$\Box$' on the front of
it to get anything strong enough to be obviously equivalent to \AxC.

\begin{rem}
$\Box 5$ and \AxC\ are equivalent. 
\end{rem}

(We already know that
\AxC\ and $\poss 1$ are equivalent).


\Proof 

L $\to$ R: (By contraposition) If \AxC\ fails, there is $n > 2^{Tn}
\in \Nn$.  Since whenever $x \not\in x$, $B`x$ is a set of size
$|V|$ disjoint from its power set, we can find, for any cardinal
$n$, a set of size $n$ disjoint from its power set.  In particular if
$n$ is the finite cardinal promised above (so $2^{Tn} < n)$ then we
have a set $x$ of size $n$ disjoint from its power set and an
injection $p$ from $\pow x)$ into $x$.  This can be extended to a
permutation $\pi$ of $V$. This proves $\poss \neg 5$.

R $\to$ L: If $\pi$ is a permutation such that $V^\pi$ thinks that
some set $x$ is finite and a superset of its power set, then $V$
contains a map (namely a suitable restriction of $\pi$) from some
finite power set $\pow x)$ into $x$ and therefore a natural number $n = |x|$
such that $2^{Tn} < n$, which contradicts \AxC. \endproof



sept 2003: a brilliant idea.  Clearly if there is an infinite
wellfounded set then there can be no finite $x$ with $\pow x) \subseteq
x$.  However, we can even show, in those circumstances, that $\in$
restricted to finite power sets is wellfounded. Indeed we can prove
even that if $A$ be a family of power sets without a $\in$-minimal
member then every member of $A$ is infinite.  Let $A$ be a set of
power sets with no $\in$-minimal member.  We prove by $\in$-induction
that every wellfounded set belongs to every member of $A$.  (reality
check: $\emptyset$ obviously does, so we are pointing in the right
direction!!).  Let $\pow x)$ be an arbitrary member of $A$, and $a$ a
family of sets each of which belongs to every $\pow y) \in A$.  We
want $a \in \pow x)$.  Beco's $A$ has no $\in$-minimal member, there
is $\pow y)$ in $A$ with $\pow y) \in \pow x)$.  Then $a \subseteq
\pow y)$ by induction hypothesis, so $a \subseteq \pow y) \in \pow x)$
but $\pow x)$, being a power set, is $\subseteq$-downward closed, so
$a \in \pow x)$ as desired.  This means that if $A$ contains even one
finite set then every wellfounded set is finite.  So if there is an
infinite wellfounded set then not only is $\in$ restricted to finite
power sets wellfounded but any bottomless set of power sets consists
entirely of infinite sets.  Indeed we can even prove the following:

\begin{rem}
If $A$ is a bottomless set of power sets, then $\bigcap A$ is
a self-membered power set.
\end{rem}

\Proof Clearly any intersection of a lot of power sets is a power set,
  so $\bigcap A$ is a power set.  We want $\bigcap A \in \bigcap A$.
  So we want $\bigcap A \in \pow x)$ for every power set $\pow x) \in
  A$.  Let $\pow x)$ be an arbitrary member of $A$.  Now $A$ is
  bottomless, so there is another power set $\pow y)$ in $A$ such that
  $\pow y \in \pow x)$, which is to say $\pow y) \subseteq x$.  Now
  $\pow y) \in A$ gives $\bigcap A \subseteq \pow y)$.  But then
  $\bigcap A \subseteq \pow y) \subseteq x$ and $\bigcap A \subseteq
  x$ and $\bigcap A \in \pow x)$.  But $\pow x)$ was an arbitrary
  member of $A$, so $\bigcap A \in \bigcap A$ as desired. \endproof

So, to recapitulate:

If every finite cardinal contains a wellfounded set, then there can be
no finite self-membered power set.  So every bottomless set of power
sets consists entirely of infinite sets.  So membership restricted to
finite power sets is wellfounded.  So using the clever Boffa-style
permutatation of remark \ref{rembottomlesssetsofpowersets} (i think
this reference is correct!) we get a model in which membership
restricted to finite sets is wellfounded.



(Sse $A$ is a set of finite power sets with no $\in$-minimal element.
Let $\pow M)$ be an element of $A$ of minimal size.  Then $\pow N) \in
\pow M)$ for some $N$.  ($|M| = m$, $|N| = n$ of course).  Then $\pow
N) \subseteq M)$ so $Tn < 2^{Tn} \leq m \leq n$ (this last by
minimality of $\pow M)$).  This contradicts \AxC.  Thus \AxC $\to$
$\in\restric$ finite power sets is wellfounded. A similar argument will show
that if there is a bottomless set $A$ of power sets, with $\kappa =
inf(\aleph``\{|x|: x \in A\}$, then $\kappa > T\kappa$.  But this
isn't big news.   We know stronger results than this already.)

So if there are arbitrarily large finite wellfounded sets, then every
bottomless set of power sets consists entirely of infinite sets.  How
surprising is this? Are there any bottomless sets of power sets at
all??  Yes: $\{V\}$ is one!  


So the general argument now goes as follows.  Let $\kappa$ be strongly
inaccessible, and suppose that there are wellfounded sets of arbitrarily
large size below $\kappa$.  So if $B$ is a collection of self-membered
power sets then nothing in $B$ is $\kappa$-large.  So $\in\restric\{\pow x):
|x| < \kappa\}$ is wellfounded.  Then use a Boffa permutation as above
to obtain a model in which  $\in\restric \{x: |x| < \kappa\}$ is wellfounded.

We can also show 
\begin{rem}\label{rembottomlesssetsofpowersets}

$\poss 5' \bic \poss(\in$ restricted to finite sets is wellfounded$)$
\end{rem}

\Proof

This requires a Boffa-style permutation.

R $\to$ L.
\begin{quote}
We can prove this even with the $\poss$ stripped off, which we will
now do.  The right-hand side implies that no finite set is selfmembered 
and in particular that no finite power set is selfmembered.  Now let $A$ be
a set of power sets with no $\in$-minimal member.  Then $\bigcap A$ is
a self-membered power set by remark \ref{bottomless}.  If any member 
of $A$ had been finite, then $\bigcap A$ would be finite too.  So $A$ 
consists entirely of infinite sets.  This is 5$'$. \endproof

[This takes us very close to a proof of a result of Tonny Hurkens for
  Zermelo set theory: that the relation $F(x,y)$ iff $\pow x \cap y)
  \subseteq y$ is wellfounded.  Suppose not, and that there is a set
  $X$ with no $F$-minimal element.  Consider $\bigcap X$.  Let $y$ be
  an arbitrary member of $X$.  Then there is $x \in X$ with $F(x,y))$.
  We have $\bigcap X \subseteq x$ and $\bigcap X \subseteq y$ whence
  $\pow \bigcap X) \subseteq \pow x)$ and $\pow \bigcap X) \subseteq
  \pow y)$.  These last two imply $\pow \bigcap X) \subseteq \pow x)
  \cap \pow y = \pow x \cap y) \subseteq y)$. But $y$ was an arbitrary
  member of $X$; so $\pow \bigcap X$ is included in every member of
  $X$, so $\pow \bigcap X) \subseteq \bigcap X$, contradicting
  Zermelo's axioms.]


\end{quote}


L $\to$ R.

\begin{quote}
Let $\pi$ be the permutation

$$\prod_{|x| \in \smallNn}(\tuple{\pow(\bigcup (\fst``x \cap FIN)), V \setminus x},x)$$ 

($FIN$ is the set of finite sets). Clearly if $x$ is finite then $\pi(x)$ isn't.  

Suppose $V^\pi \models x \in y$, both finite.  Then $x \in \pi(y)$,
and $\pi(x)$ and  $\pi(y)$ are both finite, so $$x = \tuple{\pow
  \bigcup\fst``(\pi(x))) \cap FIN,  V \setminus\pi(x)}$$ and $$y = \tuple{\pow
  \bigcup\fst``(\pi(y))) \cap FIN,  V \setminus\pi(y)}.$$


$x \in \pi(y)$ so $\fst(x) \in \fst``\pi(y)$.  That is to say
$$\tuple{\pow \bigcup\fst``(\pi(x))) \cap FIN,  V \setminus \pi(x)} \in
\fst``(\pi(y)).$$ Now $\fst``\pi(y)$ is a subset of $\pow
\bigcup\fst``(\pi(y)))$ and consists entirely of finite sets so
$$\fst``\pi(y) \subseteq \pow \bigcup\fst``(\pi(y))) \cap FIN$$ which is
$\fst(y)$.

This tells us that $\fst$ is a homomorphism from $\tuple{FIN^\pi,
  \in_\pi}$ to $\tuple{PFIN, \in}$ where $PFIN$ is the set of finite
power sets.  And 5$'$ certainly implies that this second structure is
wellfounded. So the first must be wellfounded too. \endproof
\end{quote}


It's natural to wonder if we can do this for properties other than
finiteness, for other notions of smallness.  Remark
\ref{rembottomlesssetsofpowersets} exploits the fact that a union of
finitely many finite sets is finite, which is a bit of a downer.  In
general we will have difficulties beco's $|\powk{\kappa} X)| > \kappa$
so $\powk{\kappa} X) \subseteq X$ is not the same as $\powk{\kappa} X)
\in \powk{\kappa} X)$.  We seem to need $\kappa$ to be strong limit.

So, if there is an infinite wellfounded set, $\poss( \in$ restricted to
finite sets is wellfounded$)$.  The first of several interesting 
questions this suggests is: is there a converse?
\begin{quote} If $\poss(\in$ restricted to
finite sets is wellfounded) is there an infinite wellfounded
set?\end{quote}

A second question arises from the observation that of course the
interesting assertion is not `` $\poss( \in$ restricted to finite sets
is wellfounded$)$'' but $\Delta_{\aleph_0}$, which says ``$\poss( \in$
restricted to finite sets is wellfounded and the graph of the rank
function is a set$)$''. The second question is
\begin{quote}
Might $\Delta_{\aleph_0}$ follow from ``There is an infinite
{\sl transitive} wellfounded set?''?  \end{quote}

This is suggested to me by the way in which Holmes' permutation
highlights the r\^ole of transitivity in this setting.  Is it time to
review the question of whether or not transitive closures of
wellfounded sets (when they exist) are likewise wellfounded?



\subsubsection{A funny translation task}

As always happens when i encounter a new idea, i cannot leave it
alone.  Here are some tho'rts on how to take it further.

Suppose we are in a model $\M$ where we have some stratified property
$\phi$ such that $\M \models \phi(x) \to x \not\in x$.  A good example would
be Boffa's model, where $\phi$ is $|\fst``x| \not\geq_* \aleph_0$.
$\phi$ is closed under surjections.  Consider the permutation
$$\pi = \prod_{|\fst``x| \not\geq_* \aleph_0}(x, \tuple{ V \setminus \fst``x,x})$$
We want to show
$$V^\pi \models \psi(x) \to x \not\in x$$
for some suitable $\psi$.  This is
$$V \models \psi(\pi_n`x) \to x \not\in \pi`x$$

Now we have constructed $\pi$ so that, for instance: $$V \models
|\fst``x| \not\geq_* \aleph_0 \to x \not\in \pi`x$$ so what we
want is to find $\psi$ s.t. $(\forall x)(\psi(\pi_n`x) \to |\fst``x|
\not\geq_* \aleph_0)$.  This is equivalent to 
$$(\forall x)(\psi(x) \to |\fst``(\pi_n`x)| \not\geq_* \aleph_0).$$

Therefore we want to see if the property $$|\fst``(\pi_n`x)|
\not\geq_* \aleph_0$$ turns out to be implied by anything sensible.
(Remember $\pi$ is definable, so `$x$' is the only free variable!

\section[Infinite WF sets consis wrt NF?]{Is it consistent relative to \nf\ that there should be an infinite wellfounded set?}

I tho'rt i'd proved it:

Let us work in Friederike's model which contains a natural number $k$ s.t.
$(\forall n \in \Nn)(n > k \to n < Tn)$.  Let $\pi$ be the permutation that
for $n \in \Nn$ swaps $\{n\cdot k\}$ with $(Tn +1)\cdot k$ for $n > 0$,
swaps $\Lambda$ with $0$ and fixes everything else.  In $V^\pi$ the set
that was $\{n \cdot k: n \in \Nn\}$ (let us call this set \verb#b#) has
become the Zermelo integers, which is to say the intersection of all sets
containing the empty set and closed under singleton.  Suppose $V^\pi
\models 0 \in y \wedge (\forall x \in y)(\{x\} \in y)$, we want $V^\pi
\models \verb#b# \subseteq y$.  That is to say, $\pi(\verb#b#) \subseteq
\pi(y)$.  $\pi(\verb#b#) =$ \verb#b#.

$V^\pi \models \Lambda \in y \wedge (\forall x \in y)(\{x\} \in y)$ is
just $\pi`\Lambda \in \pi`y \wedge (\forall x \in \pi`y)(\pi`\{x\} \in
\pi`y)$.  We know $\pi`\Lambda = 0$. So we must show $$(\forall y)(0 \in y
\wedge (\forall x \in y)(\pi`\{x\} \in y) \to \verb#b# \subseteq y)$$

We cannot prove by induction on \verb#b# that if $n \in \verb#b#$\ then
$(\forall y)((0 \in y \wedge (\forall x \in y)(\pi`\{x\} \in y)) \to n \in
y)$ because this is not a stratified induction. We do it instead by UG on
`$y$'.  Let $y$ be a set such that $0 \in y \wedge (\forall x \in
y)(\pi`\{x\} \in y)$.  We want $\verb#b# \subseteq y$.

0 is in $y$, by hypothesis.  Let $n \cdot k$ be minimal such that $n\cdot k
\not \in y$.  Now $n = \pi`\{T^{-1}(n -1) \cdot k\}$.  But---since
$(\forall x \in y)(\pi`\{x\} \in y)$---we must have $T^{-1}(n -1)\cdot k
\not\in y$ too.  But $T^{-1}(n -1)\cdot k$ is bigger than $k$ so $T^{-1}(n
-1)\cdot k < (n -1)\cdot k$ and $(n -1)\cdot k < n\cdot k$ contradicting
minimality of $n \cdot k$.

This tells us that, in $V^\pi$, \verb#b# is the intersection of all sets
containing the empty set and closed under singleton.  This set is clearly
wellfounded, because if $\pow X) \subseteq X$ then $X$ contains the empty
set and is closed under singleton.  Now to show it is infinite.  We have
$|\verb#b#| = T|\verb#b#| + 1$, so clearly $|b|$ is
not a natural number.


\ldots but of course in the last paragraph but one the sentence beginning
``But $T^{-1}(n -1)\cdot k$ is bigger than $k$ \ldots" should go on to say
that $T^{-1}(n -1)\cdot k < (n-1)\cdot Tk$ which of course is buggerall use
to man or beast.

Nevertheless, the idea of trying to prove $\poss \exists \zi$ from
nothing seems a good one.
All we need is a $k$ such that $(\forall n \in \Nn)(n\cdot k < T n \cdot
k)$ rather than $(\forall n > k)(n < T n)$.  But this is just \AxC.

Inductively define a subset \verb#X# of \Nn\ as follows: 0 $\in$ \verb#X#;
if $x \subseteq$ \verb#X# then $k\cdot \Sigma_{n \in x} 2^n$ $\in$
\verb#X#.

Define $E'$ on \verb#X# by $x E' y$ iff the $x$th bit of $y/k$ is 1.

Now swap $Tn$ with $\{m: m E' n\}$, for $n$ and $Tn$ in \verb#X#.  The
trouble is, for this to work we seem to need \verb#X# to be closed under
$T$.

There is this idea abroad that if $\in$ restricted to FIN is wellfounded,
then we should be able to get an infinite wellfounded set.  Let $f: FIN \to
\Nn$.  Define $f^*:FIN \to \Nn$ by $f^*`x= T$sup$\{f`y + 1: y \in x \cap
FIN\}$.  If we could show that * had a fixed point we would be able to infer
that $<^T$ is wellfounded.  But this is far too strong.  So the obvious
approach doesn't work!

Think again about trying to get an infinite wellfounded set at no cost.
What does a natural have to do to be a wellfounded set in the Ackermann
permutation model?  Clearly the restriction of $E^T$ to $E^T``\{n\}$ has to
be wellfounded.  One way of ensuring this is to require that $(\exists
m)(\forall k)(T^k n \leq m)$.  (Forgive abuse of notation!).

So we are led to the proposition that the collection of naturals $n$ such that there is such an $m$ is unbounded.

$$(\forall m')(\exists m \geq m')(\exists k)(\exists A \subseteq \Nn)(T^{-1}``A \subseteq A \wedge m \in A \wedge (\forall a \in A)(a < k))$$ 

Of course one could be more kosher about it and concentrate on the property

$$V^\alpha \models (\forall x)(\powk{\aleph_0} x) \subseteq x \to n \in x)$$

where $\alpha$ is the Ackermann permutation.  This is $(\forall
x)(\powk{\aleph_0} x) \subseteq x \to n \in x)^\alpha$ $(\forall
x)(\powk{\aleph_0} x) \subseteq \alpha``x \to n \in x)$.  I think \wwlog\ we
can restrict attention to subsets of \Nn.

$(\forall x \subseteq \Nn)((\forall$ finite $x'\subseteq x)((\Sigma_{y \in x'} 2^{Ty}) \in x) \to n \in x)$

We want there to be an infinite set of such $n$.  An obvious question is:
is this collection downward-closed? I think it is clear that if $x$ is a
member and the $Tk$th bit of $x$ is 1 then $k$ is a member.

The obvious thing to do is a recursive definition:

\begin{quote} 0 is a wellfounded* natural; if $X$ is a finite set of
wellfounded* naturals, then $T(\Sigma_{n \in X} 2^n)$ is wellfounded*
\end{quote}

Is this class unbounded?  Does it have an unbounded subset? This is
something to do with finite sets extending their own power sets. I suspect
it is unbounded iff the following class is unbounded:


\begin{quote} 0 is a widget; if $n$ is a widget, so is $2^{Tn}$.
\end{quote}

(This takes us back to the Zermelo naturals!)

What we seem to have done is shown that in $V^\alpha$ there are
infinitely many hereditarily finite sets iff the collection of
wellfounded* naturals is unbounded.

Think about the family of inductively defined collections: $\{0, Tf(0),
T(f(Tf(0))) \ldots \}$ indexed by the family of definable homogeneous
maps $f: \Nn \to \Nn$\ which commute with $T$. Are these equally
unbounded, as it were? Start with the closure of $\{0\}$ under
$\lambda n. T(n+1)$.  The assumption that this is unbounded implies
\AxC.  For sse there is $x > Tx$. Then the initial segment bounded by $x$
contains 0 and is closed under $\lambda n. T(n+1)$.  So this is a strong
assumption.  Where $f$ is more rapidly increasing this could be a weaker
assumption.  So how about: $\exists f: \Nn \to \Nn$ which commutes with $T$
such that $\{0, Tf`0, T(f`Tf`0) \ldots \}$ is unbounded?  Is this the same
as $\exists f: \Nn \to \Nn$ which commutes with $T$ such that $(\forall
n)(f`Tn \geq n)$?

So suppose there is an $x$ s.t. $x > f`Tx$.  Then the initial segment
bounded by $x$ contains 0 and is closed under $\lambda n. T(f`n)$, so
the sequence $\{0, Tf`0, T(f`Tf`0) \ldots \}$ is bounded.
Contraposing, if $\{0, Tf`0, T(f`Tf`0) \ldots \}$ is unbounded, then
there is no $x$ s.t. $x > f`Tx$, so $(\forall n \in \Nn)(n \leq
f`Tn)$.  But, as we have seen earlier (lemma ~\ref{lem:weakhenson}), if $f$
commutes with $T$ then $(\forall n \in \Nn)(n \leq f`Tn) \to
\AxC$. (If there is an $n > Tn$ then consider the $Tn$th member of the
sequence $\{0, f`0, f^2`0, \ldots f^n`0 \ldots\}$.  This will be a
counterexample to the antecedent.)

So we seem to have proved: 

\begin{thm}
Let $f$ be a definable homogeneous map \Nn $\to$ \Nn\ which commutes with
$T$. The inductively defined collection: $\{0, Tf`0, T(f`Tf`0) \ldots \}$
is unbounded iff \AxC.  \hole{What happens if $f$ isn't a unary thing like
this?}
\end{thm}


No, we haven't shown that: there is a gap in the proof.  We shouldn't
be considering an $n$ s.t. $n > f(Tn)$ but an $n$ which bounds the
collection.  But i think what is true is that if $f: \Nn \to \Nn$ is
monotone increasing and commutes with $F$ then \AxC\ is equivalent to
the assertion that the only initial segment of \Nn\ closed under
$f\circ T$ is \Nn\ itself.  We reason as follows. (This will need a
bit of tidying up) Sse $f: \Nn \to \Nn$ is monotone increasing and
commutes with $F$.  If $f$ isn't the identity then we will have $n + 1
\leq f(n)$.  Sse now that $[0,n+1]$ is closed under $f \circ T$. Then 
$$f(Tn) < n+1 \leq f(n)$$ whence
$$f(Tn) < f(n)$$ and $Tn < n$.   So, unless \AxC\ fails, no proper 
initial segment of \Nn\ can be closed under $f \circ T$.  So the
intersection of all initial segments closed under $f \circ T$ is \Nn\
itself. Is this {\sl exactly} the same as saying that the closure of
$\{0\}$ under $f\circ T$ is unbounded\ldots?  This relies on the
downwards closure of the closure of $\{0\}$ under $f\circ T$ being the
same as the intersection of all initial segments of \Nn\ closed ubnder
$f \circ T$.  This should be easy to check one way or another.






Even that case is easy.  Suppose there is an $x$ s.t. $x > T(\Sigma_{y
  < x} 2^y)$.  Then the collection of wellfounded* naturals is
bounded.  But there will be such an $x$ unless \AxC.

\begin{coroll}
If $V^\alpha$ contains infinitely many hereditarily finite sets then 
\AxC. (The converse is easy because if \AxC\ holds $V^\alpha$ contains
$V_\omega$). 
\end{coroll}

So if we want infinitely many wellfounded sets cheaply we will have to
try something else.  For example: work in a model where $\in$
restricted to finite sets is wellfounded.  

One might think that one should be able to collapse $FIN$ to get
$H_{fin}$.  The collapsing function is a fixed point of a TRO so one
might hope to add it by permutation while keeping FIN wellfounded. $j$
is a map from $FIN \to FIN$ into itself, and the collapsing function
is the lfp.  We would need an $f$ that was $n$ similar to $jf$ and
that might require weak forms of choice.  Sounds hard.  In any case
the output of this construction (if there is one) would be $H_{fin}$
and this is much stronger than the result we are looking for, namely
the existence of an infinite welfounded set.




 Instead ask: is
there an infinite extensional set of finite sets? If $X$ is such a set
consider the permutation $\prod_{x \in X}(x, x \cap X)$.
Extnsionality of $X$ ensures that these transpositions are disjoint.
In the resulting permutation model $X$ has become an infinite
wellfounded set.

So we have shown:

\begin{rem}

If $\in \restric FIN$ is wellfounded, and there is an infinite extensional 
subset of $FIN$, then $\poss \exists $ infinite wellfounded set.
\end{rem}

$\in \restric FIN$ can be made wellfounded at no cost, so the only hard part
is getting an infinite extensional subset of $FIN$.  If we can't do
that, then every infinite set $X$ of finite sets contains $x$ and $y$
s.t. $x \cap X = y \cap X$. Can we do anything with this?




%Maurice says: $\iota``FIN$ is extensional and infinite but that doesn't seem to help.  What has gone wrong?  What's gone wrong is that Maurice has made a mistake: it's not  extensional

Here's an idea i had while invigilating one day.

Suppose no permutation model contains an infinite wellfounded set.  Let's derive something nasty from it.

$$\Box\forall x((\forall y)(\pow y) \subseteq y \to x \subseteq y) \to Fin(x))$$


$$\forall \sigma\forall x((\forall y)(\pow \sigma``y)) \subseteq y \to x \subseteq y) \to Fin(x))$$




$$(\forall x)(\ (\forall \sigma)(\forall y)(\pow \sigma``y)) \subseteq y \to x \subseteq y) \to Fin(x))$$


$$(\forall x)(\ (\exists \sigma)(\forall y)(\pow \sigma``y)) \subseteq y \to x \subseteq y) \to Fin(x))$$

Snooze \ldots does this become



$$(\forall \alpha)(\ (\exists \sigma)(\forall y)(\pow \sigma``y)) \subseteq y \to \alpha \leq |y|) \to \alpha \in \Nn)$$


Idea: show that every singleton is $\{\Lambda\}$ in some permutation
model.  This is easy.  $\{x\}$ is $\{\Lambda\}$ in $V^{(x,\Lambda)}$.
Then show that every pair is wellfounded in $V^\sigma$ for some
$\sigma$, and so on.



What we now have to do is find something slightly stronger than
`$(\exists \sigma)(\forall y)(\pow \sigma``y) \subseteq y \to x
\subseteq y)$' that might enable us to deduce ``any set that embeds into all
fat sets is finite''?  Either there are arbitrarily large cardinals of
such finite sets in which case $\aleph_0$ is such a cardinal (any fat
set that is not inductive is dedekind-infinite) which contradicts
hypothesis, or there aren't.  If not, there finite sets too big to
embed in a fat set.  But then there are smaller fat sets.  But this
was any old base model.  So we would have established that if no
permutation model contains an infinite wellfounded set, every
permutation model contains a fat finite set.  The consequent sounds
improbable!!!!


Mind you, i'd've tho'rt that 
$(\exists \sigma)(\forall y)(\pow \sigma``y) \subseteq y \to x \subseteq y)$
is strictly stronger than $x$ embedding into all fat sets.

Randall sez:

Wed Sep 23 11:42:23 1998


Note on permutation idea

Aim is to make an infinite well-founded set.  The idea is to ensure that
all sets which include their power sets include \Nn.

The strategy is to permute in such a way that any set which excludes 
part of \Nn\ has a subset mapped outside itself.  We postulate further that
this subset will be a subset of N (necessarily a proper subset).  We
postulate further still that this subset of \Nn\ will be mapped to an element
of \Nn.

So the situation we envisage in one in which each proper subset of \Nn\
(the part of \Nn\ included in a set missing part of \Nn) itself has a subset
which is assigned by the permutation as the extension of a natural number
not in the original proper subset.

For each proper subset $A$ of \Nn\ there is a subset $B$ of $A$ such
that $\pi`n = B$ for some natural number $n$ in $\Nn\setminus A$.

It is not clear that this is possible, but it is also not clear that it
is impossible.




\subsection{}

How about this for a clever idea.  We need to think about permutation
models with more wellfounded sets than $V^\alpha$, yes?  Now say


Try $\sigma \leq \tau$ iff

$(\forall Y)((\pow Y) \subseteq Y)^\tau \to (\exists X)((\pow X) \subseteq X)^\sigma \wedge \sigma`X \subseteq \tau Y))$

which simplifies to

$$(\forall Y)((\pow Y) \subseteq \tau``Y) \to (\exists X)((\pow X) \subseteq\sigma``X)\wedge X\subseteq Y))$$

 
Now some questions about $\leq$.\begin{itemize} 

\item Is $\leq_w$ wellfounded?  It certainly should be!

\item Is it connected? This is really a question about how ragged $\WF$ can
be.  If it can't be ragged then $\leq$ might be connected.

\item Is it invariant (in the sense that closed formul{\ae} containing only 
$\leq_w$ and = are invariant? (need a word for this!)

\item Can we show that $\leq$ is not the universal relation?  Slightly more likely is
the assertion that $\phi^{WF}$ is invariant for all $\phi$.  How about $\phi^{H_{fin}}$
being invariant?  That ought to be easy!

\end{itemize}


Let's have a look at this last one. We might be able to do something if
$\phi$ is stratified.

Expand $\phi^{WF}$.  The variables in it have types.  The quantifiers become things like
$(Qx) ((\forall X)(\pow X)) \subseteq X \to x \in X) \to \ldots)^\sigma$

which is

$(Qx) ((\forall X)(\pow \sigma_n `X)) \subseteq \sigma_{n+1}`X \to \sigma_{n-1}`x \in \sigma_n `X) \to \ldots)$

relettering we get:

$(Qx)((\forall X)(\pow X)) \subseteq (j^{n+1}\sigma)`X \to \sigma_{n-1}`x \in X) \to \ldots)$

and the trouble now is that saying that something at type $n$ is
wellfounded is different from saying that something at type $n+1$ is
wellfounded, unless there is something really clever we can do.


The obvious way to show that $WF^\sigma$ and $WF^\pi$ are elementarily
equivalent w.r.t. stratified expressions is to show that they are
stratimorphic.  For them to be stratimorphic one would expect there to
be a permutation of $V$ that maps one onto the other.  Such a
permutation one would expect to be definable in terms of $\sigma$ and
$\pi$ and there isn't anything obvious.

The same difficulty occurs when trying to show $\phi^{H_{fin}}$ invariant.
This suggests that even the stratified theory of hereditarily finite sets
isn't clean!  Clearly there is a gap here.  I can imagine no way of showing
that the stratified theory of hereditarily finite sets is invariant and no
way of showing that it isn't.



One might think it would be worth trying

$\pi \leq \sigma$ iff $\exists$ partial injection $f:V \rightharpoondown V$ such that 
$f`x = \sigma^{-1}`\{f`z: z \in_\pi x\}$, which is to say

$\pi \leq \sigma \bic (\exists f:V \rightharpoondown V)(\sigma^{-1} (j`f)\pi \subseteq f)$.
That is to say $\sigma \leq \tau$ iff $(\exists
h:V \rightharpoondown V)(\tau^{-1}\circ(j`h)\circ\sigma \subseteq h$.

\ldots which looks nice because these $h$'s look a bit like germs.
The trouble with this is that it is symmetrical!  We should also
presumably have the banally obvious: $(\exists h:V \rightharpoondown
V)$ s.t. if $V^\sigma$ thinks $x$ is wellfounded then $h`x$ is defined
and believed by $V^\tau$ to be wellfounded.



 
\subsection*{}

Let $\Gamma$ be the set of sentences in the language of arithmetic-with-$T$
which become truths of arithmetic when one erases `$T$' from them.

\begin{enumerate}
\item Are the $f \in \Nn^{\Nn}$ that commute with $T$ cofinal in the partial 
      order under dominance?
      
    \item \AxC$\to (\forall \alpha < \omega_1)(\alpha \leq T\alpha)$?
      Is this perhaps related to the question of whether or not there
      is an assignment of fundamental sequences to ctbl limit ordinals
      that commutes with $T$. But this formula is in $\Gamma$.  I have
      the very strong feeling that no form of AC (eg fns assigning
      fundamental sequences) will help prove them.

       
     \item If $\Phi$ is a sentence in arithmetic-with-$T$ that is true
       of the identity but not provable in arithmetic-with-$T$ is
       there an Ehrenfeucht-Mostowski model in which it fails?

\item Andr\'e's question.  
      $(\exists n \in \Nn)(n \not = Tn \wedge (\forall m < n)(m \leq Tm))$ 
      
\item Does ``$\in$ restricted to FIN is well founded" imply
     ``$(\exists f \in \Nn^{\Nn})(\forall n \in \Nn)(f(Tn)>n))$,
      namely the existence of a K\"orner function?  Is the existence
      of K\"orner functions provable in \nf\ already? It's certainly
      invariant.

\item Is it consistent with \nf\  that there should be a $f: T``NO \to NO$
s.t. $(\forall \alpha)(f`\alpha \geq T^{-1} \alpha)$?

\end{enumerate}


If there is $f: T``NO \to NO$ s.t. $(\forall \alpha)(f(T \alpha) \geq
\alpha)$---namely a K\"orner function on the ordinals---then surely
something interesting must happen.  For suppose $\alpha$ is an ordinal
to which $f$ can be applied as often as we like, and think about
$F(\alpha) := sup A$ where $A := \{f^n(\alpha): n \in \Nn\}$.  We have
$f^n(\alpha) \geq T^{-1}f^{n-1}(\alpha)$ for each $n$, so everything
in $T^{-1}``A$ is $\leq$ something in $A$ so $F(\alpha) \leq
T(F(\alpha))$.  Without extra assumptions we're not going to be able
to do much: commutes with $T$, monotone, nondecreasing....  Let's face
it, might there not be such a function with no assumptions at all:
$\lambda \alpha. \Omega + \alpha$?  No---it doesn't work for $\alpha >
\Omega$.

Suppose there is a K\"orner function on the ordinals: $f: T``NO \to
NO$ s.t. $(\forall \alpha)(f(T \alpha) \geq \alpha)$.  Then we can
make it continuous by filling it in, and we can make it nondecreasing,
by recursively setting $g(\alpha) :=$ max$(f(\alpha), \sum\limits_{\beta <
  \alpha}^{} g(\beta))$.  Or at least that's what i tho'rt.  The trouble
is, that sup might not always be defined, as $\{g(\beta): \beta <
\alpha\}$ might be cofinal in the ordinals.  So we have to use my
version of Erd\"os-Rado.  Two-colour the complete graph on $NO$
according to whether or not $(\alpha < \beta) \bic (f(\alpha) <
f(\beta))$.  There will be a monochromatic set of size $T^3|NO|$ and on it $f$
must be monotone.  Then things get a bit tricky, but the idea is to
pull the map back via $T$.  No, not even that will work, as there is
no guarantee that enough of the image under $f$ of the monochromatic set is in
the range of $T$.  We have to do various man{\oe}uvres like work not
on $f$ but on $f|\{\alpha: \alpha > \Omega\} \cup f^T$.


6 is obviously something to do with $\in$ restricted to something quite
being wellfounded: set $\Omega_x = \bigvee_{On}\{\alpha +1: \alpha$ is a
second component of an ordered pair in $x\}$.  That sort of thing.

It implies that $cf(\Omega)$ is cantorian.  Is the converse true?

Is there any connection between 4 and the cofinality of $\Omega$?  (Must
check that if there is such a function there is one that is continuous)
Suppose there is a map $h$ from the ordinals below $\alpha$ cofinally into
$T``NO$.  Suppose also that $\alpha = T\alpha$.  Then there is also a map
$g$ from the ordinals below $\alpha$ cofinally into $NO$.  We prove by
induction that $(\forall \beta < \alpha)(Tg`\beta = h`T\beta)$ We now
define an $f$ as in 4 as follows.  First we define it on things in the
range of $h$.  For $\beta < \alpha$ set $f$ of $h`\beta$ to be $g`\beta$.




Whatever happened to the idea that every assertion of cardinal arithmetic
is $\Box$ some assertion about sets?  For example, \AxC\ is equivalent to
\begin{quote}$\Box$(if $x$ is a finite set then $x$ is not a proper subset of
$\iota``x$).\end{quote}

I think that in the \nf\ case it is complicated by the fact that
equinumerosity and 1-equivalence are not the same, but in \zf\ they
are.  Obviously their negations are $\poss$ of some piece of
combinatorics.

Andr\'e once cheered me greatly by saying that it was very significant that
cardinal arithmetic is invariant.  I'd always tho'rt i was the only
solipsist.  It now occurs to me that the fact that it has an implementation
that is invariant must be something to do with the fact that it is a theory
of virtual entities. 



%\chapter{The undecidability of intuitionistic monadic predicate logic as proved by Saul Kripke and presented by Thomas Forster on the basis of a recollection of conversations with Martin Hyland and Max Cresswell}

%\input monadic
\chapter{Typed lambda calculus}

\subsection*{Lambda calculus in NF}

It is generally assumed that NF gives rise to a natural model of the Lambda
calculus, not least beco's it's pretty plausible that $V$ and $V \to V$ are
the same size.  This is true, and a little bit fiddlier than one would
like, so i provide a proof to save others the trouble.


For a first try, let's send each set $x$ to that function which sends
everything in $x$ to itself, and everything else to $\Lambda$, the
empty set.

\begin{quote} {\bf F}: Input $x$, output $\lambda y.$\verb#if # $y \in x$\verb# then #$y$\verb# else #$\Lambda$. \end{quote}

We can recover $x$ from {\bf F}$`x$ as long as $\Lambda \not\in x$ and 
$V \setminus x$ has at least two elements.  In particular, if there are 
$|V|$ things that are of size $|V|$ (not containing $\Lambda$) and whose
complements are of size $|V|$, then we're OK.

If $\Lambda \not\in x$ then $x \times V$ satisfies this.  There are
$|V|$ things not containing $\Lambda$, so there are, in fact,
$|V|$ things that are of size $|V|$ (not containing $\Lambda$)
and whose complements are of size $|V|$ as desired.

{\bf F} restricted to these things is 1-1 ({\bf F}$^{-1}(x) =$ {\bf F}$``x
\setminus \{\Lambda\}$) so there are $|V|$ functions from $V$ into
$V$.

The trouble now is that the $K$ combinator cannot be a set. If it were,
then $\lambda x.(V \times \{x\})$ would be a set and so would $\iota$.
Presumably $S$ can't be a set either, tho' i can't see such a cute proof
offhand.  The only combinators that ought to be sets in \nf\ are those of
(polymorphic) type $\alpha \to \alpha$.

\section{How easy is it to interpret typed set theory in the typed $\lambda$-calculus?}

This is related to the question of ascertaining the relative strengths of
things like \verb#HOL# and simply typed set theory.

If we can decide on elements $0$ and $1$ at each type, then we can
regard a model of this $\lambda$-calculus as an extension of a model
of \TZT: we restrict attention to the hereditarily two-valued functions. 
(A hereditarily 2-valued function is an $f$ such that range$(f) =
\{0,1\}$ and $\forall y. f`y = 1 \to$ $y$ hereditarily 2-valued.)
This is o.k. when we have atomic types, for we can take 0 and 1 at the
atomic types to be whatever they are, and then procede to define
$1_{\beta \to \alpha}$ and $0_{\beta \to \alpha}$ by recursion.  This
is slightly more delicate than one might think, since we want $1_\alpha$ 
and $0_\alpha$ to be hereditarily two-valued. The definition of
$0_{\beta \to \alpha}$ as $\lambda x_\beta.0_{\alpha}$ is perfectly
satisfactory but $1_{\beta \to \alpha}$ has to be $\lambda x_\beta.$
hereditarily-two-valued$(x) \to 1_\alpha \restric 0_\alpha$.


It is not clear how to do this when there are no atomic types to start
the recursion, but a compactness argument will probably save the day.

To complete the interpretation we will need to have, for each type
$\alpha \to \beta$, and each type $\gamma$, a $\lambda$-term
$F_{\alpha \beta\gamma}$: $(\alpha \to \beta) \to \gamma$ such that,
for any $t: \alpha \to \beta$, $Ft = 1_\gamma$ iff $t$ is hereditarily
two-valued and = $0_\gamma$ otherwise. Presumably this can be done but
not uniformly.

%\chapter{Constructive NF} \input rna

\chapter{Miscellaneous Cardinal Arithmetic}

\subsection{Some factoids useful in connection with \TZT\ and Bowler-Forster}

For the moment let the variable `$\kappa$' range over alephs.  Then we
can prove things like: a union of $\kappa$-many $\kappa$-sized sets
cannot be of cardinality $\geq \kappa^{++}$; a union of $\kappa$-many
$< \kappa$-sized sets cannot be of cardinality $\geq \kappa^{+}$.  In
the study of \TZT\ we sometimes need arguments that rely on facts like
these.  Can a union of $T|V|$ many sets each of size $T|V|$ be of size
$|V|$?  Well, yes---obviously.  But how about a union of $T^2|V|$ many
sets each of size $T^2|V|$?  We need to worry about things like that.

It turns out that the old methods work quite well.  Drop the
assumption that $\kappa$ is an aleph, but assume $\kappa^2 = \kappa$.
This is actually reasonable, because it holds of $|V|$.

Here's a taster.  Suppose $\kappa^2 = \kappa$.  Then a union of
$\kappa$-many things of size $\kappa$ cannot be of size $2^{2^\kappa}$.  
Here's why.  By Sierpinski-Hartogs any set ${\cal P}^2(K)$ of size
$2^{2^\kappa}$ has a subset of size $(\aleph(\kappa))^{++}$.  (Well,
you might need an extra exponent but you get the idea) The subset is a
union of $\leq \aleph(\kappa)$ sets each of size $\leq \aleph(\kappa)$
and therefore cannot be of size $(\aleph(\kappa))^{++}$ after all.

We need to push the boat out a bit. Suppose $\kappa^2 = \kappa$ as
before.  Then can a union of $\leq \kappa$-many things of size $\leq
\kappa$ be of size $2^{2^\kappa}$?  Presumably not, and by the same
proof.  But then can a union of $\leq^* \kappa$-many things of size
$\leq^* \kappa$ be of size $2^{2^\kappa}$?  We may well need results
like that and they may well be much harder to obtain.

Perhaps we should spare some thought for the parenthetical remark a
couple of paragraphs ago.  I write there as if $(\aleph(\kappa))^{++}
\leq 2^{2^\kappa}$ as long as $\kappa = \kappa^2$.  But that's not
secure.  Let $K$ be a set of size $\kappa$. Send every prewellordering
of a subset of $K$ to its length and send everything not a
prewellordering to $0$.  This maps ${\cal P}(K \times K)$ onto a set
of size $\aleph^*(K)$ and---since $\kappa^2 = \kappa$---we get
$\aleph^*(\kappa) \leq^* 2^\kappa$.  Analogously we of course also get
$\aleph^*(2^\kappa) \leq^* 2^{2^\kappa}$.  So certainly
$(\aleph^*(\kappa))^+ \leq \aleph^*(2^\kappa) \leq^* 2^{2^\kappa}$.
It doesn't seem to want to come out at the moment.  It probably
doesn't matter (for \TZT\ applications at any rate) having to have another layer
of exponentiation.


  
Here's why.  By Sierpinski-Hartogs any set ${\cal P}^2(K)$ of size
$2^{2^\kappa}$ has a subset of size $(\aleph(\kappa))^{++}$.  (Well,
you might need an extra exponent but you get the idea) The subset is a
union of $\leq \aleph(\kappa)$ sets each of size $\leq \aleph(\kappa)$
and therefore cannot be of size $(\aleph(\kappa))^{++}$ after all.

\subsection*{Other Stuff to fit in}

Cardinality of $\Sigma_V$?

While thinking about the question of whether or not \AxC implied that
$(\forall \alpha \leq \omega_1)(\alpha \leq T\alpha)$ I found that
this would follow from the assumption that every ordinal (in
$T^2``NO$) contains a wellordering that commutes with $T$.  This
should make us think of the term model, because although there are
clearly definable functions (definable as stratified set abstracts)
$NO \to NO$ which do not commute with $T$ (send $\alpha$ to 0 if
$T^{-1}\alpha$ is not defined and to 1 o/w) something along the lines:
every definable wellordering of ordinals commutes with $T$. But isn't
$(\forall \alpha \in NO)(\alpha \geq T\alpha)$ strong?  This {\sl
might} enable us to show that \nf\ has no term models.

\section{Wellfounded extensional relations}

I broadcast a message yesterday which got lost.  That was probably just as
well for in the intervening 24 hours i have had time to collect my thoughts
on this subject and give a better summary.  I owe thanks to Randall Holmes
and Bob Solovay for pushing me in the right direction.

    Randall has been thinking for some time about whether or not $\pow NO)$
can be wellordered.  Since if we can wellorder $\pow NO)$ we can wellorder
the power set of any wellordered set, this reminded me that there is an old
theorem of Rubin's that if the power set of a wellordered set is always
wellordered, then every wellfounded set is wellordered.  Since one of my
current preoccupations is the theory of wellfounded sets in NF (I
conjecture that it is precisely KF) I was intrigued! However the induction
Rubin used is highly unstratified and there seems no hope at all of
reproducing it in NF.  Something Rob Solovay said made it clear to me that
the correct thing to do with Rubin's proof is to use it to prove something
about domains of wellfounded extensional relations rather than about
wellfounded sets.  This i do below.  

There is an old problem of Hinnion's: in his thesis he did a lot of work on
relational types of wellfounded extensional relations and asked whether one
could show that there is no wellfounded extensional relation on V.  I know
of no progress with this problem.  However, what we can now say (if i've
got this right - and i am not staking my life on the correctness of this
broadcast!) then if $\pow NO)$ can be wellordered there is no wellfounded
extensional relation on $V$.  Since Holmes has recently pointed out that if
$\pow NO)$ cannot be wellordered there is a cantorian wellordered set whose
power set is not wellorderable something interesting is doomed to come out
of this one way or another.  To prove Holmes' result, simply consider the
least aleph $\alpha$ such that $2^\alpha$ is not an aleph.  If $\alpha$ is
such an aleph so is $T\alpha$, so $\alpha \leq T\alpha$.  Therefore
$T^{-1}\alpha$ is defined and is another such aleph, so $\alpha \leq
T^{-1}\alpha$ whence $\alpha = T\alpha$.   Finally Richard Kay remarked to me
some time ago that there seems a natural way in which models of NF with
wellfounded extensional relations on $V$ might arise, and i append his
message on the end of this broadcast with his permission.

I shall prove the following
\begin{thm}
\tfae \begin{enumerate}
\item [1] $\pow NO)$ is wellorderable
\item [2] The power set of a wellordered set can be wellordered
\item [3] The domain of a wellfounded extensional relation is wellorderable
\item [4] $|\pow NO)| < |NO|$
\end{enumerate}
\end{thm}

\Proof

1 $\to$ 2 is fairly easy.  Let $X$ be an arbitrary wellordered set.  The
$\iota``X$ is the same size as some subset of $NO$ and therefore its power
set is wellordered.  4 comes into the picture because it is a theorem of
Henson's that $|NO| \not\leq |\pow NO)|$.

To prove 3 $\to$ 1 notice that we can define a wellfounded extensional
relation on $\pow NO)$.  For starters, we can define a relation $E$ on
subsets of $NO$ that are not initial segments by setting $\{\alpha\} E X$ iff
$\alpha \in X$ (so that the only things that $E$ anything are singletons)
and distinguishing between singletons by saying $\alpha E \beta$ iff
$\alpha < \beta$.  Now a simple application of Bernstein's lemma shows that
$NO$ has as many subsets that aren't initial segments as it has subsets,
and we use the bijection to pull back the relation to the whole of $\pow
NO)$.

To prove 2 $\to$ 3 we need the induction in Rubin.  This is lifted
wholesale from Rubin (or rather the version of it in Jech {\sl The Axiom of
Choice}) the only difference being that here it is cast in the more general
setting of an arbitrary wellfounded extensional relation. It seems highly
unlikely that one could prove it over $\in$ in \nf, since the induction is
unstratified and $\in$ is not a set. 

Assume 2.  Let $R$ be a wellfounded extensional relation with domain
$X$.  We will show that $X$ can be wellordered.  Without serious loss
of generality we can assume that the rank of $R$ is reasonably small,
by considering $RUSC^n(R)$ for $n$ sufficiently large (3 will be large
enough) because $X$ can be wellordered iff $\iota^n``X$ can be.

    To each member $x$ of $X$ we can associate the rank of $*R^{-1}``\{x\}$.
Call this the {\bf rank} of $x$. Let $N_\alpha$ be the set of things of rank
$\leq \alpha$.  We will need to know that there is an ordinal too big to be
the rank of any element of $X$. (This is the reason for reasoning with
$RUSC^3(R)$ instead of $R$, just to be on the safe side).  Let $K$ be some
set of size $\aleph`|x|$, and fix $\leq_K$ and $\leq_{PK}$
wellorderings of $K$ and $\pow K)$ respectively.  We are going to show that
there is a canonical injection $$i_\alpha: N_\alpha \inj K$$ where the 
range of $i_\alpha$ is an initial segment of $K$ in the sense of $\leq_K$.

For $\alpha = 0$ it is easy.  For the induction step from $\alpha$ to
$\alpha + 1$ notice that $i_\alpha$ lifts to 

$$j`(i_\alpha): \pow N_\alpha) \inj \pow K)$$

Since $R$ is extensional there is a canonical map $\iota``N_{\alpha + 1} \inj
\pow N_{\alpha+1})$ so we compose the two to get a map $\iota``N_{\alpha+1} 
\inj \pow K)$.  Since $\pow K)$ is wellordered by $\leq_{PK}$ this gives us a
(canonical) wellordering of $N_{\alpha + 1}$.  Now compare this
wellordering of $N_{\alpha + 1}$ with $\tuple{K, \leq_K}$.  Remember
that $K$ has been chosen so that it has a wellordering $\leq_K$ too
long to be isomorphic to any wellordering of any subset of $X$.
Therefore there is a (canonical) injection $N_{\alpha + 1} \inj K$
obtained by the recursive construction of the canonical map between
two wellorderings.  

This is not the end of the story, because we want to ensure that the
various $i_\alpha$ agree on their intersections, so that we can take
sums at limits.  Therefore we have to ensure that everything in
$N_{\alpha + 1}$ goes after everything in $N_\alpha$.  So, given our
injection from $N_{\alpha + 1}$ into $K$, use it to order things in
$N_{\alpha + 1} \setminus N_\alpha$ (by pulling back $\leq_K$) and map them to
the terminal segment of $\tuple{K,\leq_K}$ consisting of things not in
the range of $i_\alpha$.

The case where $\alpha$ is a limit is easy as long as each $i_\alpha$
is an end-extension of all earlier $i_\beta$, and we have arranged for
this by construction.

This shows that $N_\alpha$ is wellordered for all $\alpha$.  Since there
is some ordinal too big to be the rank of any member of $X$, (call it $\gamma$)
we know that $N_\gamma$ must be the whole of $X$.  Therefore $X$ is wellordered.

\endproof

A metamathematical remark.  Many people find it difficult, on being
told Rubin's result, to reconstruct the proof.  If you are told it
relies on foundation, you try to prove by induction on $\in$ that
every wellfounded set is wellordered. {\bf But is isn't a proof by
induction on $\in$, it's a proof by induction on rank.}


Don't forget that Henson proved that $|NO| \not\leq |\pow NO)|$.


\begin{coroll}
Either $\pow NO)$ is wellordered, in which case there is no bfext on $V$ or 
it isn't, in which case there is $\aleph =T\aleph$ s.t. $2^\aleph$ isn't an aleph

So if there is a wellfounded extensional relation on $V$ there is a bad aleph.
\end{coroll}

Can we find a proof that is a bit more effective?  This one uses cut (the
cut formula is `$|\pow NO)| \in WC$').

Consider the minimal rank of wellfounded relations on $x$.

We need a notion of relative jaggedness of a wellfounded relation. We have a
notion of {\sl hole}, and of {\sl rank} of holes.  We can make a relation less
jagged by chipping off some elements that do not bear $R$ to anything, and
putting them in holes. We say $R < S$ if some of the holes in $R$ are filled
in $S$, and any of the holes in $S$ that are not holes in $R$ are of higher
rank than those in $R$ but not $S$. It should not be too hard to show that
every descending chain under $<$ has a lower bound.  Any minimal element is
something very like a $V_{\alpha}$.  We can consider a version of $<$ on
isomorphism types.

Suppose every set has a wellfounded extensional relation on it. Does this
follow from the assertion that $V$ has a wellfounded extensional relation 
on it?  In either case consider the least ordinal $\alpha$ s.t. $\exists$
a wellfounded extensional relation on $V$.  Should be easy to show $\alpha
\leq T\alpha + 1$


Suppose there is a wellfounded extensional relation on $x$. Then there is
also one on $\iota ``x$. How about $\pow x)$? Some of the holes we would
want to fill up with elements of $\pow x)$ are already occupied, so we
can only accomodate $2^{T|x|} - T|x|$. But this is likely to
be at least $2^{T|x|}$, at least if $2.|x| = |x|$.
                     

Existence of wellfounded extensional relations on $V$ generalises upward in
models of \TZT, and is $\SiP_1$.

\subsection{Inhomogeneous wellfounded extensional relations on $V$}

Given a set $X$ we say that a relation $R
\subseteq \iota``X \times X$ such that if $x_1 \not= x_2$ both in $X$ then
there is a singleton $R$-related to one but not the other is  {\bf skew-extensional}.  

If $X$ admits such a relation then there is a map $f:X \inj \pow X)$ defined by
$\lambda x_X. \bigcup\{z: z R x\}$.  Since not all sets can be embedded
into their power sets, this is nontrivial.  The corresponding move with
bfexts does nothing.

Say $R \subseteq \iota``X \times X$ is {\bf skew-wellfounded} iff
$(\forall X'\subseteq X)(\exists x \in X')(\forall y \in
X')(\neg(\{y\} R x))$.

  We shall say that $R$ is a skew-extensional skew-wellfounded
relation {\bf on} $X$ if its range is $X$, and let us call these relations `Kbfext's.


  Naturally the existence of Kbfexts is related to the existence of
transitive wellfounded sets.  For example, $V$ is the same size as a
transitive welllfounded set iff $\poss$ there is a kbfext on $V$.

We'd better have a proof of this.

If $b: V \to X$ is a bijection between $V$ and a transitive
wellfounded set $X$, 
\Wlog\ $X$ is a power set.  Now $\{\tuple{\{x\}, y}: \b(x) \in \b(y)\}$ 
is skew-extensional and skew-wellfounded.




Conversely, if $R \subseteq \iota``V \times V$ codes a Kbfext, and $f$
is a bijection between $V$ and a moiety, then $\{\tuple{\{f`x\}, f`y}:
\tuple{\{x\},y} \in R\}$ codes a Kbfext on a moiety.

If $R$ is a Kbfext on a moiety $X$, let $\pi$ be a permutation of $V$
extending the map $\lambda x, \bigcup R^{-1}``\{x\}$. Then in $V^\pi$
$\pi^{-1}`X$ has become a transitive wellfounded set the same size as
the universe.


It seems so extraordinarily unlikely that $V$ should even be the same
size as a wellfounded set, let alone a {\sl transitive} wellfounded
set, that i've never taken much interest in Kbfexts on $V$.


Now i claim the following.  $\poss \exists H_{\aleph_0}$ iff there is
a skew-wellfounded skew-extensional structure satisfying the obvious.
And generalisations of this are true.

For suppose

$V^\pi \models \exists x \forall y(y \in x \bic (\forall z)(\powk{\aleph_0} z) \subseteq z \to y\in x)$
this is 

$\exists x \forall y(y \in x \bic (\forall z)(\powk{\aleph_0} z) \subseteq \pi``z \to y\in x))$

Fix {\bf a} a witness to this. We then prove that {\bf a} with the relation $x R y$ if
$x \in \pi`y$ is a skewthingie.   The way to do this is to consider

$Z = \{y \in${\bf a}$: (\forall w \subseteq${\bf a}$)(y \in w \to (\exists
x \in w)(\forall u \in w)(\neg(u \in \pi`x)))\}$.  It is easy to check that
$Z$ is a $z$ such that $\powk{\aleph_0} z) \subseteq \pi``z$ and therefore
contains everything in {\bf a}.  All we have to do is verify that if $v \subseteq Z$ is finite
then it is $\pi$ of something in $Z$.

The other direction is easy.  Suppose $\tuple{X,R}$ is a skewthingie.  \Wlog\ we can assume $X \cap \pow X)$ is empty, so that the product of
transpositions

$$\prod_{x \in X}(x, \{y: \{y\}Rx\})$$ is well-defined.  That does it.

Suppose we have a set $X$ and a map $i$ that accepts small subsets of $X$
and returns members of $X$. Suppose further that the relation $x R y$ iff
$(\exists X' \subseteq X)(x \in X' \wedge f`X' = y)$ is wellfounded. \Wlog\
we can assume that all members of $X$ are the size of the universe.

Then consider the product $\pi$ of transpositions $(x, \tuple{x, i`(X \cap
\snd``x)})$ over all sets $x$ with the property that all partitions of $x$
are small.  Notice that if $x$ is small $\pi`x$ isn't.

Notice that if $n$ is a K\"orner number we can take $X$ to be \Nn\ and $i`x
= Tsup(x) + 1$.  

In $V^\pi$ membership restricted to sets all of whose partitions are small is wellfounded.
(Write this out)

Now is it possible to have such an $X$ where ``small" means ``cannot be
mapped onto $V$?

\subsection{A message from Richard Kaye}
\label{A message from Richard Kaye}

If $x$ is a transitive set in a model $\M$ of ZF (say), $J$ is an automorphism
of $\M$ and $f \in \M$ is a bijection from $y = J(x)$ to $\pow x)$.  Then
$\{ u \in \M: \M\models u\in x \}$ is the domain of a model of \nf, the
epsilon being $u\in_{new} v$ iff $u \in fJ(v)$.  This much is standard.

The point is, since $\bigcup x \subseteq x$,

         $$R = \{\tuple{u,v} : u \in v \in x \} \subset {\cal P}^n`x$$ 

is a relation on the universe, actually a set (or rather, you probably want
$(fJ)^{-n}(R)$ for some suitable $n$) and is wellfounded (but certainly won't
be the new $\in$ relation ).  There is some minor trouble in checking that
this set really is a well-founded relation in the sense of the new model,
but this shouldn't be too bad, as it is certainly wf in the original.
It doesn't seem to contradict anything particular, so one might think 
that if models of \nf\  exist at all, they might arise in this way.  Incidently
models of $NFU$ like this do exist.  That's why it occurred to me.

I think I need the original model to satisfy rather more than KF.  Foundation
is obviously necessary.  Perhaps this is enough.  I'm not sure exactly
what you've written,  (i.e. what base theory is implied) so maybe you should
check this point.  Otherwise it sounds OK.

       Best wishes,        Richard

I'm pretty sure it should be $R = \{\tuple{\{u\},v} : u \in v \in x \}$.


Consider also the situation (which admittedly seems rather unlikely)
of a transitive wellfounded set $X$ the same size as its power set,
with some bijection $\pi$. This of course gives us a model of \nf.
Now consider the fate of the set $\{\tuple{\iota`x, y}:x \in y \in
X\}$ which is going to be a set of the new model, $Y$, say. Clearly
the relation $\tuple{\iota`x,y} \in Y$ is going to be wellfounded.
However it doesn't give rise to a wellfounded extensional relation on
the new universe because it isn't homogeneous, and so (and here we
return to the metamathematical remark) it doesn't enable us to carry
out Rubin's proof beco's Rubin's proof is an induction on rank not on
the wellfounded relation itself. A pity, really.

However there is an old observation (i think it is in the yellow book)
that if there is a definable wellfounded extensional relation on $V$
then there is no nontrivial automorphism of $\tuple{V, \in}$.  This
works even if the definable wellfounded extensional relation is not
homogeneous.  Therefore, if there is a Kaye model, it has an element
that is moved by all automorphisms.



\subsubsection{Wellfounded sets all over the place!}

Remarks on wellfounded sets are scattered all over the place! Here is
another one to go somewhere one day.

\begin{rem}
We cannot prove that if $\aleph_0$ contains a wellfounded set 
then so does every other aleph.
\end{rem}

\Proof

Suppose we could prove that if $\aleph_0$ contains a wellfounded set then
so does every other aleph.  Then we could prove $\Box$(if $\aleph_0$
contains a wellfounded set then so does every other aleph), and therefore
if $\Box$($\aleph_0$ contains a wellfounded set) $\to$ $\Box$(every aleph
contains a wellfounded set).  Now it is easy to arrange for a permutation
model with an $x \in T^2|V|$ extending its own power set, which makes
the consequent false, so the antecedent would be refutable in \nf, which
seems rather unlikely. \endproof

There are other observations of this kind.

We can prove that every concrete natural contains a wellfounded set.  We
know (because Hinnion has done it) that we can at least prove in \nf\ (as
opposed to \nf+\AxC) that $\poss$(every strongly cantorian natural contains
a wellfounded set).  Can we prove that every strongly cantorian natural
contains a wellfounded set?  If \nf + \AxC + $\neg$ AxCount is consistent
then there are models of \nf\ with finite noncantorian wellfounded sets

\section{Does the universe have a wellordered partition into finite sets?}

Suppose it does: we hope to show that the universe is wellordered.  It is
obvious that if the universe has a wellordered partition into finite sets
then any set has a wellordered partition into finite sets.  So any ordered
set can be wellordered: consider a wellordered partition into finite
pieces, order all the pieces uniformly and the result is a wellordering.
In particular, the power set of a wellordered set is wellorderable.

So far so good. We will now use the assumption that every set has a
wellordered partition into finite sets to derive a version of the
order-extension principle (I hope!)

Let $X$ be an arbitrary set, and $\leq$ a partial order on it.  Let ${\cal
X}$ be the set of partial orderings of $X$ that refine $\leq$.  ${\cal X}$
has a wellordered partition {\bf P} into finite sets, and {\bf P} is in
fact the set of atoms of an atomic subalgebra {\bf B} of $\pow {\cal X})$.
Now {\bf B} is, up to isomorphism, the power set of {\bf P}, which is
wellordered, so {\bf B} is wellordered too.  The idea is that we can use
the fact that {\bf B} is wellordered to show that every filter in {\bf B}
can be extended to an ultrafilter in {\bf B} and then rely on the fact that
{\bf B} is nearly the same as $\pow {\cal X})$ to be able to extend any
filter $\subseteq \pow {\cal X})$ to an ultrafilter $\subseteq \pow {\cal
X})$.  Unfortunately this doesn't work. Consider the simple case where a set
$Y$ has a countable partition into pairs, and $\Re$ is wordered. Then there
is an ultrafilter on the index set (\Nn) but not---or not obviously---on
$Y$. No dice.


For each pair $x$, $y$ $\in X$, set $N_{\tuple{x,y}}$ be the set of
partial orders refining $\leq$ that decide whether or not $x < y$ or
$y < x$. $N_{\tuple{x,y}}$ is not in general going to be an element of
{\bf B}, but $\bigcup\{z \in$ {\bf P}$:z \cap N_{\tuple{x,y}} \not = \Lambda\}$ 
is. Let us abbreviate it to ${\cal N}_{\tuple{x,y}}$.  It is obvious
that the $N_{\tuple{x,y}}$ form a filter base in $\pow {\cal X})$, so
it follows that the ${\cal N}_{\tuple{x,y}}$ form a filter base in
{\bf B}.  Now {\bf B} can be wellordered, so we can extend this filter
base to an ultrafilter ${\cal U} \subseteq$ {\bf B}.

So this bombs out.

However, if we put a finite bound (any bound) on the size of the
pieces we get the result we need.  They don't even have to be
disjoint.  This is beco's of a result in Jaune 5 to the effect that if
$|x| = |x|^2$ and $x$ is a union of a wellordered family of
$n$-tuples then $x$ can be wellordered.  In fact a trivial reworking
of the proof in Jaune 5 allows us to weaken the hypothesis to
$|x| \geq_* |x|^2$.  If there is no finite bound on the size
of the tuples it doesn't seem to work.  All we get is that $V$ is the
union of countable many very funny much smaller sets.


%However the technique in Jaune 5 will prove the following:\begin{thm}  If NCI is finite then if $\{P_i: i \in I\}$ is a family of disjoint  finite sets of sizes bounded below $\aleph_0$ whre $I$ is totally  ordered, then $$|{\displaystyle\bigcup_{i \in I} P_i}| = |I|$$ \end{thm}

%\Proof The technique of Jaune 5 requires $|I|$ to be idemmultiple. However if NCI is finite, then every totally ordered $I$ is included in some totally ordered $I'$ s.t. $|I'|$ is idemmultiple (ust square it a few times) so we reason with $I'$ instead \endproof




Some random tho'rts.  If $V$ is the union of a wellordered set of
finite sets then the power set of a wellordered set is wellordered.
Does this show there is no last aleph and that the cofinality of
$\Omega$ is uncountable?  If $V$ is indeed a union of countably many
finite sets one can ask about the cardinality of the number of
$n$-tuples. This gives us an $\omega$-sequence of alephs, and one
should think about its sup.  Notice that a union of $\aleph$ finite
sets has no partition of size $\aleph^+$ so one should be able to do
something there \ldots

Thinking aloud.  If $V$ is the union of a wellordered family of finite
sets then the power set of every wellordered set is wellordered.  Now
let $\alpha_n$ be the sup of those alephs that are $\beth_n$ of
something.  These sets get smaller so the $\alpha$s form a
nonincreasing sequence and must be eventually constant. (We can do
this anyway but perhaps if the power set of a wellordered set is
wellordered something interesting will happen)

Let $n_0$ be the least $n$ s.t. $\alpha_m$ is constant for $m > n$.
Then every cardinal that is $\beth_{n_0}$ of something is $\leq$ a
cardinal that is $\beth_{n_0 +1}$ of something.

If $V$ is a union of a wellordered family of finite sets then we can
use the fact that $V = V \times V$ to refine the partition in various
ways until we reach a partition whose corresponding equivalence
relation is a sort of congruence relation for \verb#pair#, \verb#fst#
and \verb#snd#.  We can do things like this.  Let $<$ be the
prewellorder and $\sim$ the equivalence relation.  Let $P$ be a piece
of the partition and ordain that, for $x, y \in P$, $x <' y$ iff
$\{x\} \times P' <^+ \{y\} \times P'$ where $P'$ is the first piece of
the partition that can tell then apart.  Of course we can do
multiplication on the $L$ too.  Similarly any piece $P$ can be
prewellordered lexicographically by $<$ since every set is a pair.
When we reach a fixed point we must have that, for all pieces $P$,
\verb#fst#``$P$ is a single piece and $|$\verb#fst#``$P = |P|$---and
of course \verb#snd#``$P$, too, is a single piece and
$|$\verb#snd#``$P = |P|$.


The trouble is, I don't seem to be able to show that a fixed point for
all these refinements must be a wellorder!!!

\section{A theorem of Tarski's}

We know from this result of Tarski that every set has more wellordered
subsets than singletons.  So consider the operation that sends
$T|x|$ to $|\{y \subseteq x: {\mbox {\rm
wellorderable}}(y)\}|$.  This behaves in various ways like
exponentiation.  Can we work tricks on it like we do with ordinary
exponentiation?  First (silly) problem: how do we notate it?  Try
$wexp`\alpha$.  Perhaps there is some mileage to be got out of
considering operations $f$ which---like $wexp$ and ordinary
exponentiation---satisfy $$f(x + y) = f`x \times f`y$$ and suchlike.
Do categorists have anything to say about this?

\section{The Attic}

This is what Andrei Bovykin calls the big sets of NF.


Developments in set theory since the 1960s have shown that large
cardinal axioms (which talk about sets of high rank) can tell us
things about sets of low rank.  (This story is usually told as {\sl
  large sets giving us information about small sets} but my take is
that it is the {\sl rank} (rather than the size) that is doing the
work.  Given that large sets have to have large rank it might be
complained that I am arguing about nothing, but I shall press on).
This matters to people beyond set theory because these sets of low
rank are the sets that we use to implement mathematical objects of 
the kind that most mathematicians care about, and the information 
they give us might solve old problems about the reals and other 
similar small objects.

Illfounded sets are sets whose internal $\in$-structure is so
complicated that they have not so much {\sl high} rank as rank that
is---in Cantor's sense---absolutely infinite.  Seeing them in this
light one would expect the sets of the attic to have things to tell us
about the sets of low rank that implement reals etc, just as the sets
of high rank do.  However, things are not entirely straightforward,
since there can be sets that lack rank for silly reasons: Quine atoms
for example.  Clearly illfounded sets {\sl per se} do not necessarily
have anything to tell us about sets of low rank.  ZF + antifoundation
gives us no new stratified theorems (which is to say no new facts
about reals).  \marginpar{\say\ CO models here} If we want novel
information about sets of low rank, or about reals, then we will have
to look to illfounded sets of a kind not compatible with ZF, to wit,
the sets that NF keeps in the attic.  So: does the attic tell us
anything about arithmetic?  Well, yes: the obvious example is the
proof of the axiom of infinity!  That's not much use, beco's we knew
that already, but---by showing that the attic {\sl does} have things
to tell us---it may be a harbinger of results of the kind we seek.

But when these results start coming in, should we believe them?  In
short, do we/are-we-going-to believe that NF is consistent?  Most set
theorists would exhibit scepticism and caution in response to this
question.  There is an instructive parallel here with the early days
of large cardinal axioms.  The initial reaction to them was caution
and scepticism: for example it is clear, reading between the lines of
Keisler-Tarski, that the authors expected measureable cardinals to be
proved inconsistent.  Back in those days rumours of inconsistency
proofs received a much more attentive and respectful hearing than they
do nowadays. What has brought about the change?  Man is a sense-making
animal, as Quine says, and the mere fact that no inconsistency has
turned up in sixty years spurs us to find explanations for this
absence, and stories about cumulative hierarchies are co-opted to
provide them. It is clear how a belief that the cumulative hierarchy
can and should be extended as far as possible can explain the Mahlo
cardinals, but measurables are another matter. One cannot altogether
escape the unworthy thought that the real reason why measurables,
supercompacts etc are now accepted as part of the set-theoretic zoo is
simply that nobody has yet refuted them---so it seems reasonable to
adopt them.  To quote another American: ``so convenient a thing it is
to be a reasonable creature, for one can always find or make a reason
for that which one has a mind to do''. The moral of this {\sl null
  hypothesis} is that what goes for measurables and supercompacts and
the rest of them goes also for NF. In sixty years time, when NF has
still not been proved inconsistent, people will accept whatever
consequences NF has for wellfounded sets, just as my generation
accepted that there must be nonconstructible sets of reals, because
measureable cardinals say so.

It's worth asking why this hasn't happened {\sl already}.

My guess is that it's merely that taking a universal set on board is a
more radical departure than taking a measurable cardinal on board, or
at least is generally felt to be.


{\sl Summary: (i) Most of the mathematical entities that people care
  about can be implemented in a theory of sets of low rank; (ii)
  theories of sets of high rank tell us important things about the
  sets of low rank that perform the implementations; (iii) illfounded
  sets are like sets of high rank only more so, so they might tell us
  yet more about sets of low rank; the illfounded sets we can find in
  models of ZF-minus-foundation don't tell us anything new, but (iv)
  the sets we find in the attic of NF just might.  Certainly worth a
  rummage.}


There is a temptation to think that wellfounded sets and illfounded
sets are such different kinds of chap that there is an
interpolation-lemma argument to show that facts about the second
cannot tell you anything about the first.  However, a close inspection
reveals no lemma corresponding to the intuition.

*****************************************************************

NF knows about certain structures (Specker trees like $\tree{|V|}$))
which can be seen from outside to be illfounded, but which it can
prove to be wellfounded.  Thus any model of NF contains structures
which it steadfastly believes to wellfounded (and therefore to have a
rank) but which the outside world knows to be illfounded.  This means
that the more the model knows about the world outside it, the bigger
it believes those ranks to be.  This is a source of large ordinals.
(Might it be that all the information we get about sets of low rank
from the attic is channeled through large ordinals in this way?)

*****************************************************************

Assumptions about natural numbers tell us things about the attic:
\AxC\ implies that $\rho(\tree\kern-3pt(|V|)) > \omega$, for example.
But i don't think that's what people mean.  Here are three ways in
which we can use cardinal trees to extract information from the attic.
\begin{itemize}

\item Assume the axiom of counting.  Then there are lots of cardinals
  (whose Specker trees are) of infinite rank.  Observe that a tree
  (whose top element is) of rank $\lambda$ (where $\lambda$ is limit)
  has nodes of all ranks below $\lambda$, so there are lots of
  (cardinal) trees of rank $\omega$.  If you are a node of rank
  $\omega$ then the set of ranks of your children is an unbounded
  subset of \Nn, which is to say (in some sense) a real---definable
  with a single parameter.  Similarly if you are node of rank $\omega
  + \omega$ you have children of rank $\omega + n$ for arbitrarily
  large $n$.  Below each of these children is a node of rank $\omega$
  and of course a real as before.  So every cardinal of rank $\omega +
  \omega$ gives us a set of reals---again, definable with a single
  parameter.  Since counting (or even \AxC) tells us that there are
  lots of such cardinals inside $\tree\kern-3pt{|V|}$ we have sets of
  reals definable with parameters {\sl from the attic}.

\item Let $\kappa$ be any cardinal of infinite rank. Recall that
  $\tree\kern-3pt(\kappa)\restric_{NO} \beta$ is the tree consisting
  of those elements of $\tree\kern-3pt(\kappa)$ that are of rank at
  least $\beta$.  All these trees are wellfounded, and therefore
  support games.  So to any $\beta < \rho(\tree\kern-3pt(\kappa))$ we
  can associate \one\ or \two\ depending on who has a winning strategy
  in the game over $\tree\kern-3pt(\kappa)\restric_{NO} \beta$.  Thus
  $\kappa$ comes to define a subset of the ordinals below
  $\rho(\tree\kern-3pt(\kappa))$.

\item Every cardinal not in $SM$ corresponds to an $\omega$-sequence
  of ordinals, as follows.  $\alpha \mapsto (\lambda n \in
  \Nn)(\rho(\beth_n(\alpha)))$.  But there are other tricks we can do.
  $\tree\kern-3pt \alpha$ is a wellfounded tree and gives rise to a
  determinate game.  (``pick a logarithm-to-base-2 and lose if you
  can't!'').  For ordinals below $\rho(\alpha)$ we can do the
  following recursive construction.  $[\tree\kern-3pt \alpha]_0 :=
  \tree\kern-3pt \alpha$; thereafter remove endpoints at successor
  stages and take intersections at limits.  Each tree $[\tree\kern-3pt
  \alpha]_\zeta$ is either a Win for \one\ or for \two, so $\alpha$
  gives us a sequence of length $\rho(\alpha)$ of \one's and \two's.

There is a relation between the sequence for $\alpha$ and that for
$2^\alpha$.  If we let $((\alpha,\zeta))$ be \one\ or \two\ depending
on where the result of removing from $\tree\kern-3pt \alpha$ all cardinals of
rank less than $\zeta$ is a win for \one\ or for \two, then
$((\alpha,\zeta)) = \two \to ((2^\alpha, \zeta))= \one$.

In general, how much information about a tree can one code by this
sequence of \one's and \two's?

\item But there is yet more we can do.  The extensional quotient of
  $\tree\kern-3pt(\kappa)$ is a $BFEXT$, a wellfounded set picture.  If
  $\kappa$ is a cardinal of infinite rank this BFEXT is of infinite
  rank, since the rank of the extensional quotient is the same as the
  rank of the original tree.  Now assume \AxC\ or something of that
  nature, in order to ensure that $\rho(\tree\kern-3pt(|V|))$ is infinite.
  Then there will be cardinals in $\tree\kern-3pt(|V|)$ of infinite strongly
  cantorian rank, and their extensional quotients will be of strongly
  cantorian rank.  We have to do a little bit of work to ensure that
  their carrier sets are likewise strongly cantorian.  (We can show
  that any $BFEXT$ of rank $\omega$ has a countable carrier set and is
  therefore strongly cantorian. It'll be harder in general but even
  the rank $\omega$ case serves to make the point.) Once we have
  established that, Rieger-Bernays permutation constructions will then
  give us actual wellfounded sets isomorphic to these set pictures
  ($BFEXTS$).  And these wellfounded sets are defined using parameters
  from the attic.

\end{itemize}

For the last item to give us wellfounded sets of large transfinite 
rank with attic provenance we will need the following
\begin{lem}
Every $BFEXT$ of strongly cantorian rank has strongly cantorian
carrier set.\end{lem} \Proof All in good time! \endproof

 
Of course there is no reason to suppose that sets definable with
attic-parameters in this way cannot be defined in other ways, but
equally there is no reason to suppose that they can.



\section{$NCI$ finite}

\subsection{How many socks?}

Let $S$ be a union of countably many pairs, and assume $|S| = |S| + |S|$.
(This last happens automatically if $NCI$ is finite.)

We have two functions $\pi_L$ and $\pi_R$: $S \inj S$ where $\pi_L ``S
\cup \pi_R ``S = S$ and $\pi_L ``S \cap \pi_R ``S = \emptyset$.  Thus
every sock $s$ can be thought of as the ordered pair $\tuple{\pi_L(s),
  \pi_R(s)}$.  (Not every ordered pair of socks is a sock, tho').


There is a quasi-order on the socks, beco's the socks come in
countably many pairs.  We want to refine this quasiorder into a total
order.  What do we do with the pair $\{s_1, s_2\}$?  We exploit the
fact that we can extend the quasiorder to ordered pairs of socks and
ask which of $\tuple{\pi_L(s_1), \pi_R(s_1)}$ and $\tuple{\pi_L(s_2),
  \pi_R(s_2)}$ comes first.  We iterate until we reach a fixed point.
Is this fixed point antisymmetric?  Suppose we have been unable to
split the pair $\{a,b\}$, and let us suppose it is the first one we
cannot split.  This must mean there are two pairs $\{u,v\}$ and
$\{x,y\}$ with $a = \tuple{u,x}$ and $b = \tuple{v,y}$. Our pair
$\{a,b\}$ now represents a bijection between these two pairs.  It does
not give us a choice from either of them, but it has reduced the task
of choosing from two pairs to a task of choosing only from one.  Now
we look at the second unsplit pair, and so on, getting more and more
bijections between pairs.  Notice that we don't have to worry about
the possibility of $a$ being the pair $\tuple{x,x}$ and $b$ being the
pair $\tuple{y,y}$ (in which case the pair $\{x,y\}$ would have
contained no information$(^*)$) beco's the set of first components is $\pi_L
``S$ and the set of second components is $\pi_R ``S$ and these two are
disjoint.  Nor do we have to worry that $a$ might be $\tuple{x,y}$ and $b$ be
$\tuple{y,x}$ beco's nothing can be both a first component and a second component.


The idea is that eventually we will build a family of commuting
bijections, so with one choice from the first pair we will be able to
wellorder the whole of $S$.  The major problem with this is that since
every sock is a component of precisely one ordered pair, no pair of
socks lands in the range of more than one bijection!  It may be that
with a bit more work we can get round this, perhaps by using more than
one pair of mappings, so that we can prove: Sse $|S| + |S| = |S|$ and
$S$ is a union of countably many pairs, then $S$ is countable.  (This
would presumably also prove that if $|S|\cdot n = |S|$ and $S$ is a
union of countably many unordered $n$-tuples, then $S$ is countable.
That would be nice!!)

But for now let's assume not only that every sock is an ordered pair
of socks but that every ordered pair of socks is another sock, in
other words $|S| = |S| \times |S|$.  What now?  This time we can use
pairing ``in the other direction'' as well.  If we want to separate
$a$ from $b$ we can compare $\{a\} \times S$ and $\{b\} \times S$
lexicographically.

Now think of the first unsplit pair, which is $\{a,b\}$, and let
$\{x,y\}$ be any other unsplit pair.  Think about the four ordered
pairs in $\{a,b\} \times \{x,y\}$.  They can't belong to a quadruple
co's there are no quadruples, and they must come in two pairs
$\{\tuple{a,x}, \tuple{b,y}\}$ and $\{\tuple{a,y}, \tuple{b,x}\}$
(without loss of generality) and {\bf one of these pairs comes first}!
This pair is simply the graph of a bijection between $\{a,b\}$ and
$\{x,y\}$.  That way we have reduced the problem of choosing from
$\{x,y\}$ to the problem of choosing from $\{a,b\}$.


Pretty, isn't it?!  Now how about things that come in bundles larger
than two?  Let $S$ be a union of countably many unordered $k$-tuples,
and do the same.  This time we reason not about the first surviving
pair, but the first surviving $j$-bundle, where $j$ is the maximal
size of surviving bundles. Let $A$ be the first surviving $j$-bundle
and let $B$ an arbitrary other $j$-bundle.  $A \times B$ must be split
into $j$-bundles.  None of the bundles can be $i$-bundles with $i < j$
beco's we would have been able to use that to split $A$ or $B$.  In
each bundle $\subseteq A \times B$ each member of $A$ must be the
first component of precisely one ordered pair and each member of $B$
must be the second component of precisely one ordered pair.  In other
words, each bundles is the graph of a bijection---as in the case of
the socks.

So we can match up all the $j$-tuples in such a way that one single
choice suffices to order them all.  Then we work on the next size
down.  So only finitely many choices needed.  This is the correct
proof of the allegation in the yellow book: the proof there is
fallacious.


Can we do the same if $S$ is a union of countably many finite sets
without any bound on the size of the finite sets?

This time let's not assume that every ordered pair of socks is a sock,
but that every ordered pair of distinct socks is a sock, and that
every sock is an ordered pair of distinct socks. (This addresses the
concerns above at *) This time there may well be no maximal size of
surviving bundles, so we cannot use the useful boldface trick of last
time to get bijections---though we might sometimes be lucky and get
bijections or at least constraints on bijections: if $A \times B$ gets
split we get a constraint on a bijection: the earliest bundle to be
included in it represents a constraint.  Also, a bijection or
constraint-on-a-bijection between $A$ and $A'$, together with a
bijection or constraint-on-a-bijection between $B$ and $B'$ will lift
to a bijection or constraint-on-a-bijection between $A \times B$ and
$A'\times B'$.

This time we look at surviving bundles of {\sl minimal} size.  Just as
in the original development, with $\pi_L$ and $\pi_R$ we can say that
a $j$-bundle can only be a bijection between two $j$-bundles.  Now it
becomes clear that it could really be worth trying very hard to show
that in that development there really is enuff info to obtain
bijections between all the pairs, because if it works there, it might
work here.  If it did, we could reduce the problem of splitting all
$j$-bundles to the problem of splitting one.   Then we use the fact 
that bijections and constraints on bijections can be lifted to cartesian 
products and hope that we can then attack larger bundles.

I am deeply pessimistic about this.  Even supposing that we can
exploit the fact that everything is an ordered pair to build up
bijections between all surviving $j$-bundles, where $j$ is minimal,
and that we can (well, we obviously can) use this to build up
bijections between cartesian products, i don't see any reason why
there shouldn't be infinitely many surviving bundles of every size.
For each $p$, we might be able to build bijections between all the
$p$-bundles, but they don't interfere helpfully at all.

So the best we can hope is that we hang onto the finite bound in the
assumption, and weaken the assumption to $|S| = |S|\cdot n$.

March 2009: i now think that this method will show that if $|S| = |S|^2$ 
and $S$ has a totally ordered partition into pieces of bounded finite 
size then $S$ is totally ordered.


\begin{thm}
  If $NCI$ is finite, $\tuple{NCI,\leq}$ is a complete distributive
  lattice, and $a \vee b = a + b$.  \end{thm} 

\Proof Observe that if $a \leq c$ and $b \leq c$ then $a+b \leq c + c
= c$, so $a+ b$ really is $a \vee b$. This makes $\tuple{NCI,\leq}$
into a complete poset, so $a \wedge b$ is defined. All that remains to
be shown is distributivity.


It is clear that $c \wedge a + c \wedge b \leq c \wedge (a+b)$. A set
that is a union of a piece that embeds into both $C$ and $A$, and a
piece that embeds into both $C$ and $B$ embeds into both $C \sqcup C$
(which is $C$) and into $A \sqcup B$.

For the other direction ($c \wedge (a+b)\ \leq\ c \wedge a + c \wedge b$)
we reason as follows.  Consider subsets of $A \sqcup B$ of size $\leq
c$.  Such a subset $D \subseteq A \sqcup B$ comes in two parts: $D
\cap A$ and $D \cap B$, and thereby defines two cardinals: $|D \cap
A|$ and $|D \cap B|$. There are only finitely many such pairs of
cardinals so for each such pair pick a $D$ and ensure that they are all
disjoint.  Then take the union of all the $D\cap A$.  It will be of
size $c \wedge a$. And the union of all the $D\cap N$ will be of size
$c \wedge b$.  But then the union of all the $D$ will clearly be of
size $(c \wedge a) \vee (c \wedge b)$. But the union of all the $D$ 
is obviously the largest things thing that can be embedded in both 
$C$ and $A \sqcup B$, and is therefore of size $c \wedge (a \vee b)$.   




\endproof

I suspect there are general reasons why $NC$ should be a distributive
lattice if it is a lattice at all, but in this case we can exploit $a
= 2 \cdot a$.

\bigskip

Now that we know that $\tuple{NCI, \leq}$ is a complete distributive
lattice consider the function $f: NCI \to NCI$ defined by $f(a) =
\bigvee \{b: b \not\geq a\}$. If $a \leq a'$ then $\{b: b \not\geq a\}
\subseteq \{b: b \not\geq a'\}$ whence $f(a) = \bigvee \{b: b \not\geq
a\} \leq \bigvee \{b: b \not\geq a'\} = f(a')$.  So $a \leq a' \to
f(a) \leq f(a')$.  (Can we have $a \leq f(a)$?  I don't see why
not\ldots)

Now start with an arbitrary cardinal $a$ and consider $\tuple{f^n(a):
  n \in \Nn}$.  This sequence can take only finitely many values, so
it must repeat.  Any loop must be an antichain, because of the
monotonicity.  Suppose it is $\{a, f(a), f^2(a) \ldots f^{(n-1)}(a),
f^n(a) = a\}$ with $n > 2$.

But then $(f(a) + f(f(a)) + \ldots f^n(a))$ is a sum of things
$\not\geq a$ and so must be $ \leq f(a)$, so $f(f(a)) \leq f(a)$
contradicting the fact that we have an antichain.

Thus the antichain must be of width 2 at most.  It could be a singleton.

\medskip

Now, it doesn't have to be an antichain.  It could be a chain ending at a fixed point!


Suppose $a = f(b)\ \wedge\ b = f(a)$ is such an antichain.  What
happens above $a$ and $b$?  Suppose $c > a$.  If $c\not\geq b$ we have
$c \leq f(b) = a$ so we must have $c \geq b$.  Thus $c > a \to c \geq
b$ and $c > b \to c \geq a$ analogously. So everything above either
$a$ or $b$ must be above $a \vee b$ which is therefore a {\bf
  pinchpoint}.  It could be $|V|$ of course\ldots


How about analogously defining $g(a)$ to be $\bigwedge\{b: b \not\leq a\}$?

\bigskip

\bigskip

\bigskip



Let $\alpha$ be a cardinal with a unique successor $\alpha^+$. That is to
say, anything $> \alpha$ is $\geq \alpha^+$. Now suppose there are
cardinals incomparable with $\alpha$. This makes the following definition
sensible. Set

$$\alpha^- = \bigwedge\{\beta: \alpha \not\leq\beta\}.$$ 
By distributivity (after all, $\tuple{NCI,\leq}$ is a complete distributive lattice)
$$\alpha \vee \bigwedge\{\beta: \alpha \not\leq\beta \not\leq\alpha\} = 
\bigwedge\{\alpha \vee \beta:\alpha \not\leq\beta\not\leq\alpha\}$$ 
This cardinal must be $\geq \alpha^+$ since it is an inf of things all $> \alpha$. 
But if $\alpha \vee$ splat is bigger than $\alpha$, splat must be bigger than
$\alpha$ or at least incomparable with it. It can't be bigger than $\alpha$ (it is
the inf of thing incomparable with it) so it must be incomparable with it. So
$\alpha^-$ is incomparable with $\alpha$.  

This proves that if $\alpha$ has a unique successor, and is not a
pinch-point, there is a unique minimal thing incomparable with it.

(Let's try the dual of this. Suppose as before that $\alpha$ is a
cardinal with a unique predecessor $\alpha^-$. That is to say,
anything incomparable with $\alpha$ is $\leq \alpha^+$. Now suppose
there are cardinals incomparable with $\alpha$. This makes the
following definition sensible. This time

$$\alpha^+ = \bigvee\{\beta: \alpha \not\leq\beta\}.$$
By distributivity
$$\alpha \wedge \bigvee\{\beta: \alpha \not\leq\beta\} = 
\bigvee\{\alpha \wedge \beta:\alpha \not\leq\beta\not\leq\alpha\}$$
etc)

\subsection{The Oberwolfach cardinal}
 {\sl At the Oberwolfach meeting in 1987 John Truss and I had a look at the old question of whether or not $NCI$ can be proved to be infinite and we briefly thought we had proved it.  If $NCI$ is finite there is a $*$-unique $*$-maximal cardinal $\alpha$ s.t $\alpha^2 \not\leq_* \alpha$.}

\medskip

Assume $NCI$ finite \ldots now read on \ldots

\medskip

 
Suppose $\alpha$ is *-maximal so that $\alpha^2 \not\leq_* \alpha$. We
will show that it is *-unique. (We may also have to consider a
$\leq$-maximal version.)

Suppose $\alpha$ and $\beta$ are both $\leq_*$-maximal with this
property, and are *-incomparable, so $\alpha$, $\beta, < \alpha +
\beta$.  Therefore, by *-maximality, $(\alpha + \beta)^2 \leq_*
(\alpha + \beta)$.  Therefore, by Bernstein's lemma, $\alpha$ and
$\beta$ are *-comparable.

Therefore, if $\alpha$ is *-maximal so that $\alpha^2 \not\leq_*
\alpha$ then it is unique with this property: it is the *-maximum $\alpha$
such that $\alpha^2 \not\leq_* \alpha$. 


Now let $\beta$ be any cardinal s.t. $\beta \not\leq_* \alpha$. Then
$\alpha <_* \beta + \alpha$.  Therefore, by maximality of $\alpha$ we
have $(\beta + \alpha)^2 \leq_* \beta + \alpha$ and we invoke
Bernstein's lemma to infer $\alpha \leq_* \beta$.  So $\alpha$ is a
*-pinch-point: $(\forall \beta)(\beta \leq^* \alpha \vee \alpha \leq^*
\beta)$.

Next we show that $\alpha^2$ is a *-successor of $\alpha$.
Suppose $b$ and $c$ are two cardinals $>^* \alpha$.  We must have 
$(b + c)^2 \leq^* b + c$ so by Bernstein's lemma $b$ and $c$ are 
$*$-comparable.  (this fact is worth noting on its own account!)  


Next suppose $\alpha <^* \beta <^* \alpha^2$.  This gives $\alpha^2
\leq^* \beta^2 \leq^* \beta \leq \alpha^2$.  So $\beta$ and $\alpha^2$
are *-equivalent, whence $\alpha$ is *-adjacent to $\alpha^2$.


%Suppose $b$ and $c$ are two cardinals $>^* \alpha$.  We must have $(b + c)^2 \leq^* b + c$ so by Bernstein's lemma $b$ and $c$ are $*$-comparable.  Next suppose $\alpha <^* \beta <^* \alpha^2$.  This gives $\alpha^2 \leq^* \beta^2 \leq^* \beta \leq \alpha^2$.  So $\beta$ and $\alpha^2$ are *-equivalent, whence $\alpha$ is *-adjacent to $\alpha^2$.


%Suppose $\alpha$ is a maximal lower bound (wrt $\leq^*$) for two cardinals $b \not = c$. By maximality of $\alpha$ we have $b^2 \leq^* b$ and $c^2 \leq^* c$.  We also have $\alpha^2 \leq^* b^2$ and $\alpha^2 \leq^* c^2$.  So $\alpha^2 \leq^* b$ and $\alpha^2 \leq^* c$.  So $\alpha^2$ is also a $\leq^*$-lower bound for $\{b,c\}$, and $\alpha <^* \alpha^2$ contradicting the assumption that $\alpha$ was a $\leq^*$-maximal lower bound for $\{b, c\}$.

So $(\forall b, c >^*\alpha)((\alpha <^* \alpha^2 \leq^* b, c) \wedge (b \leq^* c \vee c \leq^* b))$

\subsubsection{Let's try to get a contradiction}
\begin{dfn}

  Define $f$ on $NCI$ by $f`\alpha =_{\small df}\ \Sigma \{ \beta:
2^{\beta} \leq \alpha \}$.  
\end{dfn}
Evidently $f`\alpha \leq \alpha$. We want
$f`\alpha < \alpha$ to make $f$ pressing-down and interesting. In fact we can prove something stronger.

\begin{lem}
$f(\alpha) \leq \alpha \not\leq_* f`\alpha$. 
\end{lem}
\Proof Let $A_{\alpha}$ be $\{ \beta: 2^{\beta} \leq \alpha \}$.  
With $\alpha$ free we will show by induction on
$n$ that no subset of $A_{\alpha}$ with $n$ members has a supremum $\geq
\alpha$.

When $n = 1$ this is trivial.

Suppose it proved for $n = k$ and let $X \cup \{ \beta \}$ be a $k+1$-membered
subset of $A_{\alpha}$ whose supremum is $\geq_* \alpha$. Let $\chi$ be $\Sigma
X$.  Suppose {\sl per impossibile} that $\chi + \beta \geq_* \alpha$.  Then 
$$\chi + \beta \geq_* \alpha \geq 2^{\beta} = 2^{\beta + \beta} =
2^{\beta} \cdot 2^{\beta}$$ Now use Bernstein's lemma: $$\chi \geq_* 2^{\beta}
\vee \beta \geq_* 2^{\beta}$$ so $$\chi \geq_* 2^{\beta}$$ whence $\chi \geq
\beta$ and $\chi + \beta \leq_* \chi + \chi = \chi$.  But $\chi \not \geq_*
\alpha$ by induction hypothesis. \endproof

We note that $f`\alpha$ is defined \ala\ $\alpha \geq 2^{\aleph_0}$. 

We will eventually obtain a contradiction by considering $f$-chains.  Let
$F(\alpha,n,\beta)$ say that $\beta = f^n(\alpha)$.  By induction on
`$n$' we have $(\forall \alpha \beta)(F(\alpha,n,\beta) \bic F(T\alpha,
Tn, T\beta))$ as long as $\alpha, \beta \leq T|V|$.

Because $NCI$ is finite and $f$ is pressing-down, every $f$-chain is
finite.  Let $G(\alpha)$ be the largest $n$ such that $(\exists
\beta)(F(\alpha, n, \beta)$.  We must check that $G(T\alpha) =
TG(\alpha)$.  Consider the $\beta$ such that $F(\alpha,
G(\alpha),\beta)$.  We have $F(T\alpha, TG(\alpha),T\beta)$.  Now
$\beta \not\geq 2^{\aleph_0}$ (since $f(\beta)$ is not defined) and
the continuum is cantorian so $f(T\beta)$ is not defined either. So if
we do $f$ $TG(\alpha)$ times to $T\alpha$ we obtain something we
cannot do $f$ to.  So $G(T\alpha) = TG(\alpha)$.


To obtain a contradiction it will suffice to show that $G(T|V|) =
G(T^2|V|) +1$.   This might not be possible.   We note that
$T|V| > f(T|V|) \geq T^2|V|$.  (Draw a ladder.)

Now attempt to build a bijection, leaving out $T|V|$ to get the
parity argument.  Pair $f(T|V|)$ with $T^2|V|$ and, once
you've paired $x$ with $y$, pair $f(x)$ with $f(y)$.  That is to say,
we endeavour to pair off $f^{n+1}(T|V|)$ with $f^n(T^2|V|)$.
Since we know $f^{n+1}(T|V|) \geq f^n(T^2|V|)$ this process
can come adrift (the two arms of the ladder run out at different
times) only if we reach an $n$ such that $f^{n+1}(T|V|)$ is big
enuff to feed to $f$ but $f^n(T^2|V|)$ isn't.  But if
$f^{n+1}(T|V|) \geq 2^{\aleph_0}$ then also 
$T(f^{n+1}(T|V|)) \geq 2^{\aleph_0}$. 

All this shows is that this $n$ isn't cantorian.  Bugger.

\subsubsection{A bit of fun}


Assume $NCI$ finite as usual.  Let $A_n := \{\alpha \in NC: \aleph_n
\leq \alpha \not\geq \aleph_{n+1}\}$. Since $NCI$ is a distributive
lattice we can show that each $A_n$ is a sublattice, with a top
element and a bottom element and is closed under $\times$. 

We can do something clever by exploiting theorem \ref{thm:tarski} to
show that the map $\alpha \mapsto \alpha + \aleph_{n+1}: A_n \to A_{n+1}$
must be an {\sl injection}.  What about a map coming down?  consider
$\alpha \mapsto \bigvee\{\beta \in A_n: \beta \leq \alpha\}$. I think this is a right-inverse to the last map. (miniexercise: check this)  It wouldn't be onto by any chance would it?  No reason to suppose that.  But at least we show that $NCI$ is the union of a family of finite distributive lattices, with a sequence of retracts....

\subsubsection{leftovers}


 Now suppose $\alpha^2 = \alpha$. Then $2^{f`\alpha} \leq \alpha$ 
$$2^{f`\alpha} = \prod_{2^{\beta} \leq \alpha} 2^{\beta}$$ 
so $2^{f`\alpha}$ is a product of things $\leq \alpha$ and so is $\leq \alpha^n$ which is $\alpha$ 

We ought to be able to prove something like this. Let $\alpha$ be a
cardinal of infinite rank.  Let [$\alpha$]$_0$ be $\{ \alpha \}$ and let
[$\alpha$]$_{n+1}$ be $\{ \beta : 2^{\beta} \in $[$\alpha$]$_n \}$.  Let
$\oplus_0$ be + and $\kappa \oplus_{n+1} \mu =_{\small df}\ 2^{(\gamma
\oplus_n \zeta)}$ where $\kappa = 2^{\gamma}$ and $\mu = 2^{\zeta}$. It
would be nice to show by induction on n that

$\rho(\alpha) \geq n+4 \to (\forall k \leq n)($[$\alpha$]$_k$ is closed 
under $\oplus_{n-k})$ 

Unfortunately this doesn't seem to work.  Suppose $2^{\aleph_{17}} =
\aleph_{\omega+1}$, $2^{\aleph_{\omega}} =
\aleph_{\omega+3}$. $\aleph_{\omega}^{\aleph_{17}} > \aleph_{\omega}$ 
{\sl to be continued} 
  
\subsection{$\alpha$ of infinite rank or $2^{T\alpha} \leq \alpha$}
\label{infinite rank}

Does $2^{T\alpha} \leq \alpha$ have the same consequences for $\alpha$ (not
being an aleph, for example) as $2^{T\alpha} = \alpha$?  Well, it certainly
doesn't if $\alpha \in \Nn$ for then we can have $n > 2^{Tn}$ but $n$ is
the cardinal of a wellordered set.  So the conjecture should be something
like: if $\alpha$ is infinite, or if \AxC, then $2^{T\alpha} \leq \alpha$
has the same consequences for $\alpha$ as $2^{T\alpha} = \alpha$?  

The idea is this: use the singleton function, given $|x| > |\pow x)|$,
to get a setlike bijection (which will---obviously---not be a set)
between $x$ and $\pow x)$ so that $\m{x} \simeq \m{\pow x)}$ and thus
$\m{x}$ is a model {\sl glissant} of \tsti.  So what we need, given
$\alpha > 2^{T\alpha}$, is that there should be $x$ and $y$ such that
$$\alpha = x + y$$
$$2^{T\alpha} = x + Ty$$

$x$ and $Ty$ are both odd or both even, since their sum is even.  Either way $\alpha$ is even.
Then whenever we have a thing $A$ of size $\alpha$ we can partition $A = A_1
\sqcup A_2, \ \ \pow A) = B_1 \sqcup B_2$ with maps $f: B_1 \to A_1$ and $g:
\iota``A_2 \to B_2$ with $f$ a set $\iota^{-1}g$ a set. We use this to
construct a bijection $h:A \bic \pow A)$ by $h = f \ \cup \ \iota^{-1}g$.
We would like this to be setlike. If it is we have shown that $M_A$ and
$M_{\pow A)}$ are isomorphic.

    Now we do know that if $\alpha \in \Nn$ we have no hope of partitioning
$x$ in this way to get a setlike bijection, so either (i) the construction
of $x$ and $y$ must depend on $\alpha$ being infinite, or (ii) the fact that
the bijection constructed is setlike must depend of $\alpha$ being infinite,
or on \AxC, or something.

Now (i) doesn't seem possible.  It is true that a parity argument shows that 
$\alpha$ would have to be even but i don't see any way of excluding the
possibility of finite solutions to this pair of equations.

So it is probably (ii) and we have to think about strong axiom would be
available to make the partition have the desired property.  It's worth
pointing out that as long as $\alpha$ is infinite there are such $x$ and
$y$, for set $x = 2^{T\alpha}$ and $y = \alpha - 2^{T\alpha}$ (unless
$\alpha = 2^{T\alpha}$ in which case there is nothing to prove!)  For we
want $$2^{T\alpha} = 2^{T\alpha} + T(\alpha - 2^{T\alpha})$$ This will
follow from $$2^{T\alpha} = 2^{T\alpha} + T(\alpha)$$ which will follow
from $$2^{T\alpha} = 2^{T\alpha} + 2^{T(\alpha)}$$ which follows from
$$\alpha = \alpha + 1$$ But if $2^{T\alpha} \leq \alpha$ we must have
$$2^{2^{2^{T^3\alpha}}} \leq \alpha$$ so $\alpha$ must be dedekind infinite
as desired.

\hole{Tidy this up}
\begin{enumerate}

\item Can we show that $\alpha$ of infinite rank $\to$ $\alpha$ not $\beth_n$ 
of any aleph?
      Assuming $AC_{wo}$ then $\beth_{\alpha}$ is defined for some
$\alpha > T\alpha$ so in such a case, no.

\item If \AxC fails, then $\beth_n$ will be a counterexample for some $n$.

\item \AxC $\bic \ (\Box \forall x(\pow x) \subseteq x \ \to \ x$ not
      wellordered))? 
                    
      Suppose $2^{T\alpha} \leq \alpha$.  Then $\alpha$ is infinite. Suppose it
      is an aleph.  Let $\Phi_{\alpha}`\beta$ be $\{\beta,2^{\beta} \ldots\}$
      as far as the powers remain below $\alpha$. Can we do anything with this?

\item \AxC $\to (2^{T|x|} \leq |x| \to M_x \models Amb$)?  \AxC is
      needed to prove $(2^{T|x|} \leq |x| \to M_x \models Amb$ 
      because $2^{Tn} < n$ can happen otherwise and this would give a model of 
      $Amb$ + $\neg$ AxInf.

\item \AxC $\to (2^{T|x|} \leq |x| \to M_x \models Amb$)?  Assume
      $AC_{wo}$. So all $\beth$  numbers exist. Now for some $\alpha \in On$ 
      with $\alpha > T\alpha$ we will have the corresponding $\beth$ number 
      $\beth_{\alpha}$ with $2^{T\beth_{\alpha}} < \beth_{\alpha}$. This cannot
      give rise to a model of $NF$ + \AxC, for in any such we can prove 
      ``$|V|$ is not a $\beth$ number" But since we can use Ehrenfeucht-
      Mostowski to get models of $KF$ containing $2^{T\alpha} \leq \alpha$ 
      without any additional assumptions, we know that $2^{T\alpha} \leq\alpha$
      has no strong consequences in a stratified context.

\end{enumerate}

If (2) is to work, we want to be sure that (1) works only for $\alpha \not\in$
\Nn.  If it were to work for all finite $n \geq Tn$ then for any $n \in$ \Nn\
we would have $x$ and $y$ s.t. 
         $$x + Ty = 2^{Tn}$$
         $$x + y  = n$$
Now clearly $x + Ty$ and $x + y$ are conguent mod 2, and one of them is a power
of 2, so the other is at least even. So $n$ is even.  But if $2^{Tn} \leq n$
then certainly $2^{(Tn + 1)} \leq (n + 1)$ and $n + 1$ would have
to be even as well.  

So far so good!

That takes care of (1).  How about (2)?  Well, this is just the old problem of
showing that s-b works for setlike injections to give setlike bijections, and
there seems no reason why it should. It is quite clear that $h$ will lift once,
but there seems no reason to suppose it will lift twice.    Of course in
general we cannot expect to be able to derive interesting consequences from
$2^{T\alpha} \leq \alpha$ because this can happen in $KF$ with no knobs on.

   We have seen that \AxC $\to |V|$ is not a $\beth$ number, and
that if $\alpha$ is of infinite rank then it is not an aleph. Can we
show that if $\alpha$ is of infinite rank then $\alpha$ is not a
$\beth$ number?

Every now and then one of my part II set theory supervisees asks me ``I
know what $\omega$ is, it's the length of the positive integers. What
is $\omega_1$ the length of?"  And i always feel, when i reply ``the
set of all countable ordinals in their natural order" that i am giving
a trick answer.  And i suspect i am too, because they usually don't
seem very satisfied.  The witness is not the one one would obtain by
transformation of a constructive proof---unless it is of higher type,
where all the countable ordinals are elements---so we get $\aleph_n$
at type $n$. Therefore no proof of existence of $\aleph_n$ uniform in
$n$, and no stratified proof of existence of $\aleph_{\omega}$.  The
only proof is by induction on $n$.

This certainly seems to be the situation in NF anyway.

 Indeed even if we do have all $\aleph_n$, we cannot construct an
$\aleph_{\omega}$ of the same type without AC. (Coret\index{Coret}: all
stratified replacement provable in Zermelo, and no $\aleph_{\omega}$ in Z)


    $(\forall n)(\aleph_n$ exists) stratified but has no stratified proof.


 Is it true that whenever $\TZT \vdash F(t)$ for all terms $t$ then
there is a uniform proof in the arithmetic of $\TZT$ that such proofs
exist?  Clearly the arithmetic of $\TZT$ is typically ambiguous: the
$T$ function is an isomorphism between the naturals of level $n$ and
the naturals at level $n+1$.


     I remember now why i was so concerned about finding a nice set of
large finite size.  Consider the claim that there is a function
$\beta$ such that $(\forall n)$ every non-empty $n$-symmetric set has
an (at worst) $\beta(n)$-symmetric member. This sounds desirable, at
least, though it is {\sl prima facie} an even stronger assertion than
the one that \TZT\ has a term model.  Now consider the finite cardinals,
all of which are 2-symmetric. We are now stuck with having to produce,
for each finite cardinal $k$, a $\leq \beta(2)$-symmetric set of size
$k$.  If we now use hereditarily finite sets we run up against the
fact that $m$-symmetric hereditarily finite sets are bounded in size,
and so for $k$ large enuff we are not going to be able to find a
hereditarily finite set of rank $\beta`2$ and size $k$. The obvious thing
to do is to reach for the initial seg of \Nn\ bounded by $k$, but this
is an object of higher type.  What does the proof look like that that the
natural numbers below $n$ are a set of size $Tn$?  One way we could hack
round this is if we have an algebraic version of ``definable with $n$
alternating blocks of quantifiers" (after all, the notion of $n$-symmetric
set is an algebraic version of set-abstract-with-sole-free-vbl-of-type-n)
for then we seek instead function $\beta$, $\gamma$ such that $(\forall
n_1,n_2)$ every non-empty $n_1$-symmetric set definable with $n_2$
alternating blocks of quantifiers has an (at worst) $\beta`\langle n_1,n_2
\rangle$-symmetric member definable with $\gamma`\langle n_1,n_2 \rangle$
alternating blocks of quantifiers.

  Finding large sets disjoint from their power sets can be useful.
Suppose we wanted to prove the consistency of $\nf_3 + \Phi(\aleph_0)
\in \Nn$.  We work in $NF + \neg$\AxC\ and fix on $\alpha$ some finite
beth number $> 2^{T\alpha}$.  We cannot use \ref{prop:``existence of
permutations"} here because this only works for internal permutations,
but if we can find $x \in \alpha$ disjoint from $\pow x)$ then we can
extend the 1-setlike bijection $x \bic \pow x)$ to a 1-setlike
permutation of the universe, which gives rise to a model of $\nf_3$ in
which $\alpha$ is the size of the new universe. Mind you, if $a
\not\in a$ then $\pow B(a))$ is disjoint from its power set and of
size $|V|$. But in any case $a \not\in a \to B`a \cap \pow B(a)) =
\Lambda$ so every cardinal contains a set disjoint from its power set.

The point about finding sets disjoint from their power sets is this.
If we have a bijection between a part of $x$ and a part of $\pow x)$
then this will extend to a permutation of the universe. If $x \cap
\pow x) = \Lambda$, then the permutation is an involution, which makes
life much easier.

\section{Everything to do with Henrard's trick}
\hole{Explain why we can prove SB for Henrard maps in $\nf_3$}
If we can explain bijection we can explain injection. So (deep breath)

An {\sl bijection} from $x$ into $y$ is a set $I$ of singletons and unordered pairs from
$x \cup y$ such that
\begin{itemize}
\item every member of $x \Delta y$ belongs to precisely one pair
\item every member of $x \cap y$ belongs to precisely two pairs or to one singleton.
\item No chain can have two ends in $x \setminus y$ or two ends in $y \setminus x$.
\end{itemize}

All these conditions can be made to look horn.

If we take the set containing a pair and close under the operation
``add any pair that meets one of the things you've already got'' you
get a {\bf chain}.  There are several sorts of chains.

\begin{enumerate}
\item Chains consisting of one pair only. 
\item Chains consisting of more than one pair beginning in $x$ and ending in $y$.
\item Chains with one end in $x$ and no other end.
\item Chains with one end in $y$ and no other end.
\item Chains with no ends at all.
\end{enumerate}



If we want to use Henrard bijections to talk about orderisomorphisms
then we will need to allow chains like those in (2).  This is because
such bijections have to do extra things, and we will explain this later.

 Normally we can assume \wwlog\ that all chains with two ends consist of
precisely one pair. This is because you simply pair off the two
endpoints (to get a pair in 1) and rejoin the severed ends to get a
chain in 5)


Let us make explicit the connections with a decomposition theorem of Tarski's.
It's old and elementary but not commonly taught nowadays.  

Suppose we have three sets $A$, $B$ and $C$, all disjoint of sizes
$\alpha$, $\beta$ and $\gamma$, and a henrard bijection between $A
\cup B$ and $A \cup C$.  We assume---as we always can \wwlog, and this
time we need it---that all chains with two ends contain precisely one
pair.  We are going to partition these sets.
\begin{itemize}
\item Some things in $B$ are paired directly with things in $C$. Put these into $B_1$ 
\item Some things in $B$ start in $B$ and belong to chains with only one end. Put these in $B_2$.  
\end{itemize}

Similarly

\begin{itemize}

\item Some things in $C$ are paired directly with things in $B$. Put these into $C_1$
\item Some things in $C$ start in $C$ and belong to chains with only one end. Put these in $C_2$.
\end{itemize}

\begin{itemize}
\item Some things in $A$ belong to singletons or to chains without ends; Put them in $A_1$
\item Some things in $A$ belong to single-ended chains ending in $C$; put them in $A_2$
\item Some things in $A$ belong to single-ended chains ending in $B$; put them in $A_3$
\end{itemize} 

Clearly we have $|B_1| = |C_1|$. Call this cardinal $\delta$.
Let $|B_2|$ be $\beta'$ and $|C_2|$ be $\gamma'$ and $|A_1|$ be $\alpha'$.

Then we have
\begin{itemize}

\item $\beta = \beta' + \delta$
\item $\gamma = \gamma' + \delta$
\item $\alpha = \alpha' + \aleph_0 \cdot (\beta' +\gamma')$
\end{itemize}

That is to say, we have proved:
\begin{thm}\label{thm:tarski} (Tarski)
Whenever $\alpha + \beta = \alpha + \gamma$ there are $\delta$, $\alpha'$,
$\beta'$ and $\gamma'$ such that $\beta = \beta' + \delta$, $\gamma =
\gamma' + \delta$ and $\alpha = \alpha' + \aleph_0 \cdot (\beta'
+\gamma')$
\end{thm}

Must check whether or not this can be done in $\nf_3$.

Now we have to consider the problem of composing bijections!

Note that the slick proof of S-B works in \nf$_3$ for Henrard maps.

\subsection{Orderisomorphisms}
We can represent partial orders without using ordered pairs by talking
about the set of initial segments.  We want to use Henrard bijections
to talk about {\sl isomorphisms} between orderings.  If we do this, then
we cannot assume that all chains with two ends have precisely one pair.

Consider the two wellorderings. \begin{itemize}

\item $\A = $ set of even naturals in their usual order followed by the 
      set of odd naturals in their usual order
\item $\B$ = $\{4n:n \in \Nn\}$ in its usual order followed by $\{4n+2:n \in \Nn\}$
      in its usual order followed by the set of odd naturals in their usual order.
\end{itemize}
The set \Nn\  belongs to a pair with the set of evens, because---in
$\A$---\Nn\  is the initial segment of length $\omega.2$ and in $\B$ the
initial segment of length $\omega.2$ and in $\B$ is the evens.  The set
of evens---in $A$---is the initial segment of length $\omega$ and is therefore
paired with the initial segment of $\B$ that is of length $\omega$, namely the
set $\{4n:n \in \Nn\}$.

On the other hand a henrard bijection that codes a bijection between
two total orderings $\A$ and $\B$ cannot contain any chains without
endpoints. For suppose it does.  One of the features of partial orders
coded-as-initial-segments is that the intersection of any subset of
this representation is another element of it. (Careful: this is true
always! The clause that codes wellfoundedness---at least in the case
of total orders---is that, for any initial segment, the intersection
of all its supersets in the code has precisely one more element) If we
have a chain with no ends we know that all the objects appearing in
its pairs are initial segments of both $\A$ and $\B$.  If $C$ is such
a chain, look at its intersection $I$. This is an initial segment of
both $\A$ and $\B$. If these two initial segments are the same length,
then $I$ would belong only to a singleton.  So one of them is shorter
than the other (here we use the fact that the orders are total), and
\wwlog\ it is the occurrence in $\A$.  Since $C$ has no ends, we know
this occurrence of $I$ is paired with some initial segment $I'$ of
$\B$. But then, since the $\A$-occurrence of $I$ was shorter than the
$\B$-occurrence of $I$, we know that $I'$ is shorter than (and
therefore a $\B$-initial segment of) $I$, contradicting
$\subseteq$-minimality of $I$.

There may be easier ways to procede from here, but this does at least
mean that every pair in every chain can be said to have an `$\A$-end'
and a `$\B$-end', and once we have that we can characterise
isomorphisms between two wellorderings by adding the clause that if a
bijection contains a pair $\{A,B\}$ where $A$ is the $\A$-end of
$\{A,B\}$ and $B$ is the $\B$-end, then it must also contain the pair
$\{A',B'\}$ where $A'$ is the `next' initial segment after $A$ (the
intersection of all its supersets in $\A$) \ldots, and a similar
condition for limits.  This part depends on the partial order being
wellfounded.

Unfortunately all this seems to need four types, so we aren't really 
any further forward!


\chapter{Miscellaneous other proof theory}

Dear Daniel,

     Randall and i are thinking about constructive TST and have a
couple of questions.  We want to understand your remark that the weak
+-rule is consistent wrt constructive type theory.  I am (and i think
we both are) thinking in terms of sequent calculus.  There are two
obvious formulations (i hope you can understand my LaTeX!)

$$\frac{\Gamma \vdash \Delta}{\Gamma^+\vdash \Delta^+}$$

where $\Gamma^+$ is the result of raising the type of every variable
in $\Gamma$ by exactly one.   

Holmes and i are agreed that this is a derivable rule: simply push all
occurrences of this rule up the proof tree until they disappear off
the top.  Presumably your rule is rather


$$\frac{\Gamma \vdash \Delta}{\Gamma \vdash \Delta^+}$$

which is obviously stronger!  We don't see why this is consistent.
Also i am not sure what this means if there are free variables common
to $\Gamma$ and $\Delta$.  (Randall is happier about this than i am)
Is there a side-condition on the free variables, and if so what is it?

       best wishes

            Thomas

Dear Thomas and Randall,

I'm very confused.  I wrote that TST + weak ambiguity is consistent
because the classical theory is.  I guess I was asleep when I wrote
that.  Of course, we don't know whether the classical theory is
consistent; that's ``the" question. (I think that I had TST alone
in mind and some mist in my brain mixed it up with ambiguity...
I'm no so proud about all this...  I hope that the company where I
work isn't destroying my logic...)

Humm...  Anyway, I don't know if intuitionistic TST + weak ambiguity
is consistent.  By the way, why are you using the rule
$\Gamma \vdash \sigma$ gives $\Gamma \vdash \sigma+$ instead of the 
axioms $\vdash (\sigma \bic \sigma+)$?  I'm not sure they are equivalent (try to prove the axioms
from the rule).  Notice that the axiom is ``symmetric" in sigma and
sigma+, while the rule is not.  In classical logic, $p\bic q$ is equivalent
to $((p \to  q) \wedge (\neg p \to  \neg q))$.  That's why, in classical logic,
ambiguity can be axiomatized with non symmetric axioms $\sigma \to \sigma^+\ 
\vee\  \sigma^+ \to  \sigma$.

Notice that for full ambiguity, you can use the rule
$\Gamma \vdash \sigma$ gives $ \Gamma \vdash \sigma^*$ or the axioms $\vdash (\sigma \bic \sigma^*)$,
where $\sigma^*$ denotes any variant of $\sigma$.  The reason is that
$(\sigma \bic \sigma)$ is a tautology and $(\sigma \bic \sigma^*)$ is a
$(\sigma \bic \sigma)^*$.

In classical logic, Hilbert style, ambiguity can be axiomatized with
the rule $\vdash \sigma$ gives $ \vdash \sigma^*$, or with the axioms
$(\sigma \bic \sigma^*)$.  It can also be axiomatized with the usual
axioms $(\sigma \bic \sigma^+)$, but I don't know if it can be
axiomatized with some ``+" rule: $\vdash \sigma$ implies $\vdash
\sigma^+$ is of course not correct and I cannot remember the point with
$\vdash \sigma^+$ implies $\vdash \sigma$.

Also, I could prove that *in pure predicate calculus* weak ambiguity
is not equivalent to full ambiguity, in intuitionistic logic.
But I was not able to prove that intuitionistic TST + weak ambiguity
is not equivalent to int. TST + full ambiguity.  I think it is not
but I also think that it is as hard to prove as to prove that it is
consistent!
                        More later...
                          Daniel.

and some tho'rts of mine, on wed 5/viii

How do we set up the sequent calculus rules for ambiguity in constructive 
type theory?  Presumably

$$\proof{\Gamma \vdash p}{\Gamma \vdash p^+}$$

Now what about the case where $p$ and $p^+$ are contraries?  This might arise
in various ways, through derivations of any of the following

$\proof{\Gamma, p}{\neg p^+}$ $\proof{\Gamma, \neg p}{p^+}$ $\proof{\Gamma, p^+}{\neg p}$ $\proof{\Gamma, \neg p^+}{p}$

Now the case that worries me is (there may be others) is (the
following notational variant): $\Gamma, p \vdash \neg p^-$.
This gives $\proof{\Gamma, p}{\neg p}$ by the +-rule.  Then
$\proof{\Gamma, p, \neg,\neg p}{}$ by $\neg$-L; $\proof{\Gamma,
\neg\neg p}{\neg p}$ by $\neg$-R; $\proof{\Gamma, \neg\neg p, \neg\neg
p}{}$ by $\neg$-L; $\proof{\Gamma, \neg \neg p}{}$ by contr-L;
$\proof{\Gamma}{\neg\neg\neg p}$ by $\neg$-R.  Then we cut against
$\proof{\neg\neg\neg p}{\neg p}$ to get $\proof{\Gamma}{\neg p}$ which
we would expect all along.   

Now this occurrence of cut cannot be eliminated.  This sounds pretty
dire, but---at least in LPC---it isn't.  The point is that if we have
a constructive proof of $\proof{\Gamma, p}{\neg p}$ we cannot have
obtained this by assembling $p$ on the left and leaving $\neg p$
undecomposed on the R.  This is beco's we would have had no initial
sequent with $\neg p$ on the R.  So the $\neg p$ on the R must have
come from a $p$ on the left, and given that we can use contr-L.

This saves the day for cut-elimination for LPC, but it is clear that
the cuts cannot be eliminated from the proof above!


The $\in$ rules for sequent calculus.


These must be $\proof{\Gamma \vdash \phi(t)}{\Gamma \vdash t \in \{x:
\phi\}}$ and $\proof{\Gamma, \phi(t)\vdash p}{\Gamma, t \in
\{x:\phi\}\vdash p}$. 

That way we can prove $\proof{t \in \{x:\phi\}}{t \not\in
\{x:\neg\phi\}}$ and $\proof{t \not\in \{x:\phi\}}{t \in
\{x:\neg\phi\}}$.

\ldots but not---of course!---$\proof{t \not\in \{x:\neg\phi\}}{t \in
\{x:\phi\}}$


The point is that to keep things simple one does not allow oneself to
introduce anything like $t \not\in \{x:\phi\}$ on either side except
by the negation rules.  But the restriction doesn't prevent us from
proving any of the things we want.

Andre,

I remember years ago you saying that the interpolation theorem was a
problem for stratified languages. I have been thinking about this
recently in connection with cut-elimination for type theory with
ambiguity.  (Holmes is here and we are going to prove the consistency
of constructive NF!)  Consider the following situation.  We have a
proof that $\phi \to \phi^+$, and therefore of $\phi \to \phi^n$ for
$n$ as large as one wants, in particular for $n$ so large that $\phi$
and $\phi^n$ have no predicate letters in common (all the $\in_k$ are
different predicates of course!).  If we apply interpolation to this
we get an absurd result.  Presumably this shows that interpolation
fails.  But how can this be?  Interpolation follows from
cut-elimination, and didn't someone (Takeuti?) prove cut elimination
for simple type theory (or perhaps it was for type theory without the
$\in$-rules


Have you thought about this


     very best wishes


         Thomas


\chapter{The Universal-Existential Problem}

(One day i am going to write a novel in which the world 
is being taken over by a nasty yank megacorporation which 
peddles psychotherapeutic bullshit.  It will be called 
{\sl Universal Existential} and its logo looks something like

\definecolor{zzttqq}{rgb}{0.6,0.2,0}
\begin{tikzpicture}[line cap=round,line join=round,>=triangle 45,x=1.0cm,y=1.0cm]
%\clip(-3.84,-4.7) rectangle (17.8,7.47);
\clip(-3.84,-2.0) rectangle (17.8,2.0);
\fill[color=zzttqq,fill=zzttqq,fill opacity=1] (4.41,0.17) -- (4.01,0.36) -- (4.01,-0.02) -- cycle;
\draw (2.15,1.24)-- (1.06,1.24);
\draw (2.1,-1.15)-- (1.09,-1.15);
\draw (-1.8,1.24)-- (-0.75,-1.15);
\draw (-0.75,-1.15)-- (0.1,1.24);
\draw (-1.34,0.17) -- (4.41,0.17);
\draw [color=zzttqq] (4.41,0.17)-- (4.01,0.36);
\draw [color=zzttqq] (4.01,0.36)-- (4.01,0);
\draw [color=zzttqq] (4.01,0)-- (4.41,0.17);
\draw (2.15,1.24)-- (2.1,-1.16);
\draw (2.1,-1.16)-- (2.1,-1.16);
\end{tikzpicture}


Its mission statement will contain a promise to free the world
from {\sl Angst}.)



\section{Stuff to fit in}



\bigskip


Don't we prove somewhere that an $\in$-loop cannot consist entirely of
finite sets?  (yes: it's lemma 15 of Bowler-Forster). Is there an AE
version of this allegation?

The natural assertion ``Every bottomless set contains $V$'' is 
$\forall^*\exists^*\forall^*$
which is the wrong way round.

\bigskip



Zachiri,


Thanks for this. You have started me thinking, and reminded me of old
tho'rts...

Cast your mind back to the proof you showed me on wednesday. You have a big
model of TST, and a family of [points in it, and you want to find a small
model of TST and an injection from the small model into the big model which
hits all those points. It's easy if the family is extensional, so the idea
is to plump up the family to an extensional one, You show how to do that.
Fine

I have two tho'rts on this

(i) I recall having had the same idea myself once, but i got stuck, beco's
what i was trying was too ambitious. Suppose the big model is a model of
**TZT**! How can you be sure that the downward propagation ever terminates?
There's no reason why it should, but might it happen, if you are very clever
in your choice of witnesses to symmetric difference, that it eventually
hits the empty set. Suppose one had an E*A* sentence such that, whatever
witnesses one chose, and however one propagated downward, one never reached
the empty set. Wouldn't there be something really weird going on?

(ii) Your downward propagation idea is fine. Tickety-boo. However, i think
one can do something even better. Recall that every level of your model of
TST is a boolean algebra. So, when you propagate, add enuff stuff to ensure
that at any one level of the extended family, the things at that level form
a sub-boolean-algebra of that Level. As far as i can see, this is entirely
painless. And what does it get for us? Presumably it means that: whatever
we could prove for sentences of the form $\exists \vec y \forall \vec x
\phi$ where $\phi$ is quantifier-free, we can now prove for such formulae
where phi is allowed to contain $\cap$, $\cup$, $\setminus$ and
$\subseteq$.

   Is that not so?

\bigskip

On Nov 1 2013, Zachiri McKenzie wrote:

 Dear Anuj (cc'ed Thomas),
 
 I hope that this finds you both well!
 
 It is Friday afternoon and perhaps a good time to make a summary of where
 we are at:
 
 So far we have shown that every EA sentence it either true in finitely many
 finitely generated models or cofinitely many finitely generated models.
 Moreover, if an EA sentence is true in any `infinitely generated model'
 (model with an infinite base) then it is true in cofinitely many finitely
 generated models. This has the following consequences:
 
 * Every pseudo-finite model of TST satisfies the same EA sentences and this
 set of sentences is decidable (I suppose we already knew the latter).
 
 * The set of EA sentences true in any model of TST must be contained in the
 set of EA sentences true in the pseudo-finite models.
 
 Thomas has also proved the following: Any AE sentence that is true in
 some model of $\TZT$ is true in the term model of $\TZT$0.
 
 Therefore, what we would like to do is show that the term model of $\TZT$0
 only satisfies the AE sentences true in the pseudo-finite models of TST...
 
 Anuj -- just to let you know, I will be in Scotland on Tuesday, Wednesday
 and Thursday next week.
 
 Very best wishes,
 
 Zach.



\bigskip

5/xii/2013

I've been thinking some more about these recent tho'rts of Zachiri's. Here
is my take on them.

We have in our left hand a large model $\M$ of TST, one with an infinite
bottom level. (To keep things simple, but large-finite might come later).
We want to establish that $\M \models (\forall \vec x)(\exists \vec
y)\phi(\vec x \vec y)$, where $\phi$ belongs to some syntactic class
$\Gamma$.

To this end we point to a tuple of things in $\M$ and think of them as
inputs $\vec x$ to $\phi$, and hope to find a tuple $\vec y$. The strategy
for doing this involves finding a smaller model $\M'$ (one that satisfies
$(\forall \vec x)(\exists \vec y)\phi)\vec x \vec y)$) plus an injection
$h:\M' \to \M$, where $h$ does two things. (i) everything in our tuple must
be in the range of $h$; and (ii) $h$ preserves all formulae in $\Gamma$.
Then we copy our tuple down into $\M'$ (using the fact that everything in
the tuple is hit by $h$); then we find witnesses to the $\vec y$ inside
$\M'$, and then we copy them upstairs. Job done.

That, as i understand it, is Zachiri's Cunning Plan.  And here is my take.


We have our tuple of $\vec x$ in $\M$. The idea is to use these elements to
build a substructure of $\M$. We start at the top level of $\M$ at which
elements from $\vec x$ appear. This top level is a boolean algebra, and we
consider the subalgebra generated by those top-level members of $\vec x$.
The atoms of this algebra constitute a partition of this top level, and we
add to the $\vec x$s of the next level down a representative from each
element off the partition, and we carry on downwards until we have reached
the bottom level of $\M$ at which $\vec x$s appear. Now comes the clever
bit. The boolean subalgebra we have at this level is still only finite, and
it has only finitely many atoms. So we can find a partition of the same
size as this partition-into-atoms-of-the-partition which is mapped onto it
by a permutation, and such that each element of the image of the partition
under this permutation contains a hereditarily finite set. We now continue
our downward march, but this ruse has ensured that we eventually reach the
empty set. The substructure we have thus constructed is a copy of the
canonical model of TST with empty bottom level, with a twist in the middle
induced by the permutation.

So $M'$ is just the canonical model of TST with empty level 0. Now copy the
$\vec x$ down and find $\vec y$ and copy them back up. But what formulae
does our $h$ preserve? Not just atomic formulae, but also all $\bigcup$,
$\bigcap$, $\setminus$, $\emptyset$ and $\subseteq$.

The permutation of course doesn't change anything, so we seem to have
proved:

Any $\forall^*\exists^* \Gamma$ sentence true in arbitrarily large fingen
models of TST is true in all infinite gen models, where $\Gamma$ is the
language containing not just = and $\in$ but also $\bigcup$, $\bigcap$,
$\setminus$, $\emptyset$ and $\subseteq$.

   How does this sound?

\bigskip

We seem to need the permutation to get round the possibility that the
downward propagation doesn't reliably seem to reach the empty set.  But
perhaps we can show that there is always a way of propagating downwards
so as to reach the empty set.

\section{The Conjectures}

\begin{conjecture}\label{conj:universalexistential1}
Every $\forall^1\exists^*$ sentence refutable in \nf\ is refutable
already in \nf$_2$.
\end{conjecture}

\begin{conjecture}\label{conj:universalexistential2}
Every $\forall^*\exists^*$ sentence refutable in \nf\ is refutable
already in \nf O.
\end{conjecture}

\begin{conjecture}\label{conj:universalexistential3}
$\nf O$ decides all stratified $\forall^*\exists^*$ sentences.
\end{conjecture}

\begin{conjecture}\label{conj:universalexistential4}
Any term model for \nf\ and any model for \nf\ in which all sets are
symmetric satisfies every $\forall^*\exists^*$ sentence consistent
with \nf O.
\end{conjecture}

\begin{conjecture}\label{conj:universalexistential5}
All unstratified $\forall^*\exists^*$ sentences are either decided by
\nf\ or can be proved consistent by permutations.
\end{conjecture}

\begin{conjecture}\label{henkin}
Let us say a {\sl Henkin sentence} is a branching quantifier sentence where
every prefix is  $\forall^*\exists^*$.  Then \TZT\ has a model satisfying
all consistent Henkin sentences.
\end{conjecture}

We cannot strengthen this last conjecture to ``\TZT\ decides all Henkin
formul{\ae}'' beco's there is a Henkin formula that says there is an
external tsau.  And that is true in some models of TST but not all!


\bigskip



\bigskip


Throughout this discussion we will try to keep to the cute mnemonic
habit---due to Quine---of writing a typical universal-existential
sentence with the initial---{universally quantified}---variables as
$\vec y$ (`$y$' for $y$ouniversal) and the existentially quantified
variables as $\vec x$---for E$x$istential).  That was so we can talk
about $y$ variables and $x$ variables.


In earlier versions, conjecture ~\ref{conj:universalexistential3} used
to be ``$\nf_2$ decides all stratified $\forall^*\exists^*$ sentences.


It is known that the term model for \nf O satifies all consistent
$\forall^*\exists^*$ sentences consistent with \nf O.  Putting this
together with conjecture \ref{conj:universalexistential2} suggests
that \nf\ might have a model satisfying all the $\forall^*\exists^*$
sentences consistent with \nf.  (In fact we conjecture that a term
model for \nf\ would be such a model). At the very least it suggests
that the class of $\forall^*\exists^*$ sentences consistent with \nf\
is closed under conjunction.  This also suggests that if conjecture
\ref{conj:universalexistential2} is correct then whenever $\phi$ is a
consistent $\forall^*\exists^*$ sentence consistent with \nf\ then
$\{\pi: \phi^\pi\}$ belongs to some class $\Gamma$ of sets of
permutations that is closed under intersection.  Is $\Gamma$ nicely
defined in terms of a natural topology on the symmetric group on $V$?
It clearly can't mean ``open'' in the usual topology.




\subsection*{A factoid to be fitted in}

Write $D(x)$ for $x \Delta B(x)$.  I think i can show that $D$ has no
finite cycles.  That is to say, we can prove by meta-induction on $n$ that
\begin{rem}
$(\forall x)(D^n(x) \not= x)$.
\end{rem}

\Proof  If $x = D(x)$ then $x = x \Delta B(x)$, which is clearly impossible.
 

Let $\{d_1 \ldots d_n = d_1\}$ be an $n$-cycle where $d_{i+1} = D(d_i)$ for 
$1 \leq i \leq n$.

Now $(\forall x)(x \in d_n \bic (x \in d_{n-1} \bic d_{n-1}\not\in x))$.

But $x \in d_{n-1}$ is the same as $(x \in d_{n-2} \bic d_{n-2} \not\in x)$
so we get 

 $(\forall x)(x \in d_n \bic ((x \in d_{n-2} \bic d_{n-2} \not\in x) \bic d_{n-1}\not\in x))$

and so on getting

 $(\forall x)(x \in d_n \bic ((x \in d_{n-i} \ldots \bic d_{n-2} \not\in x) \bic d_{n-1}\not\in x))$

where the number of negation signs has opposite parity to $n$.  So we end up with


$$(\forall x)(d_1 \in x \bic d_2 \in x \bic d_3 \in x \ldots d_n \in x)$$

which says that for any $x$, an odd number of $d$s are not in, or an
even number are in, depending on the parity of $n$.  Picking $x$ to be
a suitable finite set can bugger this up completely.
\marginpar{Write this out properly}
So $D$ has no finite cycles.\endproof

\medskip

Let's illustrate with an odd $n$ and an even $n$.  Sse $d_4 = d_1$

then

$$(\forall x)(x \in d_1 \bic (d_4 \not\in x \bic (d_3 \not\in x \bic (d_2 \not\in x \bic (d_1 \not\in x \bic x \in d_1)))))$$

which simplifies to 

$$(\forall x)(d_3 \in x \bic (d_2 \in x))$$

or, in plain language $d_3 = d_2$, contradicting our inductive hyp that
$d_2$ and $d_3$ are distinct.

\endproof

\medskip

``$(\forall x)(D(x)$ exists$)$'' is an unstratified $\forall_3$
sentence which, together with extensionality, has no finite models.


We can't show that $D$ is injective, sadly.  After all, if $x = B^2(x)$
we have $D(x) = D(B(x))$.

But the assertion that $D$ is injective is universal-existential.  Is it consistent ?


\section{A note on the first two conjectures}

The background to these conjectures is that \nf O proves all
$\exists^*$ sentences consistent with LPC, and one naturally wants to
speculate about what happens with formul{\ae} with more quantifiers.

Notice that ``every superset of a self-membered set is self-membered''
is a $\forall^*\exists^*$ sentence consistent with \nf$_2$ (it's true
in the term model) that is not consistent with \nf O, so we cannot
strengthen `\nf O' to `\nf$_2$' in conjecture
~\ref{conj:universalexistential2}.


Every $\forall^*\exists^*$ sentence has a canonical normal form. If we
take the disjunction of all possible conjunctions of atomic and
negatomic formul{\ae} built up from all the $x$ and $y$ variables by
means of $\in$ and $=$, then any $\forall^*\exists^*$ sentence can be
put in the form $(\forall \vec y)(\exists \vec x)$ followed by a
disjunction of some of those conjunctions.  

Let us assume this done.  Now suppose we had started with a
$\forall^1\exists^*$ sentence, and put it into this normal form.
There is only one $y$ variable, and every value that it takes either
is or is not a member of itself, so we know that if our
$\forall^1\exists^*$ sentence is to be satisfiable at all then at
least one of its disjuncts must be a conjunction containing the atomic
conjunct `$y \in y$' and at least one of its disjuncts must be a
conjunction containing the negatomic conjunct `$y \not\in y$'.  This
is because (since $V$ is a set) `$y$' might be interpreted by
something that is a member of itself, and (since $\emptyset$ is a set)
`$y$' might be interpreted by something that is not a member of
itself.

Anything else is going to be false in all models of any theory in
which we can prove the existence of $V$ and $\Lambda$.  Also, this
seems to be about all we can do in the way of weeding out formul{\ae}
that are not going to be satisfiable.  Notice that this line of talk
relies only on things we can prove in $\nf_2$. Hence conjecture
\ref{conj:universalexistential1}.

Now let us consider $\forall^2\exists^*$ sentences. We now have to
consider not just the two formul{\ae} `$y \in y$ and `$y \not\in y$'
but the 32 conjunctions we get by assigning truth values to `$y_1 \in
y_1$', `$y_1 \in y_2$', `$y_2 \in y_1$', `$y_2 \in y_2$' and `$y_1 = y_2$'.

Now in any set theory in which we can find objects satisfying, for
example, $t_1 \in t_1 \wedge t_1 \not\in t_2 \wedge t_2 \not\in t_1
\wedge t_2 \in t_2$ we can argue that if a $\forall^2\exists^*$ is to
be satisfiable at all then at least one of its disjuncts must be a
conjunction containing $y_1 \in y_1 \wedge y_1 \not\in y_2 \wedge y_2
\not\in y_1 \wedge y_2 \in y_2$, because otherwise it could be
falsified in any model by interpreting each `$y_i$' by $t_i$.  Such a
theory is \nf O. As before, this seems to be the only thing we can do
to weed out formul{\ae} that are not going to be satisfiable, so the
corresponding conjecture for $\forall^2\exists^*$ sentences will be
that every $\forall^2\exists^*$ sentence refutable in \nf\ is
refutable in \nf O.  As it happens, \nf O proves every consistent
$\exists^*$ sentence so we do not need to reach for more complicated
theories when considering $\forall^3\exists^*$ sentences. This is why
conjecture ~\ref{conj:universalexistential2} takes the form that it does.  


We can prove that every $\forall^*\exists^*$ sentence consistent with
\nf O is true in the term model of \nf O. (This is proved in the book
somewhere).  What about \nf$_2$?  There is a complication with
$\nf_2$, namely that the term model doesn't satisfy the $\exists^*$
sentence $(\exists x_1 x_2)(x_1 \in x_1 \not\in x_2 \in x_2 \not\in
x_1)$.  So it isn't true that the term model for $\nf_2$ satisfies
every consistent $\forall^*\exists^*$ sentence.  (I think it proves
that, given two self membered sets, one is a member of the other)

OTOH, we do get this:  

\begin{rem}
The term model for $\nf_2$ satisfies every $\forall^*\exists^1$
sentence consistent with $\nf_2$.
\end{rem}
\Proof

Let $(\forall \vec y)(\exists x)\Phi$ be a $\forall^*\exists^1$
 sentence consistent with $\nf_2$.  Then for every vector $\vec t$ of
 terms there is an $x$ such that $\Phi$, so all we have to do is
 establish that such a witness can be found among the terms.

$(\forall \vec y)(\exists x)\Phi$ is satisfiable, so fix a model in
which it is true. (It doesn't matter which one, as the term model is
unique, and embeds in all models) Express $\Phi$ in DNF, and fix the
vector $\vec t$.  One of the disjuncts is true.  So there is a witness
$x$ which has certain $t$s as members, is distinct from certain other
$t$s (if there is a clause requiring it to be equal to a $t$ then we
are done) lacks certain other $t$s, and belongs to a final $t$.  This
last simplification arises beco's a finite conjunction of things like
$x \in t$ and $x \not\in t$ is equivalent to one of them, complements
and intersections of $t$s being $t$s.  If this final $t$ is a low set
then the witness is already a term. If it isn't, then we are looking
inside a cofinite set for a set satisfying conditions which exclude
only a moiety of sets.  So there must be a witness. \endproof


This certainly won't work for $\forall^*\exists^2$ sentences
consistent with $\nf_2$, since we might be trying to find two
witnesses $x_1$ and $x_2$ satisfying $x_1 \in x_1 \not\in x_2 \in x_2
\not\in x_1$: after all $(\exists x_1 x_2)(x_1 \in x_1 \not\in x_2 \in x_2
\not\in x_1)$ is a $\forall^*\exists^2$ sentence consistent with
$\nf_2$!


\subsection{??!}

I now think this this is gibberish and i wonder what i was on about.   
What sense there might have been is something along the following lines.  
Every $\forall^*\exists^*$ sentence is a conjunction of things of the form

$$(\forall \vec y)(A(\vec y) \to (\exists \vec x)(B(\vec x, \vec y)))$$

where $A$ is a conjunction of $\in$ and $\not\in$ between the $\vec y$ and in $B$ all atomics involve at least one $x$.


The point is that if there is more than one $y$ we can get $A$ to
describe a finite structure that is not a substructure of the term
model for \nf$_2$, which means that any $\forall^*\exists^*$ sentence
built up using that $A$ is trivially true in the term model for
nf$_2$.

But we might be working our way back.  Suppose 
$(\forall \vec y)(A(\vec y) \to (\exists \vec x)(B(\vec x, \vec y)))$ 
is a $\forall^*\exists^*$ sentence refutable in \nf$_2$.  Then $A$ must
describe a substructure of \ldots

\section{A note on conjecture ~\ref{conj:universalexistential2} and conjecture ~\ref{conj:universalexistential3}}


%I think i can prove conjecture ~\ref{conj:universalexistential3}.

The {\bf finitely generated} models of \TST O are those whose type 0 has only finitely many atoms.

A partition $\Pi$ of a set $X$ is a subset of $\pow X)$ such that $\bigcup
\Pi= X$ and the members of $\Pi$ are pairwise disjoint.  If $\Pi_1$ and
$\Pi_2$ are two partitions of the same set we say $\Pi_1$ {\bf refines}
$\Pi_2$ if every piece of $\Pi_1$ is a subset of a piece of $\Pi_2$.  A
subset $X' \subseteq X$ {\bf crosses} another subset $p \subseteq X$ if $X'
\cap p$ and $X' \setminus p$ are both nonempty.  (That is to say, $X'$ is
not in the field of sets generated by $\Pi$ if $X'$ crosses a piece of
$\Pi$).

We first prove that every countable model of \TST O is a direct limit of
all the finitely generated models of \TST O.  (The ``all" is important.)

We do this by induction on the number of types.  For reasons which will
become clear we will regard the finitely generated models as starting with
a base type $T_1$ with ${2^n}$ elements and a boolean algebra structure
rather than starting with a base type $T_0$ with $n$ elements and no
structure.  In effect we forget about the bottom type. So the thing we are
going to prove by induction on $k$ is that every countable model of $\TST
O_k$ is a direct limit of {\bf all} finitely generated models of $\TST O_k$.

For the base case we prove that every countable atomic boolean algebra $\B$
there is a family $\B_i: i \in \Nn$ of subalgebras of $\B$ where $\B_i$ has
$i$ atoms, where the inclusion map is a boolean homomorphism and the union
$\bigcup_{i \in \Nn}\B_i$ is $\B$.

We obtain $\B_{i+1}$ from $\B_i$ by splitting one of the $i$ atoms into
two, effectively adding two new atoms.  To decide which atom to split, and
how to split it, depends on how we wellorder $\B$. We have a fixed
wellordering of $\B$ to order type $\omega$.  At stage 0 we consider $\B_0$
which of course is just the two element boolean algebra containing the top
element and the bottom element.  We make $x_1$ an atom and set $\B_1$ to be
the four element boolean algebra with $x_1$ and $V\setminus x_1$ as atoms.


Thereafter at any stage we have two things in hand. (i) a
most-recently-constructed algebra $\B_i$ and (ii) an $x_k$ which is to be
an element of an algebra soon to be constructed. (Notice that $i$ and $k$
are not assumed to be the same! In general $i$ is likely to be much bigger
than $k$.)

The set of atoms of $\B_i$ that we have is simply a partition of the atoms
of $\B$ into $i$ pieces. At stage $k$ we consider $x_k$.  $x_k$ will be a
superset of some atoms and disjoint from others. These we do nothing to.
The remaining atoms it crosses.  The atoms are ordered by the
canonical worder of $\B$.  Suppose for example $x_k$ crosses five of the $i$
atoms of $\B_i$, to wit: $c$, $d$, $e$, $f$, $g$ in order.  Then we obtain succesively
$\B_{i+1}$ by splitting $c$ into $c \cap x_k$ and $c \setminus x_k$; then 
$\B_{i+2}$ by splitting $d$ into $d \cap x_k$ and $d \setminus x_k$; then
$\B_{i+3}$ by splitting $e$ into $e \cap x_k$ and $e \setminus x_k$; then
$\B_{i+4}$ by splitting $f$ into $f \cap x_k$ and $f \setminus x_k$; and finally
$\B_{i+5}$ by splitting $g$ into $g \cap x_k$ and $g \setminus x_k$.

What has this achieved?  We now have constructed our sequence of
subalgebras as far as $\B_{i+5}$ and we have ensured that $x_k$ is in the
direct limit.  By iterating this we will eventually ensure that every
element of $\B$ appears, so the direct limit of the sequence of subalgebras
generated in this way is $\B$.

The induction step is similar but messier.

   Let $\B$ be a countable atomic boolean algebra which is the union of
$\tuple{\B_n : n < \omega }$: a $\subseteq$-nested sequence of finite
subalgebras of $\B$. Let $\B^+$ be a countable atomic subalgebra of $\pow
\B)$ containing all singletons. Then there is a sequence $\tuple{\Pi_i : i <
\omega}$ of finite partitions of $\B$ such that $\Pi_0$ is the trivial partition with only one piece
and for each $i \geq 1$,

\medskip

$\Pi_i$ refines $\Pi_{i-1}$.
$\Pi_i \subseteq {\B^+}$

$\B_i$ is a selection set for $\Pi_i$.

\medskip

Further, if we let $\B^+_i$ be the subalgebra of $\B^+$ whose atoms are the
pieces of $\Pi_i$ (so that $\pow \B_i) \simeq \B^+_i$) then the union of the
$\B^+_i$ is ${\B^+}$.  As before we have a well-ordering $\tuple{x_n : n <
\omega }$ of ${\B^+}$.

We construct the sequence of partitions by recursion. As noted above, $\Pi_0$
is the trivial partition with only one piece.  Thereafter we procede as follows.

Suppose we have constructed partitions up to $\Pi_{i-1}$, and we have $x_j$
in hand, where $x_j$ is the first element of $\B^+$ (in the sense of the
canonical ordering) not already a finite union of pieces of $\Pi_{i-1}$.
We seek a refinement $\Pi_i$ of $\Pi_{i-1}$ such that each piece of $\Pi_i$
contains precisely one element of $\B_i$ and such that $x_j$ is a union of
pieces of $\Pi_i$.

   How are we to subdivide the pieces of $\Pi_{i-1}$ to get pieces of
$\Pi_i$?  Clearly whenever $x_j$ extends, or is disjoint from, a piece of 
$\Pi_{i-1}$ then we do not need to subdivide that piece in order to get 
pieces for $\Pi_i$ such that $x_j$ is a union of some of them.  However, 
if $x_j$ crosses a piece $p$ of $\Pi_{i-1}$ we need to take steps. There 
are other things that may cause us to subdivide $p$ and that is the need 
to ensure that every member of $\Pi_i$ contains precisely one element of 
$\B_i$.  If $p$ meets $x_j$ and $p \cap x_j$ contains elements from
$\B_i \setminus \B_{i-1}$ then we can partition $p$ into pieces each of which
contains precisely one element of $\B_i \setminus \B_{i-1}$.  Naturally if we can
do this for every piece that meets $x_j$ or its complement success is assured.

However there remains the possibility that $p$ crosses $x_j$ but that $p
\cap x_j$ contains {\sl no} elements from $\B_i \setminus \B_{i-1}$.  This
is grave, because then there is no means of partitioning $p$ into pieces
each of which contains precisely one element of $\B_i$ and whose union is
$p \cap x_j$ (and this of course excludes the possibility of refining
$\Pi_{i-1}$ into a partition every piece of which contains precisely one
element of $\B_i$ and such that $x_j$ is a union of
pieces).

This means that in these circumstances we have to lower our ambitions.  It
has turned out to be too much to expect $x_j$ to be a union of pieces of
$\Pi_i$ but we can expect to be able to make it a union of pieces of
$\Pi_{i+k}$ for some finite $k$.  That will be sufficient, because that way
every $x_j$ will get used up eventually, but can it be done?  We have to go
on throwing in elements of $\B_i$, $\B_{i+1}$, $\B_{i+2}$\ldots $\B_{i+k}$,
until $p \cap x_j$ and $p \setminus x_j$ both meet $\B_{i+k}$.  But
this must happen sooner or later because $\B$ is a union of all the $\B_i$
so any subset of $\B$ (such as $p \cap x_j$) must meet cofinitely many of
them.

   So, to sum up, the step from $\Pi_{i-1}$ to $\Pi_i$ is made with an
$x_j$ in mind.  If we can refine $\Pi_{i-1}$ in such a way that every piece
of the new partition contains precisely one element of $\B_i$ and $x_j$ is
a union of the new pieces, well and good.  Set $\Pi_i$ to be the new
partition and worry next about $x_{j+1}$.  If we cannot do this, we can at
least refine $\Pi_{i-1}$ in such a way that every piece of the new
partition contains precisely one element of $\B_i$, and we call that
$\Pi_i$.  We then attempt the same, starting this time with $\Pi_i$ and
continuing to worry about $x_j$.

\endproof


Every countable model of \TST\ is a direct limit of {\bf all} finitely
generated models of \TST.

Let $\M$, a countable model of simple type theory, have as its domain a
family $\tuple{\B_n : n < \omega }$ of countable atomic boolean algebras,
where $\B_{n+1}$ is a countable atomic subalgebra of $\pow\B_n)$.  Let
$\B_1$ be a union of an $\omega$-sequence $\tuple{\B_1^i : i < \omega }$.
We then invoke the induction step above repeatedly to obtain, for each $n$,
families $\tuple{\B_n^i : i < \omega }$ of subalgebras and $\tuple{\Pi_n^i
: i < \omega }$ of partitions as above. Now for each $i < \omega$ consider
the structures $\tuple{\tuple{\B_n^i : n < \omega }, \in }$.  We have
constructed the $\B_n^i$ so that $\B_n^{i+1}$ is an atomic boolean algebra
whose atoms are elements of a partition for which $\B_n^i$ is a selection
set. Thus, if we want to turn the $\tuple{\B_n^i : n < \omega }$ into a
model of simple type theory the obvious membership relation to take is
$\in$ itself. They are models of simple type theory without the axiom of
infinity, and by construction their direct limit is pointwise the $n$th
type of $M$, so the direct limit is $M$ as desired.  \blob\footnote{This
suggests that the obvious product topology on the space of countable models
of \TST\ might be useful \ldots}

There is an obvious modification for \TZT.  Every countable model of \TST\
is a direct limit of all finitely generated models of \TST\ and so is
certainly a direct limit of $\M_1$, $\M_2$ \ldots $\M_n$ where $\M_1$ is
the canonical model where $T_0$ has one element and $\M_{n+1}$ is the
result of deleting the bottom type off $\M_n$ and relabelling.  Now let
$\N$ be an arbitrary countable model of \TZT, and consider a terminal
segment of it.  We have just shown that this terminal segment is a direct
limit of the $\M_n$.  It is a simple exercise to extend this network of
embeddings downwards \ldots

This tells us that every countable model of \TZT\ is a direct limit of
an $\omega^*$ sequence of copies of $\M_1$.

The missing link is a proof of the assertion that there is no
universal-existential sentence (in the language of boolean algebra or
perhaps set theory) which has infinite models but no finite models.

The intention is that once we have this we wrap up the proof as follows.

Let $\phi$ be a existential-universal sentence with an infinite model.
Therefore it has a countable model. Then $\neg \phi$ cannot be true in
arbitrarily large finitely generated models because otherwise $\neg \phi$
would be true in all countable models.  So $\phi$ is true in all suff large
finitely generated models, say all models with at least $n$ atoms.


 If $\neg \phi$ has an infinite model so does the expression ``$\neg \phi
\wedge$ there are at least $n$ atoms". (Indeed they have the same infinite
models!)  But this expression has no finite models.  But unless $\phi$ is
true in all countable models, ``$\neg \phi \wedge$ there are at least $n$
atoms" is an example of a universal-existential sentence with an infinite
model but no finite models.

So if there is no universal-existential sentence (in which language?) which
has infinite models but no finite models then \TST\ decides all
universal-existential sentences.

\subsection{Some other observations that might turn out to be helpful}


First prove that if $(\forall \vec y \in V_\omega)(\exists \vec
x)\Phi(\vec x ,\vec y)$ is a universal-existential sentence consistent
with \TST\ then its type-free version is true in some transitive
wellfounded model of \kf.  Next prove that if $(\forall \vec y \in
V_\omega)(\exists \vec x)\Phi(\vec x, \vec y)$ is a
universal-existential sentence true in some transitive wellfounded
model of \kf then it is true in $V_\omega$.  (This ought to be true
beco's every transitive wellfounded model of \kf is an end-extension
of $V_\omega$---also rud functions increase rank by only a finite
amount may 1998).  Then we argue that if $(\forall \vec y \in
V_\omega)(\exists \vec x)\Phi(\vec x ,\vec y)$ is a stratified
universal-existential sentence true in $V_\omega$ then its typed
version has a finitely generated model.

Do we mean true at one type or true at all types?

  Suppose $(\forall \vec y \in V_\omega)(\exists \vec x)\Phi(\vec x, \vec
y)$.  Assume that $\Phi$ is stratified and is in disjunctive normal form.

Since $\Phi$ is stratified there is a stratification of its variables.
This suggests an obvious conjecture.  If `$u$' is of type $n$ why not
restrict the quantifier binding `$u$' to $V_n$ and hope the result to be
true?  How might this go wrong?  One obvious way is exemplified by the
following formula: $$(\forall y)(\exists x_1 \ldots
x_{10^{10}})(\bigwedge_{1 \leq i < j \leq 10^{10}} (x_i \not= x_j \wedge
(y \in x_i \wedge y \in x_j)))$$

This is only going to be true at sufficiently high types.  What we have to
establish is that this is the {\sl only} way things can go wrong.

First, by reasoning in ZF or Zermelo plus foundation, we argue that every
universal-existential sentence true in $V$ is true in $V_\omega$.

Pause briefly to think about the graph of $\Phi$, by which i mean the
digraph whose vertices are variables with a directed edge from `$y$' to
`$x$' if `$y \in x$' occurs somewhere.  If this digraph has no loops
involving `$x$' variables we procede as follows, making use of the obvious
rank function on `$x$' variables which is available in these circumstances.

Instantiate all the `$y$' variables to names of individual hereditarily
finite sets.  We can import the existential quantifiers past the
disjunctions so that each disjunct is now a string of existential
quantifiers outside a conjunction of atomics and negatomics.  At least one
of these disjunctions is true: grab it. We now want to find witnesses
for---{\sl instantiate}---the `$x$' variables bound by the existential
quantifiers leading that disjunct, and we want to find these inside
$V_\omega$.  (Notice that at this stage we can assume there are no positive
occurrences of `=' within this disjunct, because `$(\exists u)(\exists
v)(\ldots u = v \ldots)$ can be rewritten to remove one of the two
variables.) Some of these variables ``point to" `$y$' variables in the sense
that there is a directed edge from them to one or more `$y$' variables.
Such `$x$' variables must be instantiated by hereditarily finite sets if
they can be instantiated at all, and we know they can be so instantiated
because we are assuming that $(\forall \vec y \in V_\omega)(\exists \vec
x)\Phi(\vec x \vec y)$. So instantiate them all simultaneously with a tuple
of witnesses in virtue of which we knew that particular instance of
$(\forall \vec y \in V_\omega)(\exists \vec x)\Phi(\vec x, \vec y)$.

Now to instantiate the remaining `$x$' variables. A witness for this sort
of variable must have certain given things as members and certain other
things not as members, so why not simply take it to be the set of things
that it has to have as its members?  Because we might end up thereby
instantiating both `$x_1$' and `$x_2$' to $\{a_1,a_2,\{\emptyset\}\}$, say,
while elsewhere in the formula we are trying to make $x_1 \not= x_2$ true,
so sometimes we have to add silly elements to things to make them
different.  

We then continue by recursion on the rank of `$x$' variables.

This tells us that any true stratified universal-existential expression in
the language of set theory is true in $V_\omega$.

In this connection it may be worth noting that every model of \tst O
${\cal P}$-extends every finitely generated model, so any $\SiP_1$
sentence true in even one finitely generated model is true in all
infinitely generated models.

It is almost certainly time to use the theorem of Ramsey that says 
that there is a decision proceedure to establish whether or not an arbitrary
$\Pi_1$ sentence has an infinite model. Ramsey claims a generalisation to
$\Sigma_2$ formul{\ae}.

The following remark probably belongs here:

\begin{rem}  $\TZT \vdash Amb(\Sigma_1^{Levy}$)
\end{rem}
                                     
\Proof   It falls into two cases 

\begin{enumerate}
\item

All models of $\TZT + \neg$ AxInf satisfy Amb($\SiP_1$). For any
$\SiP_1$ sentence $\Phi$ either it is false in all finitely generated
models or there is an $n$ such that it is true in all models bigger
than $n$.  This $n$ is standard if the G\"odel\index{G\"odel} number
of $\Phi$ is. So if $\M \models \TZT \wedge \neg$ AxInf then $|M|$ is
non-standard finite, so bigger than all the $n$. This shows that {\sl
all} models of $\TZT + \neg$ AxInf satisfy the {\sl same} $\SiP_1$
sentences.

\item

Now consider $\M$ such that $\M \models$ AxInf. $\M$ and $\M^*$ have
the same integers because Specker\index{Specker}'s $T$ function is an
isomorphism as long as each universe is at least countable.  This
shows that any model of $\TZT +$ AxInf must have the same arithmetic at
each type.  We will need this and the fact that$\Sigma^{L\acute{e}vy}_1$ 
sentences generalise upward. The next step is to show that if 
$Th(\M) \vdash Con(\phi)$ where $\phi$ is $\Sigma^{L\acute{e}vy}_1$ in 
the language of $\TZT$ then $M$ contains an $\in$-model of $\phi$.  
First we prove in the arithmetic of $Th(\M)$ that $\phi + Ext$ has a 
model $N$ in the integers.  The elements of this model have a type 
discipline in a natural way, and only finitely many types are mentioned. 
We construct an $\in$-model $N'$ essentially by a Mostowski collapse as 
follows: the elements of minimal (internal)
type are the same as they were in $N$, namely particular integers at
(external) type $k$, or whatever.  The objects of (internal) type 1 in
$N'$ are to be the appropriate sets of things of (internal) type 0,
and these will of course be of (external) type $k+1$. And so on, for
finitely many types.  Note that this construction cannot work for
$\SiP_1$! Thus $\TZT +$ AxInf $\vdash Con(\phi) \ \to \ \TZT +$ AxInf
$\vdash \phi$ for $\phi \in str(\Sigma^{L\acute{e}vy}_1)$, in slang
$\TZT +$ AxInf reflects $\Sigma^{L\acute{e}vy}_1$ sentences.

Next we need a converse. Suppose $\phi \in \Sigma^{L\acute{e}vy}_1$ is
true at some level of $M$.  Therefore $\phi$ has a model and this
model can in fact be coded by a set of $\M$.  Therefore $Th(\M)$ knows
that $\phi + Ext$ is a consistent theory.  This allegation is
expressible in the arithmetic of $\M$ and so $Th(\M) \vdash Con(\Phi)$.
\end{enumerate}\endproof

\section{Conjecture ~\ref{conj:universalexistential5}: finding permutation models}

Given a $\forall^*\exists^*$ sentence $S$, import all the $\exists$'s
and export all the $\forall$'s.  The result is a formula with $\forall
\vec y$ outside a conjunction of implications each of the form $$Y \to
(\exists \vec x)(\phi(\vec x, \vec y))$$ where $\phi$ is a boolean
combination of atomics and negatomics each one containing an $x$
variable, and $Y$ is of the form $$(\bigwedge_{\tuple{i,j} \in J
\subseteq I^2} y_i R y_j)$$, where the `$R$' is either `$\in$' or
`$\not\in$'.  The disjunction of all the $Y$s must be valid, since
every consistent $\exists^*$ formula of LPC is a theorem of \nf.  We
can now export the conjunctions, and this shows that $S$ is a
conjunction of things of the form $$(\forall \vec
y)((\bigwedge_{\tuple{i,j} \in J \subseteq I^2} y_i R y_j) \to
(\exists \vec x)(\phi(\vec x, \vec y))$$


Now a conjunction of two formul{\ae} of this form is another formula
of this form.  This means that without loss of generality we need
consider only formul{\ae} of this form.

Can we restrict attention even further to $\forall^*\exists^*$
sentences of this form where the consequent is the existential closure
of a {\bf conjunction} of atomics and negatomics rather than a boolean
combinations? Sadly, no.  Consider $y_1 \in y_1$ and $y_2 \not\in y_2$.
There is something in $y_1 \Delta y_2$ but is it in $y_1 \setminus
y_2$ or in $y_2 \setminus y_1$?  No reason to suppose either.  But
perhaps if we supply more information, about whether or not $y_1 \in
y_2$ and $y_2 \in y_1$ then we might be able to cut down to a single
disjunct.

[This problem is nothing to do with these things being unstratified: 
the same happens with $y_1 \in y_2 \wedge y_3 \not\in y_2$ There is 
either something in $y_1 \setminus y_3$ or something in 
$y_3 \setminus y_1$ but we don't know which.]

That this is not true is shown by the following case.

$(\forall y_1 y_2)(y_1 \not\in y_1 \wedge y_2 \in y_2 \wedge y_2 \in y_1 \wedge y_1 \in y_2 \to
(\exists x)(x \in y_1 \wedge x \not\in y_2) \vee (\exists x)(x \not\in y_1 \wedge x \in y_2))$

This is provable beco's of extensionality, but neither $$(\forall y_1
y_2)(y_1 \not\in y_1 \wedge y_2 \in y_2 \wedge y_2 \in y_1 \wedge y_1
\in y_2 \to (\exists x)(x \not\in y_1 \wedge x \in y_2))$$ nor
$$(\forall y_1 y_2)(y_1 \not\in y_1 \wedge y_2 \in y_2 \wedge y_2 \in
y_1 \wedge y_1 \in y_2 \to (\exists x)(x \in y_1 \wedge x \not\in
y_2))$$ are provable because we can find $y_1$ and $y_2$ satisfying the
antecedent with $y_1 \subseteq y_2$ and $y_1$ and $y_2$ satisfying the
antecedent with $y_2 \subseteq y_1$.  Try $y_2 := \bbar{V};\ y_1 :=
\{\bbar{V}\}$ for the first case and $y_2 := \bbar{V};\ y_1 :=\bbar{V} \cup \{V\}$ for the second.

{\bf Anyway} the idea now is that all we have to do to prove the
$\forall^*\exists^*$ conjecture is to show how to get a permutation
model of anything of the form $$(\forall \vec
y)((\bigwedge_{\tuple{i,j} \in J \subseteq I^2} y_i \in y_j) \to
(\exists \vec x)(\phi(\vec x, \vec y))$$ as long as it's consistent
with \nf O.  But is it not the case that every $\forall^*\exists^*$
sentence consistent with \nf 0 is true in the term model for \nf 0?

That suggests considering only permutations that leave \nf O
terms alone, since they have witnesses anyway!

We can't just move things that aren't \nf O terms, since being an \nf
O term is not stratified, so we have to move things are not
``sufficiently like'' \nf O terms.  The idea is that anything
sufficiently like an \nf O term will satisfy the $\forall^*\exists^*$
formula we have in mind at any one time, where ``sufficiently alike''
depends on the formula in question.  So we consider only those
permutations that, say, swap with their complements those things that
are not \nf 0 terms of rank at most $k$ for some concrete $k$.


   Illustrate this by thinking about the assertion that there are no
Boffa atoms.  What witness is there?  


There is something very odd about the case 


$(\forall y_1 y_2)(y_1 \not\in y_1 \wedge y_2 \in y_2 \wedge y_2 \in y_1 \wedge y_1 \in y_2 \to
(\exists x)(x \in y_1 \wedge x \not\in y_2) \vee (\exists x)(x \not\in y_1 \wedge x \in y_2))$

The point is that this is true not because of the behaviour of \nf O
terms, but because of extensionality and classical logic.  There is no
reason to suppose that the witnesses will be easy to find.


\section{Positive results obtained by permutations}
%\section{Results positive for conjecture ~\ref{conj:universalexistential5} obtained by permutations}


Many of these are published, and collected in Forster [1991].  Here are
some new ones.

\subsection{The size of a self-membered set is not a concrete natural}

Boffa has made some progress on this front.  He has proved that, if
the axiom of counting holds, there is a permutation $\pi$ such that in
$V^\pi$ there is no self-membered finite set.  A little adjustment
strengthens the conclusion and weakens the assumption slightly.

\begin{rem}\label{ref:boffapermutation}
If \nf+\AxC\ is consistent so is \nf+\AxC+ ``Every self-membered set has a countable partition''.
\end{rem}

\Proof
Let $X$ be the collection of sets that lack a countable partition.  If
$x$ is such a set, then the set of $n \in \Nn$ such that $\{n\} \times
V$ meets $x$ is finite, and will have a last member.  Add 1 to this
last member to get a number we will call $n_x$. $n_x$ has the feature
that $(\forall m \geq n_x)((x \cap (\{m\} \times V)) = \Lambda)$.  $Tn_x$
is the same type as $x$ and so the permutation $$\prod_{x \in
X}(x,\tuple{Tn_x,x})$$ is a set.  Notice that if $x \in X$
then $\tau`x$ is infinite and not equal to $x$.

Now suppose $x \in \tau`x$. To prove that in $V^\tau$ every
self-membered set is infinite it will suffice to show that $\tau `x$
is infinite. We will assume \AxC\ and prove that $\tau `x$ has a
countable partition. 

If $x$ is fixed then $x$ is infinite so $\tau`x$ (which is $x$) is
infinite as desired.  If $x$ is not fixed there are two cases to
consider.

(i) $x \in X$.  Then $\tau`x$ is infinite by construction.

(ii) $\tau`x \in X$.  Then $x = \tuple{Tn_{\tau`x},\tau`x}$. But also
$x \in \tau`x$ so $\tuple{Tn_{\tau`x},\tau`x} \in \tau`x$.  Now
$n_{\tau`x}$ has been chosen to be so large that no ordered pair
$\tuple{m, y}$ is a member of $\tau`x$ for any $ \geq n_{\tau`x}$.  So, to get
a contradiction all we need is $Tn_{\tau`x} \geq n_{\tau`x}$.  The simplest way
to get this is to assume \AxC.

\endproof

(Originally Boffa had taken $n_x$ to be the {\sl first} $n$ s.t.
$\{n\} \times V$ does not meet $x$.  That way he needs the whole of
the axiom of counting.)  Friederike K\"orner and i both noticed that
to make this proof work it is sufficient to have a (set) function $f:
\Nn \to \Nn$ such that $(\forall n)(f(Tn) \geq n)$.  I propose to
call such functions {\bf K\"orner functions}. If we have such a
function we swap $x$ (when $x$ is finite) with $\tuple{f(Tn_x),x}$
instead of $\tuple{(Tn_x),x}$.  Indeed in those circumstances we can
do something even better.

\begin{rem}

If there is a function $f: \Nn \to \Nn$ such that $(\forall n)(f(Tn) \geq
n)$ then (letting $\pi$ be the permutation 

$$\prod_{|x| \in \smallNn}(\tuple{f(Tn_x),x},x)$$

that swaps $x$ with
$\tuple{f(Tn_x),x}$ for $x$ finite) we find that in $V^\pi$ the 
membership relation restricted to finite sets is wellfounded.

\end{rem}

\Proof 

Suppose $V^\pi \models x \in y \wedge |x|\in \Nn \wedge  |y| \in \Nn$.
Then $\pi(x)$ and $\pi(y)$ are both finite and $x \in \pi(y)$.  We will show
$n_{\pi(x)} < n_{\pi(y)}$.  Since $\pi(x)$ is finite, $x$ must be
$\tuple{f(Tn_{\pi(x)}), \pi(x)}$.  But then, since $x \in \pi(y)$, the first
component of $x$ must be less than $n_{\pi(y)}$, so $f(Tn_{\pi(x)}) <
n_{\pi(y)}$.  But we have $n_{\pi(x)} < f(Tn_{\pi(x)})$ by choice of $f$ so
$n_{\pi(x)} < n_{\pi(y)}$ as desired.  \endproof

(In fact we can swap $x$ and $\tuple{x,f(Tn_x)}$ as long as $x$ does
not map onto \Nn. So we can set $\pi := \prod (x,\tuple{x,f(Tn_x)})$
taking $(x,\tuple{x,f(Tn_x)})$ to be the identity if $n_x$ is
undefined.)

Friederike K\"orner then showed that it is consistent relative to \nf\ that
there should be $n \in \Nn$ such that for all greater $m$ we have $m < Tm$,
and that means there is such an $f$, namely $\lambda x.($if $x < n$ then
$n$ else $x)$. Let us call natural numbers $k$ s.t. 
$(\forall n \in \Nn)((n+k) < T(n+k))$ {\bf K\"orner numbers}.

The significance of K\"orner numbers is that if there is a K\"orner
number then there is a K\"orner function, a function $f:\Nn\to \Nn$
such that $(\forall n \in \Nn)(n \leq f`Tn)$.  The existence of such a
function commuting with $T$ is of course equivalent to \AxC, but this
is weaker, and implies that there is a permutation model in which $\in
\restric FIN$ is wellfounded (which indeed was how we found it!).  Given the
desire to find cardinal arithm\'etic equivalents for all modalised sentences
it is natural to try to find a converse \ldots

\begin{rem}
If \nf\ is consistent so is \nf\ + ``No strongly cantorian set is self-membered''.
\end{rem} 
\Proof

For $\alpha \in T``NO$ set 

\begin{itemize}
\item $F(\alpha, x) = \{u \in x: (\exists y)(u = \tuple{V,T^{-1}\alpha,y})\}$

\item $\mu(x) = $ the least $\alpha \in T``NO$ such that $F(\alpha,x) = \emptyset$ if there
      is one, = $V$ otherwise.
\end{itemize}

Note the following:
\begin{enumerate}
\item $stcan(x) \to (\exists \alpha \in T``NO)(F(\alpha,x) = \emptyset))$;
\item If $stcan(x)$ then $\mu(x)$ is a strongly cantorian ordinal;
\item For all $x$, $(\forall y)(\tuple{V,T^{-1}(\mu(x)),y} \not\in x)$.
\end{enumerate}


\marginpar{If we alter the definition of $\mu$ so it picks up the sup
of the nonempty $F$s rather than the first empty $F$ we have to be
sure that (ii) remains true.  It will be true if only strongly
cantorian ordinals can have strongly cantorian cofinality.  But perhaps
that's not even plausible....}


Then set $$\pi = \prod_{x \not\in |V|}(x, \tuple{V, \mu(x),x})$$


I now think that---assuming that this works at all---it establishes
that $\in$ restricted to strongly cantorian sets is wellfounded.  To
that end, suppose $V^\pi$ believes that $x$ is a member of $y$ and
both are strongly cantorian.  We will show that $\mu(\pi(x)) <
\mu(\pi(y))$

So $\pi(x)$ and $\pi(y)$ are both strongly cantorian and therefore
cannot be nasty ordered triples.  So it is $x$ and $y$ that are the
nasty triples, and we must have $x = \tuple{V,\mu(\pi(x)),\pi(x)})$
and $y = \tuple{V, \mu(\pi(y)),\pi(y)})$

We also have $x \in \pi(y)$, which is to say that the triple $x =
\tuple{V,\mu(\pi(x)),\pi(x)})$ is one of the triples in $\pi(y)$.  

We want $\mu(\pi(x)) < \mu(\pi(y))$.  $\pi(x)$ is strongly cantorian,
so $\mu(\pi(x))$ is a strongly cantorian ordinal.  \marginpar{I think
  we have to modify the definition of $\mu(x)$ to be the sup of
  nonempty $F$s rather than the first nonempty one \ldots}


This will work as long as $cf(\Omega)$ is not strongly cantorian.  In
fact i suspect that it will show that membership restricted to small
sets is wellfounded as long as $cf(\omega)$ is not small.


\endproof



So let's try to generalise the Boffa-P\'etry construction

\smallskip

For $\alpha \in T``NO$ set 

\begin{itemize}
\item $F(\alpha, x) := \{u \in x: (\exists y)(u = \tuple{V,T^{-1}\alpha,y})\}$

\item $\mu(x) := $sup$\{\alpha + 1 \in T``NO: F(\alpha,x) = \emptyset\}$ 
if this sup is defined, = $V$ otherwise.
\end{itemize}

Note the following:
\begin{enumerate}
\item If $x$ is small then $\mu(x)$ is not $V$;
\item For all $x$, $(\forall y)(\tuple{V,T^{-1}(\mu(x)),y} \not\in x)$.
\end{enumerate}

Then set $$\pi = \prod_{x \not\in |V|}(x, \tuple{V, \mu(x),x})$$

(Perhaps we don't need to swap everything smaller than $V$: it may be
that swapping only small things will do; but we shall see.)


We shall attempt to show that, in $V^\pi$, $\in$ restricted to small
sets is wellfounded.  So let $x$ and $y$ be such that $V^\pi$ believes
$x \in y$ and that both $x$ and $y$ are small.  We will (we hope)
infer from this that $\mu(\pi(x)) < \mu(\pi(y))$.

Assuming that smallness is a property preserved under surjection we
know that $V^\sigma$ believes $x$ to be small iff $\sigma(x)$ was
small in $V$.  So in this context we infer that $\pi(y)$ and $\pi(x)$
are both small and so cannot be nasty ordered triples.  So it is $x$
and $y$ that are the nasty triples, and we must have $x =
\tuple{V,\mu(\pi(x)),\pi(x)})$ and $y = \tuple{V,
  \mu(\pi(y)),\pi(y)})$

We also have $x \in \pi(y)$, which is to say that the triple $x =
\tuple{V,\mu(\pi(x)),\pi(x)})$ is one of the triples in $\pi(y)$.  
from which $\mu(\pi(x)) < \mu(\pi(y))$ is immediate \endproof

So we seem to have shown that: 
\begin{quote} if $cf(\Omega)$ is not small, then
$\poss(\in$ restricted to small sets is wellfounded$)$.
\end{quote}  

There doesn't seem to be anything special about the choice of $\Omega$ here

\bigskip


It's worth remembering that in Boffa's original construction $\mu$
picks up the first empty $F$ rather than the sup of the nonempty $F$s.
I thought this was wasteful but actually the difference between his
definition and my modification of it is the same as the difference
between the definition of grundyrank on a wellfounded structure and
the definition of rank, so it might be something natural and meaningful.


\medskip

H I A T U S

\medskip

It seems that we should be able to do better than this. Suppose there
is a function $f: NO \to NO$ such that $(\forall \alpha)(f(T\alpha)
\geq \alpha)$.   Let $\alpha_x$ be the first ordinal that is bigger 
than every ordinal in \fst$``x$. $\alpha_x$ is defined as long as $x$ 
is small in the sense of not being mappable onto a cofinal subset of 
$NO$.  Then let $\pi$ be the permutation that swaps $x$ with 
$\tuple{f`(T \alpha_x),x}$ for $x$ small then in $V^\pi$ the 
membership relation restricted to small sets is wellfounded.

{\bf Can we tweak Andr\'e's proof to show that Con(\nf) $\to$ Con(\nf +
$\in\restric stcan$ is wellfounded)?}

What can we say about the idea that there is an $f: NO \to NO$ s.t.
$f(T\alpha) \geq \alpha$?  Suppose there is such a function, and let
$X$ be a cofinal subset of $T``NO$. Then $f``X$ is a cofinal subset of
$NO$ so $cf(NO) \leq cf(T``NO) = T(cf(NO))$.  For the other direction
sse, to take a straightforward case, that $cf(NO) = \omega$.  To get
such an $f$ (try it!) we would need \AxC.

Boffa has a conjecture that 
\begin{conjecture}
It is consistent with \nf\ that $(\forall x)(x \in x \to |x| = |V|)$
\end{conjecture}

The dual of this is $(\forall x)(x \not\in x \to |V\setminus x| =
|V|)$.  Now if these two hold simultaneously we infer $(\forall
x)(|x| = |V| \vee |V\setminus x| = |V|)$.  This is
stratified and so is certainly not going to be provably consistent by
means of permutations. It is known that there are models of ZF in
which the real line can be split into two smaller pieces.  Richard
Kaye's idea for a counterexample to $(\forall x)(|x| = |V|
\vee |-x| = |V|)$ is $\{y: |y| < |V|\}$.  In view
of what follows we should also consider $\{y: |y| \not\geq^*
|V|\}$.


If we think of Bernstein's
lemma, all it tells us is that $(\forall x)(|x| = |V| \vee
|-x| \geq_* |V|)$.  

 If
$|x| \not\geq_* |V|$ we say that $x$ is {\bf small} and if
$|-x| \not\geq_* |V|$ we say $x$ is {\bf co-small}. By
Bernstein's lemma a set cannot be simultaneously co-small and
small. (Beware: not everything the same size as a co-small set is
co-small: every co-small set is of size $|V|$ but not {\sl vice
versa}. However, nothing the size of a co-small set is small.)

  This suggests that we
might be able to tackle a weaker version by permutations, namely:

\begin{conjecture}\label{conj:boffa1}
\nf$\vdash \poss (\forall x)(x \in x \to |x| \geq_* |V|)$
\end{conjecture} 

We can make a small amount of progress with this version of the
conjecture.  

\verb#\begin{digression}#
\begin{small}
\begin{quote}
\begin{quote}{\bf Some remarks on Quine pairs}\end{quote}
In what follows we will be using ordered pairs in the
style of Quine.  That is to say, we set $\tuple{x,y} = \theta_1``x 
\cup \theta_2``y$, where $\theta_1$ and $\theta_2$ are homogeneous 
bijections between $V$ and two other sets $\theta_1``V$ and
$\theta_2``V$ s.t. $\theta_1``V = -\theta_2``V$. Quine actually
provides two such functions $\theta_1$ and $\theta_2$ but we do not
need to know anything more about them than i have just said. \fst$(x)$
is the first component of the ordered pair $x$.

The advantage Quine pairs are usually supposed to have is that they
ensure the ``$x = \tuple{y,z}$'' is homogeneous.  There are other
advantages as well. If we need a disjoint union function $x \sqcup y$
then $\tuple{x,y}$ would do.  $\tuple{V, \subseteq, - \ldots}$ is a
boolean algebra, and so is $V \times V$.  The Quine pairing function
is actually an {\bf isomorphism} between $V \times V$ and $V$.  Thus,
$-\tuple{x,y} = \tuple{-x,-y}$, $\tuple{x\cap y, z} = \tuple{x,z} \cap
\tuple{y,z}$, and so on.  Some of this will be useful in what follows.

Of course this is less attractive in the context of \zf, but similar
results hold.  One should also think about the smallest number of types
with which one can define the two theta functions. Is now the time to go
back and look at Joel Friedman Some set-theoretical partition theorems
suggested by the structure of Spinoza's God. SYNTHESE v 27 (1974) pp
199-210

\end{quote}
\end{small}
\verb#\end{digression}#
\begin{rem}\label{rem:metaboffa}
If $(\forall x)(x \in x \to |x| \geq_* |V|)$ is consistent
with \nf, so is

$(\forall x)(x \in x \to |x| \geq_* |V|)\wedge 
 (\forall x)(| V\setminus x| \not\geq_* |V| \to x \in x)$
\end{rem}

\Proof

The two conjuncts are duals of each other, so one is consistent iff
the other is.  So let us start with a model $V$ satisfying
$$(\forall x)(|V\setminus x| \not\geq_* |V| \to x \in x)$$ We want to
swap every small set $x$ with $\langle V\setminus \fst``x,x \rangle$
but to do this we must check that if $x$ is small then $\langle
V\setminus \fst``x,x \rangle$ isn't (otherwise we would have to swap
that with $\langle V\setminus \fst``\langle V\setminus \fst``x,x
\rangle,\langle V\setminus \fst``x,x \rangle \rangle$ and the
definition would not be consistent.) We will show that if $x$ is small
$\langle V\setminus \fst``x,x \rangle$ is not small, and {\sl vice
  versa}.

Suppose $x$ is small. $\langle V\setminus \fst``x,x \rangle$ is a
superset of $\theta_1``(V\setminus \fst``x)$.  Now $\fst``x$ is a
surjective image of a small set and is therefore small. Therefore
$V\setminus\fst``x$ is a co-small set, and $\theta_1``(V\setminus
\fst``x)$, being the same size as a co-small set, is at least not
small, so its superset $\langle V\setminus \fst``x,x \rangle$ is not
small either.

For the converse suppose $\langle V\setminus \fst``x,x \rangle$ is
small.  If $\langle V\setminus \fst``x,x \rangle$ is small, then so is
its subset $\theta_1``(V\setminus \fst``x)$.  But if
$\theta_1``(V\setminus \fst``x)$ does not map onto $V$ neither does
$V\setminus\fst``x$.  So $\fst``x$, being the complement of a small
set, is co-small.  But if $\fst``x$ is co-small, $x$ cannot be small.

(If we were to try to prove an analogous result with ``small'' meaning
``smaller than V'', this is where the proof would break down. We
cannot show that if $x$ is smaller than $V$ then $\langle V\setminus
\fst``x,x \rangle$ isn't. As far as we know $x$ could be smaller than
$V$ but $\fst``x$ could be the whole of $V$)


Now we can safely set

$$\pi = \prod_{|x| \not\geq_* |V|}(x,\langle V\setminus \fst``x,x \rangle)$$



We will verify the two conjuncts separately.

$$V^\pi \models (\forall x)(|x| \not\geq_* |V| \to x \not\in x)$$
This is
$$V \models (\forall x)(|x| \not\geq_* |V| \to \pi`x \not\in x)$$
We procede by a case analysis:
\begin{itemize}
\item If $x = \pi`x$ then $x$ was not small, because all small things are moved. Therefore
      the antecedent is false and the conditional is true.
\item If $x \not = \pi(x)$ and $x$ is small, then $\pi(x) = \langle V \setminus \fst``x,x \rangle$.
      Since $x$ is small, $\fst``x$ (which is a surjective image of $x$) is also small,
      so $V\setminus\fst``x$ is co-small, and therefore---by hypothesis---a member of itself.
      Therefore $\langle V \setminus \fst``x,x \rangle \not\in x$, which is to say $\pi(x) \not\in x$.
\item If $x \not = \pi(x)$ and $x$ is not small, then the antecedent is false and the conditional
      is true.
\end{itemize}

We also want the dual to hold in $V^\pi$, as it did in $V$. So we want
$$V^\pi \models (\forall x)(|V \setminus x| \not\geq_* |V| \to x \in x)$$
This is
$$V \models (\forall x)(|V \setminus \pi(x)| \not\geq_* |V| \to x \in \pi(x))$$
As before, we do a case analysis. \begin{itemize}
\item If $x$ is fixed, the result is true because it was true in the base model by hypothesis.
\item If $x$ is small, then $\pi(x) = \langle V \setminus \fst``x,x
  \rangle$. This is $\theta_1``(V \setminus \fst``x) \cup
  \theta_2``x$, so $V\setminus\pi(x) = \theta_1``(\fst``x) \cup
  \theta_2``(V \setminus x)$. But if $x$ is small, $V\setminus x$ is
  co-small, and so $\theta_2``(V \setminus x)$---being the same size as
  a co-small set---cannot be small. So its superset
  $\theta_1``(\fst``x) \cup \theta_2``(V \setminus x)$ isn't small
  either.  But $\theta_1``(\fst``x) \cup \theta_2``(V \setminus x)$ is
  $V\setminus\pi(x)$. Therefore $V\setminus\pi(x)$ is not small so the
  antecedent is false, and the conditional true.
\item If $x$ is not small, it is $\pi(y)$ for some small set $y$.  So 
$\pi(x)$ is small, and so $V\setminus\pi(x)$ is co-small and the 
antecedent is false.
\end{itemize}

\endproof

Another observation in the same style is the following:

\begin{rem}\label{rem:small}
If there is a wellfounded set $X$ s.t. $\powk{\kappa} X) \subseteq X$
then there is a permutation model in which $\in$ restricted to sets
without partitions of size $\kappa$ is wellfounded.
\end{rem}
\Proof ($\kappa$ actually has to satisfy the extra condition: $\alpha
\leq^* \kappa \to \alpha \leq \kappa$, but $\kappa$ will be an aleph
in all current applications---for the moment at least.)  Let $\pi$ be 
the product 
$$\prod_{|x| \not\leq^* \kappa}(x, \tuple{V \setminus x, (\snd``x \cap X)})$$ 
of the transpositions $(x, \tuple{V \setminus x, (\snd``x \cap X)})$ 
over all $x$ without partitions of size $\kappa$.

Let such sets be ``$\kappa$-small'', at least for the duration of this
proof.  This is basically a Boffa permutation (as in remark
\ref{ref:boffapermutation}).  However, there is a slight wrinkle.
With Boffa's original permutation much use was silently made of the
fact that the second components of the ordered pairs in the story were
{\sl large}, being natural numbers.  This ensured that whenever $\pi$
moved $x$, then $\pi(x)$ was large iff $x$ was small.  This was essential 
to the plot, and remains essential here.  Now $\snd``x \cap X$ is small 
if $x$ is, so in order to achieve ``whenever $\pi$ moves $x$, then 
$\pi(x)$ is large iff $x$ is small'' we need to do something to the 
$\fst$ element of the pair it make {\sl it} large instead. This is what 
complementation is doing.

Let's just check this.  If $x$ is $\kappa$-small then $\pi(x)$ is an
ordered pair one of whose components is $V\setminus x$ wot ain't nohow 
$\kappa$-small, so $\pi(x)$ is not $\kappa$-small.  Now suppose 
$\pi(x) \not= x$ and $\pi(x)$ is not small.  Then it is 
$\tuple{V \setminus x, \snd``x \cap X}$.  By design of $\pi$, this 
object can only have been moved from $x$, so $x$ was $\kappa$-small.

Suppose $V^\pi$ thinks that that $x \in y$ and both are
$\kappa$-small.  This last tells us---as we have seen---that $y$ must
be $\tuple{V \setminus \pi(y), (\snd``\pi(y) \cap X)}$, and $x$ must be
$\tuple{V \setminus \pi(x),(\snd``\pi(x) \cap X)}$.  Now $x \in \pi(y)$ so
$\snd(x) \in \snd``\pi(y)$.  Now $\snd(x) = \snd``\pi(x)\cap X$ so
$\snd(x)$ is at least a subset of $X$, and it's $\kappa$-small because
it's a subset of $\snd``\pi(x)$ which is a surjective image of
$\pi(x)$ which is $\kappa$-small.  So it's a $\kappa$-small subset of
$X$ and is therefore a member of $X$, since $\powk{\kappa} X)
\subseteq X$.  So $\snd(x)$ is a member of both $\snd``\pi(y)$ and
$X$, so it's a member of $\snd``\pi(y) \cap X$, which is $\snd(y)$ so
$\snd(x)\in \snd(y)$.

Thus we have shown that: whenever $V^\pi$ thinks that $x \in y$ and both 
$x$ and $y$ are $\kappa$-small, then $\snd(x) \in \snd(y)$, and we also 
know that both of these things are in $X$.  In other words, if we let $K$
be the set of things that $V^\pi$ believes to be $\kappa$-small, then
$\snd$ is a homomorphism from $\tuple{K, \in_\pi}$ to $\tuple{X,\in}$.
$X$ is wellfounded by assumption, so $\tuple{K, \in_\pi}$ must be too.

\endproof

It might be worth considering an indexed family of permutation models
generated as follows.  Given an $X$ as above (minus the wellfoundedness
condition) let $\pi_X$ be the permutation defined as above.  Order them
according to the partial order on the $X$'s.  The result is a Kripke model
of something-or-other.

It would be very nice to have a converse to remark ~\ref{rem:small}.

My version of Boffa's conjecture is: co-small implies self-membered.
(A special case of) the universal-existential conjecture is:
self-membered implies meets everything in the sublattice generated by
the values of $B$.  The conjunction of these two implies that every
co-small set meets everything in the sublattice generated by the
values of $B$.  This we know to be true.

\subsubsection{Can we spice this up to lattices generated by free bases for $V$?}

%Dick and Valeria's leaving party nov 1998

How many bases are there? How big are they?  How big are their elements?

Given any basis i can swap any element with its complement, so the
number of bases is at least two-to-the size of any basis.

The $\forall^*\exists^*$ conjecture implies that if $x\in x$ then $x$
meets every element of the standard basis.  How about every element
of every basis?  Doesn't that sound a bit like ``Every self-membered
set generates $\tuple{V, \subseteq, -}$''?

I claim the following\begin{enumerate}
\item ``Every self-membered set generates $\tuple{V, \subseteq, -}$'' 
is $\forall^*\exists^*$;
\item If $\alpha$ is the size of a generating set then $T|V| \leq 2^\alpha$;
\item If $2^\alpha = T|V|$ then there is a basis of size $\alpha$.
\end{enumerate}

The first is easy to check.  It is $\forall y \in y)(\forall y_1
y_2)(\exists x \in y)(y_1 \in x \bic y_2 \not\in x \vee y_1 =
y_2)$. The following generalisation of item (i) merits attention: $x
\in x \to x \cap \pow x)$ generates $\pow x)$.  It's not
$\forall^*\exists^*$ but it's natural.

(ii) Follows beco's every singleton is an intersection of basis 
elements and complements of basis elements.

(iii) Sse $F:\iota``V \bic \pow X)$ is a bijection.  Each singleton
$\{y\}$ corresponds to a subset $X'$ of $X$, and we deem that $\{y\}$
is the intersection of the basis elements belonging to $X'$ and the
complements of the basis elements in $X \setminus X'$.  So $f`x$ must
be $\bigcup\{y \in \iota``V: x \in F(y)\}$. Then $f``X$ is a basis.


small$(x) \to x \not\in x$; Hsmall$(x) \to WF(x)$; $\in\restric$small is
wellfounded.
\subsection{Bases for the irregular sets}

Something about this in coret.tex

A set is {\bf irregular} iff it meets all its members. A basis for the
irregular sets is a set that meets every irregular set.  Some of the
theorems we have proved can be expressed as facts about bases.
Membership restricted to finite sets being wellfounded is the same as
the infinite sets forming a basis.  Can the uncountable sets form a
basis?  We shall see!  However the set of co-small sets isn't big
enuff to be a basis.  If $X$ is irregular so is $B``X$, and no member
of $B``X$ is co-small!

Still, there is a large gap between the set of uncountable sets and 
the set of co-small sets.


\subsection{Membership restricted to ideals and their filtres}
\label{history} 

History seems to lead us thus.  We start off with a notion of
smallness (like {\sl finite}) and notice that no small set seems to be
a member of itself.  We then conjecture that $\in$ restricted to small
sets is wellfounded, and finally that $R$ (defined by $R(x,y)$ iff $x$
and $y$ are both small or co-small and $x \in y \bic y$ is small) is 
wellfounded. But it's no good if the ideal of small sets is prime:

\begin{rem}
Let $I$ be a prime ideal and consider the relation $x R y$ defined as 
$x \in y \bic y \in I$.  Then $R$ is not wellfounded.
\end{rem}

\Proof In those circumstances $\in \restric I$ is wellfounded and 
$\not\in \restric I$ is wellfounded.  Find somehow sets $a$ and $b$ 
such that $a \not\in a \cup b$ and $b \in a \cap b$. (This is easy 
to arrange: set $a := \bbar{\Lambda}$; $b:= B`V$.)  Then $a \not\in a$ 
so $a \in I$ and $b\ R\ a$ `cos $b \in a$.  $b\in b$ so $b \not\in I$ 
and $a\ R\ b$ `cos $a \not\in b$.  Then $R$ is not wellfounded. \endproof

(Notice that this refutation uses the \nf O axiom, so we might get away
with the following relation over Church-Oswald models of $\nf_2$ might
be wellfounded (at least when the $k$oding function is nice): $x \inn
y \bic (\snd(k^{(-1)}(y)) = 0)$.)


There are two steps involved:  (i) Move from ``$\in$ restricted to $I$
has no loops of diameter 1, 2 \ldots'' to ``$\in$ restricted to $I$
is wellfounded'' (ii) to  Move from ``$\in$ restricted to $I$
is wellfounded'' to ``$R$ is wellfounded''.

How difficult are these?  Where $I = FIN$, (i) seems clear enuff. How
about (ii)? Perhaps the permutation making $\in \restric FIN$ wellfounded 
(or some variant of it) will also make this other relation wellfounded.

Try the following permutation: if $x$ is finite, swap $x$ with
$\tuple{f(Tn_x), x}$; if $x$ is cofinite swap $x$ with
$\tuple{f(Tn_{V\setminus x}), V \setminus x}$.  (I think we will need a sort-of 
rank function that sends $x$ to $n_{V \setminus x}$ if $x$ is finite and 
to $n_{V \setminus x}$ if $n$ is cofinite. Call this $n'_x$)


We want $V^\pi \models $``$R$ is wellfounded''.  Now $V^\pi \models\ \ x\ R\
y$ iff

$\pi(x)$ and $\pi(y)$ are both finite-or-cofinite and $x \in \pi(y) \bic
\pi(y)$ is finite. 

Want to show that if $V^\pi \models\ \ x\ R\ y$ then $n'_{\pi(x)} < n'_{\pi(y)}$.

case 1: $\pi(y)$ is finite.  Then $y = \tuple{f(Tn_{\pi(y)}), \pi(y)}$. 

\begin{enumerate}\item{Case 1a}
$\pi(x)$ is finite. Then $x$ must be $\tuple{f(Tn_{\pi`x}), \pi(x)}$.
But then, since $x \in \pi(y)$, the first component of $x$ must be less
than $n_{\pi(y)}$, so $f(Tn_{\pi(x)}) < n_{\pi(y)}$.  But we have
$n_{\pi(x)} < f(Tn_{\pi(x)})$ by choice of $f$ so $n_{\pi(x)} <
n_{\pi(y)}$ as desired.


\item{case 1b}  $\pi(x)$ is cofinite. Then $x$ must be
$\tuple{f(Tn_{V \setminus \pi(x)}), V \setminus \pi(x)}$. But then, since 
$x \in \pi(y)$, the first component of $x$ must be less than $n_{\pi(y)}$, 
which is to say $f(Tn_{V \setminus \pi(x)}) < n_{\pi(y)}$ and therefore (by 
choice of $f$) $n_{V \setminus \pi(x)} < n_{\pi(y)}$.
\end{enumerate}

case 2 $\pi(y)$ is cofinite.  Then $y = \tuple{f(Tn_{V \setminus \pi(y)}), V \setminus \pi(y)}$.  
Case 2a $\pi(x)$ is finite. Then $x$ must be $\tuple{f(Tn_{\pi(x)}),\pi(x)}$.  
But then, since $x \in \pi(y)$, the first component of $x$ must be less than err......

\ldots will get to the bottom of this.



At any rate (when $I = FIN$) the assertion that $R$ has no loops is a
$\forall^*\exists^*$ scheme. For example here is the subscheme that
says there are no loops of diameter 2.

$$\forall \vec x \forall \vec y \bigwedge_{i, j}(x_i = -\{y_1 \ldots
y_n\}\to y_j \not = \{x_1 \ldots x_m\})$$




\subsection{a bit of duplication here}

Why does Boffa's permutation work?  The reason is that there is a set $X$
with a wellfounded relation on it, and a map which accepts a bounded subset
of $X$ and returns a bound.  So here's an idea.  Force with the following
family.  Let $X$ satisfy $\pow X) \subseteq X$ (tho' perhaps i mean
$\powk{\alpha} X)$ for some $\alpha$---wait and see!).  Let $A$ be the set
of things $x$ so small that any map from $x$ to $X$ has bounded range.
Remember that in Boffa's original treatment $X$ was the set of naturals and
it was very important that naturals {\sl qua} sets, are very big.  To
preserve this feature we will deal not with members of $X$ but with members
of $X$ {\bf labelled} to be big.  A {\bf widget} is a pair $\tuple{x, V}$
with $x \in X$.  Let $Y$ be a set of widgets.  Then $\bigvee Y$ is
$\tuple{\bigcup\fst``Y, V}$.  (``Peel off the labels, take the sup, put 
a label on again").  Then consider the permutation

$$\prod_{x \in A} (x, \tuple{x, \bigvee((X \times V) \cap \snd``x)})$$

This is not enuff to show that comparatively small things are not 
self-membered, but if we force over all such $X$ we might end up with a 
model in which: $\in \restric \{x: (\exists y)(WF(y) \wedge |y| =
|x|)\}$ is wellfounded.  I see no reason why this should not be true.
I have actually managed to show that every model of ZF is the
wellfounded part of a model of $\nf_2$ in which the membership
relation restricted to low sets is wellfounded.

Maybe we should start from below and have a large wellfounded set $X$ \ldots

Suppose $H_\kappa$ were a set.  Label its elements as above to get
widgets.  Let $A$ be the set of things $x$ such that no map $x \to
H_\kappa$ is unbounded.  Consider the permutation

$$\prod_{x \in A} (x, \tuple{x, \bigvee((X \times V) \cap \snd``x)})$$

Isn't this remark \ref{rem:small}?


\section{Some provable special cases or weak versions}

The two following results are already in print:
\begin{rem}
Every $\forall^*\exists^*$ sentence consistent with \nf O is true in
the term model for \nf O.
\end{rem}


We should show that this holds for branching-quantifier formul{\ae} 
all of whose quantifier prefixes are $\forall^*\exists^*$.

But this is immediate---the same proof works!
%23/x/13

\medskip

I think we should be able to prove that every stratified
$\forall^*\exists^*$ sentence consistent with $\nf_2$ is true in the
term model for $\nf_2$.  In fact it's quite a nice question how much
we can weaken ``stratified".


\begin{rem}  
Every countable binary structure can be embedded in the term model for $NF0$.
\end{rem}

Think of this last remark as saying that every $\exists^\infty$ expression
consistent with \nf O is true in the term model.

There are also these two very similar lemmas on term models

\begin{rem} \label{rem:verysim1}
  Let $M$ be the ($NF$-)term model from some model $N$ of $NF$, and
  suppose $M$ is extensional.  Let `$(\exists \vec y)(\Phi (\vec
  x,\vec y))$' be weakly stratified and suppose that `($\forall \vec
  x)(\exists \vec y)\Phi (\vec x,\vec y)$' is true in $N$. Then it is
  true in $M$.
\end{rem} 

\Proof 

Assume the hypotheses. $(\exists \vec y)(\Phi (t,\vec y))$ for any choice
$\vec t$ of terms. We now want to be sure that witnesses for the $\vec y$
can be found in $M$. To do this, consider $\{\vec y$: $\Phi (t,\vec y)\}$.
This is a term if we can stratify the $\vec y$, as the matrix will be
stratified since the $t_i$ (being closed terms) can be given any type. $M$
is an extensional substructure of $N$, and so there must be such a witness
in $M$. \endproof

And now the second theorem.

\begin{rem} \label{rem:verysim2}
Let $N$ be a model of some subsystem $T$ of \nf\ extending $NF\forall^*$, 
and $M$ be the $T$-term model from $N$, with $M$ extensional.  Let 
`$\exists \vec y \Phi(\vec x, \vec y)$' be weakly stratified with `$\Phi$' 
quantifier-free. Suppose
$$N \models \forall \vec x \exists \vec y \Phi(\vec x, \vec y)$$
then 
$$M \models \forall \vec x \exists \vec y \Phi(\vec x, \vec y)$$
\end{rem}

\Proof 

Assume the hypotheses. 

We start counting the $\vec y$ at $y_0$.  Then for each $\vec t \in M$, 
$N \models \exists \vec y \Phi(\vec t, \vec y)$ and the question is, 
can these $\vec y$ be found inside $M$?  Consider $\{y_0: \exists
y_1 \ldots y_n \Phi(\vec t, \vec y)\}$.  Now since `$\Phi$' is
quantifier-free, this thing is actually an $NF\forall^*$ term over the
$\vec t$ and therefore certainly a $T$-term and is in $M$. We also
know that it is nonempty in $N$ and therefore nonempty in $M$ since
$M$ is extensional. Therefore, for some $m_0$ in $M$, 
$\exists y_1 \ldots y_n \Phi(\vec t, m_0, y_1 \ldots y_n)$ and the task 
now is to find witnesses for the $y_1 \ldots y_n$ in $M$.  This is the 
same problem as before, but with one fewer $y$-variable to deal with.  
So we have a proof by induction on the length of `$\vec y$'. \endproof

\section{Some consequences of conjecture ~\ref{conj:universalexistential1}}

\begin{small}
STUFF TO FIT IN

If $WF(x)$ we do not expect there to be a $y = x \cup \{y\}$. This gives 
an axiom
$$A_\omega:\ \ \ \ (\forall x y)(WF(x) \to y \not= x \cup\{y\})$$

which is $\forall_4$ or something horrid anyway.  Are there
$\forall_2$ versions obtained by thinking about loops?

$$A_1:\ \ \ \ (\forall x y)(x \not\in x \to y \not= x \cup\{y\})$$
This is stronger (antecedent weaker)---perhaps {\sl much} stronger.
It's like the version that is true in the term model of \nf$_2$ but
not I\nf, but weaker.  That was ``every superset of a self-membered set
is self-membered''.  This one {\sl is} true in the term model of \nf
O---think about the least rank of a counterexample.

If so, then perhaps we should consider the other finite versions:

$$A_n:\ \ \ \ (\forall x y)(x \not\in^{\leq n} x \to y \not= x \cup\{y\})$$

which get weaker as $n$ gets larger, and they're all $\forall^*\exists^1$.
Perhaps there is an infinite family of systems between \nf$_2$ and \nf O, 
and $A_n$ is true in the term model for the $n$th but not in the term model
for \nf O, or something like that.

It would be nice to prove $A_\omega$ by $\in$-induction but of course
we can't.  We would be able to if we could show that for any $y \in
y$, the set $\{x: x \cup \{y\} \not = y\}$ is fat.  It isn't of
course, but the assertion that it is is $\forall^*\exists^*$.***  So the
universal-existential conjecture implies $A_\omega$ by $\in$-induction.

Later: i don't believe the starred allegation. (Too many quantifiers\ldots?)
 Let's check.  The following formula asserts that 
$\{x: x \cup \{y\} \not = y\}$ is fat:

$\pow\{x: x \cup \{y\} \not = y\}) \subseteq \{x: x \cup \{y\} \not = y\}$

$(\forall z)( z \subseteq \{x: x \cup \{y\} \not = y\}) \to z \in \{x: x \cup \{y\} \not = y\})$

$(\forall z)( z \subseteq \{x: x \cup \{y\} \not = y\}) \to z \cup \{y\} \not = y\})$

$(\forall z)((\forall w \in z)(w \cup \{y\} \not = y\}) \to z \cup \{y\} \not = y\})$

\ldots so it's $\forall^*\exists^*\forall^*$



But we expect $(\forall y)(x \not= y\setminus \{y\})$ to hold for nice $x$.  
(The assertion that it holds for $x = \emptyset$ is a repudiation of Quine 
atoms and is $\forall\exists$!)  Is this a property worth considering?  It 
says ``$x$ cannot be capped off''   Can we prove by $\in$-induction that all 
wellfounded sets have it??  Can't see how \ldots


\end{small}

I used to think that one consequence of conjecture
~\ref{conj:universalexistential1} is that $\{x: x \in x \}$ is an
upper set in $\tuple{V, \subseteq}$.  However, this can be refuted by
considering $ V\setminus B(V)$ (which is selfmembered) and its superset 
$(V\setminus B(V)) \cup \{V\}$ (which isn't). 


The scheme of assertions: ``$x \in x \to y \Delta x$ is finite $\to y \in
y$'' is $\forall^*\exists^*$ but doesn't (despite what i initially tho'rt)
come under the conjecture because---altho' consistent with $\nf_2$, it's
not consistent with \nf O, and for similar reasons.  There is an
interesting formula that comes out of this, tho'.  ``$x \in x \to y \Delta
x$ is finite $\to y \in y$'' would follow from ``$\{x: x \in x \}$ is an
upper set in $\tuple{V, \subseteq}$'' and $\forall x \forall y \ \ (x \in x
\ \wedge \ y \in x \to (x -\{y\}) \in (x -\{y\})$ which is
$\forall^*\exists^*$ too. This second is equivalent to the conjunction of
$$(\forall x \forall y)(x \in x \ \wedge \ y \in x \ \to (x -\{y\}) \in
x)$$ $$(\forall x \forall y)(x \in x \ \wedge \ y \in x \ \to (x -\{y\})
\not= y)$$

\begin{small} (This is because the conjunction of these two implies
that if $x \in x$ and $y \in x$ then 
$x \setminus \{y\} \in x \setminus \{y\}$. Then of course we can use 
them any standard number of times to conclude that if $x \in x$ and 
$y \subseteq x$ with $x \setminus y$ standardly finite,
then $y \in y$ too.  The we want to know that $\{x: x \in x\}$ is an
upper set to infer that if $x \in x$ and $x \Delta y$ is standardly
finite, then $y \in y$.) 
\end{small}

As noted, we can forget about the first (try $x := B(V)$ and $y := V$), but
the second is interesting. It is an assertion that there are no generalised
Quine\index{Quine} antiatoms.  The dual assertion, that there are $n$
generalised Quine\index{Quine} atoms, is

$$\forall x \forall y \ (x \cup \{y\}) = y \ \to \ (y \in x \vee x \in x)$$

Actually we can simplify this a bit.  The `$y \in x$' in the
consequent implies the other disjunct in the consequent, so this is
really

$$\forall x \forall y \ ((x \cup \{y\}) = y \ \to \ x \in x)$$

Notice that in the case where $x = \Lambda$ this becomes the assertion
that there are no Quine atoms.


This admits generalisation, and in two ways. 

If $x \cup \{y\} = y$ we say that $y$ {\bf caps} $x$.  Only
self-membered sets can be capped, and even then the cap is unique.

\begin{enumerate}

\item For some $x$ we can find $y$ such that $y \setminus \{y\} = x$.
But for any $x$ there should be at most one such $y$.  This is
$\forall^*\exists^*$ and presumably true in all term models but don't
quote me on that.  


$$(\forall y_1 \in y_1)(\forall y_2 \in y_2)(y_1 \setminus \{y_1\} = y_2 \setminus \{y_2\} \to y_1 = y_2)$$


which is

$$(\forall y_1 \in y_1)(\forall y_2 \in y_2)((\forall z)(z \in y_1 \setminus \{y_1\} \bic z \in y_2 \setminus \{y_2\}) \to y_1 = y_2)$$

which is $\forall^2\exists^1$.

We dislike counterexamples to this for the same reason that we dislike
Quine atoms: there is no recursive way of telling them apart.  (in
fact the nonexistence of Quine atoms is a special case)

What about the situation where $x_1 \setminus \{y_1\} = x_2 \setminus \{y_2\}$.

This might be perfectly innocent with all four objects different.  But
funny things start to happen if enough of them are self membered or
members of each other. (Trouble is: the number of cases is huge!)

let's try classifying them like this.

$$(\forall x_1 x_2 y_1 y_2)(x_1 \setminus \{y_1\} = x_2 \setminus \{y_2\} \wedge \Phi(x_1,x_2, y_1, y_2) \to x_1 = x_2)$$

where $\Phi$ is a boolean combination of atomics in the language 
${\cal L}($`$x_1$', `$x_2$', `$y_1$', `$y_2$', =, $\in)$.


These are all universal-existential.  If $\phi$ is stratifed then the
whole formula is stratified and not interesting.  We assume $\Phi$
contains $y_1 \in x_1$ and $y_2 \in x_2$.

I think what i was trying to get at was the following generalisation.

We have a set $X$ (which started off being finite) with the graph of
$\in$ restricted to $X$.  We are then given some equation between
words in the members of $X$ with operations like singleton, union and
difference.  (The equation must be $\forall^*$) and invited to infer
an equation between two members of $X$.  This conditional is
$\forall^* \exists^*$ and should be consistent according to the
universal-existential conjecture.


E D I T\ \   B E L O W\ \ H E R E
 
But we can claim more than this in a
$\forall^*\exists^*$ way. 


$$(\forall X)(\forall y_1 y_2)((\forall z_1 z_2 \in X)(((z_1\setminus x) = (z_2 \setminus x)) \to y_1 = y_2)$$

\marginpar{This doesn't make sense:the $y$'s aren't doing anything.  What did i mean?}

Notice that the assertion at the start of this paragraph (that if $x
\cup \{y\} = y$ and $x\cup \{z\} = z$ implies $y = z$) is the special
case where $X = \{y, z\}$.  We might need to insert into the displayed
formula a condition like $z \in X \to z \cap x$ not self-membered,
beco's of course if $x \in x$ we might be able to ``cap'' $x$ in more
than one way. (Check this!)  As it stands it's not true: $X :=V$ is a
counterexample, and so is an initial segment of $WF$.  But we should
be able to recover something.  After all: this condition is just: $\in
|\restric $ is extensional plus a little bit extra.  There might be other
examples too. One could take $X$ to be inductively defined by $\{V\setminus X\}
\in X$ and $y \subseteq X \to y \cup \{X\} \in X$.  If one inserts a
condition that $\in \restric X$ is strongly illfounded then one could require
that $X$ be empty.  But this is no longer $\forall^*\exists^*$.

\bigskip

A S\ F A R\ A S \ H E R E 

\bigskip

\item Is there an infinite family of analogues of this where the
conclusion is $x \in^n x$?  If you can obtain $y$ from $x$ by
inserting $y$ into the transitive closure of $x$ $n$ levels down then
$x \in^n x$?  Doesn't seem to be $\forall^*\exists^*$ tho'.  The key
might be to look at the dual, namely
$$\forall x \forall y \ ((x \setminus \{y\}) = y \ \to \ x \not\in x)$$
Do not make the mistake i made of assuming that $(x \setminus \{y\}) = y$
is the same as $(y \cup \{y\}) = x$ \ldots `cos $y \in y$ is a possibility!
We would need to look at 
$$9\forall x \forall y)((y \not\in y \wedge (y \cup \{y\}) = x) \to x \not\in x)$$

\end{enumerate}

\section{Some $\forall^*\exists^*$ sentences true in all term models}

There is a lemma (see lemma ~\ref{rem:verysim1} and lemma
~\ref{rem:verysim2}) that covers 1- $4^k$ below, though in fact we can
at present use it to prove only that $4^k$ must hold in DEF,
permutation models for the others not being forthcoming at present. In
fact we can show by other methods that 1-3 hold in DEF and SYMM (that
2 is true in DEF was proved directly by Boffa\index{Boffa} [1]).

\begin{enumerate}
\item[1] All $x \in x$ are infinite (which is a scheme) 
\item[2] $\bigcup x \subseteq x \in \ x \to \ x = V$
\item[3] $x \in^n x \to \ \bigcup^n x = V$
\item[$4^n$] ($\forall x)(x \not = \iota^n(x)$) 
    \end{enumerate} 

Observe that [3] and [4] are stratifiable-mod-$n$.

\subsubsection{Item 3}

About [3] one can say the following.  Let $x$ be co-small (a small
set is one that doesn't map onto $V$).  Then $x$ meets every set that
is not small.  So it meets every $B$-word (as it were!).  

We can do better than this, for if $y$ is small, the set of its
supersets isn't, and so $x$ contains a superset of $y$. If $y$ isn't
small, nor is $\pow y)$ and so $x$ contains a subset of $y$.  Let's
abbreviate this to $F(x)$.  \marginpar{So $F(x)$ means ``$x$ meets
  every non-small set''?  I don't know what i meant here...}

So we have co-small$(x) \to F(x)$.  Can we interpolate $x\in x$ into
this conditional?  Perhaps with the help of the universal-existential
conjection we can get $F(x)$ to imply $x \in x$ and even {\sl vice
versa}.

\subsubsection{item 1}

Friederike has solved 1.  She has shown that it is consistent relative to
\nf\ that the membership relation restricted to sets without a countable
partition is wellfounded.  After seeing her model i proved a similar result
about symmetric sets (proposition \ref{prop:six} below).

  First we consider direct proofs that some of the things we want must be
true in DEF or SYMM. Propositions \ref{prop:four} to \ref{prop:six} below
are best seen as statements about the behaviour of the substructure
$SYMM^M$ of an arbitrary model $M$ of NF. There is no very satisfactory way
of representing these as first-order theorems of NF.

\begin{prop}\label{prop:four}
For all symmetric sets $x$, $(x \in x\ \to \ (\exists y \in x)(y \not\in y))$ 
\end{prop}
\Proof  

Suppose not and that $$x \in x$$ is a symmetric set such that
$(\forall y \in x)(y \in y)$.  Let $x$ be $n$-symmetric and $k$ be
some power of 2 $> n$ where $x$ is $n$-symmetric. Then
$x \in x$ implies---since $x$ is $\leq k$-symmetric---that 
$(j^kc)(x) \in (j^kc)(x)$.

Now for a useful {\sl factoid} which we are going to use repeatedly:
for any $u$ and $v$ and for any permutation $f$ we have $u \in
(j`f)`v$ iff $f^{-1}`u \in v$. In fact in the only cases we are going
to use it on here $f$ is an involution so we can forget about the -1.
Using the factoid we infer

$$(j^{k-1}`c)\circ (j^k`c)`x \in x$$

Now, by hypothesis everything in $x$ is self-membered so we infer
$(j^{k-1}`c)\circ (j^k`c)`x \in (j^{k-1}`c)\circ (j^k`c)`x$. Now we
use the factoid again to rearrange this to:
$(j^{k-2}`c)\circ(j^{k-1}`c)\circ(j^{k-1}`c)\circ (j^k`c)`x \in x$ The
$k$-1`s cancel, since $j^n`c$, like $c$, is of order 2 for any $n$,
giving

$$(j^{k-2}`c) \circ (j^k`c)`x \in x.$$ 

Now consider the sequence of the three displayed formul{\ae}.  They
are all of the form $W_x \in x$. To get from the first to the second
(and to get from the second to the third) we first appealed to the
fact that everything in $x$ is self-membered, and then to the factoid
to infer that some other $W_x$ was a member of $x$.  The reader is
invited to think for a minute or so about what happens when we repeat
this process, bearing in mind that important simplifications can be
made when we exploit the fact that complementation commutes with every
permutation that is $j$ of something, and that if $\pi$ and $\sigma$
commute so do $j`\pi$ and $j`\sigma$. Thus an easy induction tells us
that all $j^n`c$, $j^k`c$ commute with each other.  This enables us to
tidy up the $W_x$ satisfactorily.  Consider the following picture (i
have written numbers in hex to make it prettier):

\begin{array}{ccccccccccccccccc}
 &10 &10 &10 &10 &10 &10 &10 &10 &10 &10 &10 &10 &10 &10 &10 & 10\\
 &   &   &   &   &   &   &   &   &   &   &   &   &   &   &   & 10\\
 &   &   &   &   &   &   &   &   &   &   &   &   &   &   & F & 10\\
 &   &   &   &   &   &   &   &   &   &   &   &   &   & E &   & 10\\
 &   &   &   &   &   &   &   &   &   &   &   &   & D & E & F & 10\\
 &   &   &   &   &   &   &   &   &   &   &   & C &   &   &   & 10\\
 &   &   &   &   &   &   &   &   &   &   & B & C &   &   & F & 10\\
 &   &   &   &   &   &   &   &   &   & A &   & C &   & E &   & 10\\
 &   &   &   &   &   &   &   &   & 9 & A & B & C & D & E & F & 10\\
 &   &   &   &   &   &   &   & 8 &   &   &   &   &   &   &   & 10\\
 &   &   &   &   &   &   & 7 & 8 &   &   &   &   &   &   & F & 10\\
 &   &   &   &   &   & 6 &   & 8 &   &   &   &   &   & E &   & 10\\
 &   &   &   &   & 5 & 6 & 7 & 8 &   &   &   &   & D & E & F & 10\\
 &   &   &   & 4 &   &   &   & 8 &   &   &   & C &   &   &   & 10\\
 &   &   & 3 & 4 &   &   & 7 & 8 &   &   & B & C &   &   & F & 10\\
 &   & 2 &   & 4 &   & 6 &   & 8 &   & A &   & C &   & E &   & 10\\
 & 1 & 2 & 3 & 4 & 5 & 6 & 7 & 8 & 9 & A & B & C & D & E & F & 10\\
0&   &   &   &   &   &   &   &   &   &   &   &   &   &   &   & 10\\
 \end{array}


 (Get each row from its predecessor by ``subtract 1 pointwise and take
symmetric difference").  Think of the elements of each row as the
indices on $j$ that appear in the prefix $W_x$ referred to above. I
have talked through the construction of the first three rows (with
$k=16$). The picture makes it plain to the eye that if $k$ is a power
of $2$ we end up with ($j^k`c)(V\setminus x) \in \ x$. Now
$(j^k`c)`(V\setminus x) = (V\setminus x)$ since $x$ is $\leq k$-symmetric 
so $V\setminus x \in \ x$ and (since all members of $x$ are self-membered) 
$V\setminus x \in \ V\setminus x$ which cannot be simultaneously true.

(This is actually a digitised picture of a Sierpinski sponge, tho'
this probably does not matter!)

\endproof

This assertion considered in the next proposition is actually a
consequence of the $\forall^*\exists^*$ expression $(\forall x)(x \in
x \to (\forall y)(\exists z \in x)(y \not\in z))$.  Can we prove that
this is true in the symmetric sets?

\begin{prop} \label{prop:five}: 
(${\forall}x \in\ { SYMM})(x \in \ x$. $\to \ .(\exists y)(y \in \ x
  \wedge x \not\in \ y$))
\end{prop}
    \Proof  

Suppose $x$ is $m$-symmetric and belongs to all its members. Then, for any
permutation $\tau $, and any $n \geq \ m$, $x \in \ x$ iff 

 $(j^n`\tau )`x \in \ (j^{n+1}`\tau )`x$ 

 iff 

$(j^n`\tau )`x \in \ x$ ($=(j^{n+1}`\tau)`x$ because $x$ is $\leq n$-symmetric)
 iff  
    \begin{enumerate}
   \item [(i)] $x \in \ (j^n`\tau )`x$ iff 
    ($j^{n-1}`\tau $)$^{-1}`x \in \ x$ whence 
     \item [(ii)] $x \in \ (j^{n-1}`\tau $)$^{-1}`x$ 
     since $x$ belongs to all its members. 
    \end{enumerate}
   We now repeat the line of reasoning that led us from (i) to (ii),
   decreasing exponents on $j$ at each step until

       $x \in \ \tau ^{-1} x$ ($n$ odd) or $x \in \ \tau `x$ ($n$ even). 

But $\tau $ was arbitrary, and it is easy enough, given $x$, to devise a 
permutation $\tau $ so that $x \in \ \tau `x \wedge x \in \ \tau^{-1}`x$. 
\endproof

\bigskip

I proved proposition ~\ref{prop:six} after Friederike K\"orner
produced a construction of a model of \nf\ in which the membership
relation restricted to finite sets is wellfounded.  It is sensible to
ask if this can be proved for larger sets too.  Let us say $I$ is a
{\sl notion of smallness} if \begin{enumerate}
\item Any subset of an $I$ thing is also $I$
\item Any union of $I$-many $I$-sets is $I$
\item $V$ is not $I$
\end{enumerate}

Finiteness is a notion of smallness, so is dedekind-finiteness. There's 
not a great deal more!  In particular {\bf smallness} (as in ``can't be 
mapped onto the universe'') isn't a notion of smallness. (We should
perhaps consider here Boffa's question about the sequence: $W_1$ = set
of wellorderable sets, $W_{i+1} =$ sumsets of wellordered subsets of
$W_i$.  This isn't {\sl directly} applicable here because the natural
application would be: $W_{i+1} =$ sumsets of $W_i$ subsets of the set
of all wellordered sets.  However, if $W_\infty$ is not $V$ we can
prove proposition ~\ref{prop:six} for $W_\infty$ too.)

Finally we should note that the first list approximant to the
branching quantifier formula saying that there is an (external)
antimorphism is true in the definable or symmetric sets.

\section{Strengthening the conjecture}

We can't extend this to formul{\ae} with bounded quantifiers because 
the assertion ``There is a dense linear order" is $\Sigma_1$.

Some $\forall^*\exists^*$ sentences are theorems of \nf\ beco's they
are consequences of extensionality.  In these cases we cannot expect
to be able to prove the formula in a nice way by witnessing the
existential quantifiers with terms.  We don't have this problem with
$\exists^*\forall^*$ expressions, so perhaps we should strengthen the
conjecture to:

For every stratified $\exists^*\forall^*$ sentence either it is provable in SF or if it isn't its negation is a theorem of \nf.


So look at the ways in which wee could fail to prove a given $\exists^*\forall^*$ sentence \ldots



One might have hoped that one could have developed \nf O as a
\verb#PROLOG# theory with the expectation that whenever \nf O proves a
universal-existential sentence the witnesses to the $x$ variables can
be found as words in the $y$ variables. The following
$\forall^3\exists^1$ example shows that this is doomed.

$$(\forall y_1 y_2 y_3)((y_1 \in y_2 \wedge y_3 \not\in y_2) \to 
(\exists x)(x \in y_1 \bic x \not\in y_3))$$



Idea:

(i) Show that if \nf O proves something existential-universal it
    exhibits a witness

(ii) Use a \verb#PROLOG#-style treatment. An attempt to prove an
     existential-universal  assertion corresponds to an attempt to
     make the youniversal vbls into constants and to instantiate the
     xistential vbls with closed terms.  ivo (i) this is sufficient.

(iii) Transform a failure to \nf O-prove your existential-universal
    assertion into an \nf O-proof of its negation.

Let's apply this to the nasty example above.  We fail to find a closed
term to do for `$x$' in

$$(\exists x)(\forall y_1 y_2 y_3)((y_1 \in y_2 \wedge y_3 \not\in y_2) \to (x \in y_1 \bic x \not\in y_3))$$
so (if this works) we expect to be able to prove its negation, namely

$$(\forall x)(\exists y_1 y_2 y_3)((y_1 \in y_2 \wedge y_3 \not\in y_2) \to (x \in y_1 \bic x \in y_3))$$
which seems innocent enough.


\subsubsection{$\forall^*\exists^*$ witnesses?}

If we should be able to prove consistent by permutations all consistent
$\forall_2$-sentences, we can ask whether there are definable skolem
functions that do the business for us. For example, is there a skolem
function witnessing $$\Psi: \ (\forall y \in y)(\exists x \in \ y)(x \not =
y)?$$ A good guess is that
$x$ could consistently be taken to be $y \setminus \{y\}$. $\Psi$ is
$\forall \exists$, and a natural extension of this conjecture would be that
for any $\forall\exists$ sentence there is some $\nf_2$ word (or finite
disjunction of $\nf_2$ words) we can consistently assume to uniformly
provide witnesses for it. This certainly {\sl looks} plausible for $\Psi$.
And, pleasingly, the assertion that $y \setminus \{y\}$ works for $\Psi$
is itself $\forall_2$, namely 
$$(\forall y)(y \in y \ \to \ (y \setminus \{y\})\in y)$$ which is


$$\Psi^*: \ \forall x \forall y \ x \in x \wedge y \not\in x \ \to \ \exists z \ \ (z \in (y \Delta (x - \iota`x)))$$

But presumably in general this cannot work, because otherwise we would be
committed to producing a disjunction of terms which would be candidate
witnesses to the $x \Delta y$ if $x \not= y$, because of 
$$(\forall x y)((x \in x \wedge y \not\in y) \to (\exists z)(z \in x \bic z
\not\in y))$$ and this would presumably imply AC.

Let's think about this a bit. Extensionality implies that if $x \not\in x$
and $y \in y$ then $x \Delta y$ is inhabited.  But by what? Not
provably by $x$ or $y$ beco's of $B`V$ and $\{B`V\}$.  I can't see any
\nf O word in $x$ and $y$ that can be relied upon to inhabit $x \Delta
y$ in these circs, nor any finite set of words one of which must.  It
would be nice to have a proof of this fact.


\subsection{Extending the conjecture to sentences with more blocks}
   
If the $\forall^*\exists^*$ conjecture is true, we will have an
extension of Hinnions old result on $\exists^*$ sentences.  What is
the appropriate extension of these conjectures to formul{\ae} with
three blocks of quantifiers?  Presumably it would be to
$\exists^*\forall^*\exists^*$ formul{\ae}, keeping going the pattern
of having the {\sl innermost} block a block of existential
quantifiers.

And what is the conjecture to be?  Let us define $\Gamma_n$ to be the
set of formul{\ae} with $n$ blocks of quantifiers, with the innnermost
existential.  The strongest form of the conjecture would be to set
$\nf^1 := \nf$; $\nf^{n+1} := \nf^n \cup$ all the $\Gamma_{n+1}$ sentences
consistent with $\nf^n$.  Finally we would hope that the complete
theory which is a union of all these is consistent and has a term
model.  


Unfortunately there are some pretty obvious {\sl prima facie}
counterexamples.  Wellfoundedness is a source of lots of hard cases
for the three-quantifier case, since ``$x$ is wellfounded'' is
$\forall^*\exists^*\forall^*$,

\begin{enumerate}
\item The axiom of $\in$-determinacy is $\forall^*\exists^*\forall^*$ 
      but is true in all term models.
\item There is a $\exists^*\forall^*\exists^*$ sentence $POL$ that
  asserts that there is an antimorphism of the universe which is an
  involution (a polarity).  ``$X$ is a partition of $V$'' is
$(\forall u \exists v \in X)((u \in v)\wedge (\forall v' \in X)(u \in v' \to v' = v))$ \hfill (B)

  $\forall^*\exists^*$.  What we do is assert that and add the clause

$(\forall y z)(\forall u v)[((\exists a \in X)(y \in a \wedge z \in a) \wedge
(\exists b \in X)(u \in b \wedge v \in b)) \to\\
 (u \in y \bic v \not\in z)]$ \hfill (A)



It is a simple exercise using extensionality to check that (A)
implies that every member of $X$ is a pair.  (If we had to assert
specifically that every member of $X$ is a pair it would cost an extra
alternation of quantifiers.)  $POL$ is the conjunction of (A) and (B)

No term model can contain an antimorphism.  So we must hope that $POL$
is refuted by the $\forall^*\exists^*$ scheme.  But that can happen
only if the existence of a polarity is $\exists^*\forall^*$.


\item ``Every transitive set that is not self-membered is wellfounded'' 
is $\forall^*\exists^*\forall^*$ but is true in all term models.

\item $x = \bigcap x$ is $\exists^*\forall^*\exists^*$ but not true in
      any term model.


 \end{enumerate}

Ad item (1). I'd like to see this spelled out.

\subsection{Perhaps the key is to doctor the logic}

There are other logics that have a notion of quantifier hierarchy that
we might be able to use. The cofinite logic for example.   With a
two-block formula we can say something like ``Every transitive set
that isn't self-membered is finite'': and this ought to be true in the
nice models

$$(\forall_\infty y_1)((\forall_\infty x_1 \in y_1)(x_1 \subseteq_\infty y_1) \to  (\forall_\infty y_2)(y_2 \in y_1))$$

(Is there a prenex normal form theorem for the logic with the cofinite quantifier?)

(3)


$$\Phi_{\infty}: (\exists x)(\bigcup x\subseteq x \not = V \wedge \neg WF(x))$$

This is $\exists^*\forall^*\exists^*$, and apparently consistent with
$\nf^2$, but it is obviously pathological, e.g., it is demonstrably
false in DEF and SYMM, because of Boffa\index{Boffa}'s theorem that
there are no definable transitive sets other than $V$ and a smattering
of hereditarily finite sets.

However it does not appear to be consistent with $\nf^2 \cup
\ \Gamma $, where $\Gamma $ is the formula:

($\forall x)($No circles  $x \in^n x \to $no$ \ \omega$-descending $\in$-chains starting at $x$) 

  To see this consider the formul{\ae} 
\begin{quote}
  $\Phi _n$: ($\forall x)( \bigcup x \subseteq x\   \wedge \ x \in^n x. \to \ x =  V$) \end{quote}

as $n$ varies over the positive integers. Each $\Phi_n$ is certainly
$\forall^*\exists^*$ and appears to be consistent with \nf\ (no Proof
to hand, but they are demonstrably true in DEF or SYMM, because of
Boffa\index{Boffa}'s theorem just alluded to). But in any model in
which the $\Phi _n$ and $\Gamma $ all hold, $\Phi_{\infty}$ must be
false.
 
Now although $\Gamma$ is certainly infinitary it is in some sense
$\forall_2$ in $L_{\omega_1\omega_1}$.  This suggests that we should
consider cutting down the number of $\exists^*\forall^*\exists^*$
sentences we have to add to $\nf^2$ to get $\nf^3$ by defining $\nf^2$
to be not: \nf\ $\cup$ all $\forall^*\exists^*$ sentences consistent
with \nf, but: \nf\ $\cup$ all $\forall^\infty\exists^\infty$
sentences consistent with \nf.



(4) $x = \bigcap x$ is a candidate pathology because (i) it is true of
no symmetric set. (We can establish this easily enuff by asking about
the least $n$ such that there is an $n$-symmetric $x$ such that $x =
\bigcap x$.) (ii) it looks possible that the existence of such an $x$
should be consistent with all the $\forall^*\exists^*$ formul{\ae}
true in all term models.  $x \subseteq \bigcap x$ is $\forall^*$ (it's
$(\forall z)(z \in x \to (\forall w)(w \in x \to z \in w))$).  So
$(\forall x)(x \subseteq \bigcap x \to x = \emptyset)$ is
$\forall^*\exists^*$ and would solve our problem if it is consistent.
However, life isn't that easy. $\{y\} \subseteq \bigcap \{y\}$ is
just $y \in y$. Cofinite sets tend to be members of each other.  Consider
$x =\{V \setminus \{\{y\}\}: y \in V\}$.  Clearly $x \subseteq \bigcap x$.


$x = \bigcap x$ is equivalent to the assertion that $\in \restric x$ is just
$x \times x$ (So it follows immediately that the proper class of $x$
s.t. $x \subseteq \bigcap x$ is downward closed and closed under
directed unions.) In particular, $(\forall y \in x)(y \in y)$.  The
contrapositive of Proposition ~\ref{prop:four} tells us that any such
(symmetric) $x$ will not be a member of itself. Also by propositions
~\ref{prop:four} and ~\ref{prop:five} this is true in symmetric
models.  Unfortunately this does not seem to tell us any more than
that $x \subseteq \bigcap x \to x \not\in x$, and this does not seem
to be impossible.  


I keep having the feeling that if $x = \bigcap x$ then stcan$(x)$.





\section{Remains of some failed proofs of conjecture ~\ref{conj:universalexistential3}}

Richard reminds me that if $\M$ satisfies every $\forall^*$-consequence of
$T$ then $\M$ has an extension which is a model of $T$.  (Do we have a
${\cal P}$-version?). So, he says, can we show that if $\M \subseteq N$,
both models of $NF$, then $\M \prec_{str(\exists^*)} N$?  To do this by the
method below we would want something along the following lines:

$N$ a model of \nf\ is an $NF$-term model over some list of generators
$\vec n$.  The question is, can we represent an arbitrary model $\M
\subseteq N$ as a term model over some subset of $\vec n$ in such a way
that terms have the same meaning in both models?  Why the hell should we
expect this?    
                                                    

                                                                        
\subsection{A failed proof}

Suppose ($\forall \vec x)(\exists \vec y)(\Phi (\vec x,\vec y))$, with
$\Phi$ stratified, is true in the term model of \nf O. We are now going to
show that it is true in an arbitrary sufficiently large finite model of
$TST$.

The proof is laborious and we spare the reader and ourselves some details.
Consider first of all the $x$ variables of lowest type.  Suppose there are
$n$ of them. We want to show that any $n$ objects from that level of any
model satisfy a certain property.  Consider such an $n$-tuple of objects in
$T_k$.  Any set of objects in $T_{k-2}$ will generate a boolean subalgebra
of $T_k$.  Consider the minimal set $\vec a \subseteq T_{k-2}$ such that
all elements of the $n$-tuple belong to the free subalgebra of $T_k$
generated by the $B`a_i$, so that each object in the $n$-tuple is a unique
$\cup$, $\cap$, $B$, $-$, word in the $\vec a$. As before, we expand `$y_i
\in t_j$' until they have all been eliminated, and recast the matrix into
DNF. As before we know that not all the disjuncts can trivially violate the
theory of identity since all results of substituting \nf O words for the
$a_i$ in `$(\exists \vec y)(\Phi (\vec x,\vec y))$' are satisfiable.
Fasten on one good disjunct. Look at the $\vec y$ of minimal type. The
conditions like `$y \in t_j$' have been replaced by boolean combinations of
conditions saying that such-and-such $a_i$ are $\in\ y$ or not, as the case
may be.  Now how many conditions of this sort are there on any of these
minimal $y$?  Clearly at most as many as there are things in $\vec a$, so
the desired witness is a member of the boolean interval $[\{\vec a_i: i \in
I\}, -\{\vec a_j:j \in J\}]$ for some sets $I$, $J$ of $a$'s. $I$ and $J$
must be disjoint, since we know that the disjunct we are contemplating does
not violate the elementary theory of $=$. So if there were no inequations
around we would have shown that $(\exists \vec y)(\Phi (\vec x,\vec y))$.
However we now have to accomodate some family of inequations $y \not= x_i$.
These may exclude some more elements of $[\vec a_i, -\vec a_j]$ and we no
longer know that $[\{\vec a_i: i \in I\}, -\{\vec a_j:j \in J\}]$ is
infinite.  What this tells us is that if we have inequations to deal with
we wish $[\{\vec a_i: i \in I\}, -\{\vec a_j:j \in J\}]$ to be big enough
for us to satisfy them all {\sl and that it is only the number of
inequations that we have to worry about}. If $\vec a$ is a proper subset of
$T_{k-2}$ then $[\{\vec a_i: i \in I\}, -\{\vec a_j:j \in J\}]$ will have
many members, and $\vec a$ will be a proper subset of $T_{k-2}$ if $2^{2^n}
< |{T_k}|$.

So as long as $T_k$ is sufficiently large in relation to the number of
inequations (which is bounded by (length of $\vec x\ +$ length of $\vec
y)^2$, we will be able to find witnesses.  In short we can see:

\begin{quote}
   For all $m$ and $n$ there is $k$ such that for all $\Phi$, if 
    $(\forall x_1\ldots x_m)(\exists y_1\ldots y_n)\Phi(\vec x ,\vec y)$ 
    is true in the term model of \nf O, then $(\forall \vec x)(\exists \vec y)
    \Phi\#(\vec x,\vec y)$
    is true in all models of $TST$ where $T_0$ has at least $k$ elements.
\end{quote}

\subsection{Another failed proof}

We are trying to show that any $\exists_2$ stratified sentence true in $M
\models NF0$ is witnessed by a term. Idea:

look at $\exists \vec x \bigwedge \forall \vec y \bigvee$ atomics or negatomics.

We can rewrite to get rid of equations and inequations but it won't help.
We think of each of the conjuncts as a constraint on what term the witness
has to be.  We have the impression that such conjuncts reduce to things
like ``$x$ is the complement of the singleton of something other than
$\Lambda$".  The point is that these say that $x$ must be a value of some
\nf O operation.  If so, this is good news, because a conjunction of
finitely many such conditions can be satisfied in the term model if at all.

each conjunct give rise to something like

$\vec x = \vec t(\vec u)$ subject to finitely many exceptions $\vec u \not= \vec s$ for some \nf O terms $\vec s$

which we shall call a {\sl constraint.}

For example $\forall y_0 y_1 y_2 (y_0 \in y_1 \vee y_0 \in y_2 \vee y_2 \in x \vee y_1 \in x)$

gives rise to 

$x = V - \iota `u$ with exception $u \not= \Lambda$


If this works we then hope that any finite set of constraints has either a
solution containing a parameter (like the singleton list above) in which
case it will certainly have infinitely many solutions, or it will have none
at all, in which case it wasn't true in $M$ in the first place.

{\sl later}   

Actually it seems that we have to use $NF\forall^1$ words for this

\subsection{A third failed proof}

\begin{prop} : NF decides all stratified $\forall^*\exists$ sentences. 
\end{prop}

         In fact we will prove something significantly stronger. Let us
descend to simple type theory for a while, and accordingly impose type
subscripts on our variables. We will show that every wff

  ($\forall \vec x_0)(\exists y_2)(\forall z_1)\Phi (\vec x_0,y_2,z_1$) 
   with $\Phi (\vec x_0,y_2,z_1$) quantifier-free is true in all
sufficiently large finite models of simple type theory.

    I shall not provide a proof in full, for it makes use of tricks that we
cannot use to show that all stratified $\forall_2$ sentences are decided by
simple type theory.

     First we note that it makes no difference whether the initial
quantifier $(\forall \vec x_0$) in $$(\forall \vec x_0)(\exists
y_2)(\forall z_1)\Phi (\vec x_0,y_2,z_1)$$ is $\exists$ or $\forall$, since
all $n$-tuples will satisfy the matrix if any do.  Next we notice that each
object $x_0$ of type $0$ gives rise to an object ($B`x_0$) of type $2$ and
the subalgebra of the boolean algebra $\langle T_2,\subseteq \rangle$
generated by these elements is free. Having it in mind to make use of this
we invent a one-place predicate $g$ on objects of type 2, whose intended
reading is ``is a member of a set of free generators for $\langle
T_2,\subseteq \rangle$". We now rewrite $$(\forall \vec x_0)(\exists
y_2)(\forall z_1)\Phi (\vec x_0,y_2,z_1)$$ as $$(\forall \vec x_2)(g(\vec
x_2) \to (\exists y_2)(\forall z_1)\Phi (\vec x_0,y_2,z_1))$$ by replacing
$``x_0 \in \ x_1"$ by $``x_1 \in \ x_2 \wedge g(x_2)"$ where the $x_2$ are
secretly the various $B`\vec x_0$. Next we show that the innermost
quantifier---($\forall z_1$)---can be assimilated into the matrix to result
in an expression

$$ {\rm A} \  (\forall x_2)(g(x) \to (\exists y_2)\Psi (x_2,y_2))$$ 

in the language of boolean algebras with the added primitive $g$,
where $\Psi$ is quantifier-free.  Finally some elementary
manipulations in boolean algebra will show that if we furnish $g$ with
this interpretation then any sentence like A above with any models at
all is true in all sufficiently large finite free boolean algebras. I
am grateful to Peter Johnstone says the witnesses to the $y_2$ can be
found among words in the $x_2$.  {\sl Peter says:

    can assume only one $y$.  Restrict ourselves to combinations of

$p(\vec x,y) \leq q(\vec x y)$ \wwlog\

$p = \bigwedge \vec x \& \neg \vec x,y$
$q = \bigcup \vec x \& $[illegible]
$y$ occurs on only one side. so reduces to
$y \leq q(\vec x)$ or $p(\vec x) \leq y$

$\bigvee p_j \leq \bigwedge q_i$

set $y = V$ or $\Lambda$.}
Can't piece it together \ldots

A hard case: consider the assertion that the meet of all the $\vec y$ is
not an atom. This is certainly satisfiable in suff big algebras, but is not
true in the algebra generated by the $\vec y$.  

\section{stuff to fit in}

If $x \in x$ then $x$ meets $\pow x)$.  So what does $x \cap \pow x)$
look like?   What can we say about it in a $\forall^*\exists^*$ way?

$(\forall a \in x \in x)((x \setminus a) \in x)$


Here's another thing.  Where do self-membered sets come from in NF?
$V$ is a member of itself, and we can get get further self-membered
sets by means of NFO operations.  The NFO operations can give us sets
that are members$^2$ of themselves but these sets usually turn out to
be self-membered anyway.  $V \in \{V\} \in V$ but then $V \in V$.  So
is it the case that---in the term model for NF---$x \in y \in x \to x \in x$?  
No, beco's of $\{V\}$.  However we could try this:

$$(\forall y_1 y_2)(y_1 \in y_2 \in y_1 \to y_1 \in y_1 \vee y_2 \in y_2)$$

Obviously not, co's it's $\forall^*$. But how about the assertion that given an $n$-loop, one of the things in it belongs to an $n-1$-loop?



\subsection*{}

Paul Studtmann writes:

Robinson's Arithmetic is complete with respect to quantifier free sentences.
I am wondering whether anyone can tell me if an analog of this holds in set
theory. Suppose, for instance, that the language contains two constants --
one for the empty set and one for the set of finite ordinals -- as well as
function symbols for the basic set theoretic operations like set union, set
difference, power set, pairing, etc. Is ZF (or a fragment thereof) or some 
other theory complete with respect to all the quantifier free sentences in
the language?

If you omit the power set operator from the list and by ``union''
binary union is meant, then ZF, but also ZF $\setminus$ Power Set
Axiom and even weaker theories, are complete with respect to
quantifier free sentences (equiv.  atomic sentences).  That can be
inferred from the decidability of truth in V for existential closures
of restricted purely universal formulae with no nesting of quantified
variables, over the primitive language of set theory with the addition
of constants for the empty set and the set of finite ordinals (as well
as a unary predicate Ord(x) for ``x is an ordinal'') (Breban M., Ferro
A., Omodeo E., Schwartz J.T. ``Decision Procedures for Elementary
Sublanguages of Set Theory II. Formulas involving restricted
quantifiers together with ordinal, integer, map and domain notions''
Comm. on Pure and Applied Mathematics XLI 221-251 (1988) - see also
Ch.7 in Cantone D., Ferro A., Omodeo E ``Computable Set Theory Vol 1''
Oxford University Press, 1989 and my [FOM] of june 3th, 2003)

In fact the operations of (binary) union, intersection and
set-difference as well as the operation of n-tuple formation have a
restricted purely universal definition with no nesting of quantified
variables, so that an atomic sentence which also involves them, turns
out to be equivalent to a sentence which belong to the decidable class
described above.  Completeness follows since the proof of the
decidability of the class in question, which exhibits an actual
algorithm that shows that it does what it is supposed to do, can be
formalized inside ZF$\setminus$ Power Set Axiom (and even weaker
theories).



Franco Parlamento


Dipartimento di Matematica e Informatica

Universita' di udine

Via delle Scienze 208

33100 UDINE

Italy

parlamen@dimi.uniud



\subsection{The term model of NF0}


The term model of \nf O can be thought of as the algebra of all words
in \nf O operations reduced by \nf O-provable equations. This quotient
is unique and well-defined; a proof can be found on p. 376 of
\cite{forster}.  It behaves in some ways like a countably categorical
structure.

\begin{thm}\label{thm:angus} 
For every countable binary structure $\M$ there is a nice family of
embeddings into the term model for \nf O.
\end{thm}

\Proof 

We will prove this by refining the construction of my 1987 paper to
obtain a construction of a nice family of embeddings.
 
The 1987 construction takes a countable binary structure $\M =
\tuple{M,R}$ equipped with a wellordering of length $\omega$ and gives
to each initial segment (or more strictly, its domain) an injection
into the term model.  We will do something slightly more complicated.
We will not be providing injections-into-the-term-model to (domains
of) initial segments of a fixed wellordering: our
injections-into-the-term-model will be defined on the domains of
finite partial functions from $M$ to $\Nn$.  We will think of these
finite partial functions as lists of ordered pairs so that we can
construct the nice family of injections by primitive recursion on
lists.  Doubled colons is our notation for consing things onto the
front of lists, so that---to take a pertinant
example---$\tuple{x,k}::s$ is the finite map that agrees with $s$ on
its domain and additionally sends $x$ to $k$.  We will construct for
each $s$ an injective homomorphism $i_s$ from $dom(s) \in$ 
term-model-for-\nf O, and this family of maps will be nice.


We will need an infinite supply of distinct selfmembered sets and an
infinite supply of distinct non-selfmembered sets: such a supply can
easily be found with the help of the $B$ function. Let the $n$th {\sl
  left} object be $B^n(V)$ and the $n$th {\sl right} object be
$B^n(\emptyset)$. All left objects are self-membered and no right
objects are.  The exponent gives us a convenient notion of {\sl rank}
of these left and right objects.  It will be important in what follows
that every value of any $i_s$ has finite symmetric difference with a
left object or a right object.  It will also be important that any two
left or right objects have infinite symmetric difference.

For $s$ a finite partial map $M \to \Nn$ we will construct $i_s$ from
$dom(s)$ to the term model by primitive recursion on lists.

We start with the empty map from the empty substructure (the domain of
the empty partial map).

The variable `$s$' will range over finite partial maps $M \to \Nn$ and
for each $s$, $i_s$ will be an injective homomorphism from $dom(s)$ to
the term model for \nf O.

For the recursion (primitive recursion on lists) let us suppose we
have constructed a map $i_s$ and we want to construct
$i_{\langle x,k\rangle::s}$.  And we must have $i_{s'} \not= i_{s''}$
whenever $s \not= s''$.

The construction of $i_{\langle x,k\rangle::s}$ from $i_s$ is uniform in $x$
and $k$.  $i_{\langle x,k\rangle::s}$ will agree with $i_s$ on $dom(s)$ of
course.  During the earlier construction of $i_s$ we will have used
some left objects and some right objects.  Let $n_s$ be the least $n$
such that the only left or right objects touched so far in the
construction of $i_s$ have indices below $n$. Now, given $k \in \Nn$,
we want $X$ to be a left object or a right object, depending on
whether $\M \models R(x,x)$ or not, and we set it to be the $(n_s +
k)$th such object, or the $(n_s + k + 1)$th, if $(n_s + k)$ is odd.
$X$ is thus a left or right object, with a subscript that is even and
is larger than any subscript we have seen so far.

$i_{\langle x,k\rangle::s}(x)$ will be obtained from $X$ by adding and
removing only finitely many things. We have to add things in $A$ and
delete things that are in $B$:
\begin{quote}
$A$: $\{i_s(m): m \in dom(s) \ \wedge \M \models  R(m,x)\}$ 

$B$: $\{i_s(m): m \in dom(s) \ \wedge \M \models  \neg R(m,x)\}$ 
\end{quote}

$C$ and $D$ are harder to deal with:

\begin{quote}
$C$: $\{i_s(m): m \in dom(s) \ \wedge\ \M \models  R(x,m)\}$ 

$E$: $\{i_s(m): m \in dom(s) \ \wedge\ \M \models  \neg R(x,m)\}$ 
\end{quote}
Our final choice for $ i_{\langle x,k\rangle::s}(x)$ must extend $A$,
be disjoint from $B$, belong to everything in $C$, and to nothing in
$E$.  There is no guarantee that $X$ will do, but it's a point of
departure; our first approximation to $ i_{\langle x,k\rangle::s}(x)$
is $(X \setminus B) \cup A$.

For each $m$ in $dom(s)$ let $X_m$ be that left or right object from
which $i_s(m)$ was obtained by the finite tweaking that we are about
to explain. We want to control the truth-value of $i_{\langle
  x,k\rangle::s}(x) \in i_s(m)$.  It's hard to see how to do this
directly, but one thing we {\sl can} control is the truth-value of 
$i_{\langle x,k\rangle::s}(x) \in X_m$, because this is the same as the
truth-value of $B^{-1}X_m \in i_{\langle x,k\rangle::s}(m)$ and we can
easily add or delete the various $B^{-1}(X_m)$ from $(X \setminus B)
\cup A$.

Suppose for some particular $m$ we want to arrange that $i_{\langle
  x,k\rangle::s}(x) \in i_s(m)$.  We put $B^{-1}(X_m)$ into
$i_{\langle x,k\rangle::s}(x)$.  This ensures that $i_{\langle
  x,k\rangle::s}(x) \in X_m$.  This is very nearly what we want, since
the symmetric difference $X_m\ \Delta\ i_s(m)$ is finite.  Now because
we chose $n_s$ to be larger than the subscript on any left or right
object we had used so far in building $i_s$ we can be sure that $
i_{\langle x,k\rangle::s}(x)$ is not one of the finitely many things
in $X_m\ \Delta\ i_s(m)$.  So $i_{\langle x,k\rangle::s}(x) \in X_m$
and $i_{\langle x,k\rangle::s}(x) \in i_s(m)$ have the same
truth-value.

In the light of this, we obtain $ i_{\langle x,k\rangle::s}(x)$ from our 
first approximation---$(X \setminus B) \cup A$---by adding everything 
in $\{B^{-1}(X_m): \M \models R(x,m)\}$ and deleting everything in
$\{B^{-1}(X_m): \M \models \neg R(x,m)\}$.  Just a final check to
ensure that this doesn't interfere with the adding and deleting we did
initially, by adding everything in $A$ and deleting everything in
$B$: this last stage adds and deletes left-or-right objects with {\sl
  odd} subscripts, whereas the initial tweaking added and deleted
left-or-right objects (if any) with {\sl even} subscripts only.  \endproof

\begin{coroll}
Every countable binary structure embeds into the term model of 
\nf O in $2^{\aleph_0}$ ways.
\end{coroll}


The general theme of this note is extending to the logic of the
cofinite quantifier the various known results about ordinary logic 
and the Quine systems.  We know that every $\exists^*$ sentence 
consistent with \nf O holds in the term model.  To get a version 
for the cofinite quantifier we need to get straight the idea of a 
$\exists_\infty^*$ formula consistent with \nf O.

``Being consistent'' in this sense for a formula $(\exists_\infty x_1
 \ldots x_n)\phi$ where $\phi$ is quantifier-free means the following.
Suppose $\phi$ has $n$ free variables.  Then we invent constants whose
suffixes come from $\Nn^{\leq n}$.  For each sequence $c_{i_1} \ldots 
c_{i_n}$ where the suffix $i_{k+1}$ is of length $k+1$ and is an
end-extension of the suffix $i_k$, we adopt the axiom $\phi(c_{i_1}
 \ldots c_{i_n})$.  Call this theory $T$.  Then $T$ is equivalent to
$(\exists_\infty x_1 \ldots x_n)\phi$ in the sense that every model of
$T$ is an expansion of a model of $(\exists_\infty x_1 \ldots 
x_n)\phi$ and {\sl vice versa}.  


\begin{thm}
  Every $\exists_\infty^*$ formula consistent with \nf O is true in
  all models of \nf O.
\end{thm}
 
\Proof Let $(\exists_\infty x_1 \ldots x_n)\phi$ be such a formula,
and $T$ the theory obtained from it as above. Now every axiom of $T$
is a consistent $\exists^*$ formula, and so is true in the term model,
and so is a theorem of \nf O.


\endproof

Notice that we haven't yet had to exploit the clever construction of
nice embeddings.  That happens next.

\begin{rem}\label{angus2}
The term model for \nf O satisfies every $\forall_\infty^*\exists_\infty^*$ 
formula consistent with \nf O.
\end{rem}

\Proof Consider $(\forall_\infty x_1 \ldots x_n)(\exists_\infty y_1 \ldots 
y_k)\phi(\vec x, \vec y)$.  Suppose this has a model $\M$.  We want to
show that it is true in the term model.  For this it will suffice to
show that if $\vec t$ is any tuple of terms such that $\M \models
(\exists_\infty y_1 \ldots y_k)\phi(\vec t, \vec y)$ then there are
infinitely many many {\sl terms} $s_1$ such that there are infinitely
many terms $s_2$ etc such that $\phi(\vec t, \vec s)$.

The first step is to simplify $(\exists_\infty y_1 \ldots
y_k)\phi(\vec t, \vec y)$ to the limits of our ingenuity.  We know
that atomic formul{\ae} in $\phi$ need never be of the form `$y_j \in
\ t_i$', because any such atomic wff can be expanded until it becomes
a boolean combination of atomic wffs like `$y_i = t_j$', `$y_j \in
y_i$', and `$t_j \in \ y_i$'. Then we can recast the matrix into
disjunctive normal form.  We know that $\M \models (\forall_\infty
\vec x)(\exists_\infty \vec y)(\Phi (\vec x,\vec y))$ so there is at
least one disjunct that does not trivially violate the theory of
identity. This disjunct is a conjunction of things like `$y_i = t_j$',
`$y_j \in y_i$', and `$t_j \in \ y_i$' and their negations, atomic
wffs not containing any $\vec y$ having vanished since they are
decidable.

We now have to find ways of substituting $\nf O$ terms $\vec w$ for
the $\vec y$ to make every conjunct in the disjunct true.  To do this
we return to the constructions seen in the proof of theorem
\ref{thm:angus}.  We construct witnesses for the $\vec y$ in the way
we constructed values of the function $i$ in the proof of theorem
\ref{thm:angus}. Let $n_0$ be some fixed integer such that all the
$t_i$ that appear in our disjunct have $B$s nested less deeply than
$n$.  We know of (the infinitely many witnesses that we have to find
for) $y_0$ that they is to have certain $t$s as members and certain
others not. For each $k \in \Nn$ we construct a word $w_0$ which is
the $n_0 + k$th left member (if `$y_0 \in y_0$' is a conjunct) or the
$n_0$th right object (otherwise) $\cup$ (the tuple of $t_i$ such that
`$t_i \in y_0$' is a conjunct) minus (the tuple of $t_j$ such that
`$t_j \not\in y_0$' is a conjunct).  From here on, we construct words
$w_i$ to be witnesses for $y_i$ in exactly the same way as we proved
theorem \ref{thm:angus}.  \endproof

%(Irvine 11.iv.03)



Actually we can exploit the theorem (Yasuhara?) that says that all
occurrences of `=' within the scope of a `$\forall_\infty$' can be
massaged away.
 

\begin{thm}\label{another}
  If $\nf O \vdash \exists \vec x\forall \vec y \phi(\vec x, \vec y)$
  where $\phi$ is quantifier-free then for some tuple $\vec t$ of \nf O 
  words, we have $\nf O \vdash \forall \vec y \phi(\vec t, \vec y)$.
\end{thm}

\Proof
 Let $\exists \vec x\forall \vec y \phi(\vec x, \vec y)$ be a
$\exists^*\forall^*$ sentence, and suppose that for every tuple $\vec
t$ of \nf O terms it is consistent that the tuple $\vec t$ is not a
witness to the $\vec x$.   Then the scheme 

\begin{equation}\label{eqnscheme}
(\exists \vec y)(\neg\phi(\vec t, \vec y)) 
\mbox{\rm\  over all tuples of terms\ } \vec t\end{equation}
 is consistent.
 
How complicated is scheme \ref{eqnscheme}?  Well, each instance is
equivalent to a disjunction of things of the form $(\exists \vec
y)(\psi(\vec t, \vec y))$ where $\psi$ is a conjunction of atomics and
negatomics.  What sort of atomics and negatomics?  Well, equations and
inequations between the $t$s disappear beco's they are all $T$ or $F$
by elementary means.  Equations $y=t$ can be removed by replacing all
occurrences of `$y$' by `$t$'.  What's left?  Inequations $y \not= t$
and $y \in t$, $t \in y$, $y \not\in t$, $t \not\in y$.  We attack
those recursively.  $y \in t$ might be $y \in t_1 \wedge y \in t_2$,
in which case we recurse further.  If it is $y \in t_1 \vee y \in t_2$
then the $\exists^*$ formula in which it occurs gets split into two
such formul{\ae}.  If we keep on doing this we will end up with a
disjunction of $\exists^*$ formul{\ae} with terms appearing, but only
in inequations or to the left of an `$\in$'.  Clearly any such
disjunction, if satisfiable at all, is satisfiable with the witnesses
being finite tuples of terms, and is therefore true in the term model.
So each instance of scheme \ref{eqnscheme} is true in the term model.
That is to say, the term model believes $(\forall \vec t)(\exists \vec
y)(\neg\phi(\vec t, \vec y))$. So the original $\exists^*\forall^*$
sentence is not true in the term model, contradicting our assumption
that $\nf O \vdash \exists \vec x\forall \vec y \phi(\vec x, \vec y)$.

So if \nf O proves a $\exists^*\forall^*$ sentence, there are provably 
witnesses that are \nf O terms. \endproof

 
By now the reader will have thought enough about extending these
results to isomorphic formul{\ae} in the language with the cofinite
quantifier to have spotted that in the last para of the last proof
there are of course {\sl infinitely many} ways of satisfying such
disjunctions.  Accordingly I hope that later draughts of this note
will contain a proof of


\begin{thm}
If $\nf O \vdash (\exists_\infty\vec x)(\forall_\infty \vec y)
\phi(\vec x, \vec y)$ where $\phi$ is quantifier-free then for a
suitable infinity of tuples $\vec t$ of \nf O words, we have $\nf O
\vdash (\forall_\infty \vec y)\phi(\vec t, \vec y)$.
\end{thm}


We must think a bit about the scenario that the theorem describes.
``$\nf O \vdash (\exists_\infty \vec x)(\forall_\infty \vec y)\phi(\vec
x, \vec y)$'' means simply that in every model of \nf O we can find
infinitely many $x_1$ such that for each of them we can find
infinitely many $x_2$ etc.  The claim then is that, whenever this
happens, we can take this network of $x$s to be \nf O terms.

Now suppose the claim is false, and that altho' in every model of \nf
O we can find infinitely many $x_1$ such that for each of them we can
find infinitely many $x_2$ etc., we cannot take all of these witnesses
to be terms.

That is to say, if we take any set of countably many terms---and think
of them as $t_s$ where $s$ is a sequence of natural numbers of length
at most the length of $\vec x$---then the scheme

\begin{equation}\label{next}
(\forall_\infty \vec y) \phi(t_i, t_{i,j}, t_{i,j,k} \ldots \vec y)  \mbox{\rm\  over all tuples of terms\ } \vec t\end{equation}

is not a theorem scheme.  We wish to show that this scheme fails in
the term model.  So let $(\forall_\infty \vec y) \phi(t_i, t_{i,j},
t_{i,j,k} \ldots \vec y)$ be one of the instances that is not a theorem.
Its negation is 
$$(\exists_\infty \vec y) \phi(t_i, t_{i,j}, t_{i,j,k} \ldots \vec y)$$

and we wish to show that this is true in the term model.  But this can
be done by the constructions of theorem \ref{thm:angus} and remark
\ref{angus2}.

See section \ref{truss} of quantifiertalk.tex for a discussion of the
correct generalisation of this to random/generic/countably categorical
structures.

\bigskip

It's worth asking whether or not we can prove that every Henkin
sentence consistent with \nf 0 is true in the term model for \nf
0. And of course there is the same question about \TZT 0.


But this is immediate!

%13/x/13

\bigskip

%16/vi/2017

It has been a puzzle to me for many years how the term model for \nf O
could have all these homogeneity properties exploited so nicely above
and yet be rigid!  I think the answer is that the homogeneity comes
from the axioms giving $\cup$, $\cap$, set difference, $\{x\}$ and
$B(x)$ and the rigidity happens only once you get $\emptyset$ and $V$.
You can get the infinite sequence of left and right objects starting
with any $x \in x$ and $y \not\in y$.

Let's sort this out properly.  Exactly what do we need to prove theorem
\ref{thm:angus}?   We need infinitely many left and right objects of course
but beyond that i think we need only $B$ and adjunction and subscission.

So it looks to me as tho' all we need is two constants $a$ and $b$ with 
$a \in a$, $b \not\in b$ and $B^n(a) \not= a$ and $B^n(b) \not= b$ for 
all $n$, plus adjunction, subscission and existence of $B(x)$. (I'm guessing 
that this theory is sufficient to show that any two $B$s have infinite 
symmetric difference, which we do need). Do we need $\bbar$ as well?   I think not.


\section{Friederike K\"orner on Model Companions of Stratified Theories: notes by Thomas Forster}


Let $T$ be a theory in the language {$\cal L$} of set theory (= and
$\in$) with (at least) the axioms of extensionality and
$$\forall x_1 \ldots x_n \exists y (y = \{x_1, \ldots x_n\})$$
existence of unordered $n$-tuples.

We assume that $T$ has an infinite model in which every transposition
is setlike.

\subsection{The set of universal consequences of $T$}

\begin{prop}\label{prop1.1}
Every finite {$\cal L$}-structure can be isomorphically embedded in 
some model of $T$.
\end{prop}
\Proof

Let $\A = \tuple{A, \in_{\A}}$ be an arbitrary finite {$\cal
  L$}-structure.\footnote{Something like this is in Hinnion's thesis}
($A = \{a_1 \ldots a_n\}$). Let $\M$ be a model of $T$ where every
permutation of finite suport is setlike. Choose distinct $c_1 \ldots
c_n \in \M$ and define a permutation $\tau$ by $$\tau(c_i) = \{c_j: \A
\models a_i \in c_j\}$$ for $1 \leq i \leq n$. Then $\M^\tau \models
c_j \in c_i$ iff $\A \models a_j \in a_i$ (for $1 \leq i \leq n$ as
before).  So the function $f$ sending each $a_i$ to the corresponding
$c_i$ is an isomorphic embedding.

\endproof

\begin{coroll}\label{coroll1.2}
Any {$\cal L$}-structure can be isomorphically embedded in a model of $T$.
\end{coroll}
\Proof Compactness \endproof
\begin{dfn}
$T_\forall$ is the set of $\forall^*$ theorems of $T$.
\end{dfn}

\begin{coroll}\label{coroll1.3}
$T$ is an extension of $LPC$ conservative for $\forall^*$ formul{\ae}
\end{coroll}

%\Proof (We could also say $T_\forall \subseteq LPC$.) \endproof



\section{The Model Companion of $T$}
\begin{dfn}
A theory $T$ is {\bf model-complete} iff every embedding between
models is an elementary embedding. (Equivalently, every first-order
formula is equivalent to a universal formula. This notion was
introduced by Abraham Robinson.)
\end{dfn}

\begin{dfn}
A theory $T^*$ in {$\cal L$} is the {\bf model companion} of $T$ if
$(T^*)_\forall = T_\forall$ and $T^*$ is model-complete
\end{dfn}

If $T$ has a model companion at all then it is unique.

Now we are going to define the theory $T^*$ which will turn out to 
be the model companion of $T$.  Let $\gamma(x, y_1 \ldots y_n)$ be 
a conjunction of some of the following atomic and negatomic formul{\ae}: 
$x \in x$, $x \not\in x$, $x \in y_i$, $x \not\in y_i$ ($1 \leq i \leq n$) 
$y_i \in x$, $y_i\not\in x$ ($1 \leq i \leq n$).  Now if
$$\bigwedge_{1 \leq i < j \leq n}(y_i \not= y_j \wedge x \not= y_i) 
\wedge \gamma(x, y_1 \ldots y_n)$$ is 
satisfiable\footnote{Does she mean ``consistent with $T$?"} then
$$(\forall y_1 \ldots y_n)(\exists x)\bigwedge_{1 \leq i < j \leq n}y_i \not= y_j \to (\bigwedge_{1 \leq i \leq n} x \not= y_i \wedge \gamma(x, y_1 \ldots y_n))$$
is an axiom of $T^*$. $T^*$ has no other axioms.

\begin{prop}\label{prop2.3}
$T^*$ is consistent
\end{prop}

\Proof Since all the axioms of $T^*$ are $\forall^*\exists$ sentences we
have grounds to hope that we can devise a model which is a union of a
countable chain of models.  Enumerate all the axioms of $T^*$ as
$\tuple{\phi_n: n \in \Nn}$ in such a way that every axiom appears 
infinitely often.


Start with $\M_0 = \tuple{M, R_0}$ where $M = \{a_i: i \in \Nn\}$ and 
if $i \not= j$ then $a_i \not= a_j$. Thereafter construct $\M_{n+1}$ from
$\M$ as follows:

Suppose $\phi_n$ is $(\forall y_i \ldots y_n)(\exists x)\psi(x, y_1 \ldots
y_n)$. If $\M_n \models \phi_n$ then $\M_{n+1} = \M_n$.  Otherwise let
$\tuple{a_{i_1} \ldots a_{i_k}} \in \M^k$ be the first $k$-tuple (in the
lexicographic order of $\M^k$) for which there is no $x$ such that $\exists
x \psi(x,a_{i_1} \ldots a_{i_k})$. Let $b$ be the $a_i$ of smallest index
which is not in $dom(R_n) \cup rn(R_n)$.  $R_{n+1}$ is now obtained from
$R_n$ by adding enough pairs $\tuple{a_{i_l},b}$, $\tuple{b,a_{i_l}}$ to
make $\gamma(b,a_{i_1} \ldots a_{1_k})$ true.

$\M$ is the direct limit of the $\M_i$ and is a model of $T^*$.

\footnote{Why do we want each $\phi$ to appear infinitely often?  Presumably
all this is ``standard model theoretic nonsense".}

\begin{prop}\label{prop2.4}
$(T^*)_\forall = T_\forall$.
\end{prop}

\Proof

We have to show that for every {$\cal L$}-structure $\A$ there is a model 
$\A \models T^*$ with $\A \subseteq \M$.

Let $\A$ be an arbitrary {$\cal L$}-structure and let $\Delta_{\A}$ be
the diagram of $\A$ in the language ${\cal L}_{\A}$. We claim that any
finite subset of $\Delta_{\A}$ is consistent with $T^*$.  Let $\Sigma$
be a finite subset of $\Delta_{\A}$. There are only finitely many
constants $c_1, \ldots c_n$ that occur in $\Sigma$. We may assume that
$c_i \not= c_j$ for $1 \leq i \leq j \leq n$.  Let $\gamma(c_1)$ be
the conjunction of the formul{\ae} in $\Sigma$ that contain $c_1$
only. Choose $a_1 \in \M$ such that $\M \models
\gamma(a_1)$. Thereafter, having chosen $a_1 \ldots a_i \in \M$, let
$\gamma(c_{i+1},c_1 \ldots c_i)$ be the conjunction of those of the
following formul{\ae} that are in $\Sigma$: $c_{i+1} \in c_{i+1}$,
$c_{i+1} \not\in c_{i+1}$, $c_{i+1} \in c_l$, $c_l \in c_{i+1}$ ($1
\leq l \leq i$).  Since $\gamma(c_{i+1},c_1 \ldots c_i)$ is
satisfiable, $(\forall y_1 \ldots y_n)(\exists x)\bigwedge_{y_i \not=
  y_j} \to \bigwedge_i (x \not= y_i) \wedge \gamma(x, \vec y))$ is an
axiom of $T^*$ and thus there is $a_{i+1} \in \M$ with $\M \models
\bigwedge_{k = 1}^i a_k \not= a_{i+1} \wedge \gamma(a_{i+n},a_1 \ldots
a_i)$.  Therefore $\tuple{\M,a_1 \ldots a_n} \models T^* \cup \Sigma$.

\endproof

\begin{prop}\label{prop2.5}
$T^*$ is model complete
\end{prop}

\Proof

Use Lindstr\"om's theorem. (See, for example, Chang and Keisler 3rd
edn 3.5.8.)  To do this we must show:
\begin{enumerate}
\item All models of $T^*$ are infinite.
\item $T^*$ is preserved under unions of chains.
\item $T^*$ is $\alpha$-categorical for some $\alpha \geq \aleph_0$.
\end{enumerate}

($1$) is obvious. ($2$) follows from the fact that $T^*$ has a set of
$\forall^*\exists^*$ axioms.  As for ($3$), a back-and-forth argument 
will show that $T^*$ is countably categorical.

Suppose $\A = \tuple{A,\in_A}$ and $\B = \tuple{B,\in_B}$ are countable
models of $T^*$. Wellorder $\A$ and $\B$ in order-type $\omega$ by
$\leq_{\A}$ and $\leq_{\B}$.

Let $a_0$ be the $\leq_{\A}$-first element of $A$, and let 
$\gamma_0(x) = x \in x$ (if $a_0 \in_{\A} a_0$) and $x \not\in x$ 
(otherwise). Let $b_0$ be the $\leq_{\B}$-first member of $B$ that 
satisfies $\gamma_0()$ and set $f`a_0 \ b_0$.

Now suppose we have constructed $n$ pairs in $f$.

Two cases 
\begin{itemize} 
\item $n+1$ is even.  Let $a_{n+1}$ be the $\leq_{\A}$-first element
not in the domain of the $f$-so-far. Let $\gamma(x,y_0 \ldots y_n)$ be
$x \in^* x \wedge \bigwedge_{i=0}^n x \in^* y_i \wedge
\bigwedge_{i=0}^n y_i \in^* x$ where the asterisks on top of the
epsilons mean that they should be negated, or not, so that $\A \models
\gamma(x,y_0 \ldots y_n)$. Since $(\forall y_1 \ldots y_n)(\exists
x)(\bigwedge y_i \not= y_j \to (\bigwedge x \not= y_i \wedge
\gamma(x,y_1 \ldots y_n) \in T^*$ we infer that $\B \models \exists x
\gamma(x,b_0 \ldots b_n) \wedge \bigwedge_{i = 0}^n x \not=
b_i$. Define $b_{n+1}$ to be the $\leq_{\B}$-first element $b$ of
$B\setminus \{b_0 \ldots b_n\}$ that satisfies $\gamma(b,b_0 \ldots
b_n)$ and set $f`a_{n+1} = b_{n+1}$.

\item $n+1$ is odd. Let $b_{n+1}$ be the $\leq_{\B}$-first element not in
the range of the $f$-so-far. \ldots and procede as before.

\end{itemize}\endproof

\begin{prop}\label{prop2.7}
$T^*$ is the model-completion\footnote{That is to say, $T^*$ is the model
companion of $T$ and, for any model $\M \models T$, $T\cup \Delta_{\A}$ is
complete.} of $T$.
\end{prop}

\Proof  It will be sufficient to show that $T$ has the amalgamation property.

Let $\A = \tuple{A, \in_{\A}}$, $\B = \tuple{B, \in_{\B}}$ and ${\cal
  C} = \tuple{C, \in_{\cal C}}$ be three disjoint models of $T$ with
$f: {\cal C} \inj \A$ and $g:{\cal C} \inj \B$.  Define an
{$\cal L$}-structure ${\cal D}$ as follows. The domain $D$ will be $C \cup
(A \setminus f``C) \cup (B \setminus g``C)$.  Then, for $a,b \in D$
set $a \in_{\cal D} b$ iff one of the following holds:

$a, b \in C$ and $a \in_{\cal C} b$

$a, b \in B$ and $a \in_{\cal B} b$

$a, b \in A$ and $a \in_{\cal A} b$


\hole{exercise: complete this definition!!!!}

\begin{verbatim}

From koerner@math.tu-berlin.de Fri Jun 12 15:03:12 1998

> 
>   You know i have a conjecture
> that NF remains consistent if you add to it
> every $\forall^*\exists^*$ (or $\forall_2$ if
> you prefer) sentence that is consistent with 
> it.    This is presumably something to do with
> NF having a model companion.

\end{verbatim}

Your question concerns the stuff in Ch2 of my thesis (do you have a copy ?, i forget).

If i recall correctly, the basic facts are

NF has a model companion, i.e. there is a theory $T$ which has
exactly the same universal consequences as \nf\ (i.e. no sentences
except tautologies) and is model complete. (for definitions etc. see
e.g. Chang/Keisler, 3rd ed., 3.5)

The countable model of $T$ is countably categorical and probably
should be named ``the countable universal homogeneous di-graph".

That is, it's the theory consists of all the sentences saying:
i) 

for all finite disjoint sets $I$, $J$ of points (vertices) and
all all finite disjoint sets $K$, $L$ of points (vertices) 
there is a point $x$ s.t.  \begin{itemize}
\item  
$xRy_i$ for all $y_i \in I$,

$\neg xRy_j$ for all $y_j \in J$,

$y_k R x$ for all $y_k \in K$ and

$\neg y_l R x$ for all $y_l \in L$ and

$xRx$

and \item

for all finite disjoint sets $I$, $J$ of points (vertices) and
all all finite disjoint sets $K$, $L$ of points (vertices) 
there is a point $x$ s.t.  

$xRy_i$ for all $y_i \in I$,

$\neg xRy_j$ for all $y_j \in J$,

$y_k R x$ for all $y_k \in K$ and

$\neg y_l R x$ for all $y_l \in L$ and

$\neg xRx$ .

\end{itemize}

$T$ admits elimination of quantifiers.  All $\forall_2$-sentences
which are consistent with \nf\ are true in $T$.  Unfortunately the
converse is false.

Love, Friederike

\section{More thoughts about NF0}

If we add a constant symbol `$V$' for the universe, and function symbols 
$B$, $\{,\}$ (for singletons) and the boolean operations $\setminus$ and 
$\cup$ then we can axiomatise NF0 as a $\forall^*$ theory as follows. 


$(\forall x y)(x \in B(y) \to \bic y \in x))$\\
$(\forall x y)(x \in \{y\} \bic x = y))$\\
$(\forall x y z)(x \in y \cup z \bic (x\in y \vee x\in z)))$\\
$(\forall x y)(x \in \overline{y}\bic x \not\in y)$\\
$(\forall x)(x \in V)$\\

and extensionality is 

$(\forall x y)(x \verb# xor # y = \emptyset \to x = y)$

%\input{universalexistentialnonce}

  
Do we need all the comprehension of $TZT$ to make this work?  It
suffices that every permutation of finite support (or at least every
finite product of disjoint transpositions) should be setlike.  Do we
get this in $\TZT$0? My guess is not.

Can we generalise this to theories with richer axioms than $\TZT0$. No,
or at least not straightforwardly.  We were able to obtain the
assignment $W_p$ by an iterative process that worked by recursion on
types. This was because the characteristic axioms of $\TZT0$ are type
raising.  At least one of the characteristic axioms of $\TZT0$ is
$\bigcup$, which is type-lowering.

\bigskip

 
Corollary: any $\Sigma^{\{B\}}_1$ sentence that is consistent with $\TZT$
is true in the term model for $\TZT 0$, and therefore true in every
model of $\TZT 0$.  So $\TZT 0$ decides all $\Sigma^{\{B\}}_1$ sentences.
  I think every $\forall^*\exists$ sentence is $\Pi^{\{B\}}_1$ so we
    will have proved at least that $\TZT$ decides all
    $\forall^*\exists$ sentences.



\section{Subthingies}

\begin{thm}
Every $\forall^*\exists^*$ sentence true in arbitrarily large finitely
generated model of TST is true in all infinite models of TST.\end{thm}

\Proof
The key is to show that every model of TST can be obtained as a direct
limit of finitely generated models of TST.  The hard part is to find
the correct embedding.

Let $\M$ be a model of TST.  We will be interested in finite
subthingies characterised as follows.  Pick finitely many elements
$x_1 \ldots x_k$ from level 0 of $\M$; they will be level 0 of the
finite subthingie. Then take a partition of level 1 of $\M$ for which
the $x_i$ form a selection set (a ``transversal'').  The pieces of
this partition are the atoms of a boolean algebra that is to be level
1 of the finite subthingie.  That gives us level 1 of the subthingie.
To obtain level 2 we find a partition of level 2 of $\M$ such that the
carrier set of the boolean algebra we have just constructed (which is
level 1 of the subthingie) is a selection set for it.  The pieces of
this partition are the atoms of a boolean algebra that is to be level
2 of the finite subthingie. Thereafter one obtains level $n+1$ as a
boolean alegbra whose atoms are the pieces of a partition of level
$n+1$ of $\M$ for which level $n$ of the subthingie is a transversal.

There is, at
each stage, an opportunity to choose a partition, so this process
generates not {\sl one} subthingie from the finitely many elements
$x_1 \ldots x_k$ from level 0 of $\M$, but infinitely many. This means
that the family of subthingies has not only a partial order structure
but also a topology.  Choosing $n$ things from level 0 does not
determine a single finite subthingie, co's you have a degree of
freedom at each step (when you add a new level).  It's a kind of
product topology, where each finite initial segment (a model of
$TST_k$ with $n$ things at level 0) determines an open set: the set of
its upward extensions.

Is the obvious inclusion embedding an example of what Richard calls 
an almost-$\forall$ embedding?

The long-term aim is to take a direct limit, and we want this direct
limit to be $\M$ itself, so we must check that every element of $M$
can be inserted into a subthingie somehow.

Clearly any finite set of elements of level 0 of $\M$ can be put into
a finite subthingie, but what about higher levels?  We prove by
induction on $n$ that every finite collection of things of level $n$
can be found in some finite subthingie or other.

The induction step works as follows.  We have a subthingie $\M_1$ and
we want to expand it to a subthingie $\M_2$ that at level $n+1$
contains finitely many things $x_1 \ldots x_k$.  To do this we have to
refine the partition of $V_n$ that is the set of atoms that $\M_1$ has
at level $n+1$ so that every $x_i$ is a union of pieces of the refined
partition.  There are only finitely many $x_i$ so any refinement that
does the job has only finitely many pieces.  Identify such a
refinement, and pick a transversal for it that refines the set which
is level $n$ of $\M_1$.  This transversal is a finite set of things 
of level $n$, and we can appeal to the induction hypothesis.\endproof

\bigskip

Next we ask, suppose at each level from 2 onwards, instead of picking
a partition of level $n$ of $\M$ to be the set of atoms of the boolean
algebra at level $n$, we simply take $B``$level $n-2$ to be a set of
generators for the boolean algebra of level $n$?  We lose a degree of
freedom but we get better behaviour of the embedding, since this
ensures that it preserves $B$.  Can we still ensure that every element
of $\M$ appears in the direct product?

Unfortunately the answer to this can be easily shown to be `no' since, 
for the answer to be `yes', one would have to be able to express every 
element of level $n$ of $\M$---for $n$ as big as you please---as 
a $\{B, \cup, \cap, V, \setminus\}$-word in the finitely many 
elements chosen to be level 0 of the subthingie and the elements of the 
partition that are to be level 1.  That is clearly not going to happen.

\bigskip

This proof is essentially the correct general version of the proof in
the book where the same result is claimed only for countable models.
This proof is more general and easier to follow.  The converse problem
remains: can we show that every $\forall^*\exists^*$ sentence true in
even one model of TNT is true in the term model for TNTO?  
%LA oct 2011



\chapter{How many sets are there of any given size?}


How many sets are there that are smaller than the universe?

We have two facts relevant to this question, but we still don't know the
answer.

(i) Tarski has a theorem that every set has more wellordered subsets
than singleton subsets.  This works in NF.  Since no wellordered set
maps onto $V$ this tells us that there are more sets that do not map
onto $V$ than there are singletons:

$|\{x: |x| \not\geq^* |V|\}| > T|V|$

(Funny how this usage seems to suggest that the two collections are
disjoint.  To avoid making this suggestion one seems to have to use 
the word `merely'.)

(ii) Nathan showed that the collection of sets that are surjective
images of $\iota``V$ is itself a surjective image of $\iota``V$.


He also showed that there are precisely $T|V|$ finite sets.  So there
are the same number of wellordered sets of finite sets as there are
wellordered sets of singletons. Do $T^{-1}$ to infer that the set of
wellordered unions of finite sets is a surjective image of the set of
wellordered sets.  In more detail: Sse $f:FIN \bic \iota``V$ this
lifts to a bijection ${\cal P}_{WO}(FIN) \bic {\cal P}_{WO}(\iota``V)$
and of course there is a bijection ${\cal P}_{WO}(\iota``V) \bic
\iota``WO$.  $x \mapsto \{\bigcup x\}$ maps ${\cal P}_{WO}(FIN)$ onto
$\iota``WFIN$ where $WFIN$ is the set of all those sets that are
unions of wellordered families of finite sets.


Can we connect this with the question of whether or not $V$ is a union
of a wellordered family of finite sets?  Is $WFIN$ smaller than $V$?



\subsection{cardinal ideals}
A {\bf cardinal ideal} is a set closed under subset and equipollence.
What Tarski's argument shows in an NF context is that any cardinal
ideal the same size as $\iota``V$ cannot contain all w'ordered sets.

\begin{dfn}
For $I$ a cardinal ideal, let $I^*$ be the set of surjective
images of things in $I$.  $I^*$ is another cardinal ideal, possibly a
proper superset of $I$.  \end{dfn}

 This next little lemma is the result of an idea of Nathan Bowler's.

\begin{lem}\label{lemmanathan}
For all cardinal ideals $I$, $$|I^*| \leq^* |I|.$$
\end{lem} 
\Proof  

Send each $X \in I$ to \verb#fst#$``x$. (\verb#snd#$``x$ would do just
as well). Clearly this gives us members of $I^*$ as values: we just
have to check that everything in $I^*$ is obtained in this way.  Yes,
because if $x \in I^*$ is $f``y$ for some $y \in I$, then $f\restric
y$ is also in $I$: every function is the same size as its domain!
\endproof


%If $f$ is a function defined on a member $X$ of $I$ then \verb#fst# is a bijection between $f$ and $X$---which is in $I$, so $f \in I$.  So anything in $I^*$ is the range of a function which---considered as a set of ordered pairs---is a member of  $I$.  So let $F$ be the function defined on $I$ as follows:  \begin{quote} $F(X) :=$ \verb#if# $X$ is a function \verb#then# $X``I$ \verb#else# $\emptyset$. \end{quote} then every member of $I^*$ is a value of $F$.  So $F$ maps $I$ onto $I^*$. \endproof



\begin{coroll}\label{corolldisj}
$$(\forall \alpha)(\alpha \geq^* |V| \to \alpha = |V|)\ \ \vee\ \  |\{x: |x| < |V|\}| = |V|$$
\end{coroll}

Let $S$ be the cardinal ideal $\{x: |x| < |V|\}$---mentioned in the
corollary---of sets strictly smaller than $V$.  Suppose the left
disjunct fails, so that $V$ is a surjective image of a set strictly
smaller than $V$; this implies that $S^* = V$.  Next we observe that
lemma \ref{lemmanathan} tells us that $S$ maps onto $S^*$---which is
$V$---so $|V| \leq^* |S|$.  This in turn implies that $|V| \leq |\pow
S)|$.  Now, if $|S| < |V|$, we have $\pow S) \subseteq S$, so $S$ is
not smaller than $V$ after all. \endproof


Actually there is a shorter proof.  Sse $A$ is a set with $|A| < |V|
\leq^* |A|$.  Then $\pow A)$ is a set the same size as $|V|$ consisting
entirely of sets smaller than $V$.

\begin{rem}\label{remdouble}\mbox{\negthinspace}\\

\begin{enumerate}
\item $(\forall \alpha, \beta \in NC)(\alpha \leq \beta \to |\alpha| \leq |\beta|)$

\item $(\forall \alpha, \beta \in NC)(\alpha \leq^* \beta \to |\alpha| \leq^* |\beta|)$\end{enumerate}
\end{rem}

\Proof

\begin{enumerate}
\item

%Suppose $\alpha < \beta$ are cardinals.  Fix $A \in \alpha$ and $B \in \beta$ with $A \subset B$.  WLOG we can take $B$ to be included in a moiety.  This means that there are the same number of things in $\alpha$ disjoint from $B$ as there are things in $\alpha$. (Details for the suspicious.  If $M$ is the moiety disjoint from $B$ and $\pi$ a bijection $V \bic M$ then, for any $A' \in \alpha$, $\pi``A'$ is a member of $\alpha$ disjoint from $B$, and the function $A' \mapsto \pi``A'$ is injective.)  Now let $A' \in \alpha$ be disjoint from $B$.  We send it to $(B\setminus A) \cup A'$, which is a member of $\beta$.  This map $A' \mapsto (B\setminus A) \cup A'$ too is injective.  Composing these two injections sends $\alpha$ into $\beta$.  This proves $\alpha \leq \beta \to |\alpha| \leq |\beta|$. \endproof Here's another proof.  

Suppose $\alpha < \beta$ are cardinals.  Fix a
moiety $M$.  Clearly $M$ has the same number of $\alpha$-sized subsets
as $V$ does, so if we can find an injection from $\powk{\alpha} M)$
(the set of $\alpha$-sized subsets of $M$) into $\beta$ we will be done.
Now the moiety $V \setminus M$ will contain a set $c$ of size $\beta -
\alpha$.  We have to be careful here: a set $c$ is of size $\beta -
\alpha$ if its union with a disjoint set of size $\alpha$ is of size
$\beta$.  $\alpha < \beta$ so there are such sets $c$, and $V
\setminus M$ is a moiety and so has subsets of all sizes.  But then
the function from $\powk{\alpha} M)$ defined by $a \mapsto a \cup c$
is injective and all its values are sets of size $\beta$. \endproof

\item Suppose $\alpha \leq^* \beta$.
If $f$ is a function defined on a member $B$ of $\beta$ then \verb#fst# is
a bijection between $f$ and $B$---which is in $\beta$, so $f \in \beta$.  So
anything in $\alpha$ is the range of a function which---considered as a
set of ordered pairs---is a member of  $\beta$.

So fix $A$ an arbitrary set of size $\alpha$, and let $F$ be the
function defined on $\beta$ as follows:
\begin{quote}
$F(B) :=$ \verb#if# $B$ is a function with $|B``\beta| = \alpha$
\verb#then# $B``\beta$ \verb#else# $A$.
\end{quote}
then every member of $\alpha$ is a value of $F$.  So $F$ maps $\beta$
onto $\alpha$. \endproof
\end{enumerate}





We can refine (ii) into a proof that if $\alpha \leq^* \beta$ then
$|\{x: |x| \leq^* \alpha\}| \leq^* |\{x: |x| \leq^* \beta\}|$.   
The $F$ that we need can be defined as:

\begin{quote}
$F(B) :=$ \verb#if# $B$ is a function with 
$|B``\{x: |x| \leq^* \beta\}| \leq^* \alpha$
\verb#then# $B``\{x: |x| \leq^* \beta\}$ \verb#else# $\emptyset$.
\end{quote}

Notice that $F$ has no parameters.  That is to say, we have a {\bf
canonical} construction that gives us, for all cardinals $\alpha
\leq^* \beta$, a map 

$$F_{\alpha,\beta}:\{x:|x|\leq^* \beta\}\onto \{x:|x|\leq^*\alpha\}.$$

Do the $F_{\alpha,\beta}$ commute?  I bet they don't



I think the following is OK.
\begin{rem}
  Let $I$ and $J$ be cardinal ideals with $|J| = T|V|$.  Let $K$ be
  the cardinal ideal $\{\bigcup x: x \in I \wedge x \subseteq J\}$.
  Then $|K| \leq^* |I|$.\end{rem}

\Proof Assume $|J| = T|V|$.  Then we have the following

$\iota``I = \iota``{\cal P}_I(V) \simeq {\cal P}_I(\iota``V) \simeq 
{\cal P}_I(J) \onto \iota``K$

so $|K| \leq^* |I|$. \endproof

The assumption is good (for example) in the case where $J$ is the ideal of 
finite sets.   Presumably (he says, waiting to step on a mine at any moment) 
it will show that there is a surjection from the set of wellordered sets to 
the collection of sets that are wellordered unions of finite sets.

So if $V$ really is a wellordered union of finite sets then the
collection of wellordered sets maps onto $V$.



I noticed years ago that if $x$ injects into its complement, so does
$\pow x)$.  After all, if $x$ injects into $V \setminus x$, $\pow x)$
injects into $\pow V \setminus x)$, which is a subset of $V \setminus \pow x)$.

But actually the same works for other lifts.  If $x$ injects into its
complement, so does $x \times x$ and $x \sqcup x$.  But what kind of
set does {\sl not} inject into its complement?  Some things of size
$|V|$ of course.  But if you are smaller than $V$ and still do not
embed in your complement then you are one piece of a partition of $V$
into two smaller pieces.  This tells us that if $X^2$ is one piece of
a partition of $V$ into two smaller pieces, $X$ too is one piece of
such a partition.  Does this mean that if $\alpha^2, \beta < |V|$ with
$\alpha^2 + \beta = |V|$ then $\alpha + \beta = |V|$?  It looks like
it but we have to be careful.  Even if $\alpha^2 + \beta = |V|$ it
might be the case that whenever $|A| = \alpha$ then $|V \setminus (A
\times A)| = |V|$.  The fly in the ointment is that---for all we
know---it might be that there are sets of size $\alpha^2$ whose
complements are of size $\beta$ with $\beta < |V|$ but whenever $|A| =
\alpha$ then $|V \setminus (A \times A)| = |V|$.


Another cute fact i've just noticed, which will have to be fitted in 
somehow.
\begin{rem}
Let $\alpha$ be a cardinal such that $\alpha \geq \alpha^2 \geq^* |V|$, 
Then there are $|V|$-many sets of size $\alpha$. \end{rem}

\Proof Let $\alpha$ be as in the statement of the remark, and let $A$
be a set with $|A| =\alpha$.  For each $A' \subseteq A$ we have
$\alpha \leq |A \times A'| \leq \alpha^2$ whence $|A \times A'| =
\alpha$.  There are $|V|$-many such $A'$ (beco's $\alpha \geq^* |V|$)
so there are $|V|$-many sets of size $\alpha$. \endproof


\ldots but perhaps this is subsumed by stuff already in these notes.
Of course what is really going on in this proof is the following.
Suppose $A$ is a set of size $\alpha$ and $I$ is the cardinal ideal 
$\{x: |x \times A| = \alpha\}$   Then there are at least $|I|$-many 
things of size $\alpha$.  Sounds a bit fiddly but perhaps we have to 
do fiddly things like that to get anywhere.

\subsection{An old question of Boffa's}

%Consider the recursive datatype ${\cal C}$ generated by the countable (ie countable or finite) sets as founders, and containing $Y$ whenever there is a surjection $f:Y \onto X$ where $X$ is a ${\cal   C}$-set and the preimage $f^{-1}``x$ is a ${\cal C}$-set for every $x \in X$.


%It does serve to make one interesting point.  It is a recursive datatype of infinite character, and this obstructs something one tends to take for granted.  Sitting on top of any recursive datatype is a recursive datatype of {\sl certificates}.  Notice that if the rectype has infinite character and is not free (so that a wombat can be made in more than one way) then one needs the axiom of choice to ensure that the rectype of certificates really contains everything it should. See \verb#logicrave.tex# on this matter of certificates


\subsection{Countable Unions}

Countable unions of$^n$ countable sets.  $C_0$ = set of countable
sets, $C_\alpha$ = countable unions of sets in $\bigcup_{\beta <
\alpha} C_\beta$.  The closure set is $C_\infty$.

Every countable sequence $S$ of sets can be coded up as a single set
 $K(S) = X$ such that $S(0) = \fst(X)$ and thereafter $S(n) =
 \fst(\snd^n(X))$.

This gives us $f_0:\iota``V \onto C_0$  by $f_0(\{x\}) = (K^{-1}(x))``\Nn$ 

Thereafter we can set

$f_{n+1}\{x\} = \bigcup f_n``\iota``x$

\ldots which is stratified but inhomogeneous.  So we can define it for
concrete $n$ but cannot iterate transfinitely.  $f_n: \iota``V \onto
C_n$.

To be more concrete about it: we have two bijections $\theta_1$ and
$\theta_2$ with $\theta_1``V \sqcup \theta_2``V = V$.  $\tuple{x,y}$
is usually $\theta_1``x \cup \theta_2``y$.  But we can do better than this.  
We can encode an $\omega$-sequence $\tuple{x_0, x_1, x_2 \ldots}$ as

$$\theta_1``x_0 \cup \theta_2``(\theta_1``x_1 \cup \theta_2``(\theta_1``x_2 \cup \theta_2``(\ldots$$

or, avoiding the unbounded nesting (since we can):

$$\theta_1``x_0 \cup \theta_2``(\theta_1``x_1) \cup (\theta_2)^2``(\theta_1``x_2)\cup \ldots (\theta_2)^n``(\theta_1``x_n) \ldots $$

By this means we can encode an $\omega$-sequence of things at the same type as the things in the sequence

Notice that every set encodes a countable sequence in this way: the GFP 
for the operation $X \mapsto$ set of $\omega$-sequences-from-$X$ is $V$. 
It would be nice to have a steer on the size of the LFP---or its rank.
We can reach the LFP by starting with the set of all those sequences 
whose every component has size 1 at most.

We might have to be careful.  If we only stop when we reach a
singleton (on the grounds that a ctbl union of finite sets might not
be countable we have to be sure that if we decode a finite set as a
sequence then it is a sequence of singletons, and that might not be
true.   We could just stop descending once we reach finite sets, but 
that looks a bit odd.   Let us call this set $S_\infty$.  

\ldots or we could decide to just start with those $\omega$-sequences 
that are everywhere singletons or empty, and then close under taking 
$\omega$-sequences.  Now it's no longer true that the GFP is $V$ but 
that doesn't matter.

We can probably use a modification of Jech's argument to show that
everything in $S_\infty$ has rank $< \omega_2$.  There is an obvious
projection from $S_\infty \to C_\infty$.  However there is no reason
to suppose that it is surjective.

How can we exploit Jech's construction in a model in which every limit
ordinal has cofinality $\omega$?   Instead of $HC$ we consider the 
rectype of $\omega$-sequences of $\omega$-sequences of \ldots  There 
will be a surjection fromn this family onto $On$.  Or will there?   
Does this need $AC_\omega$ (see the worries about certificates above).


Anyway the idea now is to use a trick like that i used in LIS to show
that you can embed $H_{\aleph_1}$ into $\Re$.  All you need is a set
that is as big as the set of countable sequences from itself.  However
one such set is $\iota^5``V$, and we surely don't expect $S_\infty$ to
embed into anything that small.  The point is that in LIS trick you
start from nothing.  Here you start from the collection of things that
are unions of countably many finite sets.  This is a surjective image
of $FIN^\omega$ which is of size $T|V|$

******************************************************************

$C_0$ is the set of countable sets; $C_{\alpha +1}$ is the set of
countable unions of things in $C_\alpha$; take sumsets at limits.
This recursion is stratified (homogeneous) and so is unproblematic.

At some point the $C_\alpha$ close up, and we reach a set one might
call $C_\infty$.  When do we close up?  And is $C_\infty = V$?

It might be that this body of results we are developing about the
sizes of cardinal ideals will help answer this question: perhaps we
can show that $C_\infty$ is too small to be $V$.

A key observation of course is that, for all $\alpha$, the map that
sends a countable subset $X$ of $C_\alpha$ to $\{\bigcup X\}$ is a
surjection from $\powk{\aleph_1} C_\alpha)$ (= the set of countable
subsets of $C_\alpha$) onto $\iota``C_{\alpha+1}$.

Next we show that 
\begin{rem}\label{firstC}
$|C_0| \leq^* |\iota``V|$.\end{rem}

\Proof Let $\{X_i: i \in \Nn\}$ be a partition of $V$ into $\aleph_0$
moieties, and let $\chi_n$ be a bijection between $V$ and $X_n$.

Then we can encode any sequence $f: \Nn \to V$ as the singleton 

$K(f) := \{\bigcup\{\chi_n``(f(n)): n \in \Nn\}\}$


$K$ is evidently a bijection between $\iota``V$ and the set $\Nn \to
V$ of functions $\Nn \to V$.  Clearly any singleton is the result of
encoding some---unique---$f$ or other.  Thus the map

$$\{x\} \mapsto (K(\{x\})``\Nn$$

is a surjection from $\iota``V$ to $C_0$, the set of countable sets. \endproof

I think we can actually do better than this.  let $\Pi$ be a partition
of $V$ into moieties, equipped with a function $\pi$ such that, for
each $p \in \Pi$, $\pi(p)$ is a bijection between $V$ and $p$.

Now suppose $X$ is a set the same size as $\Pi$, with $\sigma$ a bijection
$X \bic \Pi$.  Consider the singleton

$$\{\{(\pi(\sigma(x)))``x: x \in X\}\}$$

Notice that we can recover $X$ from this singleton. Any $y \in
\{(\pi(\sigma(x)))``x: x \in X\}$ is a subset of a unique $p \in \Pi$.
$(\pi(p))^{-1}``y$ is now a member of $X$

This gives us a map from $\iota``V$ onto the set of things of size
$\leq |\Pi|$

However all this gives us is a recasting of Nathan's proof that the set of
surjective images of $\iota``V$ is itself a surjective image of $\iota``V$.

We will need the following
\begin{rem}\label{CCCC}
Any surjective image of a set in $C_\alpha$ is in $C_\alpha$
\end{rem}


\Proof.  Clearly a surjective image of a countable set is countable.
If $X \in C_\alpha$ then $X = \bigcup_{i\in \smallNn} X_i$ where all
the $X_i$ are in $C_\beta$ with $\beta < \alpha$.  For any function
$f$ evidently $f``X = \bigcup_{i\in \smallNn} f``X_i$, and the
$f``X_i$ are all in $C_\beta$ with $\beta < \alpha$ by induction
hypothesis. \endproof



All this is OK so far.  This is where it starts to go wrong.


\begin{rem}
$|C_\infty| \leq^* T|V|$ and $C_\infty \not= V$.
\end{rem}


We observed in remark \ref{firstC} that $|C_0| \leq^* T|V|$. We now 
claim the following chain of inequalities.
$$|\iota``C_1|\ \ \leq^*\ \ |\powk{\aleph_1} C_0)|\ \ \leq^{*(1)}\ \ |\powk{\aleph_1} \iota``V)|\  \ =^{(2)}\ \  T|\powk{\aleph_1} V)| \ \ = \ \ T|C_0|\ \ \leq^*\ \ T^2|V|$$ 


(1) This is where the mistake is. One would think that this
star-inequality follows from $|C_0| \leq^* T|V|$, but we have to be
careful.  The problem is that altho' $h``x$ is, indeed, a countable
subset of $C_0$ we cannot be sure that every countable subset of $C_0$
is an $h$-image.

(2) Take the $T$ outside.

so

$$|C_1| \leq^* T|V|$$

and the analogous argument will work for any $\alpha$, so we have shown 
$$|C_\alpha| \leq^* T|V| \to |C_{\alpha+ 1}| \leq^* T|V|$$

Notice that this construction is canonical: if we start with a
surjection $\iota``V \onto C_0$ we can recursively give later
surjections in terms of it. How do we prove that there is a surjection
from $\iota``V$ to $C_\lambda$, given, for each $\alpha < \lambda$, a
surjection $\iota``V$ to $C_\alpha$?  The details deserve to be
spelled out.

Let us write `$C$' for $\{C_\alpha: \alpha < \lambda\}$.  Let $f$ be
the function that sends each singleton $\{x\}$ to the first $C_\alpha$
in $C$ s.t. $x \in C_\alpha$, or to $C_0$ if there is no such
$\alpha$.  Thus we have $f:\iota``V \onto C$.  Also, the canonical
nature of the construction-so-far of the surjections means that we
have a function $g$ such that, for each $c \in C$, $g(c)$ is a
surjection $\iota``V \onto c$.

Now consider $\iota``V \times \iota``V$ and define a map

$$\tuple{\{x\}, \{y\}} \mapsto g(f(\{x\}), \{y\})$$

This sends every ordered pair of singletons to something in the union
$\bigcup C$ which is of course $C_\lambda$.  Thus we can extend the
canonical sequence of surjections at limit stages.  
 

Finally this shows that $\iota``V$ can be mapped onto
$C_\infty$. \endproof




However I can now reveal that that actually wasn't Boffa's original
problem.  The original version was with ``countable'' replaced by
``wellordered''.  It is not clear that the analogous proof will go
through, because it is not clear that the set of wellordered sets is a
surjective image of the set of all singletons.  However it will go
through if we replace ``countable'' by ``is a surjective image of
$\iota``V$''.  Thus to be pedantic, say:

\begin{quote}
An $S_0$ set is a surjective image of $\iota``V$.  An
$S_{\alpha+1}$-set is a set of the form $\bigcup X$ where $X \subseteq
S_\alpha$ and $X$ is a surjective image of $\iota``V$.  Take unions at
limits, and let $S_\infty$ be the union of all the $S_\alpha$.\end{quote}

  Then $S_\infty$ is a surjective image of $\iota``V$, and therefore
$S_\infty \not= V$.


\section{leftovers}



``small'' = ``cannot be mapped onto $V$''

Is the set of small sets small?  If so, every set of small sets is
small, so the power set of a small set is small, whence 
$$(\forall \alpha)(2^{T\alpha} \geq_* |V| \to \alpha \geq_* |V|)$$ 
which is the same as 
$$(\forall \alpha)(2^{T\alpha} \geq_*|V| \to T\alpha \geq_* T|V|)$$ 
which is the same as  
$$(\forall \alpha)(2^\alpha \geq_* |V| \to \alpha \geq_* T|V|)$$ 
which is surely false \ldots After all it implies  
$$(\forall \alpha)(2^\alpha = |V| \to \alpha \geq_* T|V|).$$

This looks rather unlikely\ldots so presumably the set of small sets is not small.
\label{small}

Try again.  Suppose the set of small sets is small.  Then the power
set of a small set is small, so

$$(\forall A)(|\pow A)| \geq^* |V| \to |A| \geq^* |V|)$$
Now $|A| \geq^* |V|$ implies $|\pow V)| \leq |\pow A)|$ so we infer
$$(\forall A)(|\pow A)| \geq^* |V| \to |\pow A)| = |V|)$$
Now $|\pow A)| = |V|)$ certainly implies $|\pow A)| \geq^* |V|$ so we have proved
$$(\forall A)(|\pow A)| = |V| \to |A| \geq^* |V|)$$

Now consider the case where $A$ is $\pow B)$. This gives

$$(\forall B)(|\pow \pow B))| = |V| \to |\pow B)| \geq^* |V|)$$

and the RHS implies $ |\pow B)| = |V|)$ and thence $|B| = |V|$ so we get 

$$(\forall B)(|\pow \pow B))| = |V| \to |B| = |V|)$$

We need $\alpha \leq |\alpha|$.  Then we would
have been able to show that $|S|$ was the supremum of all the small
cardinals.

\marginpar{All that looks rather sus.  Read below here \ldots}


We can prove by induction on the ordinals that

\begin{equation}\label{one}
(\forall \kappa)((\exists x \in C_\alpha)(|x| = \kappa) \bic (\exists x \in C_{T\alpha})(|x| = T\kappa))\end{equation}

\begin{equation}(\forall x)(\forall \alpha)(x \in C_\alpha \bic \iota``x \in C_{T\alpha})\end{equation}

We'd better have a proof of this.  

We note first that all the $C_\alpha$ are closed under equinumerosity.
This we prove by induction on $\alpha$.  If $x \in C_\alpha$ and $|y|
= |x|$ then there is a bijection $\pi$ between $x$ and $y$.  If $x =
\bigcup_{i \in \smallNn}x_i$---so that $\{x_i: i \in \Nn\}$ is
a certificate that $x \in C_\alpha$---then $y = \bigcup_{i \in
\smallNn}\pi``x_i$ so so that $\{\pi``x_i: i \in \Nn\}$ is a
certificate that $y \in C_\alpha$.

Now we can prove \ref{one} by induction on $\alpha$.
Assume \ref{one} ordinals below $\alpha$.


Suppose $x \in C_\alpha$.  Then there is a certificate $\{x_i: i \in
\Nn\}$ with \begin{enumerate}
\item $x_i \in C_{\alpha_i}$ for each $i$; 
\item $\alpha = sup\{\alpha_i: i \in \Nn\}$.
\end{enumerate}
Then---by induction hypothesis---for each $\alpha_i$ we have
$\iota``x_i \in C_{\alpha_{Ti}}$.
(Here we need the fact that all the $C_\alpha$ are closed under
equinumerosity.)  So $\iota``x \in C_{T\alpha}$.

So we have proved that if $C_\alpha$ contains a set of size $|x|$ then
$C_{T\alpha}$ contains a set of size $|\iota``x|$---indeed by the
equinumerosity lemma it will contain $\iota``x$ itself.

For the other direction we want to show that if $C_{T\alpha}$ contains
a set of size $|x|$ then $C_\alpha$ contains a set of size
$T^{-1}|x|$.  This is where the gap is!  After all, if $cf(\Omega) =
\omega$ then some $C_{T\alpha}$ might contain a set not the size of a
set of singletons even tho' every smaller set {\sl is} the size of a
set of singletons.  It seems that what might happen is that the
$C_\infty = C_\alpha$ and $C_{T\alpha}$ is the first level to contain
sets that are not the same size as any set of singletons.

***************************************************************


Let us say an $S$ set is a surjective image of $\iota``V$.


How many sets are there that are unions of $S$-many finite sets?
We have to be careful what we mean by this:  $V = \bigcup\iota``V$ 
and so is a union of an $S$ set of singletons!  We are interested 
in those sets that are the ranges of functions $\iota^2``V \to V$.
Let us call this set $S^*$.  Then

$\iota``S^* \ontoreverse 
(\iota^2``V \to S) \subseteq 
\powk{T^2|V|}S)\ \ ??\ \
\powk{T^2|V|}\iota``V) \simeq \iota``\powk{T^2|V|}V) \ontoreverse \iota^2``V$

The problem comes with the stage flagged by the question mark.  One 
wants these two sets to be the same size but it's not clear that they are.


However some smaller cases work.  Let $FIN$ be the set of finite sets,
$C$ be the set of countable sets and $C^*$ the set of sets that are
unions of countably many finite sets.

$\iota``C^* \ontoreverse 
(\Nn \to FIN) \subseteq 
\powk{\aleph_1}FIN) \simeq 
\powk{\aleph_1}\iota``V) \simeq 
\iota``(\powk{\aleph_1}V)) \ontoreverse \iota^2``V$

so $|C^*| \leq^* T|V|$.

Notice that this is not a trivial corollary of Nathan's result. If $x$
is cantorian then it is certainly a surjective image of $\iota``V$.
It's not obvious that a union of countably many finite sets is a
surjective image of $\iota``V$ nor {\sl a foriori} cantorian, even if
AxCount holds.  Is a surjective image of a cantorian set cantorian?
Not unless Axcount.  Is a surjective image of a strongly cantorian set
strongly cantorian? Yes: think about the power sets.



\subsection{How many Dedekind-finite sets are there?}


If $x$ is Dedekind finite then, for any countable $y$, $y \setminus x$
is nonempty.  So, assuming GC, for any Dedekind-finite $x$ there will
be lots of functions $f$ such that $f$ picks from any
$\omega$-sequence a member of that sequence that is not in $x$.  We
ought to be able to use this to show that there is a surjection from
the set of singletons onto the set of Dedekind-finite sets.



Let $S$ be the set of wellorderings of length $\omega$. Evidently $|S|
= T|V|$.  Now let $f$ be a function from $S$ to $\iota``V$.  (Again,
there are $T|V|$ such functions.)

Then $f \mapsto \{(V \setminus \bigcup f``S)\}$ and we seem to have an extra $T$....

%\end{document}
\chapter{The General Hierarchy}

\hole{This chapter needs heavy editing!}

\begin{quote}
   It is an old puzzle whether or not Amb$^n$ (as i call it) is 
equiconsistent with Amb.  I showed that Amb$^n$, for any $n$ is enough
to refute AC, and Marcel gave a much simpler proof.  How about trying to
prove that $Amb^n \vdash Amb$ for any $n$.

   Here is a way that might work.  Think about ${\cal P}$-extensions.
These are the extensions Kaye and I wrote about in our joint JSL paper of
1990.  ${\cal B}$ is a ${\cal P}$-extension of ${\cal A}$ iff ${\cal B}$ is
an end-extension of ${\cal A}$ in which old sets do not acquire new {\bf
subsets} (not only no new members).   

   Take the case $n = 2$ for ease of illustration. If we had a model of
$Amb^2$ then we would have a model of TST that was glissant$^2$. (I hope it
is obvious what that means!).  Remind yourself of two elementary facts, and
one piece of notation. ${\cal M}_{-n}$ is the model obtained from ${\cal
M}$ by deleting the bottom $n$ levels and relabelling everything so that
the old level $n$ is now level 0.  It is not hard to check (use $\iota$)
that ${\cal M}_{-1}$ is (isomorphic to) a ${\cal P}$-extension of ${\cal
M}$ whatever ${\cal M}$ is.  Now let ${\cal M}$ be a model of TST which is
glissant$^2$.  We have

\begin{itemize}
\item  ${\cal M}$ is a ${\cal P}$-extension of ${\cal M}_{-1}$ (because
       ${\cal M}$ is isomorphic to ${\cal M}_{-2}$ (it's glissant$^2$) and
\item  ${\cal M}_{-1}$ is a  ${\cal P}$-extension of ${\cal M}$ (it always is).
\end{itemize}

So we have two structures each of which is (isomorphic to a) ${\cal
P}$-extension of the other.  What can we infer from this?  Must they be
elementarily equivalent, or what?  

   Of course there are similar examples in the one-sorted case.
\end{quote}

$\M \subseteq_e$ $\N$ says that $\N$ is a ${\cal P}$-extension of $\M$.

\section{end-extension}
\label{end-extension}

$\subseteq_e$ is obviously
transitive. Boffa\index{Boffa} has pointed out to me that $CH$ is $\DeP_1$ and
independent of $TST$ so $\subseteq_e$ lacks upper bounds (and {\sl a fortiori}
sups).  Also $\omega$-chains do not have sups in general, for the sup of
$\langle \chi^n`M: n < \omega \rangle$ would have to be a model of $Amb$.  

There is a related relation probably best written $\prec_e$ We might
wonder whether $\prec_e$ is antisymmetrical.  (We should at least be
able to prove that if $\M \prec_e \N \prec_e \M$ then $\M \equiv \N$
but even this i cannot see at the moment). This question of
antisymmetry is intimately related to whether we allow the injection
implicit in ``$\N \prec_e \M$" to be setlike or insist on it being a
set.  The problem is that the relation ``there is a setlike injection
$x \to y$" does not seem to be antisymmetrical, and it appears to go
wrong in two quite separate ways.  First, there seems to be no
guarantee that $x$ and $y$ can be split in the way required by the
proof of s-b, and second, even if we can split $x$ and $y$
appropriately the bijection doesn't seem compelled to be setlike: it
will lift once (to give a model of $TST_3$) but not twice! This keeps
cropping up.  Perhaps it is worth isolating this problem: it might be
the right context for developing $NF$-with-classes.  $ML$ is usually
overlooked, as is $GB$ and for the same reasons. However, there might
be a case for examining the halo of classes that lives around a {\sl
pair} of models, for it might help us understand $\prec_e$.

It is not clear whether or not $\prec_e$ or $\subseteq_e$ is
wellfounded.  It is probably worth noting that it doesn't really
measure size, for very small models of $TST$ do not go into very big
ones: no non-natural-model is $\subseteq_e$ a natural one! Another
topic for later development will be how $\subseteq_e$ behaves with
ultraproducts, permutation models etc. We do know that every model of
\nf\ has a permutation model which is a proper ${\cal P}$-extension of
it. We know also that end-extensions are never elementary so we cannot
ever have $M \subseteq_e M^{\kappa}/U$. To the extent that
$\subseteq_e$ is a bit like $\unlhd$ (normal subgroup) we should be
thinking about a quotient $N/M$ when $M \subseteq_e N$. Since the
images of the embeddings are ideals this is a possibility\footnote{Is
the inclusion embedding of a normal subgroup ever elementary? Probably
not in interesting cases: Any abelian group is a normal subgroup of
any ultrapower of itself \ldots}. The following is an obvious thing to
try. $T^{N/M}_0 = N - M$.  $T^{N/M}_1 = (T^N_1)/(T^M_1)$ as b.a.s,
thereafter take power sets in the sense of $N$. That way the quotient
is a substructure of $N$, unlike groups, but that is only beco's of
the greater expressive power of set theory.  I have the impression
that $N$ is the sup of $N/M$ and $M$, tho' i do not see how to prove
it.

Another remark of Boffa\index{Boffa}'s is that two models could be elementarily
equivalent and still fail to have a common end-extension because one contains
nonstandard integers and the other doesn't.  But if two models have a common
end-extension they must satisfy the same $\DeP_1$ sentences! No converse! Thus 
$\prec_e$ seems to have less to do with logic than one might have expected.

\subsection{Natural models}
\label{Natural models}

The study of $\prec_e$ is fairly easy in this case. If we take
$\subseteq_e$ in the strong sense it is simply the study of $\leq$ on
cardinals.  In $ZF$ the strong and the weak notions coincide but in \nf\ they
do not, and life can get quite difficult.  We have a $+$ well-defined on 
natural models, and it is {\sl not} defined on arbitrary models.  This is kin
to the failure of s-b for setlike embeddings.  

Question  If $\M$ and $\N$ are both ambiguous natural models, what about $\M + \N$?

Question: given a consistent extension T\# of TST, is there a model of $ZF$
containing a natural model of T\#?  Presumably the answer to this is no, because we
can make $T$ assert something pathological which is irrefutable in Zermelo but not
in \zf. One thinks of borel determinacy but that uses choice \ldots

An answer to this would help us know when we can safely restrict our
attention to natural models.

Annoying (but possibly deep?) fact: There are no natural models of $\TZT$.

(Nice models of \TZT are scarce.  Not only are there no natural models, no-one has
ever found an $\omega$-model or a term model.)
\subsection{Other models}
\label{Other models}

Let us consider the old question of whether or not $Amb^2$ implies $Amb$. 
Assume $Amb^2$. Then we have a model $M$ with a $tsau^2$ $\sigma$.  If we try
to do s-b using the obvious maps ($\iota$ and $\iota \circ \sigma^{-1})$ then
we need to know that the clever split of the bottom type into two bits
actually splits it into two sets of the model. A little calculation shows that
wht we need is that there should be $x$ s.t.  $\sigma`x = -\iota `` - \iota ``
x$.  This can certainly be arranged with the help of some model theory and no
extra axioms, but all it gives us as an isomorphism $h$ between the two bottom
types.  As usual, it will lift once (beco's $x$ is a set) but not, 
apparently, twice.  Actually for what it's worth we can get this far with 
$Amb^n$ for an old $n$.

\begin{rem}
If $\prec_e$ is antisymmetric on models of TST
then $Amb^n \vdash Amb$ for any concrete $n$.
\end{rem}

\Proof: 
 
If $\M \models TST$ and has a tsau$^n$ then $\chi^n`\M \prec_e \M$.
But $\chi`\M \prec_e \chi^n`\M$ holds for all $\M$ anyway, so we infer
$\chi`\M \prec_e \M$. But we always have $\M \prec_e \chi`\M$, so, by
antisymmetry, $\chi`\M$ and $\M$ are isomorphic.

\endproof

\subsection{The \nf\ case}
\label{The \nf\ case}

We would naturally want to consider the analogous relation on models of $NF$. 
Is it antisymmetric?  If we have two models $\M$ and $\N$ of \nf\ s.t. $\M
\prec_e \N \prec_e \M$, are they 
\begin{enumerate}
\item isomorphic? or at least 
\item stratimorphic? or, lowering our sights,
\item elementarily equivalent?  Or at worst
\item elementarily equivalent w.r.t. stratified sentences?
\end{enumerate} 

We do not seem to be able to prove any of these at the moment. Discussion must
split into 4 cases depending on whether or not the models are natural, and
whether or not we are doing this in \nf. It also depends on whether or not the
injection mentioned in $\prec_e$ has to be a set! In the next paragraph it
is allowed that it mightn't be.
\begin{itemize}
\item [ ] Natural models discussed in \nf.
If the injections are {\sl sets} then they must be isomorphic.  If they are
merely {\sl setlike} then we don't know a great deal.  Any $\pow x) \subseteq x$
will give rise to such a pair of {\sl natural} models $M$ and $N$.  If $x$ and
$\pow x)$ are distinct sizes then of course $M \not= N$.  If $\neg$\AxC there
can be finite $\pow x) \subseteq x$) so they would $\models AC$.  

\item [ ] Natural models in $ZF$

   They must be isomorphic  

\item [ ] Non-natural models.

In $ZF$ without doing any extra work we can certainly show that $N$ and $M$
satisfy the same $\SiP_1$ sentences. If we try to argue that they must
satisfy the same $\PiP_2$ sentences we would want to know that every
witness to $\exists \vec x \ \phi(\vec x, \vec y)$ can be found inside
$\m{\vec y}$ if $\phi$ is $\DeP_0$ but this just isn't true, as Adrian's
counterexample shows: $n$-sized set all of whose members are infinite and
all of different sizes.  We might be able to construct a counter-example to
(1) and (3) consisting of $M$ and $N$, each embedded in the other as the
unique maximal $x = \pow x) \not= V$ and where $M \models \exists y =
\{y\} \wedge \forall x = \pow x) \not= V \ y \not\in x$ but $N$ doesn't.
(4) looks plausible.  Unfortunately the problem of constructing a
stratimorphism in this case seems to be the usual problem of s-b with
setlike maps.  
\end{itemize}

Beware of the following trap.  Suppose $\phi$ is $\DeP_2$. Consider the obvious
direct limit.  If $\phi$ is true in $M$, then it is true in the direct limit.
If $\neg \phi$ is true in $N$ then it is also true in the direct limit. 
Therefore $M$ and $N$ agree on $\DeP_2$ sentences.  Now they are both models of
$\exists V$ (in which case everything is $\DeP_2$). But this is not much help. 
Let $\psi$ be an arbitrary expression true in $M$ and false in $N$.  Then
$$M \models \forall x \exists y \ y \not\in x \wedge \psi^x$$
$$N \models \forall x \exists y \ y \not\in x \wedge \neg\psi^x$$
So the direct limit satisfies both. This doesn't give us a contradiction unless
the direct limit doesn't contain a universal set, which it obviously doesn't. 

\section{Normal Forms}
   The idea is that everything is equivalent to a formula in {\sl normal
form} where all unrestricted quantifiers are out at the front and all
restricted quantifiers are in the matrix. We need to be able to push
restricted universal quantifiers inside unrestricted existentials (and
dually). This introduces a complication.

 Quantifier-pushing lemma:

if $$(\forall x \in y)(\exists z)\Phi(x,z,y)$$ then $$(\exists w)(\forall x
\in y)(\exists z \in w)\Phi(x,z,y)$$ The usual trick for this is the axiom
scheme of collection: $$(\forall x \in A)(\exists y)\Phi(x,y,A) \ \to \
(\exists B)(\forall x \in A)(\exists y \in B)\Phi(x,y,A)$$ (which is
equivalent to replacement)\footnote{Evidently a combination of
quantifier-squashing and quantifier-pushing will eventually get any formula
into normal form.  The point is that truth-definitions are available for
things in normal form.}.  So we need collection to do quantifier-pushing,
and this is actually ok in type theory. It is even ok in NF {\sl as long as
we are restricting attention to stratified formul{\ae}}, since stratified
collection is provable in NF---just take $B$ to be $\{y: \exists x \in A \
\Phi(x,y,A)\}$.  We do not have unstratified collection in NF for obvious
reasons, so we cannot push restricted universal quantifiers inside
unrestricted existentials (and dually) if the matrix is unstratified.  This
will mean that $\forall x \in y$ outside something $\SiP_n$ may turn out to
be $\PiP_{n+1}$ instead of $\SiP_n$ if the matrix is unstratified.  So for
the moment we shall restrict our attention to stratified formul{\ae}.  If
we do restrict our attention to stratified formul{\ae} (and we are doing
type theory for the moment) we can drop the ``$\wedge$, $\vee$ and limited
quantifiers" closure condition (that exists in some formulations) on the
levels of the $G$ hierarchy.

So, back to Z and stratified formul{\ae}.  Coret\index{Coret}'s theorem is
that we have stratified replacement in Z so can we do all this for
stratified formul{\ae} in Z?  Most of it goes over without any trouble.  We
can even squash a block of quantifiers of unlike type: if we have a block
$\exists \vec x$ we can squash $\vec x$ into one variable, by saying
$\exists$ an n-tuple (or $\forall$ n-tuple) which is $\langle \ldots
\iota^{n_i}`x_i \ldots \rangle$ and this is $\Delta_0$. The way in which
this is done is not uniform in the differences in the type indices, but
this is neither surprising nor unfortunate, since even so we are lumping
together infinitely many formul{\ae} into one form.

  In Zermelo we have stratified replacement (but not stratified collection)
so consider $$(\forall x \in y)(\exists z)\Phi(x,z,y)$$ where $\Phi(x,z,y)$
is stratified. We want an $f$ so that $f`x$ is some nonempty subset of $\{
z: \Phi(x,y,z) \}$. Then we let $w = \bigcup f``y$. (We can't just send $x$
to the set of things of minimal rank, it isn't stratified).  Now one might
think we should be able to show that $\{ z: \Phi(x,y,z) \}$ must meet
${\cal P}^n`\bigcup^k y$, but Adrian has a nice counterexample: let
$H(x,y)$ say that $y$ is a set of infinite sets all of different sizes and
$ ^=y\ = x$.  Then $$(\forall x < \omega)(\exists z)H(x,z)$$ but there is
nothing that collects all the $y$, i.e., not $$(\exists w)(\forall x <
\omega)(\exists z \in w)H(x,z)$$ This counterexample clearly shows that we
cannot bound the $z$ inside ${\cal P}^n`\bigcup^k y$, which is what one
might expect.  It may be sheerest coincidence but in NF we have almost
exactly the same problem: there doesn't seem to be any way of proving that
there are infinitely many distinct infinite cardinals.

All this quantifier-pushing and squashing is pretty easy in NF and such systems if $\Phi(x,z,y)$ is stratified. 

And what about quantifier-pushing and squashing for arithmetic?
$$(\forall x \leq y)(\exists z)(\Phi(x,z,y))$$
z has to be an $y$-tuple sending things $x \leq y$ to things $z$ such that $(\Phi(x,z,y))$  Can we do a uniform
definition of $y$-tuples?
                                                         
It is suggestive that the one $\SiP_1$ sentence ($NCI$ infinite) is used to show that

\begin{enumerate}
\item stratified replacement does not prove stratified collection
\item $H_{\beth_{\omega}} \not\prec_{\SiP_1} V$ even tho' for limit $\lambda$ 
      $H_{\beth_{\lambda}} \prec_{\Sigma_1^{L\acute{e}vy}} V$.
\item NF does not prove all consistent$^{NF}$ stratified $\SiP_1$ sentences
\end{enumerate}

  It is worth noting that $V_{\omega + \omega} \prec_{strat}
H_{\beth_{\omega}} \prec_{\Sigma_1^{L\acute{e}vy}} V$ so that any
stratified $\Sigma_1^{L\acute{e}vy}$ sentence true in $V$ is true in
$V_{\omega + \omega}$, that is, $$V_{\omega + \omega}
\prec_{str(\Sigma_1^{L\acute{e}vy})} V$$ (``str" short for ``stratified")
This is actually best possible beco's the assertion that there is an
infinite set of infinite sets no two the same size is
$\Delta_2^{L\acute{e}vy}$ and false in $V_{\omega + \omega}$ tho' true in
$V$. Can we have $$V_{\omega + \omega} \prec_{str(\SiP_1)} V\ ?$$ Since
``there is an infinite set of infinite sets no two the same size" is
$str(\SiP_1)$ this would imply that GCH fails below $\beth_{\omega}$.  It
would also mean no measurables, since ``$\exists$ measurable" is also
$str(\SiP_1)$.

How about
\begin{conjecture}\label{conj:generalhierarchy}

.

\begin{enumerate}
\item NFC proves every consistent$^{NFC}$ stratified $\SiP_1$ sentence.
\item NFC proves every consistent$^{NFC}$ $\SiP_1$ sentence.
\item Every consistent$^{NFC}$ $\SiP_1$ sentence is consistent with NFC.
\item Every consistent$^{NFC}$ stratified $\SiP_1$ sentence is consistent with NFC.
\item NF proves every consistent$^{NF}$ stratified $\SiP_1$ sentence.
\item NF proves every consistent$^{NF}$ $\SiP_1$ sentence.
\item Every consistent$^{NF}$ $\SiP_1$ sentence is consistent with NF.
\item Every consistent$^{NF}$ stratified $\SiP_1$ sentence is consistent with NF.
\end{enumerate}

\end{conjecture}

$3 \to 7$, $4 \to 8$.  We can't prove these by skolemheim.  1,2,5 and 6 are
presumably false beco's of $CH$. This is a (probably) consistent$^{NF}$
stratified $\SiP_1$ sentence that appears not to be a theorem of NF.  It
should be possible to find examples that are more obviously not theorems of
NF, though this and ``there is a nonprincipal ultrafilter" are the best i
can do at the moment. 6 is obviously false, because AxCount is a
consistent$^{NF}$ $\SiP_1$ sentence.  7 simply says NFC is consistent. 8
can be true only if it is consistent w.r.t. NF that NCI should be infinite
and there is a nonprincipal ultrafilter somewhere.

Existence of wellfounded extensional relations on $V$ generalises upward in
models of \TZT, and is $\SiP_1$.

\begin{itemize} 

\item Is $Z$ + stratified collection equiconsistent with $ZF$?  

\item $ZF$ is not an extension of $Z$ conservative for $\Sigma_1$-sentences: consider ``There 
      is a model of Z". For stratified $\Sigma_1$-sentences? 

\item Does every stratified $\Sigma_1$ consequence of $Z$ follow from Ext, $\bigcup x$, $P(x)$, 
      AxInf, $\{ x, y \}$ and {\sl stratified} replacement. 

\item What substructures of $V$ are there elementary for $\Pi_2^{L\acute{e}vy}$ sentences? 

\end{itemize}


                              
\subsection{remaining junk}

$Z$ really is stronger than $TST \ +$ AxInf so we cannot assume that the
model of $TST \ +\ Inf$ is a model of $Z$, and, even if it was, we know
that not every model of $Z$ is an inital segment of a model of $ZF$
(Martin-Friedman theorem) in the sense of being $V_{\omega + \omega}$ of
the new model.  The new model might be an end-extension of the old but that
isn't enough to ensure that no new $\Sigma_1^{L\acute{e}vy}$ sentences
become true.

Develop arithmetic in $Z$ in a stratified way (use Russell-Whitehead
cardinals at some level).  We then find that we can devise lots of nasty
{\sl stratified} $\Sigma_1^{Levy}$ sentences, such as $Con(TST)$.
This means that there is no hope of showing (in $ZF$) that any model of
$TST + Inf$ must satisfy all stratified $\Sigma_1^{Levy}$ sentences.
This also shows that $ZF$ is not an extension of $Z$ conservative for
stratified $\Sigma_1^{L\acute{e}vy}$ sentences (even).  So this trick
cannot work.
                     
 Let the scheme $E_n$ say there are at least $n$ distinct objects.  If
$\phi$ is true in all sufficiently large finite models then it follows from
some $E_n$ + {\sl whatever remaining first-order stuff all finite models
have in common}, like the negation of the axiom of infinity etc., so it
does {\sl not} automatically follow that $\phi$ is true in all infinite
models of $T$.



Let us say a map $\sigma$ between the bottom types of two models $M$ and
$N$ of $TST$ is {\sl setlike} if for all $n$, $j^n`\sigma|(T_n^M)$ is onto
$T_n^N$.  We can have a similar notion of setlike permutations of a model
of a set theory with a universal set.  There are setlike maps from $V$ onto
proper subsets of $V$ that are not sets, e.g.  $\iota$. I don't know any
setlike permutations of $V$ that are not sets.  There are setlike
permutations of $\Nn$, $NC$, $NO$ etc. that aren't sets but they do not
seem to extend to setlike permutations of $V$\footnote{Let's try. After
all, we have

$$V \stackrel{\iota}{\longrightarrow} V$$
$$V \stackrel{\iota}{\longleftarrow} V$$

So the S-B trick invites us to find $X$ such that $V\setminus X = \iota``-\iota``X$,
and look at the permutation $\iota|X \ \cup \ \iota^{-1}|(V\setminus X)$. As usual,
it seems to lift one type but not two. Now even this much is almost
certainly not possible in a term model (exercise: prove that no set
abstract $t$ satisfies $t = -\iota``-\iota``t$) so perhaps in term models
every setlike permutation is a set \ldots


Conjecture: if we have permutation $\sigma$ of $V$ so that $j`\sigma, \
j^2`\sigma$ and $j^3`\sigma$ are all permutations of $V$, then $\sigma$ is
setlike. Outer automorphims of $V$ are setlike. I do not know how to prove
the existence of setlike permutations of $V$ that are not sets, so consider
the $ML$ axiom: every setlike permutation of $V$ is a set.  Is there are
nice way of restricting this to a first-order version?}.


  Andr\'e says that you can prove omitting types
if you define $\phi(\vec x \vec y)$ realizes $\Sigma$ [ a set of fml{\ae} with 
only `$x$' free] if there is some $\vec a$ s.t.
$$\phi(\vec x, \vec a) \to \bigwedge_{\sigma\in\Sigma} \sigma(\vec x, \vec a)$$

       
\subsection{messages from james about reflection principles}       

 dear t, thanks for the message. here is all i can think of off the top of my
head about reflection principles: 
\begin{itemize} 
\item  (levy) if $|V_{\theta}|=\theta$, and $\phi$ is a $\Sigma_1$ statement
      with parameters from $V_{\theta}$, then ($V \models \phi) \to (V_{\theta}
      \models \phi$). In the jargon of model theory the inclusion map is a 
      1-embedding.  
\item (solovay?) if $\theta$ is supercompact, then the same holds for $\Sigma_2 \phi$.  
\item (reinhardt?) if $\theta$ is extendible, ditto for $\Sigma_3$.  Notice that if $\Sigma_n$ 
formul{\ae} reflect down then

a) $\Pi_{n+1}$ formul{\ae} reflect down

b) $\Pi_n$ formul{\ae} go up ( i think the model theorists say they
are preserved) see kanamori + magidor's expository paper on large
cardinals for proofs and refs relating to 1,C,3.  Another approach
could be to reason like this \ldots if j embeds V into M (not
necessarily contained in V) then $j``V$ is an elementary substructure
of $j'V=M$. 2) has amusing consequences e.g. the first huge $<$ the
first supercompact if both cardinals exist ('cos although huge is
higher in consistency strength, the defn. of huge is $\Sigma_2$ so
``there exists huge'' reflects). Not sure if this is germane (but it's
good stuff anyway). if A is a class of V, $\kappa$ is $(\Pi_1)$-strong
in A iff for all $\Pi_1 \ \phi$ (in a language with a 1-place
predicate A(x) ) $\tuple{V,A} \models \phi \to \exists j, crit`j
=\kappa$, into inner model $M$ s.t.  $\tuple{M,j(A)} \models \phi$
(n.b. $\phi$ could have parameters from V, i'm asserting that these
get into M) it's easy to check that $\kappa$ is $\Pi_1$ strong in
$\Lambda \ \bic \ \forall \lambda \exists j: V \to $inner model M
s.t. $V_{\lambda} \subseteq M$.  The point of all this is that the
$\Pi_n$ strong hierarchy fit nicely into the large cardinals (between
hypermeasures and woodins) and have a goodish inner model theory ( at
least $\Pi_1$ does..

\end{itemize}
                                                                               
another message

Let $\phi$ be $\Sigma_1$, with free variables among $x_1, \ldots x_n$.

a) $\kappa$ is regular.

 let $a_1, \ldots a_n \in H_{\kappa}$, and let $\phi(a_1, \ldots a_n)$ hold in
V. By the reflection principle it holds in some $V_\lambda$ where $\lambda$
is chosen $>> \kappa$. By Skolemheim there is $M \prec V_\lambda$ such that 
$TC(a_1 \cup \ldots \cup a_n) \subseteq M$ and $|M| < \kappa$.  Take the 
Mostowski collapse of M to N: $N\subseteq H_{\kappa}$, the collapses fix 
$a_1, \ldots a_n,$ so N thinks $\phi(a_1 \ldots a_n)$. but now $M$ does too by
upwards absoluteness. does this sound plausible?

In fact why doesn't this work for singular $\kappa$ as well? Answer beco's
for singular $\kappa$ it's not enough to be hereditarily card less than $\kappa$
to be in $H_\kappa=_{\rm def} \{ x: |TC(x)| < \kappa \}$.

b) $\kappa$ singular. $\phi$ and a's as before. As $\kappa$ is limit, all
the a's are in $H_\beta$ for $\beta < \kappa$, $\beta$ regular! $H_\beta$ thinks
$\phi$ holds so by upwards absoluteness $H_\kappa$ does.
\begin{quote}
                                            luv         j.
\end{quote}

(I asked him: is it true that every $\Pi^{Levy}_2$ theorem of $ZF$ is true in
$V_{\lambda}$ for $\lambda$ limit)

   let $\phi(x, y)$ say something like $y = x \cup \omega$.
   then your statement is false in $V_\omega$. less trivial
   examples can be concocted. 

This is the usual thing about $\Sigma_0$ functions vs rud functions; the
former can raise rank by an infinite amount, and the latter (by an easy
induction) cannot [in the sense that, if F is rud, there is $n$ finite such
that $(\forall x)(rank(F(x)) \leq rank(x) +n)$.

 there is a theorem of jensen saying
    that if $\phi(\vec x)$ is $\Sigma_0$ then for some rud
     F $\phi(\vec x) \bic F(\vec x) =0$.

\chapter{Miscellaneous junk}

\section{Another game}
Consider also the game $H_x$ played as follows.  If $x$ is empty, 
\two\ loses.  Otherwise \one\  picks $x' \in x$ and they play $H_{x'}$,
swapping r\^oles.  Thus \one\ wins iff the game ever comes to
an end.
 
Let \verb#A# be the collection of sets Won by \one, and \verb#B# the
collection of sets Won by \two.  If even one member of $x$ is a subset
of \verb#A# then for his first move \one\ can pick that element, and
then, whatever member $x''$ of it \two\ chooses, the result is a game
for which \one\ has a winning strategy.  Thus $\b(\pow \verb#A#)
\subseteq$ \verb#A#.  Similarly, if every member of $x$ contains a
member of \verb#B# then whatever \one\ does on his first move, \two\ 
can put him into a game $H_{x'}$ with $x' \in^2 x$ for which she has a
winning strategy, so $\pow \b(\verb#B#)) \subseteq \verb#B#$. Indeed,
that is the only way \two\ can win, by living on to fight another day,
so in fact we have $\pow \b(\verb#B#)) = \verb#B#$.  But wait! We don't
mean ``power set of'' $\b(\verb#B#) = \verb#B#$! we mean ``set of nonempty
subsets of ''$ \b(\verb#B#) = \verb#B#$!  Without this we would have concluded
that in this games \two\ can Win any set $x$ for which she could have won
$G_x$.  This is obviously wrong, because \two\  Wins $G_{\{\Lambda\}}$ but is
doomed to lose $H_{\{\Lambda\}}$ whatever \one\ does: \two\ cannot win $H_x$ if
$x$ is wellfounded.

\section{Non-principal ultrafilters}

See the discussion after conjecture ~\ref{conj:generalhierarchy}.  The
assertion that there is a nonprincipal ultrafilter on $V$ is $\SiP_1$
(that is to say, simple).  

Are ultrafilters extensional?  Are there any symmetric non-principal
ultrafilters? Any $\U$ on $\{x : x$ is $(n-2)$-symmetric$\}$ is
$n-$symmetric and extends to a ${\cal U}$ on $V$.


\section{A pretty picture}

\begin{tabular}{|l| l| l| l|}
\hline
\ \multicolumn{1}{c}{Recursive models} & \multicolumn{2}{c}{Decidability}& \multicolumn{1}{c}{Axiomatisability} \\
\  & Is \nf\  $\Gamma$-complete & Is $\Gamma NF$ recursive? & $NF=\Gamma NF$?\\ \hline
\  $NFO$ yes         & $\exists_1$ yes &                   & $\exists_2$ No \\ 
\  $NF\forall_1$ yes &                 &                   & $\PiP_2$ No \\\hline
\  $NF\forall^1$ yes? & $str(\forall_2)$ yes? &             & \\ 
\  & $str(\forall_3)$ No? &             &  \\ \hline
\  & $\forall_2$ No  &                  & $\SiP_2$ yes\\ 
\                     & $str(\exists^+_3)$ No & $str(\exists^+_3)$ No & $\forall^+_4$ yes \\
\                   &                  &                   & $str(\forall_4 \cup \exists^+_3)$ yes\\ \hline

\end{tabular}



\section{Mainly concerning $J_0$}

\subsection{Some remarks on permutations and bijections}
\label{Some remarks on permutations and bijections}

{\sl This section will need to be rewritten to take account of
 Henrard's trick:  }

In this section we will prove a lemma telling us under what
circumstances two sets are 1-equivalent, and show that given a modest
amount of $AC$, we can characterise equinumerosity without using any
ordered-pair function. At present this is a curiosity and, as such,
could be skipped.  It may turn out to be useful (see the remarks that
close this section).  Consider the following four
relations:\begin{enumerate}


\item $x \sim_1 y \bic_{\footnotesize \rm df} (\exists \pi \in J_0)  \pi``x = y$.

\item $x \approx y \bic_{\footnotesize \rm df}$ there is a partition $P$ of $x \Delta y$
      \label{approx} into pairs such that each pair in $P$ meets both $x$ and $y$.

\item $|x| = |y|$

\item $x$ and $y$ are {\sl $J_0$-equidecomposable (with $n$ pieces)}\label{equidecomposable}
  (written $x \sim_{J_0} y$ since we usually ignore the $n$ as long as it is finite) if\begin{enumerate}

      \item  $y$ can be partitioned into $y_1....y_n$, and $n$ elements 
      $g_1...g_n$ of $J_0$ can be found such that $x$ is the union of 
      the $g_i``y_i$, and 
      
      \item $y$ can be built similarly from a partition of $x$.
      \end{enumerate}

\end{enumerate}

In particular, 1-equivalence is $J_0$-equidecomposability with one piece. We
shall see that the other three can be expressed in terms of  $\approx$, which
does not make any mention of ordered pairs.

Let $GC$ be ({\sl group choice}: as in Forster [1987a]) be the
axiom saying that sets of finite-or-countable sets have selection
functions.\label{GCdef}
\begin{rem}{($GC$)}

      $\forall x y (x \sim_1 y \bic \exists z (x \approx z \approx y))$.
\end{rem}
{\sl Proof.}

`$x \sim_1 y$' makes the assertion that there is a permutation $\pi$
of $V$ so that $\pi``x = y$. Now $GC\index{GC}$ implies that every
permutation is a product of two involutions, as
follows. Given a permutation $\tau$ we construct two
involutions $\sigma$ and $\pi$ such that $\tau =
\sigma \cdot \tau$ cyclewise.  We think of any infinite $\tau$-cycle
as a copy of $\Z$, by choosing an element $w$ ``to be" $0$ (using
$GC$). The restriction of $\pi$ to this cycle is $\lambda x.(-x)$ (to
be explicit: send $x$ to the unique $z$ such that $\exists n \in \Z
(\tau^n`z = w)$ and $\tau^n`w = x$), and the restriction of $\sigma$
is $\lambda x.(1-x)$.  (The notation `$\tau^n$' is legitimate because
it can be defined uniformly for all $n \in \Nn$\ by recursion on the
integers since functional composition is homogeneous.) For an
$n$-cycle we do the same mod $n$. So $\exists z$ such that $\pi \sigma
\pi^2 =\sigma^2 = 1 \ \wedge \ \pi``x = z \ \wedge \ \sigma``z =
y$. But obviously $u \approx v$ iff $\exists \pi \in J_0$ such that
$\pi``u = v \ \wedge \pi^2 = 1$. The remark follows immediately. \blob
\index{$J_n$}

Next we show that we can express equi\-numerosity in terms of
1-equivalence.  It will turn out that $|x| =|y| \bic$ $x$ and $y$ are
$J_0$-equidecomposable with at most two pieces. That $x\sim_{J_0} y$
implies $|x| =|y|$ is obvious. The converse is easy to prove for
equinumerous $x$, $y$ whose complements are also equinumerous, but
the result is of some interest in its own right as it enables us to
make use of Lemma \ref{lem:``The permutation lemma for set abstracts"}.
Indeed, if $|x| = |y|$ and $|V \setminus x| = |V \setminus y|$, then
$x$ and $y$ are $J_0$-equidecomposable with one piece, that is to say,
they are 1-equivalent.

\begin{prop}\label{prop:``existence of permutations"}
: If $|X| = |Y|$, and $|V \setminus X| = |V \setminus Y|$, then there is a 
permutation of $V$ mapping $X$ onto $Y$.
\end{prop}
\Proof 
Simply take the union of the two bijections considered as sets of ordered
pairs. They are disjoint, total, and onto. \endproof

Richard Kaye pointed out this obvious fact to me.  It's probably worth
noticing that there is a nice generalisation.  For each $n$ we can
show that for all $n$-tuples $\vec a$ and $\vec b$ there is a permutation 
$\pi$ of $V$ s.t., for each $i \leq n$, $\pi``a_i = b_i$ iff each of
the $2^n$ boolean combinations of the $a$s is the same size as the
corresponding boolean combination of the $b$s.  (Equally obvious!)


\begin{rem} 
For all $x$ and $y$ $|x| =|y|$ iff $x$ and $y$ are
$J_0$-equidecomposable using two pieces.\index{$J_n$} 
\end{rem}
\Proof

The right-to-left implication is a consequence of the
Schr\"oder-Bernstein theorem. (All the standard proofs work in \nf\
since everything is stratified and there is no need for $AC$.) We now
do the converse. Let $x$ and $y$ be of size $m$ and have complements
of sizes $p$ and $q$ respectively. Proposition \ref{prop:``existence
of permutations"} deals with the case where $p = q$. To show $x$ and
$y$ are $J_0$-equidecomposable using two pieces in the remaining case,
we need to show that a set of size $m$ can be split into two sets of
size $m_1$ and $m_2$ such that $m_1 + p = m_1 + q$ and $m_2 + p = m_2
+ q$.  If we can do this, then

\begin{quote}
$x$ is the disjoint union of $x_1$ and $x_2$ with $|x_1| = m_1$ 
and $|x_2| = m_2$
\end{quote}
\begin{quote}              
$y$ is the disjoint union of $y_1$ and $y_2$ with $|y_1| = m_1$ 
and $|y_2| = m_2$
\end{quote} 
and $x_1$ is mapped onto $y_1$ by a permutation that we construct by
noting that $|x_1| = |y_1|$ and that $|V \setminus x_1|= |x_2| + |V
\setminus x|$ so $| V \setminus x_1| = m_2 + p$.  Also $|V \setminus
y_1| = |y_2| + |V \setminus y|$ so $|V \setminus y_1| = m_2 + q$,
which equals $m_2 + p$.  $x_2$ will be mapped onto $y_2$ similarly.
To find $m_1$ and $m_2$, we need a theorem of Tarski's, which we have
proved elsewhere in these notes: theorem ~\ref{thm:tarski}
  
If $m + p = m + q$ then there are $n$, $p_1$ and  $q_1$ such that
$p = n + p_1$, $q = n + q_1$, and $m = m + p_1 = m + q_1$.

In the case we are considering, $m$, $p$, and $q$ are as in the
hypothesis of the statement of this remark.  The desired 
$m_1$ and $m_2$ can be found as follows:
      $$m_1 = m$$
      $$m_2 = \aleph_0 \cdot (p_1 + q_1).$$
We need to verify that $m_1 + p = m_1 + q$, $m_2 + p = m_2 + q$, and
$m_1 + m_2 = m$. We know $m$ absorbs $p_1$ and $q_1$ so  $m_1 + p = m_1 + q$
since they are both equal to  $m + n$.  Also $m$ absorbs  $p_1 + q_1$,
so it absorbs $\aleph_0 \cdot (p_1 + q_1)$.  Thus $m_1 + m_2 = m$ as 
desired.  To verify  $m_2 + p = m_2 + q$  we expand and rearrange, noting that 
$(\forall x)(\aleph_0 \cdot x + x = \aleph_0 \cdot x)$. \endproof


\subsection{Digression on nonprincipal ultrafilters}


The more i think about the this the less chance i see of refuting it.
An $n$-symmetric nonprincipal ultrafilter on $V$ corresponds naturally
to a non-principal ultrafilter on the set of all $n$-equivalence classes.
Why should there not be such a  thing?

There is a family of generalisations of the last section that i
haven't proved or even formulated yet, but which we might need when
tackling the question of whether or not there might be a symmetric
nonprincipal ultrafilter on $V$.  Let us say that $x \leq_{J_n} y$ iff
$x$ can be partitioned into finitely many pieces which, once
translated by elements of $J_n$, give rise to some pieces of a
partition of $y$.  $x \sim_{J_0} y$ iff etc ect. What we have just
proved is that $x \sim_{J_1} y$ iff $|x| = |y|$.

Now suppose $\U$ is an $n$-symmetric ultrafilter, that $x \in \U$ and $x
\leq_{J_n} y$.  If we split $x$ into finitely many pieces one of them must
be in $\U$, since $\U$ is ultra.  So $x' \subseteq x \in \U$.  Then its
translation under anything in $J_n$ is also in $\U$, so any superset of that
is too, so $y \in \U$.


If we can find $x$ s.t. $x$ and $V \setminus x$ are $\sim_{J_n}$
equivalent then any $n$-symm ultrafilter containing one must contain
the other and we get a contradiction.  Now we can easily enuff find
$x$ s.t. $x$ and $V\setminus x$ are the same size.  Can we find $x$
s.t. $x$ and $V \setminus x$ are $1$-equivalent?  No, beco's one of
the two pieces must contain $\emptyset$ and there is no way of moving
$\emptyset$.  However that argument doesn't scupper the endeavour to
chop $V \setminus \{V, \emptyset\}$ into two pieces that are
$1$-equivalent.  How does this generalise?  Presumably if we delete
from $V$ all cardinal numbers, plus $V$ and $\emptyset$ plus $\{V\}$
plus $\{\emptyset\}$ then we can chop that into two pieces that are
${2}$-equivalent.  And so on.  So what are we doing?  We first
cut off the set of those things that are $n$-symmetric, and then chop
the rest into two $\sim_{J_n}$ equivalent halves.  Any symmetric
ultrafilter must contain precisely one of these three.  It can't
contain either of the last two beco's it would have to contain the
other, so it contains the first.

Conclusion: one element of an $n$-symmetric ultrafilter on $V$ is the set
of $n-1$-symmetric sets.  And in fact the converse is true.  If $\U$ is
an ultrafilter on (say) the set of $2$-symmetric sets, then it is $4$-symmetric
(say) and the set of supersets of its members is likewise $4$-symmetric and
is an ultrafilter on $V$.   Therefore no contradiction---so far at least!

One obvious thing to try is: what is the least $n$ such that there is
an $n$-symmetric nonprincipal ultrafilter on $V$?  Notice that if
${\cal U}$ is an ultrafilter on $\pow X)$, then $\bigcap``{\cal U}$ is
a filter on $X$.  Two things to check\begin{enumerate}
\item Upward closed.  Sse $\bigcup A \in \bigcup``{\cal U}$ and
  $\bigcup A \subseteq B \subseteq X$.  $A \cup \iota``(B \setminus
  \bigcup A)$ is now a member of ${\cal U}$.
\item Closed under finite intersection.  Sse $A,B \in {\cal U}$.  We
  want $\bigcup A \cap \bigcup B \in \bigcup``{\cal U}$.  We know
$\bigcup(A \cap B) \subseteq \bigcup A$ and 
$\bigcup(A \cap B) \subseteq \bigcup B$ so we have 
$\bigcup(A \cap B) \subseteq (\bigcup A \cap \bigcup B)$. So
$\bigcup A \cap \bigcup B$ is at least a superset of something in 
$\bigcup``{\cal U}$.  But $\bigcup``{\cal U}$ is closed under superset 
as above, so we are done.\end{enumerate}


Notice that if ${\cal U}$ is $n$-symmetric, then $\bigcup``{\cal U}$
is $(n-1)$-symmetric. However there is no reason to suppose that
$\bigcup``{\cal U}$ is ultra.  It will be if $\iota``V \in {\cal U}$
but that's not much help, beco's although it will ensure that
$\bigcup``{\cal U}$ is ultra, it won't ensure that $\iota``V
\in \bigcup``{\cal U}$.

Notice that $F = \{x: |(V \setminus x)| \not\geq^*|V|\}$ is a filter,
and it's 2-symmetric.  How can $\bigcup``F$ possibly be a filter too??
It would have to be 1-symmetric! But there {\sl is} a 1-symmetric
filtre on $V$, namely $\{V\}$.  So we seem to have proved, if $V
\setminus x$ cannot be mapped onto $V$, then $\bigcup x = V$.  But we
know that anyway: if $V \setminus x$ cannot be mapped onto $V$, then
$V \setminus x$ cannot extend any $B(y)$, so $x$ must meet every
$B(y)$.

Can we show that $\bigcup``{\cal U}$ is {\sl never} ultra?

Sse $F$ is a filter on $V$, and $X \in F$.  Why not say $\{y: (B(y)
\cap X) \in F\}$ is a typical element of the new filter?  It's too
crude.  Either $B(y) \in F$---in which case $y$ belongs to all new
elements, or it doesn't---in which case $y$ belongs to none.  We could
try: ``put $y$ into the new set obtained from $X$ if $B(y) \cap $ is
$F$-stationary'' but that probably won't fare much better.

To get a feel for this, try the filter of cofinite sets.  Then say,
for $X$ a cofinite set, $X':= \{y: |B(y) \cap X| \not\in \Nn\}$. But
then $X' =V$ so it's trivial.

 
\begin{thm}\label{thm:} ($NF + GC$)
$$(\forall x y)(|x| = |y| \bic (\exists x_1 x_2 y_1 y_2 z_1 z_2)((x = x_1 \sqcup x_2 \wedge y = y_1 \sqcup y_2 \wedge x_1 \approx z_1 \approx y_1 \wedge \ \ x_2 \approx z_2 \approx y_2)).$$
\end{thm}
{\sl Proof.}
The right-hand side is simply the assertion that $x$ and $y$ are
$J_0$-equi\-decomposable with two pieces, with `$x_i \sim_1 y_i$' replaced by
their equivalents using the preceding remarks. \blob

That is to say `$|x| = |y|$' is equivalent (assuming $GC$)
to a 3-stratified 2-formula. We have to be cautious in drawing conclusions
about the existence of (Frege/Russell$-$Whitehead) cardinals in
$\nf_3\index{NF3} + GC\index{GC}$ since some of the proofs above may
not be reproducible in $\nf_3\index{NF3} + GC\index{GC}$. Although the 
(Frege/Russell$-$Whitehead) 0, 1, 2, \ldots are all sets in $\nf_3\index{NF3}$,
in general cardinal numbers do not seem to be provably sets in
$\nf_3\index{NF3}$. It suggests, curiously, that the addition of a small amount
of choice ($GC$) to $\nf_3$ may make it much easier to conduct cardinal
arithmetic.

The group of all (inner) $\in$-automorphisms is certainly a subgroup of
$J_{\infty} = \bigcap_{n < \omega} J_n$.  $J_{\infty}$ contains all fixed
points of $j$ (if there are any) and therefore all automorphisms of
$\langle V,\in \rangle$.

What do we know about $J_{\infty}$?  There is no reason to suppose that it
is nontrivial, nor that if it is nontrivial it should be a set.  If it is a
set it is cantorian.  We know it is the nested intersection of $\omega$
symmetric groups, but this does not tell us a great deal.  We know that it
has an external automorphism ($j$) which we would wish had a fixed point.
(would a finite cycle under $j$ be any good?  Perhaps not!) For reasons
which will emerge below it would be nice if it had nontrivial centre.

We know rather more about the $J_{\infty}$ of a saturated model of $NF +
GC\index{GC}$.  In such a model it is certainly nontrivial and we know
exactly the cycle types of all its elements and can even show (tho' perhaps
we need AxCount for this) that the external automorphism $j$ of
$J_{\infty}$ is locally represented by a conjugacy relation. That is to
say, for all $\sigma$ in $J_{\infty}$ there is $\tau$ in $J_{\infty}$ s.t.
$\tau^{-1} \sigma \tau = j`\sigma$.  Chocks away.

Let $\langle V,\in \rangle$ be a saturated model, so $J_{\infty}$ is not trivial.  
Consider permutations $\tau$, $j`\tau$. Now we can show in any case that 
\begin{itemize}

\item if $\tau$ has infinite cycles, $j`\tau$ has cycles of all sizes
\item if $\tau$ has cycles of arbitrarily large finite sizes, then $j`\tau$ has 
      infinite cycles
\item if $\sigma$ has only (bounded) finite cycles whose lengths are in 
      $I \subset \Nn$ then $j^n`\sigma$  eventually has cycles of all sizes 
      that divide $LCM(I)$.

\end{itemize} 

These can be shown by fairly elementary arguments, and should be enough to
classify the cycle types of things in $J_{\infty}$ completely.


Claim:
\begin{quote}

$\forall \sigma$ for all sufficiently large $n$ $j^n`\sigma$ does either 1 or 2:

\begin{enumerate} 

\item It has $|V|$ $n$-cycles for all $n \leq \aleph_0$ (and some $\Z$-cycles);

\item For some $k \in \Nn$ it has $|V|$ $n$-cycles for all $n$ that divide $k$ (and no infinite cycles).

\end{enumerate}

\end{quote}

In fact $j^n`\sigma$ is eventually of type (1) above iff there is no finite
bound on the length of finite cycles under $\sigma$.

This will involve a fair amount of hard work. We have to use some sort of
pigeon-hole principle.  The idea is that for at least one $n$ the number of
things residing in $n$-cycles under $\sigma$ is $|V|$. It looks as if
we need to assume that $V$ is not the disjoint union of $\aleph_0$ smaller
sets, but all we actually need is for this to be eventually true of
$j^n`\sigma$.  We sketch where to go from here.  (Is there a generalisation
of Bernstein's lemma (using $GC\index{GC}$) which says that if $\alpha =
\alpha^{\aleph_0} = \Sigma_{i \in \smallNn} \beta_i$ then either some $\beta_i
\geq \alpha$ or all $\beta_i \geq_* \alpha$? That would probably do)
$GC\index{GC}$ is essential for what follows.

\begin{itemize}
\item If there are $|V|$ infinite cycles then pretty soon there are $|V|$ cycles of any length.

\item Whatever happens there will very soon be $|V|$ fixed points. 

\item If $\sigma$ has $|V|$ fixed points and some $n$-cycles then
$j^k`\sigma$ (with $k$ fairly small) will have $|V|$ $n$-cycles.  Use
ordered pairs of things of order $n$ and fixed points.

\item Eventually $j^n`\tau$ will have $T|V|$ $n$-cycles if it has any at all.
\end{itemize}

Eventually we should prove:

\begin{quote}
$(\forall \tau)(\exists m)(\forall n>m)(j^n`\tau$ and  $j^{n+1}`\tau$ have the same cycle type)
\end{quote}
and $GC\index{GC}$ then gives us 
\begin{quote}
$(\forall \tau)(\exists m)(\forall n>m)(j^n`\tau$ and $j^{n+1}`\tau$ are conjugate in $J_0$)
\end{quote}

so in particular for $\tau \in J_{\infty}$ $\tau$ and $j`\tau$ are
conjugate in $J_0$. Now consider the general case of $\sigma$, $\tau$ $\in$
$J_{\infty}$ conjugate in $J_0$. $(\forall n < \omega)$, $\sigma$, $\tau$
are $j^n$`something in $J_{\infty}$ so we can argue that $j^{-n}`\sigma$
and $j^{-n}`\tau$ are conjugated by something $\gamma \in J_0$. (We have
this beco's $GC\index{GC}$ implies that two things in $J_0$ of the same
cycle type are conjugate) so $\sigma$ and $\tau$ are conjugated by
$j^n`\gamma \in J_n$. Therefore, by saturation of $J_{\infty}$ they are
conjugated by something in $J_{\infty}$. That is to say,

$$(\forall \tau \in J_{\infty})(\exists \sigma \in J_{\infty})(\sigma \tau \sigma^{-1} = j`\tau)$$

This is very pretty: we know the cycle types of all members of $J_{\infty}$
and we know that any two elements of $J_{\infty}$ with the same cycle type
are conjugated by something in $J_{\infty}$.  Of course what we are really
after is showing if possible that $J_{\infty}$ contains a fixed point for
$j$. So what we really want is to swap the quantifiers around in the above
to get:

$$\exists \sigma \in J_{\infty}\ \forall \tau \in J_{\infty} \ \ \sigma \tau \sigma^{-1} = j`\tau$$

for then this $\sigma$ must be an $\in$-automorphism of $V$. 

Proof: 

Suppose there were such a $\sigma$. Then for any $\tau \in J_{\infty}$ 

       $$\sigma \cdot\tau \cdot\sigma^{-1} = j`\tau$$ 

so in particular 

       $$\sigma \cdot\sigma \cdot\sigma^{-1} = j`\sigma$$ 

so $\sigma$ is an $\in$-automorphism of $V$.

In fact it will be sufficient for our purposes that $j$ have a non-trivial
fixed point, because the fixed point would also be an $\in$-automorphism of
$V$.

then we would have a theorem:

\begin{thm} \label{thm:termauto} 
$NF +$ AxCount $+ GC\index{GC} \vdash \ \exists \ \in$-automorphism of $V$
\end{thm} 

The proof would go like this: Work inside a saturated model of
$NF +$ AxCount $+ GC\index{GC}$.  $J_{\infty}$ is a proper class of this
model.  $j$ is an automorphism of it, and $J_{\infty}$ is such that all
automorphisms are inner.  Then throw away the model.

Presumably this won't work beco's $J_{\infty}$ is a saturated group and any
saturated group has too many automorphisms for them all to be inner. We
could try the other extreme: add axioms to make $J_{\infty}$ (when
nontrivial) into a group for which all automorphisms are inner. Then there
will be an $\in$-automorphism of $V$ as before.

It will be sufficient for $J_{\infty}$ to have non-trivial centre.  For
then let $\tau$ belong to the centre.  Let $\sigma$ conjugate $\tau$ and
$j`\tau$. But $\tau^{\sigma} = \tau$ since $\tau$ is in the centre, so
$\tau$ is an automorphism. We can show that in $V^{\sigma}$ everything in
$J_{\infty}$ is an automorphism.  For $$(x \in J_{\infty})^\tau$$ iff $$(x
\in J_0, x \in J_1 \ldots x \in J_n)^\tau$$ Now $(x \in J_n)^\tau$ is just
$(\tau_{n+k}`x \in J_n)$ is $x \in J_n$.

But presumably it is obvious that $J_{\infty}$ has trivial centre, by some
compactness argument \ldots

                 
\subsubsection{Some more random tho'rts on $\in$-automorphisms, from May 2008, in the form of a letter to Nathan Bowler}

We are working in NF.

We start with two observations about  $\in$-automorphisms.
\begin{enumerate}
\item $\pi$ is an  $\in$-automorphism iff $\pi = j(\pi)$;
\item If $\sigma \pi \sigma^{-1} = j(\pi)$ then, in the permutation 
model $V^\sigma$, $\pi$ has become an  $\in$-automorphism.
\end{enumerate}


What must the cycle type be of an $\in$-automorphism?  As far as i can
see, everything we know about the cycle types of $\in$-automorphisms
follow from the fact that every $\in$-automorphism is $j$ of
something.

If $\pi$ is an $\in$-automorphism then either $\pi$ is of finite
order---$n$, say---in which case for each $m|n$ it has $|V|$-many
things belonging to $m$-cycles; or it is of infinite order, in which
case it has it has $|V|$-many things belonging to $m$-cycles for every
$m \leq \aleph_0$.  This seems to be all we can say---and (as i say)
it seems to follow merely from the fact that every $\in$-automorphism
is in $J_2$.  (I think i do mean $J_2$ not $J_1$: suppose $\sigma$ is
a permutation with cycles of all even orders.  Then $j(\pi)$ has
cycles of all even orders plus infinite cycles, and it isn't until
$j^2\pi$---which has cycles of all orders---that things settle down.

Thus it seems that every cycle type of a permutation in $J_2$ can be
the cycle type of an $\in$-automorphism. (The cycle type of a
permutation is how many cycles you have of each size).  With a little
bit of AC (the version i call `GC') cycle types are the same as
conjugacy classes.  So certainly if i show you a
cycle-type-aka-conjugacy-class of a member of $J_2$ you can cook up a
permutation model in which that conjugacy class contains an
$\in$-automorphism. (The example i gave above, of a $\pi$ with cycles
of all even lengths, gives us a $j(\pi)$ with a cycle type that cannot
be the cycle type of an $\in$-automorphism.  This is why it has to be
$J_2$ not $J_1$.)

Can we do this for all $J_2$ conjugacy classes simultaneously?  That
is to say, might there be a permutation model in which there are so
many $\in$-automorphisms that every conjugacy class of elements of
$J_1$ contains an $\in$-automorphism?  Might it be that for all $\pi
\in J_1$, $\pi$ and $j(\pi)$ are conjugate?  This question turns out to
be related to the question: how many conjugacy classes (wrt $J_0$) are
there of elements of $J_1$?  This is an instance of a general class of
questions for which i have no feel beco's nobody ever taught me group
theory: for $G$ a subgroup of $J_0$ how many conjugacy classes (wrt
$J_0$) can $G$ have?  It's pretty clear that there are lots of
conjugacy classes (wrt $J_0$) of elements of $J_0$---as you say it's
like ``the number of cardinals''.  I suspect it's a delicate
calculation to ascertain precisely how many conjugacy classes (wrt
$J_0$) there are of elements---even of of $J_0$ (computing the size of
quotients is hard of course) but i'm going to have a crack at it
anyway (at some point!).

The key observation now is that every set of $\in$-automorphisms is
strongly cantorian!  So if every conjugacy class of elements of $J_1$
contains an $\in$-automorphism it follows that the collection of such
conjugacy classes will be strongly cantorian!  Is this absurd?  Might
this number actually be finite? Or a sensible ZF-style number like
$2^{\aleph_0}$ or the least strong inaccessible?  If it can, then we
have only the second example of something i have been seeking for a
long time: a sensible small number emerging as the answer to a
question about big NF-style sets. And even if that isn't a sensible
number, it might nevertheless be the case that if we make $n$ large
enough then the cardinality of the set of conjugacy classes (wrt
$J_0$) of elements of $J_n$ might be sensible.  


\ldots but isn't this easy?  Surely, assuming GC, there are precisely
$\aleph_0$ conjugacy classes of $J_2$ in $J_0$---for the reasons given
above.  We described them!

So, assuming GC, the collection of things that are possible
automorphisms is actually a set, and a big set at that.  However we
can prove that every set of actual-automorphisms is strongly
cantorian.

Can we find a permutation model in which there is a proper class of
$\in$-automorphisms?  Ward Henson had this clever permutation that
gave a proper class of Quine atoms: $\prod{\alpha \in NO}(T\alpha,
\{\alpha\})$.  The point is that in $V^\pi$ $x$ is a Quine atom iff
$\pi(x) = \{x\}$, and for this permutation that happens iff $x$ is a
cantorian ordinal.  What about $\in$-automorphisms in $V^\pi$?  $x$ is
an $\in$-automorphism in $V^\pi$ iff $\pi^{-1}\cdot x \cdot \pi = jx$.
So we would be looking for a permutation $\pi$ such that
$\pi^{-1}\cdot x \cdot \pi = jx$ happens iff $x$ is a cantorian
ordinal.  This is of course absurd, but it might point us in the right
direction.  It would work equally well if $x$ were a permutation of the
form $j(T\alpha, \{\alpha\})$.  Can we cook up a permutation that,
for all $\alpha \in NO$, conjugates  $j(T\alpha, \{\alpha\})$ with  
$j^2(T\alpha, \{\alpha\})$?


Let us write `$\pi_\alpha$' for  `$j(T\alpha, \{\alpha\})$'  And let $\sigma$ be a permutation such that

$$(\forall \alpha \in NO)(\sigma^{-1}\pi_\alpha \sigma = j(\pi_\alpha)).$$



But when you put it like that there seems no reason at all why that might work.

%\end{document}

A later thought, october 2017

It sounds as if everything in $J_2$ has the right kind of cycle type
to be an automorphism. That is to say (given that conjugacy is a
congruence relation for $j$, every conguence class of $J_2$ is fixed
by the function $[\tau] \mapsto [j\tau]$.  But that means that $J_2$
has only a strongly cantorian set of congruence classes!

\section{Arithmetisation}
Let $NZF = \nf\ \cap \zf$.

NZF contains extensionality, pairing, stratified separation,
stratified replacement, transitive containment, collection, sumset,
power set, and existence of a Dedekind-infinite set.


If $T_1$ and $T_2$ are two theories arithmetised in the same way then
not both $T_1 \vdash$ Con($T_1 \cap T_2$) and $T_2 \vdash$ Con$(T_1
\cap T_2$), for otherwise $T_1 \cap T_2 \vdash$ Con($T_1 \cap T_2$).
So we can write $T_1 \geq T_2$ iff $T_1 \vdash$ Con($T_1 \cap T_2$).
This gives us a relation on all theories in one arithmetisation that
extends $\subseteq$ and even $T_1 \vdash $Con($T_2$).  (It is not
clear that this is transitive \ldots) What does this do for us in the
present instance? It is complicated by the fact that $ZF$ and \nf\ are
not arithmetised in the same way.  Presumably $\nf \not\vdash
Con(NZF)$ because we cannot do the unstratified inductions needed to
show that all the axioms are true in $WF$.

Are the two relations actually the same? Notice that $T \vdash Con(T
\cap S)$ iff $T \vdash (Con(T) \vee Con(S))$ The converse of the
following is easy, but what of the formula itself (the hard
direction)?

$$T \vdash (Con(T) \vee Con(S)) \quad \to \quad T \vdash Con(S)?$$

Try

$$T_0 = T$$ $$T_{n+1} = T_n \cup \{con(T_n) \vee Con(S)\}$$ 
$$T_\infty = \bigcup_{i \in \smallNn} T_n$$

WANT: \begin{quote} $T_\infty \vdash Con(T_\infty) \vee Con(S)$ but
$T_\infty \not\vdash Con(S)$ \end{quote} which would be a good counterexample.

It seems to me that the {\sl desideratum} $T_\infty \models Con(T_\infty) \vee 
Con(S)$ holds because we can argue in something very elementary (so presumably 
in $T$) that in every model of $T_\infty$ either $Con(S)$ holds or
$\bigwedge_{i \in \smallNn} Con(T_n)$ holds, which should be enough to show that
$Con(T_\infty)$ holds. Therefore $T_\infty \vdash Con(T_\infty) \vee Con(S)$.

Now to persuade ourselves that $T_\infty \not\vdash Con(S)$. If $T_\infty\vdash
Con(S)$ then $T_n \vdash Con(S)$ for some $n$. id est:

$$T\vdash (\bigwedge_{i \leq n} (Con(T_i) \vee Con(S))) \to Con(S)$$
or

$$T\vdash ((\bigwedge_{i \leq n} Con(T_i)) \vee Con(S)) \to Con(S)$$

whence

$$T \vdash \bigwedge_{i \leq n} Con(T_i) \to Con(S)$$

So $T_\infty \models Con(T_\infty) \vee Con(S)$ seems to hold but
$T_\infty \models Con(S)$ doesn't---as desired.  

A message from Richard Kaye
\begin{quote}
A good question.

Suppose $T$ is sufficiently strong.  ($T$ extends $\Delta_0$ induction +
exponentiation will do.  $T \supseteq$ Prim rec arithmetic is more than
enough.)  suppose also that $T$ is consistent, and $T+ \neg con(S) \vdash
con(T)$.  Then, by the assumption that $T$ is strong we have:

1.  If $\sigma$ is any $\Sigma_1$ sentence then $T \vdash$ `$T$ proves
$\sigma$' ( in fact, $T \vdash$ `$Q$ proves $\sigma$', where $Q$ is
Robinson's minimal artithmetic containing only the recursive defns of + and
. )

2.  the second incompleteness theorem can be formalised in $T$, that is:
              $T$ proves  ` $con(T)$ implies ``$T$ does-not-prove $con(T)$" '

Now consider an arbitrary model $\M$ of $T$. (This is easier than writing
things like T proves `... proves `` ... proves ..." ' !)  Suppose for the
moment that $\M \models \neg con(S)$.  Then by (2) $\M$ contains a proof
from $T$ of $\neg con(S)$, and by simple modification of these nonstandard
proofs, together with the standard proof that $T+\neg con(S)$ implies
$con(T)$ we have that $\M$ contains a proof (of nonstandard length) of
$con(T)$.  But this implies, by 2, that $\M$ satisfies $\neg con(T)$, for if 
$\M \models con(T)$ then it can't have a proof of $con(T)$.  Thus $\M \models
\neg con(T)$ and $\neg con(S)$, contradiction, so no such $\M$ exists so $T
\vdash con(S)$.

This argument seems to depend critically on the second incompleteness theorem
formalised in the model, so it seems unlikely that the $\Pi_1$ disjunction
property is possible in general.  Actually, its well known that it isn't
true in general.  We need two facts:

3.  For $T$ extending $I\Delta_0$ + exp, as before, $T$ proves the
Matijasevic theorem so any $\Delta_0$ formula is equivalent to existential and
universal forms.  This in turn means that any extension of models of $T$
automatically preserves $\Delta_0$ formulas.

4.  There is something called the JOINT EMBEDDING PROPERTY. A theory $T$
has JEP iff for every pair of models $\M$, $\N$ of $T$ there is a third
model ${\cal K}$ of $T$ and embeddings $M \inj {\cal K}$ and $\N \inj {\cal
K}$.  Plenty of theories have JEP. eg $T_p$ = the theory of fields of a given
charateristic $p$.  (You don't have to say anything else, not even that the
fields are alg closed.) Some don't, eg the theory of fields. (you cant
jointly embed two fields of different characteristic) A well know
PRESERVATION theorem says $T$ has JEP iff whenever $T$ proves a disjunction
of purely universal sentences then it proves one of them.  Unfortunately no
theory extending $I\Delta_0$ + exp has JEP. (This is proved either by a neat
argument involving Post's simple set, or by a double diagonalization
argument, i.e. producing the disjunction explicitly.)

There's a very strange and rather weak theory of arithmetic, called
Open induction + normality, which does have JEP.  It's the only one
we know of: weaker theories tend not to have it, and stronger theories
don't either. The rather surpising result that NOI has JEP was proved
by Otero recently.  Unfortunately it (NOI) is too weak to talk about
consistency, or prove the Matijasevic theorem.

Hope this is of interest,

Richard

\end{quote}

Is this the place to note the old suggestion that \nf\ might be the 
result of adding $\neg Con(T)$ to some otherwise sensible theory $T$.
Surely it shouldn't be to hard to show that this is nonsense?

\subsubsection{The Thoughts of Chairman Holmes}

In my Ph.D. thesis and in my paper ``Systems of Combinatory Logic
Related to Quine's 'New Foundations'" (Annals of Pure and Applies
Logic, vol. 53 (1991) pp. 103-33) I describe systems of combinatory
logic, equivalent to untyped lambda-calculi with ``stratification"
restrictions on abstraction, which are of precisely the same
consistency strength and expressive power as NFU + Infinity and
extendible in parallel with NFU extensions (they are weakenings of a
system equivalent to NF); this suggests computer science applications,
as this system is similar to typed systems now in use.  I have an
unpublished essay in which I develop an intuitive motivation for this
system in terms of security of the abstract data type ``program" in a
(very) abstract model of programming, along the same lines as the
argument for set theory above; I also observe that the notion of
"strongly Cantorian set" seems to translate to the general notion of
"data type" internally to the model of programming.  This is
interesting, because ``strongly Cantorian set" is a notion which has no
analogue in ZFC; it is specific to NF and its relatives, and it is
interesting to see it corresponding to anything outside that context.



\section{Typical ergodicity}

     The minimal kind of ambiguity that we expect of a model of
negative type theory is a sort of ``ergodic" ambiguity, where there is
no first-order sentence true at a unique type. Let us call this {\sl
typical ergodicity.}  This is quite easy to arrange.  Let $\M$ be any
model of \tst, and let $\kappa$, $\U$ be an infinite object and an
ultrafilter on it. Then $\M^\kappa/\U$ is also a model of \tst, and is
elementarily equivalent to $\M$.  $\M^\kappa/\U$ has many more types
of course, and they are indexed by nonstandard integers, and thus
$\M^\kappa/\U$ can be seen to split naturally into one model of \tst\
(which will be an isomorphic copy of $\M$) and lots of models of \TZT.
What we will be working towards is the claim that {\sl any} such model
of \TZT\ satisfies ergodic ambiguity.

    To do this it is convenient to express type theory not merely as a
one-sorted theory but as a one-sorted theory without type predicates. This
is probably worth doing in some detail as it has not been done in print to
my knowledge. We will need the idea of a set being a {\sl universe}.  Let us
abbreviate ``$x$ is a universe" to $U(x)$''. Then we adopt the definition:
 
\begin{dfn}
$U(x)$ iff $(\forall y)(\forall z)(z \in x \to z \in y \to y \subseteq x)$.
\end{dfn}

We can think of this, metamathematically, as $(\exists n)(x = V_n)$.
We can also think of an equivalence relation $x \sim y$ iff $(\exists
z)(x \in z \wedge y \in z)$ and then universes are equivalence classes
under $\sim$.  We will need an axiom $((\forall x)( \exists ! y)(U(y)
\wedge x \in y)$ saying that every set belongs to a unique universe,
and another saying that a universe can have at most one universe as a
member. If we want to specify that it is TST we are dealing with not
\TZT we can say that there is a universe which does not have another
universe as a member.  For \TZT\ we assert that every universe has
another universe as a member.  This enables to re-interpret all the
type-predicates we have abolished ($T_0(\ )$, $T_1(\ )$, etc.), should
we wish to: $T_0(x)$ is short for $(\forall y)(y \not\in
x)\wedge(\forall y)(\forall z)(x \in y \wedge z \in y \to (\forall w)(
w not\in z))$ and similarly $T_{n+1}(x)$ is short for $(\exists y z
w)(x \in y \wedge z \in y \wedge w \in z \wedge T_n(w))$. To obtain the
remaining axioms let $\Phi$ be an axiom of TST with type indices (or
predicates).  Delete them, and let $\Phi^x$ be the result of
relativising all variables in the stripped version of $\Phi$ to
variables in the list $x_i: i\in I$.  Then $(\forall x_i)(U(x_i) \to
\Phi^x)$ is an axiom in the one-sorted version.
\begin{prop}
Every model of \TZT\ obtained from an ultraproduct satisfies typical ergodicity. 
\end{prop}

\Proof

We can say in a first-order way ``there is a unique type at which
$\phi$ holds''.  This will be preserved by \Los. If $\phi$ is, indeed,
true at a unique type in $\M$ then that will be so in $\M^\kappa/\U$
and that unique type must be in the copy of $\M$.  Thus the behaviour
of all other types must be ``ergodic".

\endproof

%\chapter{Bibliography}
%\input bibliography


%\end{document}

Another kind of weak ambiguity that could be confused with Typical
Ergodicity (well, {\sl I} confused it) is that exhibited by a model
$\M$ (of \TZT) where, for each closed formula $\phi$, there is $n \in
\Nn$ such that $\M \models$ the scheme $\phi\bic \phi^n$ over all
levels.  Are there models of \TZT\ in which for every $\phi$ there is
an $n$ such that \ldots? Presumably yes (beco's we believe NF to be
consistent) but is ``for every $\phi$ there is $n$ \ldots'' weak
enough to not refute AC?  What happens to $\TZT + AC$ if it does?
Think about the tree of lists of pairs $\tuple{\phi,n}$ meaning the
scheme $\phi\bic \phi^n$ over all levels. Ordered by reverse
end-extension of course.  If the grand scheme is inconsistent then the
set of consistent lists is a wellfounded fragment of the tree and
every list in it has a rank.  The lower the rank the stronger the theory(!!?!)

At some point i must work out whether Van der Waerden's theorem has anything to say about Amb$^n$.


\subsection*{Bad Join}



Now in both derivations ending in the two upper sequents the cut
formula was the last formula introduced.  I suppose the most sensible
thing to do at this stage is to do some relettering so that the two
formul{\ae} remaining after the cut are the same! If we replace `$w$'
by $x$' in the second horn we get:
 

$
\proof{
\proof{\proof{\proof{\proof{z \in x \vdash z \in x}{(\forall y)(y \in x) \vdash z \in x\ \ \ \forall-L}
                    }{z \not\in x,(\forall y)(y \in x) \vdash\ \ \neg-L}
             }{(\exists z)(z \not\in x), (\forall y)(y \in x) \vdash\ \ \ \exists-L};\ \ (\forall y)(y \in x), x \in x \vdash x \in x
      }{(\forall y)(y \in x), x \in x, x \in x \to (\exists z)(z \not\in x) \vdash\ \ \ \to-L}
      }{
\proof{
  \proof{\proof{(\forall y)(y \in x), x \in x \to (\exists z)(z \not\in x) \vdash x \not\in x\ \ \ \neg-R}
               {(\forall y)(y \in x), (\forall y)(y \in x \to (\exists z)(z \not\in y)) \vdash x \not\in x\ \ \ \forall-L}}
        {(\forall y)(y \in x), (\forall y)(y \in x \to (\exists z)(z \not\in y)) \vdash (\exists z)(z \not\in x)\ \ \ \exists-R}
      }{
  \proof{(\forall y)(y \in x)\vdash (\forall y)(y \in x \to (\exists z)(z \not\in y))\to (\exists z)(z \not\in x)\ \ \ \to-R}
         {(\exists x)(\forall y)(y \in x)\vdash (\forall y)(y \in x \to (\exists z)(z \not\in y))\to (\exists z)(z \not\in x)\ \ \ \exists-L}
       }}
$

%\newcommand{\nonce1}{\proof{\proof{\proof{\proof{\proof{z \in x \vdash z \in x}{(\forall y)(y \in x) \vdash z \in x\ \ \ \forall-L}    }{z \not\in x,(\forall y)(y \in x) \vdash\ \ \neg-L}             }{(\exists z)(z \not\in x), (\forall y)(y \in x) \vdash\ \ \ \exists-L};\ \ (\forall y)(y \in x), x \in x \vdash x \in x      }{(\forall y)(y \in x), x \in x, x \in x \to (\exists z)(z \not\in x) \vdash\ \ \ \to-L}      }{\proof{  \proof{\proof{(\forall y)(y \in x), x \in x \to (\exists z)(z \not\in x) \vdash x \not\in x\ \ \ \neg-R}               {(\forall y)(y \in x), (\forall y)(y \in x \to (\exists z)(z \not\in y)) \vdash x \not\in x\ \ \ \forall-L}}        {(\forall y)(y \in x), (\forall y)(y \in x \to (\exists z)(z \not\in y)) \vdash (\exists z)(z \not\in x)\ \ \ \exists-R}      }{  \proof{(\forall y)(y \in x)\vdash (\forall y)(y \in x \to (\exists z)(z \not\in y))\to (\exists z)(z \not\in x)\ \ \ \to-R}         {(\exists x)(\forall y)(y \in x)\vdash (\forall y)(y \in x \to (\exists z)(z \not\in y))\to (\exists z)(z \not\in x)\ \ \ \exists-L}       }}}

Thomas,

Let's be precise. Consider the sequent calculus for classical first order
logic.  Then, a cut free derivation of a weakly stratified sequent contains
only weakly stratified sequents. This follows trivially from the subformula
principle (every formula in a cut free derivation is a subformula of a
formula in the final sequent) and the observation that a subformula of a
weakly stratified formula is always weakly stratified: this is not true for
stratified formulas, $x \in x \to (\forall y)(y \in x) \to \bot)$ is a
weakly stratified and unstratified formula of $(\forall w)(w \in x \to
(\forall y)(y \in w) \to \bot)$.

When I proved ``jadis" that every stratified theorem of the predicate
calculus has a stratified (but not necessarily normal!) proof, I proceeded
as follows: first I took a cut free proof of the stratified sequent, this
proof is weakly stratified but could be unstratified as in your example,
then I gave a method to introduce suitable cuts in order to obtain a
stratified proof. The resulting derivation has moreover the property that
if you remove the cuts of it in Gentzen's way you get (almost) the original
cut free derivation.

Now if you take natural deduction and/or intuitionistic logic you have 
the same results with normal instead of cut free (you can even drop the
``(almost)").

The situation is similar in the logic with terms $\{x:A\}$. But here you have
to be a little more careful to avoid triviality.

Marcel

\begin{verbatim}

From T.Forster Tue Jul  6 14:10:24 1993
Received: by moa.pmms.cam.ac.uk (UK-Smail 3.1.25.1/1); Tue, 6 Jul 93 14:10 BST
Message-Id: <m0oDCmp-0000cAC@moa.pmms.cam.ac.uk>
Date: Tue, 6 Jul 93 14:10 BST
From: Thomas Forster <T.Forster>
To: crabbe@risp.ucl.ac.be, ekman@cs.chalmers.se, T.Forster
Subject: wk strat deriv
Status: R
\end{verbatim}

I have been thinking a lot about weakly stratified epsilon-induction.  It
provides an example of a stratified theorem with (as far as i can see) no
stratified proof. The proof is weakly stratified tho'.

{\bf However} if we now introduce a $\forall$ we find a stratified theorem
for which the proof is not stratified. Indeed not even weakly stratified
unless we allow the stratification to avoid the outermost bound variable.
\begin{verbatim}

From T.Forster Wed Jul  7 11:38:33 1993
Received: by moa.pmms.cam.ac.uk (UK-Smail 3.1.25.1/1); Wed, 7 Jul 93 11:38 BST
Message-Id: <m0oDWtU-0000d0C@moa.pmms.cam.ac.uk>
Date: Wed, 7 Jul 93 11:38 BST
From: Thomas Forster <T.Forster>
To: crabbe@risp.ucl.ac.be, ekman@cs.chalmers.se, T.Forster
Subject: aha!
Status: R

I now think i understand the significance of the proof that
if no member of $x$ is the universe then neither is $x$. I think
the point is that if you ``absorb" the cut with a \exists
introduction and elimination in the style of the \vee 
introduction and elimination (as Jan showed me) you get a
stratified proof.  The only trouble is: it's not clear to me 
that it is normal.
The new proof looks like this:
\end{verbatim}

which uses the $\exists$-elimination rule Jan showed me.  Presumably this proof is not normal though.
I am still not very happy!


% \begin{equation}
 
$$\AxiomC{$[(\forall z)(z \in x)]^1$}
\forallelim{$x \in x$}
 \AxiomC{$[(\forall w)(w \in x \to \neg(\forall z)(z \in w))]^2$}
\forallelim{$x \in x \to \neg(\forall z)(z \in x)$}
\arrowelim{$\neg(\forall z)(z \in x)$}
 \AxiomC{$[(\forall z)(z \in x)]^1$}
\arrowelim{$\bot$}
 \arrowintlabel{$\neg(\forall z)(z \in x)$}{1}
 \arrowintlabel{$(\forall w)(w \in x \to \neg(\forall z)(z \in w))\to \neg(\forall z)(z \in x)$}{2}
 \DisplayProof$$


 %\end{equation}


$$\AxiomC{$[(\forall z)(z \in x)]^1$}
\forallelim{$x \in x$}
 \AxiomC{$[(\forall w)(w \in x \to ((\forall z)(z \in w) \to \bot)]^2$}
\forallelim{$x \in x \to ((\forall z)(z \in x)\to \bot)$}
\arrowelim{$(\forall z)(z \in x) \to \bot$}
 \AxiomC{$[(\forall z)(z \in x)]^1$}
\arrowelim{$\bot$}
 \arrowintlabel{$(\forall z)(z \in x) \to \bot$}{1}
 \arrowintlabel{$(\forall w)(w \in x \to ((\forall z)(z \in w)\to \bot))\to ((\forall z)(z \in x)\to \bot)$}{2}
 \DisplayProof$$

Observe: this proof is constructive.









\begin{verbatim}

From crabbe@risp.ucl.ac.be Thu Jul  8 13:38:33 1993
Received: by moa.pmms.cam.ac.uk (UK-Smail 3.1.25.1/1); Thu, 8 Jul 93 13:38 BST
Received: by emu.pmms.cam.ac.uk (UK-Smail 3.1.25.1/1); Thu, 8 Jul 93 13:38 BST
Received: from  by sci1.sri.ucl.ac.be (4.1/jpk-5.93)
        id AB15355; Thu, 8 Jul 93 14:39:13 +0200
From: crabbe@risp.ucl.ac.be (M. Crabbe)
Message-Id: <9307081239.AB15355@sci1.sri.ucl.ac.be>
Date: Thu, 8 Jul 1993 14:41:06 +0100
To: Thomas Forster <T.Forster@pmms.cam.ac.uk>
X-Sender: crabbe@sci1.sri.ucl.ac.be
Subject: Re: wk strat deriv
Status: R
\end{verbatim}

If you transform your original proof using grosso modo the method of my 78-paper, you get:

Of course you must introduce a cut. If you remove the cut then you get
the original proof. You cannot find a normal stratified proof of this,
at least in the predicate calculus.

Your second proof is OK too.

Marcel

\subsubsection{}

Let's try to get a normal derivation of $(\forall y)( y \in x \to \bot)$
from $(\forall w)(w \in x \to (\forall y)(y \in w \to \bot))$.  The last
line can only be the result of an introduction rule, and this is presumably
the $\forall y$. So we have $$\proof{\vdots}{\proof{y \in x \to
\bot}{(\forall y)( y \in x \to \bot)}}$$ and the $y \in x \to \bot$ can
only be an $\to$-introduction (How can i be sure?) so it must be
$$\proof{\proof{\vdots}{\bot}}{\proof{y \in x \to \bot}{(\forall y)( y \in
x \to \bot)}}$$ and then, since there is no rule to introduce $\bot$, the
preceding step must have been an elimination \ldots

Perhaps the correct notion is that of a weakly stratified derivation.
Suppose we allow the rule: if you can derive $\phi(x,\vec y)$ from
$(\forall z \in x)(\phi(z,\vec y)$, where the derivation is weakly
stratified, then you draw a line and call this a proof of $(\forall
x)(\phi(x,\vec y))$.  I'm not absolutely sure what a weakly stratified
derivation is, but it's presumably one with a stratification defined on all
vbls bound in it. This seems to be intermediate in strength between $SI$ and
$WSI$. In particular (and this is why it seems to me to be the ``right" notion)
this does not (apparently) imply the whole of $\in$-induction the way $WSI$
does once you have comprehension.  On the other hand we can now prove that 
no set is a member of itself as desired. 

$$\proof{\proof{\proof{\proof{(\forall y)(y \in x \to (y \in y) \to \bot)}{x \in x \to x \in x \to \bot} x \in x}{x \in x \to \bot} x \in x}{\bot}}{x \in x \to \bot}$$

Can there be a derivation that is unstratified even tho' every formula in it is stratified?

%\end{document}




$\proof{
 \proof{
 \proof{\proof{\proof{z \in x \vdash z \in x}{(\forall y)(y \in x) \vdash z \in x \ \ \forall-L}}
              {z \not\in x, (\forall y)(y \in x) \vdash \ \ \neg-L}
      }{(\exists z)(z \not\in x), (\forall y)(y \in x) \vdash \ \ \neg-L;\ \ (\forall y)(y \in x), x \in x \vdash x \in x}
       }{
 \proof{(\forall y)(y \in x), x \in x, x \in x \to (\exists z)(z \not\in x) \vdash \ \ \to-L}
          {(\forall y)(y \in x), x \in x \to (\exists z)(z \not\in x) \vdash x \not\in x \ \ \neg-R}
        }}{
 \proof{
 \proof{(\forall y)(y \in x), (\forall y)(y \in x \to (\exists z)(z \not\in y)) \vdash x \not\in x \ \ \forall-L}
       {(\forall y)(y \in x), (\forall y)(y \in x \to (\exists z)(z \not\in y)) \vdash (\exists z)(z \not\in x) \ \ \exists-R}
       }{
 \proof{(\forall y)(y \in x)\vdash (\forall y)(y \in x \to (\exists z)(z \not\in y))\to (\exists z)(z \not\in x)\ \ \to-R}
       {(\exists x)(\forall y)(y \in x)\vdash (\forall y)(y \in x \to (\exists z)(z \not\in y))\to (\exists z)(z \not\in x)\ \ \exists-L}
        } }$
\begin{thebibliography}{}
\bibitem{ST&IL} Quine, Set theory and its Logic.  Belknap Press,  Harvard, 1967
\end{thebibliography}


\end{document}
From phil-logic@bucknell.edu Thu Apr  3 02:43:31 1997
Received: by emu.dpmms.cam.ac.uk (UK-Smail 3.1.25.1/1); Thu, 3 Apr 97 02:43 BST
Received: from reef.bucknell.edu by mail.bucknell.edu; (5.65v3.2/1.1.8.2/17Jul96-0109PM)
        id AA05895; Wed, 2 Apr 1997 20:47:25 -0500
Date: Wed, 2 Apr 1997 20:47:25 -0500
Message-Id: <v01520d00af69460e6255@[132.181.33.140]>
Errors-To: fwilson@bucknell.edu
Reply-To: phil-logic@bucknell.edu
Originator: phil-logic@bucknell.edu
Sender: phil-logic@bucknell.edu
Precedence: bulk
From: g.solomon@phil.canterbury.ac.nz (Graham Solomon)
To: Multiple recipients of list <phil-logic@bucknell.edu>
Subject: Re: Set Theory Question
X-Listprocessor-Version: 6.0c -- ListProcessor by Anastasios Kotsikonas
X-Comment: Discussions of issues in the philosophy of logic
Content-Type: text/plain; charset="us-ascii"
Mime-Version: 1.0
Status: RO


I asked:
>  >Is there any relation between lurking non-normalisability and the presence
>  >of contraction?

Thanks to Torkel, for the proofs.

I'd like to elaborate a bit on the role of contraction. I'll leave comments
about the relation between natural deduction and sequent calculus for
another time.

Let's assume a naive comprehension scheme.

Let \{x:\phi\} be a name such that
$\forall y (y \in \{x:\phi\} \bic  phi(y))$

Let a = \{x:x \in x \to  A\} for any sentence A

\begin{enumerate}

\item  $a \in a \bic  (a \in a \to  A)$  by comprehension
\item $a \in a \to  (a \in a \to  A)$    from 1
\item $a \in a \to  A$                            contraction on 2
\item $(a \in a \to  A) \to  a \in a$    from 1
\item $a \in a$                                 3,4 modus ponens
\item $A$                                                  3,5 modus ponens
\end{enumerate}

The sentence $A$ can be anything. We could, like Fitch is supposed to have
urged, give up modus ponens. But if we want naive comprehension, I think it
better to give up contraction rather than modus ponens. In terms of naive
plausibility, modus ponens is surely more naively plausible than is
contraction.

 Note: Lob's ``paradoxical" tautology is (B \bic  (B \to  A)) \bic  (B & A)

Now consider the usual formulation of the Russell paradox, which involves
negation. We have A \bic  -A and derive A and -A.
begin{enumerate}
\item $A \bic  -A$
\item $A \to  -A$           from 1
\item  $(A \to  -A) \to  -A$   minimal negation
\item $-A$                2,3 modus ponens
\item $-A \to  A$           from 1
\item $A$                 4,5 modus ponens
\end{enumerate}
Minimal negation looks to be the weakest assumption available to derive the
contradiction.

Is contraction at work here?


I noticed as a result of this thread that $(B \bic  (B \to  A)) \bic  (B \wedge A)$
is odd in some way, and now Graham says's this is L\"ob's `paradoxical
tautology'.  What did he say about it?

        Thomas


From phil-logic@bucknell.edu Fri Apr  4 00:30:12 1997
From: g.solomon@phil.canterbury.ac.nz (Graham Solomon)

Let me re-write the tautology as $(A \bic (A \to B)) \bic (A & B)$ so
it's more easily comparable with $(A \bic -A) \bic (A \wedge -A)$
which this thread started with.  The former is connected to Curry's
paradox and the (better known?) related Lob's theorem. Lob didn't say
anything specifically about the tautology (at least not that I
recall), but keeping the tautology in mind can help one see that
there's no real trickery going on. At this level of analysis, the
former is a negation-free variant of the latter. I think it's
interesting (but not surprising) that contraction shows up explicitly
in the negation-free ``paradoxes".


Re: Neil's comments, which I won't quote

1. You can write contraction as a tautology, though I like to use it as the rule
 ``from $A \to  (A \to B)$ infer $A \to  B$". In so far as we are concerned with the
consistency of naive comprehension and contraction, we'd probably like to
look at a generalization which reduces $n+1$ $A$s to $n$ $A$s, as well as which
applies to any arrow-like connective. Greg Restall discusses this in print
somewhere.

2. I suspect if you think about it carefully you'll realize that your
suggested  ``$a \in a \to  (a \in (a \to  A))$" is not
well-formed.


Torkel replied:
>  Well, as I usually understand minimal logic, $-A$ is short for $A\to \bot$, which
>makes the validity of $(A \to  -A) \to  -A$ a special case of the validity
>of contraction (in your sense). How would you explain minimal negation?


I hope we aren't talking at cross-purposes. I had in mind an axiomatization
of minimal logic using negation rather than F.

At any rate, it's helpful for me to think of $(A \to  -A) \to  -A$ as a special
case of contraction. Then, is it alright with you to claim that contraction
does indeed play a significant role in both the derivation of $B$ from  
$A \bic  (A \to  B)$ and of $A \wedge -A$ from  $A \bic  -A$, in the usual axiomatics? The
use of - in the latter just buries contraction a bit.

My speculation about the normalisation stuff is that the puzzle shows up
because of contraction (which shows up whenever there's multiple use of the
same assumption)*, and that sequent calculus handles contraction better
than does natural deduction. But ``handles" has to be given some content.


* Like Thomas I've been wondering if it isn't ``always the case that where
something doesn't normalise there must be a premiss that is introduced
twice? And doesn't this mean that contraction is used somehow?"



From phil-logic@bucknell.edu Wed Apr  9 09:03:21 1997
From: Torkel Franzen <torkel@sm.luth.se>

  Graham says:

  >I suppose this should really be under the subject heading: Curry sequents
  >and contraction. But here goes. Following is a sequent calculus proof of
  >   $p\to (p\to q),(p\to q)\to p \vdash q$
  >written up using Gentzen's original rules (for sequents regarded as lists
  >rather than sets). I hope it doesn't break up in transmission (and survives
  >close scrutiny!).

  The proof as written can't be quite what you are after. Look at the first
9 lines:
\begin{enumerate}

\item  $p \vdash p$                                Axiom
\item  $q \vdash q$                                Axiom
\item  $p\to q,p \vdash q$                           1, thinning left
\item  $p,q \vdash q$                              2, thinning left
\item  $q,p \vdash q $                             4, interchange left
\item  $p\to q,p\to q,p,p \vdash q$                    3,5, \to left
\item  $p\to q,p,p \vdash q        $                 6, contraction left
\item  $p,p,p\to q \vdash q$                         7, interchange left (twice)
\item  $p,p\to q \vdash q$                           8, contraction left
\end{enumerate}

  Line 3 is not obtainable from line 1 by thinning. 

  A correct proof of $p, p\to q \vdash q$ would be
\begin{enumerate}
\item $p \vdash p$                       Axiom
\item $q \vdash q $                      Axiom
\item $p\to q,p \vdash q$                  1,2 \to left
\item $p,p\to q \vdash q $                 3, interchange left
\end{enumerate}

From phil-logic@bucknell.edu Fri Apr 11 00:54:43 1997
From: g.solomon@phil.canterbury.ac.nz (Graham Solomon)


Charlie:
>I am completely lost about what is going on here.  Is it at all relevant
>to the discussion for me to observe that on the lines below, all sentences
>to the left of the turnstiles can be true and the one to the right false?

Here's a quick sketch.

Let  $W,$ $X$, $Y$, $Z$, be finite, possibly empty, sequences of formulas.
Let $A$, $B$, be arbitrary formulas.
The sequent   $X \vdash Y$      informally reads:  if all formulas in
$X$ are true then at least one formula in $Y$ is true; or, for $X \vdash A$ : there's
a natural deduction proof of $A from X$.

Tree proofs are basically inverted sequent proofs. So formulas on the left
of $\vdash$  map to formulas signed with $T$ and formulas on the right map to
formulas signed with $F$, when moving from sequent-style to tree-style. In
classical logic by trees the $T$s and $F$s are eliminable, but seem to be
essential for nonclassical logics.

For many logics you can regard $X$, $Y$, etc as sets. But doing so will
automatically give you various structural rules you might want to reject.
So I think it's better to make them explicit. But, like Torkel notes in a
recent message, for some kinds of investigations you don't need this degree
of explicitness.

Algebraists will recognize the groupoid aspects of sequent systems.

We start derivations wth axioms of the form
$    A \vdash A$

Structural rules:

 from  $X \vdash Y$ infer  $A,X \vdash Y$            thinning left

 from  $X \vdash Y$ infer  $X \vdash Y,A $           thinning right

 from  $A,A,X \vdash Y$ infer  $A,X \vdash Y$        contraction left

 from  $X \vdash Y,A,A$ infer  $X \vdash Y,A $       contraction right

 from  $W,A,B,X \vdash Y$ infer $ W,B,A,X \vdash Y$  interchange left

 from  $X \vdash Y,A,B,Z$ infer $ X \vdash Y,B,A,Z$  interchange right

 from  $X \vdash W,A$  and  $A,Z \vdash Y$  infer  $X,Z \vdash W,Y$   cut

Operational rules:

 from  $X \vdash W,A$ and $B,Z \vdash Y$  infer $A\to B,X,Z \vdash W,Y$    $\to$ left

 from $ A,X \vdash Y,B$  infer  $X \vdash Y,A\to B$                  $\to$ right

I'll skip the other rules. You can distinguish intuitionistic logic from
classical by the number of formulas allowed on the right of \vdash (I'll let
you figure it out yourself).

Here's a proof for  $p\to (p\to q) \vdash p\to q$

\begin{enumerate}

\item  $p \vdash p$              Axiom
\item  $q \vdash q $             Axiom
\item  $p\to q,p \vdash q$         1,2, $\to$ left
\item  $p\to (p\to q),p,p \vdash q$  1,3, $\to$ left
\item  $p,p,p\to (p\to q) \vdash q $ 4, interchange left (twice)
\item  $p,p\to (p\to q) \vdash q$    5, contraction left
\item  $p\to (p\to q) \vdash p\to q$   6, $\to$ right
\begin{enumerate}
which shows how contraction as a structural rule underlies the natural
deduction rule. One more step gives us

8. $\vdash [p\to (p\to q)]\to (p\to q)$  7, $\to$ right

Okay, (given that I've typed everything in properly!), what does this tell
us about the normalization business? I'm not at all sure. I've been doing
this exercise in order to figure out where contraction shows up in the
sequent system proofs of the paradoxical sentences. It seems to me that
sometimes natural deduction doesn't handle multiple uses of one premise
well. But I want to think about Peter Milne's remark about choice of rules
and also chew over Tennant's article.

"To seek knowledge one must prefer uncertainty" -- the first Bayesian koan.



From phil-logic@bucknell.edu Fri Apr 11 09:39:17 1997
From: Torkel Franzen <torkel@sm.luth.se>


  Graham says:

  >For many logics you can regard X, Y, etc as sets. But doing so will
  >automatically give you various structural rules you might want to reject.
  >So I think it's better to make them explicit. But, like Torkel notes in a
  >recent message, for some kinds of investigations you don't need this degree
  >of explicitness.

  Although it isn't at all relevant to the question about the proof
of $-(A\bic-A)$, I would like to add that the degree of explicitness embodied
in the rule I mentioned, i.e.


                  $[A\to B],\Gamma \vdash A      B,\Gamma \vdash C
                  -----------------------------------
                         A\to B, Gamma \vdash C$

lies in between treating Gamma etc as sets and the full use of
structural rules. B,Gamma is not a set in the rule above, but a
multi-set. We can only use a formula on the left of a sequent as many
times, *in any one branch of the proof*, as it has occurrences. In
classical propositional logic, we we need never use any formula more
than once in any one branch. In intuitionistic logic, reuse of
$A \to B$ $n$ times is sometimes necessary.

From phil-logic@bucknell.edu Sat Apr 12 01:56:54 1997
From: g.solomon@phil.canterbury.ac.nz (Graham Solomon)

A small remark about Lemmon's natural deduction system. Peter Milne gave it
as an example of a system with a case where an assumption is used only once
but the proof can't be normalized.

Lemmon's system doesn't allow us to infer directly from $B$ to $A\to
B$. We need instead to do something along the following lines: assume
$A$ and $B$ and do $\wedge$-introduction, then eliminate for $B$, and
on that basis infer $A\to B$. The assumption $A$ is used only once but
the proof isn't normalizable. Let's look at the sequent calculus proof
(with the original Gentzen rules).

\begin{enumerate}
\item  $B \vdash B$      Axiom
\item  $A,B \vdash B$    1, Thinning left
\item  $B \vdash A\to B$   2, $\to$ left
\end{enumerate}
Not much to it. Contraction isn't needed, so it isn't the case that
nonnormalisability must have something to do with contraction. So
what's Lemmon doing? He must be admitting non-normalisable proofs
instead of using thinning as a structural rule. Indeed, John Slaney,
in his reconstruction of Lemmon's system as a sequent system,
explicitly draws the connection between non-normalisability and the
absence of thinning as a primitive rule ("A General Logic" AJP 68
(1990):74-88).



From phil-logic@bucknell.edu Mon Apr 21 14:21:43 1997
From: IrvAnellis@aol.com

In 1985, Alexander Abian proposed the following expression:

(1)  for all x, A is an element of x iff x is not an element of x

and the equivalent expression:

(2)   for all x, A is not an element of x iff x is an element of x

By unrestricted universal instantiation, we get 

(1') A is an element of A iff A is not an element of A

and

(2') A is not an element of A iff A is an element of A.

Looking at (1), we see that A can be neither a set nor a class because
replacing x by the empty set in (1), we get

(1'') A is an element of the empty set iff the empty set is not an element of
the empty set.


and of course ``A is an element of the empty set'' is always false --
whether A is a class or a set -- and ``the empty set is not an element
of the empty set" is of course always true, so that we have

(1''') False iff True

which Abian regards as a paradox.

Whether (1) -- or for that matter (1''') -- is a paradox or a simple
contradiction will probably depend upon one's outlook. G. E. Mints pointed
out, however, that the so-called Abian paradox has the same structure as
Curry's paradox. For his part, Abian sees the expression as indicating that
neither sets nor classes should be formulated in terms of arbitrary
unrestricted properties, and that set theory requires some axioms for
prescribing some rules for formation of sets and classes.

Irving H. Anellis

From malitzi@logic-handle.com Mon Aug 11 19:54:43 1997
Received: by emu.dpmms.cam.ac.uk (UK-Smail 3.1.25.1/1); Mon, 11 Aug 97 19:54 BST
Received: from ISAAC by mail.pronex.com (NTMail 3.01.03) id ua019338; Mon, 11 Aug 1997 12:02:24 -0700
X-Sender: malitzi@mail.pronex.com (Unverified)
X-Mailer: Windows Eudora Pro Version 2.1.2
Mime-Version: 1.0
Content-Type: text/plain; charset="us-ascii"
To: t.forster@pmms.cam.ac.uk
From: Isaac Malitz <malitzi@logic-handle.com>
Subject: Axiom of pseudofoundation
Date: Mon, 11 Aug 1997 12:02:24 -0700
Message-Id: <19022490000576@mail.pronex.com>
Status: RO

This is in response to an issue raised in your talk at the NF conference.

You were looking for some kind of ``axiom of foundation" suitable for NF.

In what follows, I will describe two axioms of pseudofoundation; I suspect
that the second one is suitable.

Both of these axioms are characterized by means of games. The first one will
look familiar, the second one is a variation on the first.

*****************************************************************************
1. Extensionality Game #1

"All there is to know about a set is its members."

This game is played by two players, Eve and Adam. The game has potentially
an infinite number of stages STAGE0, STAGE1, ... 

STAGE0 begins with two distinct sets. The objective of Eve is to
"demonstrate" that these two sets are distinct (in a finite number of
stages). The objective of Adam is frustrate Eve's efforts by causing the
game to go on without end.

The game is played as follows: At each stage,there are two sets (the
"sets-for-that-stage"). At each stage, Eve picks a member of one of the two
sets-for-that-stage; the set picked by Eve is known as ``Eve's set". Then
Adam picks a member of the other of the two sets-for-that-stage; the set
picked by Adam is known as ``Adam's set". It is required that Eve's set be a
member of exactly one of the two sets-for-that-stage; it is required that
Adam's set be distinct from Eve's set. 

The game begins with two distinct sets at STAGE0. If the game reaches a
stage where Adam is unable to respond, then Eve wins. If the game goes on
forever, then Adam wins.

(Intuitively: At each stage, Eve is saying ``I can demonstrate that the two
sets-for-this-stage are distinct. Specifically, I am picking a set EVEn that
is a member of one but not the other". Adam responds ``Well, there is a set
ADAMn that is a member of the other set-for-this-stage; demonstrate to me
that EVEn and ADAMn are distinct")

COMMENTS: If the game begins with two distinct well-founded sets, then Eve
wins. If the game begins with two distinct non-well-founded sets, Eve can
still win, provided that there are appropriate distinct well-founded sets
embedded in the respective epsilon trees. 

If the game begins with $V$ and $V-\{V\}$, Adam wins: At each stage, Eve is forced
to select V; a winning strategy for Adam is to select V-{V} at each stage.
Intuitively, there should be a variation of Extensionality Game 1 that
allows Eve to win in this circumstance.


*****************************************************************************

2. Extensionality Game 2

"Two sets can be distinguished by means of different members *or* different
non-members".

Game is same as Extensionality Game #1, except that at each stage, Eve may
(optionally) pick a set which is *not* a member of (exactly) one of the two
sets-for-that-stage. If Eve does this, then Adam is required to pick a
non-member of the other set-for-that-stage which is distinct from Eve's set.

COMMENT: If the game begins with V and V-{V}, then Eve wins: At STAGE0, she
picks a non-memb









\section{A Question of Alice Vidrine}

 As a {\sl bonne bouche} i offer this solution to a question of Alice
 Vidrine that cropped up in connection with the above: {\sl Can there
 be an infinite set of pairwise disjoint sets that has, up to finite
 difference, precisely one transversal?}  The answer is yes.

 The construction of an example will be done in KF, or rather a version
 of KF + not-AC, in which we assume that there is an infinite set of
 pairwise disjoint sets, no infinite subset of which has a transversal.
 I do not know offhand of any construction of such a model, but i
 imagine that there is a standard FM construction that effects it.

\begin{rem}
 KF + ``there is an infinite set of pairwise disjoint sets, no
 infinite subset of which has a transversal'' proves that there is an
 infinite set of pairwise disjoint sets that has, up to finite
 difference, precisely one transversal. \end{rem}

\Proof Let $\{X_i: i \in I\}$ be an infinite family of pairwise
disjoint sets s.t. for no infinite $I' \subseteq I$ does $\{X_i: i \in
I'\}$ have a transversal. Now make a copy $\{\iota``X_i: i \in I\}$ of
$\{X_i: i \in I\}$, and add one element to each $\iota``X_i$ to obtain
$\{\iota``X_i \cup \{X_i\}: i \in I\}$.

Observe that this new family, unlike $\{X_i: i \in I\}$, does actually
have a transversal, namely $\{\{X_i\}: i \in I\}$.  Observe further
that this transversal is unique up to finite difference.

\endproof

One could use $\{X_i \cup \{X_i\}: i \in I\}$ to the same effect, but
i wanted something that was stratified all the way and that therefore
worked in KF.

\bigskip

I think this is a better version of Alice's question:

``Can there be a family of pairwise disjoint sets that has a countable
infinity of transversals?''

I suspect this is equivalent to the question:

``Can there be a countably infinite profinite structure?''


\end{document}
