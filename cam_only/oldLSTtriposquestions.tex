\documentclass{book}
 \usepackage{amssymb}
 \usepackage{amsbsy}
 \usepackage{amsmath}
 \usepackage[english]{babel}
 \usepackage{bussproofs}
 \usepackage{boxedminipage}
 \usepackage{latexsym}
 \usepackage{graphicx}
 \usepackage{makeidx}
 \usepackage{lscape}
\usepackage{hyperref}
\usepackage{wasysym}
\usepackage{pifont}
\usepackage{latexsym}
\usepackage{graphicx}
\usepackage{pgf}
\usepackage{txfonts}%for \leftsquigarrow
\usepackage{tikz}
\usetikzlibrary{arrows,automata}
\usepackage[latin1]{inputenc}
\usepackage{dingbat}
\usepackage{dingbat}
\usepackage{lipsum}
\DeclareSymbolFont{symbolsC}{U}{txsyc}{m}{n}
\DeclareMathSymbol{\strictif}{\mathrel}{symbolsC}{74}
\input newlogicmacr
 \begin{document}
 \title{Discussion-answers of Old Tripos Questions}
 \author{Thomas Forster}
  \maketitle

% \tableofcontents




\chapter{1995}
\subsection*{95108} 

Let $P,Q,R$ be three pairwise-disjoint sets of primitive propositions;
for any set $S$, let ${\cal L}(S)$ denote the set of propositional
formulae whose primitive propositions are all in $S$. Let $s\in {\cal
  L}(P\cup Q)$ and $t\in{\cal L}(Q\cup R)$ be formul{\ae}\ such that
$(s\to t)$ is a theorem of the propositional calculus. Show that there
is a formula $u\in{\cal L}(Q)$ such that both $(s\to u)$ and $(u\to
t)$ are theorems. 

\subsubsection*{Discussion}

PTJ says: consider the set $U=\{u\in{\cal
    L}(Q):\ \vdash(s\to u)\}$. Suppose $U\cup\{\neg t\}$ is
  consistent: take a suitable valuation $v$ of $Q\cup R$, and show
  that the set $$\{s\}\cup\{q: q\in Q, v(q)=1\}\cup\{\neg q: q\in Q,
  v(q)=0\}$$ is consistent. Deduce that $v$ can be extended to a
  valuation of $P\cup Q\cup R$ making $s$ true and $t$ false, and
  obtain a contradiction.

\medskip

Boo to that, I say. (This is Craig's Interpolation Theorem).  There 
is a detailed constructive proof in {\sl Logic, Induction and Sets} 
pp 91--92.

(Here is another proof---that i nicked from Sam Buss. 

Think of the literals (and let there be $k$ of them) from
$Q$ that occur in $s$, and of the valuations of those $k$ literals
which can be extended to valuations that make $s$ true. Let 
$v_1 \ldots v_n$ be those valuations.  Now let $u$ be
the\marginpar{This might need relettering} disjunction 
$$\bigvee_{1 \leq i \leq n}({q_1}^{(i)} \wedge {q_2}^{(i)} \ldots {q_k}^{(i)})$$
where ${q_j}^{(i)}$ is $q_j$ or $\neg q_j$ depending on whether or not
$v_i$ believes $q_j$. $s \vdash u$ is immediate. To show $u
\vdash t$ notice that a valuation on $Q$ that satisfies $u$ can be
extended to a valuation on $P \cup Q$ that satisfies $s$.  Any such
valuation also satisfies $t$ since $s \vdash t$.  Hence $u \vdash t$
as desired.)



\chapter{2002}
\subsection*{2002 IIA paper 1 Q 7J}\label{implication}

\url{http://www.maths.cam.ac.uk/undergrad/pastpapers/2002/Part_2/index.html}



  The only part i'm going to answer is (ii)(b), since the earlier parts
 are bookwork.  
 
 I have to confess that I don't know what the subtext of this question
 is: usually I know this kind of thing but I'm not omniscient.\footnote{I 
now (2012) realise it's all about how to recover a Heyting algebra from a 
possible world model. (At least, that's what i tho'rt! PTJ says: ``As far 
as I was concerned, it was actually about constructing the subobject 
classifier in a topos of sheaves.'' Well!!--he set the question and he should know)} 
I'm going to have to fall back on appealing to background knowledge and
 sniffing around suspicious features.  So what do I spot?
 
 The first thing that catches my eye is the use of the symbol `$\Rightarrow$'
 for this operation on sets.  ``Might it'', one wonders, ``be anything to
 do with implication?''  Can we prove the inference
\begin{equation}\label{three}
\frac{T \subseteq B; \ \ \ \ \ \ \  T \subseteq (B \Rightarrow C)}{T \subseteq C}
\end{equation}?

 You will find it easy to prove this once your suspicions have been aroused.
 
 The second unfamiliar---and therefore eyecatching---part of this
 question is the condition `$(\forall t \in T)(\exists s \in S)(t \leq
 s)$' in clause (ii) of the definition.  In the literature on
 quasiorders this is sometimes written `$T \leq^+ S$''.  I shall use
 it here, as it saves space: once you spot that this `+' is useful you
 can prove some trivial but useful facts like:

\begin{equation}\label{two}
T \leq^+ S\ \wedge\ S \subseteq (B \Rightarrow C) \ \to\ \ T \subseteq (B \Rightarrow C)
\end{equation}

Now that we are armed with these I think it is safe to approach the
question.  A work of warning before we jump in: overloading of `$\to$'
for implication and the operation on sets defined in the question:
read formul{\ae} carefully!   In fact, i think i'll doctor this file 
by using `$\Rightarrow$' for the set operation.  It's obviously got 
something to do with Heyting algebras, after all.

\smallskip
 
To show that $B\Rightarrow  C$ is $R$-closed whenever $B$ and $C$ are we 
have to show that if $S \subseteq (B \Rightarrow C)$ and $\tuple{S,a} \in R$ 
then $a \in (B \Rightarrow C)$.

\smallskip

So let $S \subseteq (B \Rightarrow C)$ and $a$ s.t. $\tuple{S,a} \in R$ be 
arbitrary.   We will show that $a \in (B \Rightarrow C)$.

\smallskip

$S \subseteq B \Rightarrow C$ is 

$(\forall a \in S)(\forall b \leq a)(b \in B \to b \in C)$

We have assumed  $\tuple{S,a} \in R$.  Then we have to show that 
$a \in (B \Rightarrow C)$.  To do this we have to show 
$$(\forall b \leq a)(b \in B \to B \in C)$$

\begin{equation}\label{one}
(\forall b \leq a)(\exists T \subseteq A)(\tuple{T,b} \in R\ \wedge\ T \leq^+ S\ \wedge\ (\forall t \in t)(\tuple{\{b\},t} \in R))\end{equation}


Now suppose $b \leq a$.  By formula \ref{one} we have
$$(\exists T \subseteq A)(\tuple{T,b} \in R\ \wedge\ T \leq^+ S\ \wedge\
(\forall t \in t)(\tuple{\{b\},t} \in R))$$

Assume further that $b \in B$; we want $b \in C$.  We know that $b \in
T$ and that $T \leq^+ S$ from formula \ref{one}; formula \ref{two}
tells us that $T \subseteq (B \Rightarrow C)$, and then formula \ref{three}
tells us that $T \subseteq C$.  But then $b \in C$ as desired.  But
$b$ was arbitrary.  This establishes $(\forall b \leq a)(b \in B \to B
\in C)$.  This is precisely the condition for $a$ to be in $B \Rightarrow C$.

\chapter{2006}

\subsection*{Paper 4 Question}

In the first part they tell you to take your wellfounded relation $r$
and define on it by recursion the obvious rank function to the
ordinals.  This refines the wellfounded relation to a {\bf
  prewellorder} (a wellordered partition). That was a bit cryptic, so
let me be clearer. You have defined a rank function $\rho$ from the
domain of $r$ to the ordinals.  This process partitions the domain
into pieces $\rho^{-1}``\{\alpha\}$ for each ordinal $\alpha$. The
relation $\{\tuple{x,y}: \rho(x)< \rho(y)\}$ is wellfounded and
contains every ordered pair in $r$.  It's not a wellorder, beco's two
things in the same piece are not related to each other, tho' it is
what we call a {\sl prewellorder}.  AC tells us that everything can
wordered, so in particular each piece of the partition can be
wellordered, and we use AC to pick such a wellorder for each
piece. Then we concatenate the wellorderings

\smallskip

That's the way the examiners expect you to do it.  However there is
the clever way.  Consider the collection of wellorders that are
compatible with $r$.  Partially order them by {\bf end-extension}.
Then use Zorn.  The point is that a union of a chain {\bf under
  end-extension} of wellorderings is another wellordering. (Look at
question {2009-3-16G} below where end-extensions reappear.)

\chapter{2007}
\subsection*{2007-3-16G}

Observe that if $\gamma \geq \omega \cdot \beta$ then $\beta + \gamma
= \gamma$.  Contraposing we infer that if $\gamma$ does {\sl not}
``absorb $\beta$ on the left'' then $\neg(\gamma \geq \omega \cdot
\beta)$ whence $\gamma < \omega \cdot \beta$.  Analogously if $\beta$
does {\sl not} ``absorb $\gamma$ on the left'' then $\beta < \omega
\cdot \gamma$.  So, if $\beta$ and $\gamma$ are {\sl commensurable}
(never heard this word used this way, but never mind) we have
\begin{quote}
$\gamma < \omega \cdot \beta$ and $\beta< \omega \cdot \gamma$.\end{quote}

Now we use the division algorithm to find the largest power of $\omega$ 
$\leq$ $\beta$.

\subsection*{2007-4-II-16G}

We say (for the purposes of this question) that a function $f:A \to
{\cal P}(A)$ is {\bf recursive} if the relation $\{\tuple{a,b}: a \in
f(b)\}$ is wellfounded.  Observe that any binary relation on a set $A$
can be thought of as a map $f:A \to {\cal P}(A)$.  

(That is the first of the two points made by this question. The other
point is that not only is wellfoundedness of a relation $R$ a
sufficient condition for definitions given by recursion-over-$R$ to
have unique solutions, it is also necessary.  If every definition
given by recursion over $R$ has a unique solution then $R$ is indeed
wellfounded.  It is disconcerting---even if only slightly---that you
are being invited to prove this converse while thinking of binary
relations as functions-into-power-sets [as above].)

Suppose $g:{\cal P}(B) \to B$.  We can attempt to define 
a function $h$ recursively by:
$$h(a) := g(\{h(a'): a' \in f(a)\}).$$

Clearly we are going to be able to show (by an appeal to the recursion
theorem) that this recursion has a unique solution---as long as the
relation $\{\tuple{a,b}: a \in f(b)\}$ is wellfounded.  But what about
a converse?

Suppose $\{\tuple{a,b}: a \in f(b)\}$ is not wellfounded. We want to
find a $B$ and $g:{\cal P}(B) \to B$ such that there is more than one
$h$ satisfying

$$(\forall a \in A)(h(a) = g(\{h(a'): a' \in f(a)\})).$$

Let $B$ be a set with at least two members, and $b_1$ and $b_2$ be two
members of $B$, and define $g:{\cal P}(B) \to B$ by \begin{quote}
  $g(B') = $\verb# if# $(B' = \emptyset \vee B' = \{b_1\})$ \verb#then#
  $b_1$ \verb#else# $b_2$\end{quote}

Suppose now that $A'$ is a subset of $A$ with no minimal member under the
relation $\{\tuple{a,b}: a \in f(b)\}$.  Notice that both
\begin{quote}
$h_1(a) := b_1$\end{quote}
and
\begin{quote}
  $h_2(a) := $ \verb#if# $a \in A'$ \verb#then# $b_2$ \verb#else# $b_1$
\end{quote}

are solutions to 

$$h(a) = g(\{h(a'): a' \in f(a)\}).$$


\chapter{2008}
\subsection*{2008-3-16G}
Obviously the rank of ${\cal P}(x)$ has to be one greater than the
rank of $x$.  The rank function is $\subseteq$-monotone ($u \subseteq
v \to \rho(u) \leq \rho(v)$) so the rank of $x$ is maximal among ranks
of members of ${\cal P}(x)$.  So $\rho({\cal P}(x))$ must be $\rho(x)
+ 1$, so it is successor.

\medskip

The fun starts when you ask about the ordinals that can be ranks of
${\cal P}_{\aleph_0}(x)$ (the set of finite subsets of $x$)
or ${\cal P}_{\aleph_1}(x)$ (the set of countable subsets of $x$).
Either of those could be a fair tripos question, i think\ldots

\chapter{2009}
\subsection*{2009-2-16G}

Suppose there were an axiomatisation $T$ of the theory of groups all
of whose elements have finite order. Add, for each $n$, the axiom
$(\forall x)(x^n = 1 \to x = 1)$.  This theory has no models, but all
its finite fragments do.

\subsection*{2009-3-16G}\label{2009-3-16G}

Let $x \subseteq y \subseteq$ the ordinals below $\alpha$.  $x$ and
$y$ both inherit an ordering.  Why do we know that the length of $x$
in the inherited ordering is $\leq$ the length of $y$?  It looks
obvious, but we have to be careful.  The ordering on ordinals arises
from ``$A$ is iso to an an initial segment of $B$'' rather than from
``$A$ injects into $B$ in an order preserving fashion'', and it is the
second one that we want here.  However these two relations are the
same!  This was an example sheet question in the year this question
was set, so i'll provide a proof:

\begin{rem}
A total order is a well order iff every subordering is iso to an
initial segment.\end{rem}

\Proof Let $\tuple{A,<_A}$ be a total order where every subordering 
is iso to an initial segment.  Consider a one-point subordering.  It 
is iso to an initial segment so $\tuple{A,<_A}$ has a bottom element.  
But now every subordering is iso to an initial segment, and so has a 
least element, so $\tuple{A,<_A}$ is a well order.

For the other direction, every suborder of a well order is a well order.
So, if $\tuple{A,<_A}$ is a well order and $\tuple{A',<_A}$ a suborder of
it, then one of $\tuple{A,<_A}$ and $\tuple{A',<_A}$ is iso to an initial
segment of the other.  (The two might be iso of course).  We must establish
that $\tuple{A,<_A}$ cannot be iso to a proper initial
segment of $\tuple{A',<_A}$.  But this is easy: such an isomorphism
would be a $<_A$-decreasing injective function $A \to A$, so that
there will be sets $\{f^n(a): n \in \Nn\} \subseteq A$ with no
$<_A$-least member, contradicting wellfoundedness of $<_A$. \endproof
\label{original}

\medskip

The second paragraph requires care.  Observe that, at any rate, the
length of the union of the $x_n$ must be at least the sup of the
lengths; that much is obvious.  It's the other direction that can
fail.

The operation that sends each $x \subseteq \alpha$ to an ordinal
$\mu(x)$ does its work by sending each member of $x$ to an ordinal;
$\mu(x)$ is then the least ordinal not used.  A member $a$ of $\alpha$
can appear in lots of $x$'s. {\bf The point of the end-extension condition
is that it ensures that $a$ gets sent to the same ordinal whatever $x$
we are considering}; that ordinal is then the ordinal it is given in
the computation of $\mu(\bigcup_n x_n)$.  To illustrate the importance
of the end-extension condition consider the following family of
subsets of the ordinals below $\omega + 1$:

$$\{[0,n] \cup\{\omega\}: n < \omega\}$$

so that $x_n$ is the ordinals below $n$ with $\omega$ whacked on the
end.  Each subordering is of finite length, and the lengths are
unbounded so the sup of the lengths is $\omega$.  However the union of
the subsets is the set of ordinals below $\omega +1$ which of course
is of length $\omega + 1$.

Observe that in this case the ordinal that the point $\omega$ gets
sent to in the calculation of $\mu(x_n)$ is not constant but depends
on $n$.

It may be worth noting that {\sl any countable linear order type whatever}
can be obtained as a union of an $\omega$-chain of finite total orders.
Come to think of it that might make a good exercise for next year.


\smallskip

For the final part let $x_n$ be $\Nn \setminus [0,n]$.  Each $x_n$ is
a terminal segment of the finite ordinals and has length $\omega$, but
their intersection is empty.

\chapter{2010}
\subsection*{2010-2-16G}

Show that an ordinal is indecomposible iff it is a power of $\omega$.

\smallskip

One direction is easy: if $\alpha$ is not a power of $\omega$ then it
is decomposed by applying the division algorithm to the function
$\beta \mapsto \omega^\beta$, which give us the largest power of
$\omega \leq \alpha$.

\smallskip

The other direction is easy if $\beta + \gamma = \omega^\lambda$ with
$\lambda$ limit. Then $\beta$ and $\gamma$ are both $< \omega^\alpha$
with $\alpha < \lambda$.  But then
$$\beta+\gamma\leq\omega^\alpha\cdot 2< \omega^{\alpha+1} < \omega^\lambda$$

We are left with the successor case.  So suppose $\beta + \gamma =
\omega^{\alpha + 1} = \omega^\alpha \cdot \omega$.  

Use the division algorithm to obtain $\beta = \omega^\alpha \cdot \beta_1 + \beta_2$ with $\beta_2 < \omega^\alpha$ and

$\gamma = \omega^\alpha \cdot \gamma_1 + \gamma_2$ with $\gamma_2 < \omega^\alpha$.

Then $$\omega^\alpha \cdot \beta_1 + \beta_2 + \omega^\alpha \cdot \gamma_1 + \gamma_2 = \omega^\alpha \cdot \omega$$

Now $\beta_2 < \omega^\alpha$ so we can simplify
 
$$\omega^\alpha \cdot \beta_1 + \omega^\alpha \cdot \gamma_1 + \gamma_2 = \omega^\alpha \cdot \omega$$

Clearly $\gamma_2 = 0$ so we simplify again

$$\omega^\alpha \cdot \beta_1 + \omega^\alpha \cdot \gamma_1 = \omega^\alpha \cdot \omega$$

This tells us that $\beta_1 + \gamma_1 = \omega$ so one of them is $\omega$, from which the result follows.


\subsection*{2010-3-16G}

I do only the hard part

\medskip

Suppose $x$ is a transitive set, of rank $\alpha$.  We will show that,
for all $\beta < \alpha$, $x$ has a member of rank $\beta$.  That is
to say, $\{\beta\in\,On:x$ has a member of rank $\beta\}$ is a proper
initial segment of $On$.  It must be a set beco's it is bounded above
by the rank of $x$.

\smallskip

Observe the following:

\smallskip

(i) If $x$ has a member $y$ of rank $\beta +1$ then $y$ has a member
$y'$ of rank $\beta$.  But $x$ is transitive, so $y' \in x$ and $x$
has a member of rank $\beta$.

\smallskip

(ii) If $x$ has a member $y$ of rank $\lambda$ (where $\lambda$ is
limit) then $(\forall \beta < \lambda)(\exists \gamma < \lambda)(\beta
< \gamma \wedge y$ has a member of rank $\gamma)$.  But $x$ is
transitive, so the same goes for $x$: $(\forall \beta <
\lambda)(\exists \gamma < \lambda)(\beta < \gamma \wedge x$ has a
member of rank $\gamma)$.

\smallskip

(i) and (ii) together mean that the collection of ordinals $\beta$
such that $x$ does not have a member of rank $\beta$ is closed under
successor (beco's of (i)) and limits (beco's of (ii)), so it must be a
terminal segment of On.  So its complement must be a proper initial
segment of $On$.  But that is what we wanted.

\bigskip

The other thing one can try is to prove by $\in$-induction on $x$ that
$TC(x)$ has members of all ranks less than $\rho(x)$.  Suppose that,
for all $y \in x$, $TC(y)$ has members of all ranks less than
$\rho(y)$. $TC(x) = x \cup \bigcup_{y\in x} TC(y)$.  Now let $\alpha$
be an ordinal less than the rank of $x$.  Then $\alpha \leq \rho(y)$
for some $y \in x$.  If we have equality then $y$ is the thing we are
looking for that is in $TC(x)$ and has rank $\alpha$. If we have $<$
then we have our witness by induction hypothesis.

\chapter{2011}
\subsection*{2011 Paper 2 Q 16H}

The first two parts are bookwork.  For the third part i'm assuming
that each element of the space is a {\bf finite} sum of multiples of
basis-element--times--a-real.  There are $\aleph_1$ basis elements and
$2^{\aleph_0}$ reals, making $\aleph_1 \cdot 2^{\aleph_0}$ such
pairs. How many finite sets of such pairs?  The cardinality of ${\cal
  P}_{\aleph_0}(X)$ (the set of finite subsets of $X$) can be quite
hard to compute in terms of $|X|$, whereas $|X^{<\omega}|$ (the set of
finite sequences-without-repetitions of members of $X$) is just $|X| +
|X|^2 + |X|^3 \ldots$.  Evidently $X^{<\omega}$ surjects onto ${\cal
  P}_{\aleph_0}(X)$ and, since any set $X$ of size $\aleph_1 \cdot
2^{\aleph_0}$ admits a total ordering, we can use that total order to
uniformly order each finite set and thus inject ${\cal
  P}_{\aleph_0}(X)$ into $X^{< \omega}$.  Evidently (whatever $X$ is)
and---certainly in this case---each of the summands in $|X^{<\omega}|
= |X| + |X|^2 + |X|^3 \ldots$ is $|X|$ (which is to say $\aleph_1
\cdot 2^{\aleph_0}$) so (with the help of Cantor-Bernstein) we arrive
at $\aleph_1 \cdot 2^{\aleph_0}\cdot \aleph_0 = \aleph_1 \cdot
2^{\aleph_0}$ such finite sets, and therefore $\aleph_1 \cdot
2^{\aleph_0}$ vectors.  

All that was without any use of AC.  If we allow ourselves even mere
countable choice we can infer $\aleph_1 \leq2^{\aleph_0}$, so there
are $2^{\aleph_0}$ such pairs and $2^{\aleph_0}$ finite sets of such
pairs, so there are $2^{\aleph_0}$ vectors.

\smallskip

That seems like a lot of work, and maybe the examiners didn't expect
that much detail (you have only about half an hour per question after
all) but it can't do you any harm to see it done properly.  My guess
is that you'd've got the brownie points simply for getting the correct
answer and waving your hands artistically.

\smallskip

If an element of the space can be an {\bf arbitrary} sum of multiples
of basis-element--times--a-real then the sums are subtly different.
There are still $2^{\aleph_0}$ pairs, but now we are interested in
subsets of the collection of pairs of sizes up to $\aleph_1$, which
gives us $(2^{\aleph_0})^{\aleph_1}$ elements.  Now
$(2^{\aleph_0})^{\aleph_1} = 2^{\aleph_0 \cdot \aleph_1}$.  Now
$\aleph_1 \leq \aleph_0 \cdot \aleph_1 \leq \aleph_1 \cdot \aleph_1 =
\aleph_1$, so $(2^{\aleph_0})^{\aleph_1} = 2^{\aleph_1}$, which won't
simplify any further (in case you were wondering).  The continuum
hypothesis implies that $2^{\aleph_1} = 2^{2^{\aleph_0}}$, and Luzin's
hypothesis [don't ask] is that $2^{\aleph_1} = 2^{\aleph_0}$.  Both
are consistent with ZFC---tho' of course not jointly!

\subsection*{2011 Paper 3 Q 16H}
\ \ \ \ (i)  The theory of dense linear orderings is axiomatisable, and the axioms are what you think they are.

\medskip

(ii) The theory of countable dense linear orderings is not
axiomatisable in the language of posets.  For suppose it were, and $T$
were an axiomatisation.  We could expand the language of $T$ by adding
uncountably many constants, and add to $T$ axioms to say that the
constants all point to distinct things.  That would be a theory with
uncountable models---all of which would be models of $T$---contradicting 
the assumption that all models of $T$ are countable. Note that there is 
a first order theory which is true in each and every countable DLO; the 
problem is that it has uncountable models as well.

\medskip

(ii) The theory of uncountable dense linear orderings is axiomatisable
{\sl if you are allowed to add uncountably many constants}.  (If the
axiom of choice holds then every uncountable set has a subset of size
$\aleph_1$, so we add $\aleph_1$ constants.)  However, that takes you
out of the language of posets, so it's not allowed; downward
Skolemheim tells us that any theory of uncountable dense linear orders
{\sl in the language of posets, which is countable} will have
countable models.

\medskip

(iv)  There cannot be a first-order theory $T$ of wellorderings in the
language of posets beco's if there were one could add constants
$\tuple{a_i: i \in \Nn}$ to the language, and axioms $a_i > a_{i+1}$
for each $i \in \Nn$.  The result would be a theory all of whose models are
wellorderings (since they are models of $T$) while not being
wellorderings after all beco's of the $a_i$.

\smallskip 

Here is another proof\ldots one that you could conceivably be expected to find
given the material you have been lectured, at least in a PTJ year \ldots 

Every substructure of a wellorder is a wellorder.  So \commentblock{, by the
discussion on p \pageref{universaltheory},} the first-order theory of
wellorders (if there is one) is a universal theory. Now if $T$ is a
universal theory then---by part (iv) of sheet 3 q10 2015 (for
 example!)---the union of a $\subseteq$-chain of models of $T$
is another model of $T$; so a union of a $\subseteq$-chain of
wellorderings would have to be a wellordering.  But---for
example---every terminal segment of $\Z$ is a wellorder and the union
of all of them is not. (A union of a chain of wellorderings is a
wellordering if it is a chain under end-extension.)

\smallskip

\ldots but that looks quite hard to me.

\subsection*{Paper 4 question}

This is mostly bookwork.  To prove that every set belongs to a
$V_\alpha$ you are clearly going to have to use the foundation axiom
somehow, since if foundation fails you can have silly things like
Quine atoms that do not belong to any $V_\alpha$.  The usual way to
use foundation is to do an $\in$-induction.  This is a case in point:
you prove by $\in$-induction that every set belongs to a $V_\alpha$.
Suppose this is true for every member $y$ of $x$.  To each such $y$
associate $\alpha_y$, the least $\alpha$ s.t. $y \in V_\alpha$. We
form the set of those $\alpha_y$ and take the sup.  Probably worth
pointing out that in so doing we are using the axiom scheme of
replacement.

\chapter{2012}
\subsection*{2012 paper 3 Q 16H}

All bookwork.  Part (v) looks scary but it's actually easy.  Let
$\alpha$ be any ordinal at all.  Use the division algorithm on the
function $\beta \mapsto \omega_1\cdot \beta$ to obtain the largest
multiple of $\omega_1$ that is less than or equal to $\alpha$.  If
that number is $\alpha$ itself then $\alpha$ is a multiple of
$\omega_1$; if it isn't then there is a remainder.  This remainder is
countable and so has cofinality $\omega$ (if it is limit) or is
successor (in which case there is a cofinal map within the meaning of
the act.).


\chapter{2013}
\subsection*{2013-2-16G}\label{2013-2-16G}

{\sl Explain what is meant by a {\sl chain-complete} poset. State the
Bourbaki--Witt fixed-point theorem for such posets.

A poset $P$ is called {\sl directed} if every finite subset of $P$
(including the empty subset) has an upper bound in $P$; $P$ is called
{\sl directed-complete} if every subset of $P$ which is directed (in the
induced ordering) has a least upper bound in $P$ . Show that the set
of all chains in an arbitrary poset $P$ , ordered by inclusion, is
directed-complete.

Given a poset $P$, let $[P \to P]$ denote the set of all
order-preserving maps $P \to P$ , ordered pointwise (i.e. $f \leq g$
if and only if $f (x) \leq g(x)$ for all $x$). Show that $[P \to P]$ is
directed-complete if $P$ is.

Now suppose $P$ is directed-complete, and that $f : P \to P$ is
order-preserving and inflationary. Show that there is a unique
smallest set $C \subseteq P \to P$ satisfying
\begin{quote}
(a) $f \in C$;\\
(b) $C$ is closed under composition (i.e. $g, h \in C \to g \cdot h \in C$); and\\
(c) $C$ is closed under joins of directed subsets.\end{quote}

Show that
\begin{quote}
\begin{tabbing}
(i)\ \ \ \ \= all maps in $C$ are inflationary;\\
(ii) \> $C$ is directed;\\
(iii) \> if $g = \bigvee C$, then all values of $g$ are fixed points of $f$ ;\\
(iv) \>for every $x \in  P$ , there exists $y \in P$ with $x\leq y = f(y)$.\end{tabbing}\end{quote}
}
\subsubsection*{Discussion}

`Directed' is an important notion.  Cast your mind back to the proof
that the class of models for a $\forall^*\exists^*$ (``inductive'')
theory is closed under unions of chains.  Look carefully at that proof.
You will notice that you didn't really need chains: if $T$ is
inductive then a union of a directed set of models of $T$ is another
model of $T$.

We have to show that the chains of an arbitrary poset form a
directed-complete poset.  Not a {\sl directed} poset!  That's clearly
not going to happen unless the poset is a toset!  If i have a directed
family $F$ of chains in a poset $\tuple{P, \leq_P}$ then that family
has a sup, namely $\bigcup F$.  We have to check that $\bigcup F$ is a
chain.  Suppose $x$ and $y$ are both elements of $\bigcup F$.  They
belong to two chains $c_x$ and $c_y$, but---by directedness---there is
now a chain $\supseteq c_x \cup c_y$ to which both $x$ and $y$ belong.
So $x$ and $y$ are $\leq_P$-comparable.

\smallskip

The set $C$ we want is defined inductively: it's the intersection of
the collection ${\cal C}$ of all the subsets of $P \to P$ satisfying
conditions (a--c). ${\cal C}$ is a subset of ${\cal P}(P \to P)$ so it
is a set, and it is nonempty, since $P \to P$ itself is such a subset. 
We need also to check that conditions (a--c) are {\sl intersection-closed} 
(that is to say, an arbitrary intersection of sets satisfying (a--c) also 
satisfies (a--c)) but that is evident, beco's (a--c) are {\sl closedness} 
properties.  So the intersection $\bigcap{\cal C}$ of all such subsets 
satisfies (a--c) and is the set $C$ that we want.

Clearly $C$ is going to be the family of all iterates of $f$.  

\bigskip

(i) Since $C$ has an inductive definition it supports induction.  So
we should expect to be able to prove (i) by induction.  If the set of
all inflationary maps $P \to P$ is a member of ${\cal C}$ then we are
home and hosed.  Just check that it satisfies (a--c).  It satisfies
(a) beco's $f$ is inflationary, it satisfies (b) beco's a composition
of two inflationary functions is inflationary, and it satisfies (c),
since the [pointwise] sup of a lot of inflationary maps is
inflationary.

  \medskip

(ii) Suppose $D \subseteq [P \to P]$ is directed.  Define $h: P \to P$
by $h(x) =\bigvee\{g(x): g \in D\}$.  For this definition to succeed we need
$\{g(x): g \in D\}$ to be directed.  So let $U$ be a finite subset of
$\{g(x): g \in D\}$.  We want $U$ to have an upper bound.  Now $U$ is
$\{g(x): g \in D'\}$ for some finite $D' \subseteq D$.  But this $D'$
has an upper bound by the assumption that $D$ is directed.

\medskip

(iii) We have to show that every value of $\bigvee C$ is a fixed point
for $f$.  For $p \in P$, $\bigvee C(p)$ is the sup of $f(p)$ for all
$f \in C$.  We must compute $f(\bigvee C(p))$ for arbitrary $p \in P$
and hope that it evaluates to $\bigvee C(p)$.


What is $\bigvee C$ applied to $p$?  It's $\bigvee$ of the set $\{g(p):
g\in C\}$ of iterates of $f$ applied to $p$.  Now everything in $C$ is
inflationary (that was (i)) so $\bigvee\{g(p): g\in C\} \leq_P \bigvee\{f(g(p)):
g\in C\}$.  The other direction, namely $\bigvee\{f(g(p)): g\in C\} \leq_P
\bigvee\{g(p): g\in C\}$,  we get beco's $\{f(g(p)): g\in C\} \subseteq
\{g(p): g\in C\}$, whence
\begin{equation}\tag{A}
\bigvee\{g(p): g\in C\} =  \bigvee\{f(g(p)): g\in C\}\end{equation}
At this point we have to use the fact (unused so far, i think) that $f$ is order-preserving, to infer
\marginpar{I don't think this works \ldots oops}
$$f(\bigvee\{g(p): g\in C\}) \leq  \bigvee\{f(g(p)): g\in C\}.$$

Substituting using the equation (A) we obtain

$$f(\bigvee\{g(p): g\in C\}) \leq  \bigvee\{g(p): g\in C\}.$$

The equality in the other direction

$$\bigvee\{g(p): g\in C\} \leq f(\bigvee\{g(p): g\in C\})$$

follows from $f$ being inflationary, so we infer

$$\bigvee\{g(p): g\in C\} = f(\bigvee\{g(p): g\in C\})$$

so $\bigvee C(p)$ is a fixed point for $f$ as desired.

\medskip

(iv) This challenge reminds me very strongly of the fact that one can
infer ``In every chain complete poset every element has a maximal
element above it'' from ``Every chain-complete poset has a maximal
element''.  That hint was enough for me to know what to do, and it
should work for you too.  So you run the preceding constructions on $P
\uparrow x$, the substructure of $\tuple{P, \leq_P}$ that is its
restriction to the elements that are $\geq_P x$: replace `$f$' by
`$f\restric (P \uparrow x)$'.

\subsection*{2013-3-16G}




{\sl Suppose $P$ , $Q$ and $R$ are pairwise disjoint sets of primitive
  propositions, and let $\phi\in L(P \cup Q)$ and $\psi \in L(Q \cup
  R)$ be propositional formulae such that $\phi \to \psi$ is a theorem
  of the propositional calculus. Consider the set $X = \{\chi \in L(Q)
    : \phi \vdash \chi\}$ .  Show that $X \cup \{\neg \psi\}$ is
    inconsistent, and deduce that there exists $\chi \in L(Q)$ such
    that both $\phi \to \chi$ and $\chi \to \psi$ are
    theorems.\begin{small} [Hint: assuming $X \cup \{\neg \psi\}$ is
      consistent, take a suitable valuation $v$ of $Q \cup R$ and show
      that $\{\phi\} \cup \{q \in Q | v(q) = 1\} \cup \{\neg q : q \in
    Q, v(q) = 0\}$ is inconsistent. The Deduction Theorem may be
    assumed without proof.]\end{small}}


\subsubsection*{Discussion}

Take the hint. If $X \cup \{\neg \psi\}$ is {\sl not} inconsistent,
then there is a valuation (call it `$v$') making $X$ true and $\psi$
false.  This valuation $v$ is defined on letters in $Q \cup R$. The
plan now is to extend $v$ to a valuation $v'$ defined on all the
letters in $P$ as well, so that $v'$ makes $\phi$ true too.  Now every
valuation making $\phi$ true makes $\psi$ true, by assumption; so $X
\cup \{\neg \psi\}$ will be inconsistent as desired.  So how do we
extend $v$ to a $v'$ making $\phi$ true?  $\phi$ has propositional
letters in it that come from $P$ and from $Q$. Doctor $\phi$ by
setting to $\top$ every $q$-letter that $v$ makes true, and setting to
$\bot$ every $q$-letter that $v$ makes false, and then simplify (using
rewrites such as $\alpha \wedge \bot \mapsto \bot$, $\alpha \vee \top
\mapsto \top$, that sort of thing) obtaining a formula in $L(P)$. If
this formula hasn't simplified to $\bot$ then it has some $p$-letters
in it and we can find a valuation making it true, and then we are
happy.  But what if we can't?  Can that happen?  That would mean that
the stuff in $X$ prevented $\phi$ being true. There may be stuff that
prevents $\phi$ being true, but the stuff in $X$ consists of {\bf
  consequences} of $\phi$.  So $\phi$ is prevented from being true by
some of its consequences.  But that must mean that $\phi$ is the
negation of a truth-table tautology and that case isn't interesting.

\subsection*{2013-4-16G}



{\sl State the Axiom of Foundation and the Principle of $\in$-Induction,
and show that they are equivalent in the presence of the other axioms
of ZF set theory. [You may assume the existence of transitive
closures.]  Given a model $\tuple{V, \in}$
for all the axioms of ZF except
Foundation, show how to define a transitive class $R$ which, with the
restriction of the given relation $\in$, is a model of ZF.  Given a model
$\tuple{V, \in}$ of ZF, indicate briefly how one may modify the relation $\in$ so
that the resulting structure $\tuple{V, \in'}$ fails to satisfy Foundation,
but satisfies all the other axioms of ZF. [You need not verify that
all the other axioms hold in $\tuple{V, \in'}$.]}

The first part is bookwork.

The second part, too, is bookwork but it's possible to get it wrong.
Indeed i was contacted by a Trinmo(!) who had got into a tangle, He
wanted the class of all $x$ s.t.
$(\exists y \in x)(x \cap y = \emptyset)$ (which is {\sl not} what
you want: suppose you had $x = \{x,\emptyset\}$; then $x$ is disjoint
from one of its members but is not wellfounded).  The class you want
is the union of all the $V_\alpha$ where $V_0 = \emptyset$ and thereafter
$V_\alpha = {\cal P}(\bigcup_{\beta < \alpha}V_\beta)$.

For the final part look up permutation models in my 2017 lecture notes.
\chapter{2014}

\subsection*{Paper 2 question}

{\sl Ordinals less than a power of $\omega$ are closed under $+$; ordinals
less than $\omega^{\omega^\alpha}$ are closed under $\times$; ordinals
below an $\epsilon$-number are closed under exponentiation.}

\subsubsection*{Discussion}

Let's prove that the ordinals less than a power of $\omega$ are closed
under addition.  Suppose not, and let $\omega^\alpha = \beta+ \gamma$.
Then both $\beta$ and $\gamma$ are below $\omega^\delta \cdot n$ for
some $\delta < \alpha$.  But then $\omega^\alpha = \beta + \gamma \leq
\omega^\delta \cdot n + \omega^\delta \cdot n \leq \omega^\delta \cdot
n\cdot 2 < \omega^{\delta + 1} \leq \omega^\alpha$.

\medskip

Closure under multiplication

\smallskip

Suppose $\alpha$ and $\beta$ are both less than $\omega^{\omega^\gamma}$.  
Then there is a $\delta < \omega^\gamma$ such that both $\alpha$ and 
$\beta$ are below $\omega^\delta$.  But then $\alpha \cdot \beta$ is 
below $\omega^\delta \cdot \omega^\delta = \omega^{\delta + \delta}$ and 
now we appeal to the fact that the ordinals below $\omega^\gamma$ are 
closed under addition!

\medskip

Closure under exponentiation

\smallskip

Suppose $\omega^\epsilon = \epsilon$, and $\alpha, \beta < \epsilon$.
Then $\alpha^\beta \leq (\omega^\alpha)^\beta =^{(1)} \omega^{\alpha \cdot \beta}$.  

Now $\epsilon = \omega^{\omega^\epsilon}$ so ordinals below $\epsilon$
are closed under multiplication (see above) giving $\alpha\cdot\beta <
\epsilon$, whence $\omega^{\alpha \cdot \beta} <^{(2)} \omega^\epsilon
= \epsilon$.
\begin{quote}
  (1) It can't do any harm to think about why this equation holds. You
  could prove it from the synthetic definition of exponentiation
  (which you were presumably not lectured) or you could prove it by
  fixing $\alpha$ (``universal generalisation'') and doing an
  induction on $\beta$.

(2) holds beco's $\gamma \mapsto \omega^\gamma$ is normal.
\end{quote}

\chapter{2015}


\subsection*{Paper 1, Section II 13I}
{\sl State and prove the Completeness Theorem for Propositional Logic.

[You do not need to give definitions of the various terms
  involved. You may assume the Deduction Theorem, provided that you
  state it precisely.]

State the Compactness Theorem and the Decidability Theorem, and deduce them from the Completeness Theorem.

Let $S$ consist of the propositions $p_{n+1} \to p_n$ for $n = 1, 2,
3 \ldots$. Does $S$ prove $p_1$?  Justify your answer. [Here $p_1$,
  $p_2$, $p_3$ \ldots are primitive propositions.]}

\medskip

Bookwork

\subsection*{Paper 2, Section II 13I}

{\sl (a) Give the inductive and synthetic definitions of ordinal addition, 
and prove that they are equivalent. Give the inductive definitions of 
ordinal multiplication and ordinal exponentiation.  

(b) Answer, with brief justification, the following: 
\begin{quote}
(i) For ordinals $\alpha$, $\beta$ and $\gamma$ with $\alpha < \beta$, must
we have $\alpha + \gamma <\beta + \gamma$?  Must we have $\gamma+\alpha
< \gamma + \beta$?

(ii) For ordinals $\alpha$ and $\beta$ with $\alpha < \beta$ must we
have $\alpha^\omega <\beta^\omega$ ?

(iii) Is there an ordinal $\alpha > 1$ such that $\alpha^\omega = \alpha$ ?  

(iv) Show that $\omega^{\omega_1} = \omega_1$. Is $\omega_1$ the least
ordinal $\alpha$ such that $\omega^\alpha = \alpha$ ?  
\end{quote}
[You may use standard facts about ordinal arithmetic.]}

\bigskip

This is mostly bookwork.  However there is a subtlety to part (iv).
$\omega^{\omega_1}$ is of course sup$\{\omega^\alpha: \alpha <
\omega_1\}$ and we want this to be no more than $\omega_1$.  It's
clearly no {\sl less} than $\omega_1$ because $\alpha \leq
\omega^\alpha$ always and the $\alpha$ we are summing over are
unbounded below $\omega_1$; for it to be no {\sl greater} than
$\omega_1$ we need $\omega^\alpha$ to be countable whenever $\alpha$
is.  If we try to do this by induction on $\alpha$ we have no problem
at successor ordinals of course, co's we're just multiplying by
$\omega$, but at limit stages we are liable to find ourselves appealing
to the principle that a union of countably many countable sets is
countable.  Why is $\omega^\lambda$ countable?  Well, if $\lambda$ is
countable limit it is sup$\tuple{\lambda_n: n\in \Nn}$ and each (von
Neumann ordinal) $\omega^{\lambda_n}$ is [a] countable [set] by induction 
hypothesis, so the (von Neumann ordinal) $\omega^{\lambda}$ is [a] 
countable [set] by countable-union-of-countable-sets-is-countable.  
This use of countable choice seems to be unavoidable.

However, if we use the synthetic definition of ordinal exponentiation
we obtain a set (the set of all those functions from a wellordering of 
length $\alpha$ to $\Nn$ that take the value $0$ at all but finitely 
many arguments) equipped with a natural wellordering that is of order
type $\omega^\alpha$.  This set can be shown to be countable, as follows.  
Each such function can be thought of as a finite set of ordered pairs 
of ordinals-below-$\alpha$ paired with naturals.  There are countably 
many such pairs and therefore only countably many finite sets of such pairs.

\medskip

You probably don't care about such things (and i bet the examiners
didn't) but i didn't get where i am today by being the Old Man of Thermopyl{\ae} who never did
anything properl{\ae}.  As for whether or not you can prove the
analogous result for the next operation after exponentiation (namely
that the countable ordinals are closed under it) without using
countable choice i have to confess that i'd never thought about it.  I
have an awful feeling that the answer might be `no'.

\subsection*{Paper 3, Section II 13I}

{\sl (i) State and prove Zorn's Lemma.  [You may assume Hartogs' Lemma.]  

Where in your proof have you made use of the Axiom of Choice?

(ii) Let $<$ be a partial ordering on a set $X$.  Prove carefully that
$<$ may be extended to a total ordering of $X$.  What does it mean to
say that $<$ is well-founded?  If $<$ has an extension that is a
well-ordering, must $<$ be well-founded?  If $<$ is well-founded, must
every total ordering extending it be a well-ordering?  Justify your
answers.}

\subsubsection*{Discussion}


The first part is bookwork.  However they want you to say (and *I*
want to hear you say it, too) that The Axiom of Choice says that every
set $X$ has a choice function blah, and that in this case the set $X$
that has a choice function is that set that contains, for each $p$ in
your poset, the set $\{p' \in P: p \leq p'\}$.


\smallskip

If you discard ordered pairs from a wellfounded relation the result is
still a wellfounded relation. (If you remove ordered pairs you
perforce remove descending sequences so you are certainly not going to
{\sl add} any new infinite descending sequences!)  So: if $<$ has an
extension that is a well-ordering, $<$ must be well-founded.  The
converse is clearly not going to be true. Consider the empty relation
on $\QQ$.  It's wellfounded, but you can extend it to the usual order
on $\QQ$, and that is not wellfounded.

\subsection*{Paper 4, Section II 13I}
{\sl State the Axiom of Foundation and the Principle of $\in$-Induction,
and show that they are equivalent (in the presence of the other axioms
of ZF). [You may assume the existence of transitive closures.]

Explain briefly how the Principle of $\in$-Induction implies that
every set is a member of some $V_\alpha$.

Find the ranks of the following sets:
\begin{quote}
(i) $\{\omega+1, \omega+2, \omega+3\}$;\\
(ii) the Cartesian product $\omega \times \omega$;\\
(iii) the set of all functions from $\omega$ to $\omega^2$.
\end{quote}

[You may assume standard properties of rank.]}

\subsubsection*{Discussion}

The first part is bookwork.  For the second we'll assume that the
ordinals are von Neumann ordinals and that ordered pairs are
Wiener-Kuratowski, and that will give the answers
\begin{quote}
(i) $\omega + 4$ (since the rank of the von Neumann ordinal $\alpha$ is $\alpha$); 

(ii) $\omega$, because every member of $\omega \times \omega$ is an ordered
pair of two things of finite rank, and W-K pairs lift rank by
$2$\ldots and 

(iii) $\omega^2 + 1$, beco's every function $\omega \to \omega^2$ is a
subset of $\omega \times \omega^2$ and is therefore of rank at least
$\omega$ (since its arguments ore of unbounded finite rank) and
possibly as much as $\omega^2$ (since its values can have any rank
below $\omega^2$) by analogy with (ii).\end{quote}

In connection with (iii) it might be worth reflecting that if we were
considering functions $\omega \to \omega_1$ then the answer would be
$\omega_1$ not $\omega_1 + 1$.  This is because a map $f:\omega \to
\omega_1$ is a set of ordered pairs of
a-finite-ordinal-with-a-countable-ordinal, so the ranks of all its
members are countable.  But this $f$ is a countable set, so its rank
is the least ordinal strictly bigger than evereything in a {\sl
  countable} set of ordinals, and by countable choice this upper bound
is countable.



\chapter{2016}


\subsection*{Paper 1 section II 15F}

\subsubsection*{Part (b)}

\subsubsection*{Part (c)}
I don't know how to make it obvious that the answer is `yes', and of
course one needs the answer to be obvious if one is to know which way
to jump.  My thought for this one is that one should remember that
$\alpha \mapsto \omega^\alpha$ is a normal function, and just give it
a whirl.  Low cunning would suggest that the mention of the Division
algorithm at the end of the question is a hint that you should be
using it.  Given $\alpha = \omega\cdot\alpha$ one looks for the
largest $\beta$ (and there will be a largest $\beta$ beco's
$\alpha \mapsto \omega^\alpha$ is a normal function and we invoke the
division algorithm for normal functions) s.t. $\omega^\beta \leq \alpha$.

That should crack it, so i'm leaving the details to you.

Perhaps the moral is to always remember the division algorithm for normal functions.

\medskip

Actually, on second thoughts i think i owe it to you to provide a
worked answer.  [Thanks to Hanna G\'al for pushing me.] One shows that $\alpha = \omega^n\cdot\alpha$ for
every $n$ by induction on $n$ so certainly $\alpha \geq
\omega^\omega$.  One then uses the division algorithm to find $\beta$
the largest multiple of $\omega^\omega \leq \alpha$.  So
$$\omega^\omega\cdot \beta + \gamma = \alpha$$
where $\gamma < \omega^\omega$.  What now?  The only fact we have is that $\alpha =
\omega\cdot\alpha$ so the obvious thing is to multiply both sides on
the left by $\omega$ getting

$$\omega \cdot (\omega^\omega\cdot \beta + \gamma) = \omega\cdot \alpha = \alpha$$
But the LHS simplifies to
$$\omega^\omega\cdot \beta + \omega\gamma$$

which (by uniqueness of subtraction) gives us $\omega\cdot \gamma = \gamma$, contradicting $\gamma < \omega^\omega$.

\smallskip

A nice proof.  Not sure whether i'd find it in the heat of the exam.


\subsubsection*{Part (e)}
This is a trick question.  Since $\alpha < \alpha^\omega$ always, the 
conditional has a false antecedent and is therefore true!

\section*{Paper 2 14F}

An alternative equivalent definition of transitive set is $x \subseteq {\cal P}(x)$ which makes rthe induction a lot easier.


\subsection*{Paper 3, Section II 14F}

This looks scary but it's actually a piece of cake.  We appeal to the
banal obvious fact\ldots.  If $T$ is a universal theory then every
substructure of a model of $T$ is another model of $T$.  So in
particular the package of closed $T$-terms must be a model of $T$.

\commentblock{Now the thing we are trying to prove looks very much
  like the output of a compactness argument.  So clearly the way to
  get there is to derive an inconsistency from $T$ plus the scheme
  $\neg \psi(t_i)$ of all terms $t_i$.  Suppose we could not derive
  this inconsistency.  Then there would be a model $\M$ of $T$ plus
  this scheme.  But then consider the substructure of $\M$ consisting
  of the denotations of all the terms $t_i$.  This is a model of $T$
  plus the scheme---since none of the formul{\ae} in the scheme have
  any quantifiers.  But this substructure is a model of $T$ and
  therefore believes $(\exists x)\psi(x)$.  This contradicts the fact
  that nothing in the ``term model'' has $\psi$}.

For the last part we of course use compactness. Suppose $T \vdash
(\exists x)\psi$.  Suppose further that $T$ does not prove any finite
disjunction of the indicated flavour.  Then we can consistently add to
$T$ any finite collection of expressions of the form $\{\neg
\psi[t_i/x]: i \in I\}$.  So, by compactness, we can consistently add
all negations $\neg \psi[t_i/x]$.  This theory will have a model $\M$.
Think about the substructure of $\M$ consisting of closed terms.  It
is a substructure of a model of $T$ and is therefore a model of
$T$. So it believes $\exists x \psi$ \ldots except that there is no
term available to witness the `$x$'.

\section*{Paper 4, Section II 15F}

Part (a) gave me a bit of a fright.  Obviously you are meant to
consider the chain-complete poset of partial homomorphisms
$L\rightharpoondown \{0,1\}$.  But why is a maximal element of this
poset a total homomorphism---one defined on the whole of $L$?  Suppose
it isn't, and that there is a maximal element $f$ and a rogue element
$b$ on which $f$ is not defined.


If we cannot send $b$ (for ``bad'') to $1$ it is beco's there is an
$a_0$ with $f(a_0) = 0$ and $f(b\vee a_0) = 1$; and if we cannot send
it to $0$ it is beco's there is an $a_1$ with $f(a_1) = 1$ and
$f(b\wedge a_1) = 0$.  We have to show that these cannot both happen,
and we have to use distributivity to do it.

Distributivity gives us

$$a_1 \wedge (b\vee a_0) = (a_1 \wedge b) \vee (a_1 \wedge a_0)$$
and the fact that $f$ is a homomorphism gives us
$$f(a_1 \wedge (b\vee a_0)) = f((a_1 \wedge b) \vee (a_1 \wedge a_0))$$
Using the equations $f(a_0) = 0$, $f(b\vee a_0) = 1$, $f(a_1) = 1$
and $f(b\wedge a_1) = 0$ we can simplify the  LHS to $1$ and the RHS
to $f(a_1 \wedge b)$ getting
$$1 = f(a_1 \wedge b)$$
Now we also have (analogously)
 $$a_0 \vee (b \wedge a_1) = (a_0 \vee b) \wedge (a_0 \vee a_1)$$
and---as before---the fact that $f$ is a homomorphism gives us
 $$f(a_0) \vee f(b \wedge a_1) = f(a_0 \vee b) \wedge f(a_0 \vee a_1)$$
Similarly the LHS is $0$ and the RHS is $f(a_0 \vee b)$ giving
$$f(a_0 \vee b) = 0$$
So we have both
\begin{center}
$f(a_0 \vee b) = 0$ and $1 = f(a_1 \wedge b)$\end{center}
But we have $$a_1 \wedge b\, \leq\, a_0 \vee b$$
contradicting the fact that $f$ of the LHS of this inequality
is $1$ and $f$ of the RHS is $0$.

\medskip

There must be a simpler way of doing it, but i am an elderly
wally and can't see it.

O Wally wally, the water is wide
and you can't get o'er.



\chapter{2017}


\subsection*{Paper 2 section II 14H}


Which of the following assertions about ordinals are true?  Justify your answers

\begin{tabbing}
(i) \ \ \ \= $\alpha + \beta = \beta + \alpha$;\\

(ii)   \>$(\alpha + \beta)\cdot \gamma = \alpha \cdot \gamma + \beta \cdot \gamma$;\\

(iii)   \>$\alpha\cdot(\beta+\gamma) = \alpha \cdot \beta + \alpha \cdot \gamma$;\\

(iv)   \> $\alpha\beta = \beta\alpha$  $\to$  $\alpha^2\beta^2 = \beta^2\alpha^2$;\\

(v)      \> $\alpha^2 = \beta^2$  $\to$ $\alpha\beta = \beta\alpha$.

\end{tabbing}

\subsection*{Discussion Answer}

I address (iv) and (v) only.  (It's the least i can do, since it was
i who set the questions).  Observe that addition, multiplication and
exponentiation preserve inequalities both ways round, and one way
round they preserve {\sl strict} inequality as well.  Thus, if
$\alpha < \beta$ then $\alpha + \gamma \leq \beta+ \gamma$ and
$\gamma + \alpha < \gamma + \beta$.  Multiplication and exponentiation
similarly. We will exploit this fact in what follows.

\smallskip

(iv) is true. Suppose $\alpha\beta = \beta\alpha$.  Multiply both
sides by $\alpha$ on the left and $\beta$ on the right to obtain
$\alpha^2\beta^2 = \underline{\alpha\beta}\alpha\beta$.  Now permute
the underlined bits to obtain

$$\alpha^2\beta^2 = \beta\alpha\underline{\alpha\beta}$$  and permute again to obtain

$$\alpha^2\beta^2 = \beta\underline{\alpha\beta}\alpha$$ and yet  {\sl again} to obtain

$$\alpha^2\beta^2 = \beta^2\alpha^2.$$

Notice (and this has only just struck me) we haven't used the fact
that $\alpha$ and $\beta$ are ordinals.  This works for any linear
order types whatever---integers, rationals, reals, you name it.

\smallskip

(v)

How about the converse to (iv)?  If $\alpha^2\beta^2 = \beta^2\alpha^2$
must we have $\alpha\beta = \beta\alpha$?  (That was an old example
sheet question from before the last ice age).
\smallskip

I am quite sobered by my student Micha\l{} Mruga\l{}a finding a mistake in my original model answer.
(Sobered not just beco's it means i am fallible but also beco's it means that people haven't been reading it!!)


Assume $\alpha^2\beta^2 = \beta^2\alpha^2$ and (for a contradiction) suppose $\alpha\beta < \beta\alpha$.

You can multiply on the left by $\alpha$ to get $\alpha^2\beta < \alpha\beta\alpha$.  (Multiplication on the left preserves strict inequality).

Now
$$\alpha\underline{\alpha\beta}\beta \leq^{1} \underline{\alpha\beta}\alpha\beta \leq^{2} \underline{\beta\alpha}\alpha\beta <^{3} \beta\underline{\alpha\beta}\alpha \leq \beta\beta\alpha\alpha$$

(1) swap the two underlined terms

(2) swap the two underlined terms

(3) $\alpha\beta < \beta\alpha$ by assumption and multiplication on the left preserves strict inequality.

giving $\alpha^2\beta^2 < \beta^2\alpha^2$ contradicting assumption.

\commentblock{
You can multiply both on the left by $\alpha$ (which preserves the
{\sl strict} inequality---this is key) and on the right by $\beta$
(which just preserves the inequality) getting
$$\alpha^2\beta^2\, <\, \alpha\beta\alpha\beta.$$
Now
$$\underline{\alpha\beta}\alpha\beta\, \leq\, \underline{\beta\alpha}\alpha\beta$$
(beco's the underlined bit on the left is less than the underlined bit on the right) and then
$$\beta\alpha\underline{\alpha\beta}\, \leq\, \beta\alpha\underline{\beta\alpha}$$
(compare the two underlined bits again) and
$$\beta\underline{\alpha\beta}\alpha\, \leq\, \beta\underline{\beta\alpha}\alpha$$
(compare the two underlined bits again) finally giving
$$\alpha^2\beta^2\, <\, \alpha\beta\alpha\beta\, \leq\, \beta\alpha\alpha\beta\, \leq\, \beta\alpha\beta\alpha\, \leq\, \beta^2\alpha^2$$

contradicting assumption.}

\medskip

(v) is harder than (iv) beco's you can't do it the way you did
(iv)---merely by manipulating equations; you {\sl have to} do it by
contradiction.  Also, in (v) you need the fact that multiplication on
the left preserves strict inequality and---if you think about it---you
can see that this doesn't work for arbitrary linear order types:
$2 <3$ but $\QQ\cdot 2 = \QQ\cdot 3$.
  


\subsection*{Paper 3 section II 14F}


This seems to be the same as 2011 Paper 2, Section II 16H Logic and Set Theory


\subsection*{Paper 1, Section II 15H}

If every valuation satisfies precisely one of $A$ and $B$ then
(assuming neither is empty) we must have $A \vee B \models \bot$
whence $X \models \bot$ for some finite $X \subseteq A \cup B$.  Then
set $A' = X \cap A$ and $B' = X \cap B$.  If $A$ and $B$ are to be
finitely axiomatised one obviously looks for them to be axiomatised by
$A'$ and $B'$ respectively.  Do we have $A' \vdash A$?  If not, then
there is a valuation $v$ that satisfies $A'$ but does not satisfy $A$.
But if $v$ does not satisfy $A$ then it must satisfy $B$, so it
certainly must satisfy $B'$.  But then it satisfies $A' \cup B'$, and
this contradicts the fact that $A' \cup B' \models \bot$.  So
(contraposing) every valuation that satisfies $A'$ satisfies $A$, so
$A'$ axiomatises $A$.  Similarly for $B$ and $B'$.

\medskip

Cute.  I wonder who came up with that?

\subsection*{Paper 4, Section II 15H}

I don't know who came up with {2017 Paper 1, Section II 15H} but this one was mine.

\smallskip

The last part is not actually difficult but it requires confidence.  I
suppose one should start by asking what one wants the answer to be: is
there a fixed point or not?  We know there is no set equal to the set
of {\sl all} its subsets, but can there be a set equal to the set of
its {\sl countable} subsets?  Well, if there is, one would obtain it
by iteration---as in the question, indeed.  If one obtains the fixed
point as a $C_\alpha$ then what can one say about the ordinal
$\alpha$?  Every countable subset of $C_\alpha$ has to be a member of
$C_\alpha$, but $C_\alpha$ is the union of all previous $C_\beta$.  So
every countable subset of $C_\alpha$ is therefore a member of
$C_\beta$ for some earlier $\beta$.  Now: what can you do with that?
For $C_\alpha$ to be a fixed point it has to be that every countable
set of ordinals below $\alpha$ is bounded below $\alpha$.  $\alpha$
cannot have cofinality $\omega$.  If you have countable choice then
$\omega_1$ is such an ordinal.

There is a delightful discussion of this question in Jech's 1980
Journal of Symbolic Logic article ``On hereditarily countable sets''.
It has a very {\sl very} beautiful construction.  If you think you
might do set theory at some point, read that paper.  It's v nice, and
no-one else will tell you to read it.


\chapter{2018}
\subsection*{Paper 4, Section II 16G}
(i) {\sl State and prove the $\in$-Recursion Theorem.  [You may assume the
    Principle of $\in$-Induction.]}

\smallskip

\noindent (ii) {\sl What does it mean to say that a relation $r$ on a set $x$ is
  well-founded and extensional ?  State and prove Mostowski's
  Collapsing Theorem.  [You may use any recursion theorem from
    the course, provided you state it precisely.]}

\smallskip

\noindent (iii) {\sl For which sets $x$ is it the case that every well-founded extensional relation on $x$ is isomorphic to the relation $\in$ on some transitive subset of $V_\omega$ ?}

\medskip

\subsubsection*{Answers}

(i) and (ii) are bookwork

\smallskip

\noindent (iii)

What is this condition on $x$?  It says: Think of $X$ as a bag of
points.  Now decorate the points with directed edges (thought of as
ordered pairs from a membership relation that you are making up as you
go along.).  The result is a digraph that can be tho'rt of as a {\sl
  set picture}.   Is this picture a picture of a transitive subset of $V_\omega$?

Now clearly this project depends only on the size of $x$, since any
decoration of $x$ can be copied over to any set in 1-1 correspondence
with $x$.  So what can we say about the cardinality of $x$?  Clearly
any finite set has this property. What about infinite sets?
Well $x$ is going to have to be countable, co's it has
to be the same size as a subset of $V_\omega$, and {\sl that's}
countable.  But countability isn't sufficient, because i could inflict
on $\Nn$ (for example) a wellordering of transfinite length, and that
won't be iso to any subset of $V_\omega$.  (Any wellordering is an
extensional wellfounded relation). It's going to have to be finite.

Clev-ah!

\smallskip

Actually Pavel Turek has pointed out something that I had overlooked
(and which I am sure the examiners expected you to overlook).  What
happens if $x$ cannot be decorated by a wellfounded extensional
relation?  Then it's vacuously true that every wellfounded extensional
relation on $x$ is isomorphic to the restriction etc etc.  So, to
deduce the intended answer (namely that $X$ has to be finite) we need
to assume that every set can be decorated with a wellfounded
extensional relation.  Now this is certainly a consequence of AC,
since AC tells us that we can wellorder anything, and a wellordering
is certainly a wellfounded extensional relation as noted above.  Quite
what the strength is of the assertion ``every set supports a
wellfounded extensional relation'' i don't know.  And i should.  But i
am sure that the examiners were expecting you to assume it.

\subsection*{Paper 3, Section II 16G}


(1) {\sl State and prove the Compactness Theorem for first-order predicate
logic.}

\noindent (2) {\sl State and prove the Upward L\"owenheim-Skolem
Theorem.  [You may assume the Completeness Theorem for first-order predicate logic.]}

\noindent (3) {\sl For each of the following theories, either give
  axioms (in the specified language) for the theory or prove that the
  theory is not axiomatisable.



\begin{quote}
{\sl (i) The theory of finite groups (in the language of groups).}

{\sl (ii) The theory of groups in which every non-identity element has
infinite order (in the language of groups).}

{\sl (iii) The theory of total orders (in the language of posets).}

{\sl (iv) The theory of well-orderings (in the language of posets ).}
\end{quote}

If a theory is axiomatisable by a set $S$ of sentences, and also
by a finite set $T$ of sentences, does it follow that the theory
is axiomatisable by some finite subset of $S$ ?  Justify your answer.}

\subsubsection*{Answers}

(3)

(i) is clearly not axiomatisable, by compactness.

\smallskip

(ii) is a recent example sheet question and is discussed in my notes on that sheet

\smallskip

(iii) Of course

\smallskip

(iv) One's first thought is that it can't be, beco's the theory is
{\sl prima facie} second order. However that isn't a proof.  To obtain
a proper proof add infinitely many constants $c_i: i \in \Nn$ and
axioms $c_i > c_{i+1}$.


The rider.  Yes, of course.  $T$ must follow from $S$, and if $S$ is
infinite it must (by compactness) follow from some finite subset of
$S$.

There are two ways to write this down.  Either argue that for each
$t \in T$ there is a finite subset $S_t$ of $S$ which entails it (by
compactness).  Then take the union of the (finitely many) $S_t$ to
obtain a finite axiomatisation $\subseteq S$.  Or consider the
conjunction of all the members of $T$ and argue that that must follow
from a finite subset of $S$---again, by compactness.
 
\subsection*{Paper 2, Section  II 16G}

{\sl (i) State and prove the Knaster-Tarski Fixed-Point Theorem.

  (ii) Deduce the
  Schr\"oder-Bernstein [{\sl sic}] Theorem.

  (iii) Show that the poset $P$ of
all countable subsets of $\Re$ (ordered by inclusion) is not complete.}

(iv) {\sl Find an order-preserving function $f : P \to P$ that does not
  have a fixed point.  [Hint: Start by well-ordering the reals.]}

\subsubsection*{Answers}

(iii) Well of course it's not complete: it hasn't got a top element.  Duh!

  Annoying question: is it {\sl countably} complete?  Is a union of
  countably many countable sets of reals a countable set of reals?
  Normally one needs countable choice to show that a countable family
  of countable sets has a countable sumset, but what if you are given
  the additional information that each countable set in the family is
  a set of reals?  Does that help?  I think it is known that if AC
  fails badly enuff then $\Re$ can be a union of countably many
  countable sets---so the extra information {\sl doesn't} help\ldots
  as it happens, and we {\sl do} need countable choice to show that
  $P$ is countably complete.  But it's good to get into the habit of
  asking questions like that.

Another annoying question: is it chain-complete?  No, it isn't even
chain-complete.  It shouldn't be hard to come up with a nested
sequence of countable subsets of $\Re$ whose union is uncountable.

And it isn't {\sl directed-complete} either.  But now we are straying
too far into poset stuff \ldots which you have been told not to worry
about.
\medskip


(iv) I had to think a bit about this last one.  If you wellorder $\Re$ by
some wellordering $<<$ then you can extend any countable subset of
$\Re$ by inserting the $<<$-first real not in it.  That is to say:
$F(A)$ is to be $A \cup \{$the $<<$-first thing in $X \setminus A\}$.
(Notice that all we need is a wellordered uncountable subset $X$ of
$\Re$, beco's no countable subset of $\Re$ can contain all of $X$ so
we just add to $A$ the first thing in $X\setminus A$).

However, in a discussion with my supervisee John Dawson he brought
home to me that---although it's obvious that this function is
inflationary---it's not {\sl quite} so obvious that it is
order-preserving.  But it is, nevertheless. Suppose $A$ is a proper
subset of $B$.  We want the $<<$-first thing in $\Re \setminus A$ to be
in $F(B)$. If it is in $B$ then it is certainly in $F(B)$ so we're
happy.  If it is not in $B$ then it must be the first thing not in
$B$. Thank you jad200!

Can one find such a function without any use of choice?  Not the way
we have done it, beco's we need at least {\sl some} AC to show that
$\Re$ has an uncountable wellordered subset.  More formally: $\aleph_1
\leq 2^{\aleph_0}$ is not a theorem of ZF without AC. I would expect
that we can, but i can't see how to do it offhand.  We do know, even
without AC, that there is a surjection from $\Re$ to the set of
countable sets of reals, but i can't see how to make good use of it.
Dunno!  Pester me if you need an answer.

\subsection*{Paper 1, Section II 16G}

(i)  Give the inductive definition of ordinal exponentiation.
Use it to show that $\alpha^\beta \leq \alpha^\gamma$ whenever
$\beta \leq\gamma$ (for $\alpha \geq 1$), and also that
$\alpha^\beta <\alpha^\gamma$ whenever $\beta < \gamma$ (for
$\alpha \geq 2$).

\smallskip

\noindent(ii) Give an example of ordinals $\alpha$ and $\beta$ with
$\omega < \alpha < \beta$ such that $\alpha^\omega = \beta^\omega$ .

\smallskip

\noindent (iii) Show that
$\alpha^{\beta+\gamma}=\alpha^\beta \alpha^\gamma$, for any ordinals
$\alpha$ $\beta$ $\gamma$,

\smallskip

\noindent(iv) Give an example to show that we need not have

$(\alpha\beta)^\gamma = \alpha^\gamma \beta^\gamma$.

\smallskip

\noindent (v) For which ordinals $\alpha$ do we have $\alpha^{\omega_1} \geq \omega_1$?

\smallskip

And

\noindent (vi)  for which $\alpha$ do we have $\alpha^{\omega_1} \geq \omega_2$?

Justify your answers.
[You may assume any standard results not concerning ordinal exponentiation.]

\subsubsection*{Answers}

(i) Bookwork. Pester me if you really want an answer.

[Actually it may be p{\ae}dogogically worth spelling out that in order
  to prove that $(\forall \alpha \beta \gamma)(\alpha^{\beta+\gamma} =
  \alpha^\beta \cdot \alpha^\gamma)$ one fixes $\alpha$ and $\beta$
  and proves by induction on $\gamma$ that $(\forall
  \gamma)(\alpha^{\beta+\gamma}= \alpha^\beta \cdot \alpha^\gamma)$.
  One does not prove by induction on $\gamma$ that $(\forall \alpha
  \beta \gamma)(\alpha^{\beta+\gamma}= \alpha^\beta \cdot \alpha^\gamma)$.
  This second induction is much stronger than the one we are doing:
  it has $\forall \alpha \beta$ at the front.]

\smallskip

(ii) Try $\alpha = \omega^2$ and $\beta = \omega^3$.
$\alpha^\omega =(\omega^2)^\omega = \omega^{2\cdot\omega} = \omega^\omega$;

$\beta$ similarly.

\smallskip

(iv) For this last bit let $\alpha$ and $\beta$ be finite and let
$\gamma = \omega$.

\smallskip

(v) All of them, or at least all of them bigger than 1.   You prove by
induction that $\alpha^\beta \geq \beta$.  Fix $\alpha > 1$ and do an
induction on $\beta$. (Do not induct on $\alpha$!)

\smallskip

(vi) This is a bit tricky, but since it is the rider that makes the
difference between an $\alpha$ and a $\beta$ that is to be expected.
That said, i do think it is really quite difficult if you are not
allowed to wave your arms.

The way in is to think about the cardinals associated with these
ordinals.  (Thinking about cardinals being associated to ordinals is a
good idea anyway since it helps you to remember that cardinals and
ordinals are related things rather than identical things!)  If
$\alpha$ is an ordinal it is the order type of some wellordering
$\tuple{A,<_A}$, and the set $A$ has a cardinal---all sets do, after
all.  Also the cardinal we get doesn't depend on our choice of $A$ but
only on $\alpha$.  This cardinal-of function is well behaved:

$\bullet$ cardinal-of$(\alpha + \beta)$ is clearly cardinal-of$(\alpha)+$ cardinal-of$(\beta)$, and

$\bullet$ cardinal-of$(\alpha \cdot \beta)$ is clearly
cardinal-of$(\alpha)\cdot$ cardinal-of$(\beta)$, and\footnote{Note the
  overloading of `+' and `$\cdot$' to mean both ordinal and cardinal
  multiplication: these are {\bf not} the same operations!} in both
cases the answer is just the bigger of the two cardinals. (Sum and
product of alephs is just max).  \smallskip Much less obviously

$\bullet$ cardinal-of$(\alpha ^\beta)$ is max(cardinal-of$(\alpha)$,
cardinal-of$(\beta)$) which can be seen by considering the definition
of ordinal exponentiation, which was an example sheet question.
Recall that $\alpha^\beta$ is the order type of the set of functions
of finite support from $B$ to $A$, where $\tuple{A,<_A}$ is a worder
of otype $\alpha$ and $\tuple{B,<_B}$ is a worder of otype $\beta$.
We order the functions colex, but for present purposes all that
matters is how many of the buggers there are.  And that is going to be
the number of finite subsets of $A \times B$ which is going to be just
max$(|A|,|B|)$, because all the cardinals involved are alephs.

Notice that the answer is {\bf not}
cardinal-of$(\alpha)^{(\mbox{\footnotesize{\rm cardinal-of}}\beta)}$; this time the ordinal operation doesn't get sent to the cardinal operation.

\smallskip

Back to the question.

\smallskip

What is cardinal-of($\alpha^{\omega_1}$)?  It's (cardinal-of$(\alpha)
\cdot ($cardinal-of$(\omega_1$).  Since these cardinals are alephs,
this product is simply the bigger of the two.  So we want the bigger
of (cardinal-of$(\alpha$) and (cardinal-of$(\omega_1$) to turn out to
be cardinal-of$(\omega_2)$---which is of course $\aleph_2$.  So
cardinal-of$(\alpha$ had better be $\aleph_2$, whence we deduce
$\alpha \geq \omega_2$.

\smallskip

As i say, i think this last bit is quite hard.


\chapter{2019}


\section*{Paper 4, Section II 16 I}

(i) Define the cardinals $\aleph_\alpha$, and explain briefly why every infinite
set has cardinality an $\aleph$.

\smallskip

\noindent (ii) Show that if $\kappa$ is an infinite cardinal then $\kappa^2=\kappa$.

\smallskip

\noindent (iii) Let $X_1, X_2,...,X_n$ be infinite sets. Show that $X_1\cup X_2\cup
\ldots \cup X_n$ must have the same cardinality as $X_i$ for some $i$.

\smallskip

\noindent (iv) Let $X_1,X_2,...$ be infinite sets, no two of the same cardinality. Is
it possible that $ X_1\cup X_2\cup...$ has the same cardinality as some
$X_i$? Justify your answer.

\subsubsection*{Answers}
(i) An aleph is simply the cardinal of a(n infinite) wellordered set.
The alephs are wellordered by the obvious order relation on cardinals
(this is a nontrivial fact, which the examiners may or may not be
expecting you to prove) so you can enumerate them in increasing order
using the ordinals, starting with 0 (obviously!) so the first infinite
aleph is $\aleph_0$.

The stream of cardinals of wellordered sets is always there, and it's
always wellordered by the natural order relation on cardinals and we
always use the letter `$\aleph$' (with subscripts) to denote its
members---whether we have AC or not.  The Axiom of Choice says that
every set can be wellordered, so it tells us that every infinite
cardinal is an aleph.

\smallskip

This is a question that reveals the extent to which the examinee is
confused about cardinals.  People learning set theory are very liable
to confuse two things:
\begin{quote}
$\bullet$ Q: What are cardinals?  A: cardinals are those things such that
two things have the same cardinal iff there is a bijection between
them.

$\bullet$ (In set theory) Q: What is a cardinal?  This is a question
about how to concretise/implement these abstract mathematical objects
{\sl as sets}---given that you engaged in the fantasy of thinking that
everything is a set.  You can use Scott's trick cardinals as long as
you have foundation.  If you have replacement you can concretise every
ordinal as a von Neumann ordinal. If you have AC then every set is
wellordered and you can concretise its cardinal as the least ordinal
(length) to which it can be wellordered.  However this tells you how
to concretise cardinals {\sl if you have replacement and choice}.  It
doesn't tell you what cardinals {\sl are} (You knew {\sl that} all
along).  Do {\bf not} go away with the idea that if AC fails then
certain sets do not have cardinals.  {\sl Every} set has a bloody
cardinal.  It's just that if you don't have AC then you can concretise
those cardinals in the way you have been led to expect.
\end{quote}

\noindent (ii)  This is bookwork\ldots bloody hard bookwork but bookwork.
(Look at the discussion in my lecture notes for 2017, or my tutorial
\url{www.dpmms.cam.ac.uk/~tf/ordinalsforwelly.pdf})

\noindent For (iii) you exploit the fact that the sum of finitely many alephs is
the biggest of them.  We are assuming AC, so every cardinal is an
aleph.


\noindent(iv)

The question is vague about the size of the family of $X$s, perhaps
deliberately so.  Is it finite?  Might it be infinite?  My guess is
that they are expecting you to think it is countable.  If it's finite,
then the size of the union is simply the biggest, beco's \ldots
assuming AC (as one does) all cardinals are alephs, and the sum of
finitely many alephs is simply the biggest of the summands. OTOH,
suppose the family is infinite: alephs are wellordered by magnitude,
so the countable many cardinals of these countably many sets will form
a chain, and that chain will be wellordered by magnitude.  Might it
have a top element?  Nothing to say it can't.  What happens if i take
the union of countably many things, all but one of them smaller than
$\alpha$?  Can the union be the size of the biggest?  Again, nothing
to say that it can't.

\smallskip

I think this question is ridiculously hard. Granted, the correct
answer is `yes', and it is plausible that it should be `yes', but a
rigorous proof is beyond the resources of this course.  Showing that
the sum of infinitely many alephs is the largest of them (if there is
a largest, indeed) is quite a lot of work\ldots at least if you do it
properly.  For a start, you need the axiom of choice if the concept of
infinite sums of cardinals is to even make sense.  (You have to pick a
set of each size and take the union\ldots)

So: i think the answer they are expecting is that $|X_1| = \aleph_\omega$
and thereafter $|X_n| = \aleph_n$.  Proving rigorously that that works
is more work than there is space for under exam conditions, so you are
expected to wave your arms artistically.


\subsection*{Paper 3, Section II 16 I}

Define the von Neumann hierarchy of sets $V_\alpha$. Show that each
$V_\alpha$ is transitive, and explain why $V_\alpha \subseteq V_\beta$
whenever $\alpha \leq \beta$.

\smallskip

Prove that every set $x$ is a member of some $V_\alpha$.

\smallskip

Which of the following are true and which are false? Give proofs or
counterexamples as appropriate. [You may assume standard properties of
  rank.]

\smallskip

(i) If the rank of a set $x$ is a (non-zero) limit then $x$ is infinite.

(ii) If the rank of a set $x$ is countable then $x$ is countable.

(iii) If every finite subset of a set $x$ has rank at most $\alpha$ then $x$ has rank at most $\alpha$.

(iv) For every ordinal $\alpha$ there exists a set of rank $\alpha$.

\subsubsection*{Answers}
To prove that every $x$ belongs to a $V_\alpha$ you do an $\in$-induction.

\medskip

(i) Yes. If $x$ has infinite rank then the set of ranks of its members
has no maximal element, so $x$ must have infinitely many members.

\smallskip

(ii)  {\sl Obviously} not:  $V_{\omega +1}$ has countable rank but its
cardinality is $2^{\aleph_0}$

\smallskip

(iii) I had to think about this.  Hold on to your hat \ldots the rank
of a set is the least ordinal bigger than all the ranks of its
members.  If the rank of $x$ is successor---$\alpha+1$ for some
$\alpha$---then it has at least one member of rank $\alpha$ so it has
finite subsets of rank $\alpha + 1$.  So: yes.  What happens if the
rank of $x$ is some limit ordinal $\lambda$?  Then the rank of any
finite subset of $x$ is the sup of some finite set of ordinals below
$\lambda$ and accordingly is below $\lambda$.  But the set of such
ordinals is not bounded below $\lambda$. So the least $\alpha$
s.t. every finite subset of $x$ has rank at most $\alpha$ is
$\lambda$.

Another take:


The rank of a finite subset of $x$ is the sup of some ordinals of the
form rank$(y) +1$ for various $y \in x$, and rank$(x)$ is the sup of
all such ordinals, so rank$(x)$ is at least the sup of all the ranks
of the finite subsets.  Can it be greater? No: every member of $x$
appears in at least one such finite set.

\smallskip

(iv) Not hard to prove by induction on the ordinals that the von Neumann ordinal $\alpha$ has rank $\alpha$.

A tangential p{\ae}dogogical point at this stage\ldots.  It is
possible to worry about the legitimacy of the construction of the
cumulative hierarchy of the $V_\alpha$s by recursion over the
ordinals.  I mean, dammit, the ordinals are inhabitants of the
cumulative hierarchy so they aren't initially there for us to do the
recursion on!  This isn't a {\sl real} worry, but it does create an
opening for one to make the point that the morally correct way to
think of the recursive construction of the cumulative hierarchy is
being run on ordinals-as-abstract-mathematical-objects, which exist in
the wider world of mathematics quite independently of any set
theoretic clothing they may or may not have.  You don't have to think
of 17 or $\pi$ as sets, and you don't have to think of ordinals as
sets either, and part (iv) is a good time to exercise that right.  I
would answer (iv) with something like ``for every ordinal $\alpha$ its
von Neumann concretisation is a set of rank (birthday!) $\alpha$.''

You can think of this construction as a way in to Set Theory.  The
narrative is: you start off not knowing about sets (tho' you do know
about numbers, incl ordinals, which are the kind of number that
measures the lengths of transfinite processes).  You then execute the
Indian Rope Trick that is the cumulative hierarchy, and now you have
got your hands on some sets!

\subsection*{Paper 2, Section II  16I}

Give the inductive and synthetic definitions of ordinal addition, and
prove that they are equivalent.

Which of the following assertions about ordinals $\alpha,\beta$ and
$\gamma$ are always true, and which can be false?

Give proofs or counterexamples as appropriate.
\begin{quote}
(i)  $\alpha+ (\beta+\gamma ) = (\alpha+\beta) +\gamma$.

(ii) If $\alpha$ and $\beta$ are uncountable then $\alpha+\beta=\beta+\alpha$.

(iii) $\alpha(\beta\gamma ) = (\alpha\beta)\gamma$ .

(iv) If $\alpha$ and $\beta$ are infinite and $\alpha+\beta=\beta+\alpha$
then $\alpha\beta=\beta\alpha$.
\end{quote}

\subsubsection*{Answers}

(i) and (iii) are true for {\sl all} order types, not just ordinals, so use the synthetic definition.

(ii)  is obviously false: try $\omega_1$ and $\omega_2$;

(iv) brought me up short.  I had to think for a bit.
$\alpha + \beta = \beta + \alpha$ happens only if the two ordinals are
``close'': $\alpha < \beta\cdot \omega$ and
$\beta < \alpha\cdot\omega$.  If they are further apart than that the
bigger one will absorb the smaller one on the left.  So you might
expect the answer to be `yes', but it ain't: try $\alpha = \omega$ and
$\beta = \omega\cdot 2$.

\subsection*{Paper 1, Section II 16I}


(i) State the completeness theorem for Propositional Logic.  Explain briefly
how the proof of this theorem changes from the usual proof in the case
when the set of primitive propositions may be uncountable.

\noindent (ii) State the compactness theorem and the decidability
theorem, and deduce them from the completeness theorem.

\noindent (iii) A poset $(X,<)$ is called {\sl two-dimensional} if there exist
total orders $<_1$ and $<_2$ on $X$ such that $x < y$ if and only if
$x <_1y$ and $x <_2y$. By applying the compactness theorem for
Propositional Logic, show that if every finite subset of a poset is
two-dimensional then so is the poset itself.

[Hint: Take primitive propositions $p_{x,y}$ and $q_{x,y}$, for each distinct
  $x,y \in X$, with the intended interpretation that $p_{x,y}$ is true if
  and only if $x <_1y$ and $q_{x,y}$ is true if and only if $x <_2y$.]

\subsubsection*{Answers}

(i) and (ii) are bookwork.

\smallskip

\noindent (iii) This is a really {\sl really} nice question.  I could bore
for Britain on all the cute points one can extract from it.

\smallskip

We start with a warning: `two dimensional' is a nonce notation, dreamt
up specifically for this question.  You will not see it anywhere else.

The question setter doesn't say whether the relation $<$ is a partial
ordering or a strict partial ordering.  The use of `$<$' rather than
`$\leq$' suggests that it is the strict version that they have in mind,
but i suspect they mean the nonstrict (reflexive) version.  That's the
assumption i would make, anyway.

This is a cute illustration of how a Zorn proof of the existence
of a maximal widget can sometimes be coded up inside propositional
logic. There are some standard examples, but this one is new to me.


The intersection of (the graphs of) two partial ordering relations
on a given set is another partial ordering of that same set. (You
don't need the partial orderings to be total orderings, as here)

\begin{quote}
The {\sl logician} in me wants to emphasise that this last fact is
because of the very simple syntax of `$R$ is a partial order'.  In
contrast the intersection of two total orderings is not a total
ordering, for similar syntactic reasons.

The {\sl amateur combinatorist} in me wants to say that there is a lot
to be said about how the structure of the antichains in a partial
ordering $R_1 \cap R_2$ is controlled by the structure of the
antichains\footnote{$R_1$ and $R_2$ are two partial orderings of one
  and the same set\ldots so we mean the intersection of the two
  graphs, of the two relations\ldots} in $R_1$ and $R_2$.  You measure
the complexity of the antichains with an ordinal, and the ordinal for
the antichains in $R_1\cap R_2$ is the Hessenberg sum of the ordinals
for $R_1$ and $R_2$\footnote{Look at
  \url{www.dpmms.cam.ac.uk/~tf/ordinalsforwelly.pdf} if you want to
  know about Hessenberg sum}.
\end{quote}

For another thing, assuming AC, every partial ordering is
two-dimensional.  (The finite case of this is obvious).  This is a
sexed-up version of OEP, the order extension principle, that says that
every partial ordering can be extended (by adding ordered pairs) to a
total ordering.  The sexed-up version will say that any partial order
can be extended in two incompatible ways whose intersection is the
partial order you started with.

It's a standard result (it's probably an old example sheet question or
old tripos question) that every {\sl wellfounded} partial order can be
extended to a {\sl wellfounded} total order.  Can this be sexed-up
too?  No: if a wellfounded partial order is an intersection of two
wellorderings then it has no infinite antichains: the empty relation
on an infinite set (for example) is not the intersection of two
wellorderings. [Why not!?]

\medskip

With the hint and with the suggestion that you use propositional
compactness this is really a piece of cake.  Contact me if you get
stuck.  I promise not to tell your friends.

\smallskip

Oh, Ok, i suppose i should say {\sl something}.  The hint is good:
``Take primitive propositions $p_{x,y}$ and $q_{x,y}$, for each distinct
$x,y \in X$, with the intended interpretation that $p_{x,y}$ is true if
and only if $x <_1 y$ and $q_{x,y}$ is true if and only if $x <_2y$.''
\smallskip

Then the theory that says that $<_X$ is two-dimensional (in virtue of
$<_1$ and $<_2$) has the following axiom (schemes):\begin{quote}
$\bullet$ $p_{x,y} \wedge q_{x,y}$ for all pairs $x <_X y$;\\
$\bullet$ $p_{x,y} \wedge p_{y,z} \to p_{x,z}$ for all $x$, $y$, $z$ in $X$;\\
$\bullet$ $q_{x,y} \wedge q_{y,z} \to q_{x,z}$ for all $x$, $y$, $z$ in $X$;\\
$\bullet$ $p_{x,y} \vee p_{y,x}$ for all $x$, $y$ in $X$;\\
$\bullet$ $q_{x,y} \vee q_{y,x}$ for all $x$, $y$ in $X$.
\end{quote}

$<_X$ is two-dimensional as long as this theory has a valuation.  By
compactness it has a valuation iff every finite subset has a
valuation.  But that is going to happen iff every restriction of $<_X$
to a finite subset of $X$ is two-dimensional (in virtue of $<_1$ and
$<_2$).  And we are told that every finite substructure of
$\tuple{X,<_X}$ is two-dimensional.

\smallskip

I'm pretty sure that every finite poset is two-dimensional.

\bigskip

While we are about it, here are some standard examples of Zorn-like
things you can prove using propositional compactness:
\begin{quote}
$\bullet$ A group is orderable if all its finitely generated subgroups are.

$\bullet$ A graph is $k$-colourable if all its finite subgraphs are.

$\bullet$ Every partial ordering can be extended to a total ordering
  on the same carrier set.
\end{quote}

\chapter{2020}

\section*{Paper 2 16H}

Parts (a) and (b), (i) are bookwork.  The third part gave me a bad fright.
It makes me realise i am becoming old and slow; time to retire.

{\sl let $X$ be an infinite set.  For each $x \in X$, let $L_x$ be a
  subset of $X$. Suppose that, for any finite $Y \subseteq X$, there is
  a function $f_Y: X \to \{1,\ldots 100\}$ such that, for all $x \in Y$
  and all $y \in Y \cap L_X$, $f_Y(x) \not= f_Y(y)$.

  Show that there is a function $F:X \to \{1,\ldots 100\}$ such that,
  for all $x \in X$ and all $y \in L_x$, $F(x) \not= F(y)$.}

Answer:

My first thought is that the condition that, ``for all $x \in Y$ and
all $y \in Y \cap L_X$, $f_Y(x) \not= f_Y(y)$'' ensures that $x \not\in L_x$.

It's also the case that $f_Y$ doesn't have to be defined on all of
$X$, but only on $Y$: we never look at what it does to anything not in $Y$

I still haven't got it \ldots.  there's something simple i'm missing
\ldots Let try low cunning.  This is obviously a compactness question.
In questions of this kind the plan is to set up a propositional
language ${\cal L}$ with a propositional theory $T$ included in it in
such way that a valuation making $T$ true contains a description of
the object you are being invited to build.  Best bet is that
${\cal L}$ has a propositional letter for every pair in
$X \times [0,100]$ whose meaning is that $x \in X$ is joined to a
number less than $100$.  For each $x \in X$, $T$ contains a disjunction
of $100$ propositional letters that say that $x$ is connected to a
number $\leq 100$, plus axioms to say that it isn't additionally
connected to any others. Also, for each $x \in X$ you have (finitely
many)---negated---letters saying that $x$ is not joined to the same
number as any member of $L_x$.  Any finite set of these axioms is
consistent in virtue of the existence of the $f_Y$ for finite
$Y \subset X$.


\section{Paper 3: 16H}

{\sl Let $\tuple{V,\in}$ be a model of ZF.

(i)  Give the definition of a class and  a function class in $V$.

(ii)   Use the concept of function class to give a short,
  informal statement of the Axiom of Replacement.

(iii)   Let $z_0=\omega$ and, for
  each $n\in\omega$, let $z_{n+1}={\cal P}(z_n)$.

(iv)  Show that $y=\{z_n:n\in\omega\}$ is a set.

We say that a set $x$ is small if there is an injection from $x$ to $z_n$
for some $n\in\omega$.  Let HS be the class of sets $x$ such that every
member of $TC({x})$ is small, where $TC({x})$ is the transitive closure
of $\{x\}$.

(v) Show that $n\in HS$ for all $n\in\omega$ and deduce that
$\omega\in HS$. Show further that $z_n\in HS$ for all $n\in\omega$.

(vi) Deduce that $y\in HS$. Is $\tuple{HS,\in}$ a model of ZF?

(vii) Justify your answer. [Recall that $0 =\emptyset$ and that
  $n+1=n\cup\{n\}$ for all $n\in\omega$.]}

\subsubsection*{Answers}

I can't think of anything particularly helpful to say about this.
There are seven parts to this question, so the examiners can't be
asking for anything very detailed.

(ii)  ``The image of a set in a function is a set''.

(v) Of course you use replacement.  I don't think he's expecting you to explain attempts.

(vi) $\bigcup y$ is not in $HS$.  So $HS$ is witness to the
independence of the axiom of sumsets from ZF.  It's the standard proof
of this fact.

\section*{Paper 4: 16H}

{\sl (a) State Zorn's lemma.

[Throughout the remainder of this question, assume Zorn's lemma.]

(b) Let $P$ be a poset in which every non-empty chain has an upper
bound and let $x\in P$.

By considering the poset $P_x=\{y \in P :x \leq y\}$, show that $P$
has a maximal element $\sigma$ with $x\leq\sigma$.

(c) A filter is a non-empty subset $F \subseteq {\cal P}(\Nn)$
  satisfying the following three conditions:\begin{quote}
  $\bullet$ if $A,B \in F$ then $A\cap B \in F$;\\
  $\bullet$ if $A \in F$ and $A \subseteq B$ then $B\in F$;\\
$\bullet$ $\emptyset\not\in F$.
\end{quote}
An ultrafilter is a filter $\U$ such that, for all $A \subseteq \Nn$, we
have either $A \in \U$ or $\Nn\setminus A \in \U$.

(i) For each $n\in \Nn$, show that $\U_n= \{A \subseteq \Nn : n \in A\}$
is an ultrafilter.

(ii) Show that $F= \{ A \subseteq \Nn:\Nn\setminus A$ is finite$\}$ is
a filter but not an ultrafilter, and that, for all $n\in \Nn$, we have
$F\not\subseteq\U_n$.

(iii) Does there exist an ultrafilter $\U$ such that $\U\not= \U_n$
for any $n\in \Nn$?  Justify your answer.}

\medskip

This is all pretty routine. The only part that involves any work is
(iii). It's an application of Zorn's lemma.  Fix any filter, and
consider the set of filters that are supersets of it.  Clearly this is
a chain-complete poset, so we get an ultrafilter extending the filter
we started with.  If we start with a nonprincipal filter (such as the
``cofinite'' (or {\sl Fr\'echet}) filter consisting of the elements
with finite complement---called `$F$' in (ii) above) then the filter
we get cannot be a $\U_n$. `$\U_n$' is not a standard notation BTW.

Ultrafilters are dead cool; i use to lecture them but they aren't
actually in the syllabus.  They should be. They lead to ultraproducts
which are cooler still.

\section{Paper 2 16H}

Throughout this question, assume the axiom of choice.]

  Let $\kappa, \lambda$ and $\mu$ be cardinals.  Define $\kappa +\lambda, \kappa\cdot \lambda$ and $\kappa^\lambda$.

  What does it mean to say $\kappa \leq \lambda$ ?

  Show that $(\kappa^\lambda)^\mu =\kappa^{ \lambda \cdot\mu}$ .
  Show also that $2^\kappa > \kappa$ .

  Assume  now  that $\kappa$ and $\lambda$ are  infinite.
  Show  that $\kappa\cdot \kappa =\kappa$ .   Deduce  that
  $\kappa +\lambda =\kappa\cdot\lambda =$max$\{\kappa ,\lambda \}$.
  
  Which of the following are always true and which can be false?
  Give  proofs  or counterexamples as appropriate.
\begin{quote}
  (i)\ \ \ $\kappa^\lambda = 2^\lambda$ ;\\
  (ii)\ \ $\kappa \leq \lambda \to \kappa^\lambda = 2^\lambda$ ;\\
  (iii)\ $\kappa^\lambda =\lambda^\kappa$.\end{quote}

  (i) is obviously false c's $\kappa$ can be as big as you like;
  (iii) is obviously false---take the two cardinals to be $\aleph_0$ and $2^{\aleph_0}$. (ii), on the other hand, is true:

  $$2^\lambda \leq \kappa^\lambda \leq (2^\kappa)^\lambda \leq 2^{\kappa\cdot \lambda} \leq 2^{\kappa^2} = 2^\kappa$$

  \medskip

  Quite a nice question\ldots


  \chapter{2021}

  These are mostly fairly routine.
  
  \subsection*{Paper 3, Section II 16G}

  The first four lines of part (b) are stuff you should have done in
  1a.  Part II students shouldn't be asked to write out proofs of
  this sort of thing.  Get a 1a student to do it for you.  Does your
  college still have fagging? Probably not. I don't know what the
  world is coming to.

  \smallskip
  
  The last part is an idiomatic piece of set theory, and recalls the
  proof of Hartogs' Lemma.  On being given a relation on (the von Neuman
  ordinal) $\omega_\alpha$---which is a set of size $\aleph_\alpha$.
  send it to $0$ if it is not a wellordering.  If it is a wellordering
  send it to (the von Neumann ordinal) of its order type.  This maps
  the set of wellorderings of (the von Neuman ordinal) $\omega_\alpha$
  onto the set of (von Neumann) ordinals $< \omega_{\alpha+1}$, which
  last set is of course the (von Neumann) ordinal $\omega_{\alpha+1}$
  itself, as desired.

  Notice that---in particular---there is a map from $\Re$ onto the
  second number class, the set of all countable ordinals, {\sl and we
    know what that map is}.  Observe, in contrast, that there is no
  obvious injection from the second number class into the reals.
  There was an example sheet question about this, the question that
  asked you to embed every countable ordinal in $\Re$.


\subsection*{Paper 4, Section II 16G}

The third part of this question has an interesting history.  Years ago
i set it as an exercise on $\in$-induction in the form ``Show that any
$\in$-automorphism of $V$ must be the identity''.  The proof is pretty
routine, the only subtlety being the point that one might fix a
function-class $f$ and then prove by $\in$-induction that $f$ is the
identity, or prove it simultaneously for all function-classes.  This
raises the question of whether or not the thing one is proving by
means of $\in$-induction is allowed to contain quantifiers over
function-classes.  But perhaps it's best not to think about that!

Anyway, subsequent versions of this question were formulated without
using the word `automorphism'; apparently it's scary.  So the question
became something like: suppose $f$ is a function-class s.t., for all
$x$ and $y$, $x \in y \bic f(x) \in f(y)$; prove that $f$ is the
identity.  The problem is that to establish that it is the identity
one needs the extra condition that $f$ be surjective.  If we drop that
requirement then we can declare a counterexample recursively by
\begin{quote}
 $f(\emptyset) =: \{\emptyset\}$; thereafter $f(x) =: f``x$.
\end{quote}

My student August (he's a star) then says: it's obvious that $x \in y$
implies $f(x) \in f(y)$, but what about the other direction??  Suppose
$f(x) \in f(y) = \{f(z):z \in y\}$.  That doesn't tell us that
$x \in y$ unless $f$ is injective!  After all we might have
$x \not\in y$ but $f(x) = f(z)$ for some $z \in y$.  To my shame this
possibility had never occurred to me!!  We'd better prove that $f$ is
injective.  Presumably we do this by induction.

Suppose, for all $x \in X$, that $(\forall u)(f(x) = f(u) \to x = u)$.
We want to show $(\forall Y)(f(Y) = f(X) \to Y=X)$.  Accordingly
suppose that $f(X) = f(Y)$.  Then
$\{f(x): x \in X\} = \{f(y):y \in Y\}$ so every $f$ of a member of $X$
is an $f$ of a member of $Y$ and {\sl vice versa}.  Perhaps a little
detail wouldn't go amiss.  Suppose $u \in X$.  Then $f(u) \in f(X))$
and {\sl vice versa} beco's of the induction hypothesis on members of
$X$. But $f(X) = f(Y)$ so that's iff $f(u) \in f(Y) = f``Y$.  But then
$u \in X$ beco's the only $z$ st $f(z) = f(u)$ is $u$, by induction
hypothesis.
Whence $X = Y$ by extensionality.  Whew.

(For one ghastly moment i feared i was going to have to do it by a
wellfounded induction on $V \times V$ using the relation
$\tuple{x,y}\,R\,\tuple{x',y'}$ iff $x \in x' \wedge y \in y'$).
  
\bigskip

Izaak Mammadov had an interesting take on this question, even if only
beco's he misread it.  He read the condition as

$$(\forall x y)(x \in y \to f(x) \in f(y))$$
i.e., as a conditional not a biconditional.  And he had difficulty
showing that $f$ must be the identity---even with the assumption that $f$ is onto.  Well, he's not the only one.
I can't prove it either!  And my current guess is that it's not true.

\bigskip

One thought that i had at the back of my mind is that thinking about
$f$s like this---that satisfy the biconditional and not merely the
conditional and are injective but not required to be surjective---can
be connected to $f$s that satisfy an extra ``elementarity'' condition.
This condition says that
$$(\forall \vec x)(\phi(\vec x) \bic \phi(\vec f(x)))$$
Such an $f$ must send the empty set to the empty set. And the
singleton of the empty set to the singleton of the empty set.  In fact
it's quite hard to see how $f$ can avoid being the identity.  Indeed the
assertion that there is a nonidentity $f$ of this kind is incredibly
strong.  The least ordinal $\alpha$ s.t. $\alpha \not= f(\alpha)$ is a
monster.  Study of functions of this kind is absolutely central to
postmodern set theory with large cardinals.


\end{document}


\section{Model Tripos questions}

Need a model question on ordinals.  One fact to fit in is that 
$\alpha \mapsto \alpha^2$ is not normal.
\subsection*{A Model Propositional Logic Question}

(This is a real live example from my research)

We have a propositional language ${\cal L}$ with propositional letters
$p_i$, $q_i$ and $r_i$ for all $i \in \ZZ$.  The theory $T$ has two
schemes; one to say that, for each $i\in\ZZ$, precisely one of $p_i$,
$q_i$ and $r_i$ is true. The other scheme says
$\neg(p_i \wedge p_{i+1}) \wedge \neg(q_i \wedge q_{i+1}) \wedge \neg(r_i \wedge r_{i+1})$.

$T$, considered as the deductive closure of these axioms (a set of 
formul{\ae}) has a automorphism group generated by $S_3$ (permute 
$\{p,q,r\}$)  and $\ZZ$ (permute the subscripts).  This group is the 
group of automorphisms of $T$ in the sense that if $\sigma(\psi)$ is 
the result of doing the permutation $\sigma$ to the formula $\psi$ then 
$T \vdash \psi$ iff $T \vdash \sigma(\psi)$. Explain this situation.

Let us write $\psi^+$ for the result of increasing every subscript in
$\psi$ by $1$.  
\begin{quote}
(a) Show that, altho' for all $\psi$ we have: $T \vdash \psi$ iff 
$T \vdash \psi^+$,  nevertheless we do not have $T \vdash \psi \bic \psi^+$ 
for all $\psi$.  Indeed, if we add the  scheme of biconditionals 
$\psi \bic \psi^+$ to $T$ the resulting theory is inconsistent.  

(b) Prove that nevertheless each expression of the form 
$\phi \bic \phi^+$ is individually consistent with $T$.  
\end{quote}

\commentblock{(a) is pretty obvious; for (b) Randall has persuaded me
  that Specker's rather elliptical explanation (filtered thru' my
  hesitant translation) makes sense.  Fix an arbitrary $S$; we will find a
  valuation satisfying $S \bic S^+$. Suppose (with a view to obtaining
  a contradiction) that every valuation satisfies precisely one of $S$
  and $S^+$. Think of the valuation that goes
  $\ldots p\, q\, r\, p\, q\, r\ldots$ (with period $3$) as you ascend
  through the levels (call it $f$) and the two valuations $f^+$ and
  $f^{++}$.  Recall that $f$ sat $S$ iff $f^+$ sat $S^+$ and so on.
  Do any of these valuations actually satisfy $S \bic S^+$?  If they
  do, we are happy.  If not, then each of them satisfies precisely one
  of $\{S, S^+\}$. WLOG $f$ satisfies $S$ but not $S^+$; then $f^+$
  satisfies $S^+$ but not $S$, $f^{++}$ satisfies $S$ but not $S^+$
  and $f^{+++}$ satisfies $S^+$ but not $S$.  But $f^{+++} = f$.}

\subsection*{Another Propositional Logic Question}

Let $T$ be a propositional theory with an automorphism $\sigma$.  That
is to say, $\sigma:{\cal L}(T) \bic {\cal L}(T)$ satisfies $T\vdash A$
iff $T\vdash \sigma(A)$ for all $A \in {\cal L}(T)$. We stipulate that
$\sigma$ has no finite cycles, and we write `$A^n$' for `$\sigma^n(A)$'.
(There being only one automorphism we can grant it name suppression.)
Let us write $T^{(N)}$ for (the deductive closure of) $T$ augmented by
the scheme of all biconditionals $\phi \bic \phi^n$.
\begin{quote}
Prove that $T^{(M\cdot N)} \subseteq T^M \cap T^N$.
\end{quote}

I'm pretty sure the proof is constructive.  Mind you, I don't know
how to do the inclusion in the other direction. How in God's name do i
deduce $\phi \bic \phi^n \vee \psi \bic \psi^m$ from $T^{(n\cdot m)}$!?
In fact the more i think about it the less plausible it seems.


\subsubsection{Finite axiomatisability}

Show that the intersection of two finitely (resp recursively) axiomatisable
theories is finitely (resp recursively) axiomatisable.

\subsection{Axiomatising theories}

 Are the following theories axiomatisable in the language of posets? Finitely axiomatisable?
\begin{tabbing}
(1) \= Directed Posets;\\
(2)\> Posets in which every chain has an upper bound;\\
(3)\>Posets in which every element belongs to a unique maximal antichain.
\end{tabbing}
In each case exhibit an axiomatisation or prove that there is none.


\subsection{Omitting Types}
  A {\sl type} in a propositional
  language ${\cal L}$ is a set of formul{\ae} (a countably infinite set
  unless otherwise specified).

  For $T$ an ${\cal L}$-theory a $T$-{\sl valuation} is an ${\cal
    L}$-valuation that satisfies $T$.  A valuation $v$ {\sl realises}
  a type $\Sigma$ if $v(\sigma)$ = \verb#true# for every $\sigma \in
  \Sigma$.  Otherwise $v$ {\sl omits} $\Sigma$.  We say a theory $T$
  {\sl locally omits} a type $\Sigma$ if, whenever $\phi$ is a formula
  such that $T$ proves $\phi \to \sigma$ for every $\sigma \in
  \Sigma$, then $T \vdash \neg \phi$.


\bigskip

Now prove the following:

\bigskip


  {\sl Let $T$ be a propositional theory, and $\Sigma \subseteq {\cal
    L}(T)$ a type.  If $T$ locally omits $\Sigma$ then there is a
  $T$-valuation omitting $\Sigma$.}

\bigskip

You may also wish to prove the generalisation

\bigskip


{\sl Let $T$ be a propositional theory and, for each $i \in \Nn$,
  let $\Sigma_i \subseteq {\cal L}(T)$ be a type.  If $T$ locally omits
  every $\Sigma_i$ then there is a $T$-valuation omitting all of the
  $\Sigma_i$.}


[Hint: Show that, if you can find---for each $n$---a family
$\tuple{\phi_i: i \leq n}$, with $\phi_i$ in $\Sigma_i$ for every
$i < n$ s.t. $T \cup \{\bigwedge_{i \leq n} \neg \phi_i\}$ is
consistent, then you can extend this family to one of length $n+1$.]


\bigskip

For further reading have a look at \url{yabloomittingtypes.pdf} linked from my home page.

\commentblock{
\Proof

We prove only the hard (second) part.

We will show that whenever $T \cup \{\neg \phi_1, \ldots \neg \phi_i\}$ 
is consistent, where $\phi_n \in \Sigma_n$ for each $n \leq i$, then we 
can find $\phi_{i+1} \in \Sigma_{i+1}$ such that 
$T \cup \{\neg \phi_1, \ldots \neg \phi_i, \neg \phi_{i+1}\}$ is consistent.

Suppose not, then 
$\displaystyle{T \vdash (\bigwedge_{1 \leq j \leq i} \neg \phi_j) \to \phi_{i+1}}$
for every $\phi_{i+1} \in \Sigma_{i+1}$.  But, by assumption,
$T$ locally omits $\Sigma_{i+1}$, so we would have 
$\displaystyle{T \vdash \neg \bigwedge_{1 \leq j \leq i} \neg \phi_j}$ 
contradicting the assumption that 
$T \cup \{\neg \phi_1, \ldots \neg \phi_i\}$ is consistent.

Now, as long as there is an enumeration of the formul{\ae} in ${\cal
  L}(T)$, we can run an iterative process where at each stage we pick
for $\phi_{i+1}$ the first formula in $\Sigma_{i+1}$ such that $T \cup
\{\neg \phi_1, \ldots \neg \phi_i, \neg \phi_{i+1}\}$ is consistent.
This gives us a theory $T \cup \{\neg \phi_i: i \in \Nn\}$ which is
consistent by compactness.  Any model of $T \cup \{\neg \phi_i: i \in
\Nn\}$ is a model of $T$ that omits each $\Sigma_i$.

\endproof


There is a version of this theorem for predicate (first-order) logic. 
(It deals with $n$-types for $n >0$.)  I can't face proving it here!}

\subsection{A Model Tripos Question about Hall's Marriage Theorem}


see \url{www.math.uwo.ca/~mdawes/courses/420/asst3.pdf}

\medskip

A countable set $M$ of men, and a countable set $W$ of women
and a binary compatibility relation between them.  Assume that, 

(i) for each man, the set of women with whom he is compatible is 
finite. Assume further

(ii) that, for every finite set $W' \subset W$, 
the set of those men who are compatible only with members of 
$W'$ is no bigger than $C'$.

Use propositional compactness to show that in these circumstances
every man can be paired with a compatible woman.  You may assume 
Hall's theorem in the finite case, namely that if $W$ and $M$ are 
finite, and the condition on all $W' \subseteq M$ is satisfied 
for all such $W'$, then every man can be paired with a woman.

Use your result to prove the axiom of choice for countable sets 
of finite sets.

\commentblock{

%\end{document}


\subsection*{Discussion}

In an earlier version of the question above, i omitted the first
condition.  Trying to solve it without this condition is actually
quite instructive.  

\smallskip

OK, so what's the obvious thing to do?  For each pair $b,c$ of a bloke
and a chick, have a propositional letter $p_{bc}$ to say that $b$ and
$c$ are compatible, and for each pair $b,c$ of a bloke and a chick,
have a propositional letter $m_{bc}$ to say that $b$ and $c$ are
paired off at the end. Clearly, for each $b$ and $c$ we have either
$p_{bc}$ or $\neg p_{bc}$ as an axiom, and we have $m_{bc} \to p_{bc}$
as an axiom.  Also, for every $b \not= b'$ and every $c$ we have an
axiom $\neg(m_{bc} \wedge m_{b'c})$ and for every $c \not= c'$ and
every $b$ we have an axiom $\neg(m_{bc} \wedge m_{bc'})$.  The problem
is, how do we express the condition that says that for every finite set
$C' \subset C$, the set of those blokes who are compatible only with
members of $C'$ is no bigger than $C'$.  I mean, dammit, how do we say
that for even one bloke there is someone he's compatible with?  One
seems to need an infinite disjunction.


The key is to exploit the fact that each bloke is compatible with only
finitely many chicks, so we don't need the $p$ letters.  And that of
course was the detail i omitted.  For each bloke you have an axiom
that is a finite disjunction of $m$ letters, $m_{bc} \vee m_{bc'} \vee
m_{bc''} \vee \ldots$.  We have these axioms instead of the $m_{bc}
\to p_{bc}$ axioms.

I think it's easy from there, tho' i might supply the details if i am
at a loose end and pressed by my students.

\medskip

It does make me wonder about the infinite case. Can the infinite
case be turned into a more complex propositional logic exercise,
possibly involving propositional omitting types?  Sadly this cannot
possibly work. For how would it work?  One would prove the infinite
version of the marriage theorem by showing how to omit countably many
types, as follows.  For each bloke one omits the type that says he is
a wallflower, chicks ditto. If one is to omit all these types one has
to adopt as an axiom every formula $\phi$ that locally realises any of
these types.  But of course there is no such formula! (Any such
formula would mention only finitely many blokes and chicks and
therefore cannot imply that (s)he is never mated.)

A pity

}

%\end{document}















%\begin{thebibliography}{}
%\bibitem{sweetreason} Tymoczko and Henle.   Sweet Reason: A Field Guide to Modern Logic.  Birkh\"auser 2000  ISBN 
%\end{thebibliography}





\end{document}
\documentclass{article}
 \usepackage{amssymb}
 \usepackage{amsbsy}
 \usepackage{amsmath}
 \usepackage[english]{babel}
 \usepackage{bussproofs}
 \usepackage{boxedminipage}
 \usepackage{latexsym}
 \usepackage{graphicx}
 \usepackage{makeidx}
 \usepackage{lscape}
\usepackage{hyperref}
\usepackage{wasysym}
\usepackage{pifont}
\usepackage{latexsym}
\usepackage{graphicx}
\usepackage{pgf}
\usepackage{txfonts}%for \leftsquigarrow
\usepackage{tikz}
\usetikzlibrary{arrows,automata}
\usepackage[latin1]{inputenc}
\usepackage{dingbat}
\input newlogicmacr
 \begin{document}
 \title{Part II Set Theory and Logic Exam Questions for 2017}
 \author{Thomas Forster}
  \maketitle


\section{Question 1}
Show that the wellordering principle is equivalent to the assertion that
$|X \times X| = |X|$ for every infinite set $X$.

You may use without proof any facts about normal functions On $\to$ On
that you need provided you state them correctly.  You may use Hartogs'
Lemma and Bernstein's Lemma without proof provided you state them correctly.


\subsection*{Answer to Question 1}

First we show Left $\to$ Right.

\smallskip

[What follows is a slightly doctored extract from the notes from which
  i lectured this result.  It has been available to the students
  (linked from my home page) from the time i lectured it]

(Using the letter `$\aleph$' as  a variable to range over alephs\ldots)
 
We start by noting that $\aleph = \aleph+\aleph+\aleph$. This is
because $2\cdot \omega_\alpha = \omega_\alpha$.  Any order of limit
order-type consists of lots of concatenated copies of $\Nn$, each of
length $\omega$.  You can interleave three wellorders of length
$\omega$ to get a wellorder of length $\omega$ so you can do this for
all the copies simultaneously.

\smallskip

We start by defining a function ${\mathfrak S}: On \to On$.  Given an
ordinal $\alpha$, take a wellordering $\tuple{A, <_A}$ of order type
$\alpha$, make disjoint copies of all its proper initial segments, and
then concatenate the copies \ldots with longer things appended after
shorter things. 


The result is a wellordering and its order type is defined to be
${\mathfrak S}(\alpha)$. 

\begin{lem}\label{lem:S} \mbox{\negthinspace}\\

(i) ${\mathfrak S}: On \to On$ is a normal function;

(ii) Every initial ordinal is a value of ${\mathfrak S}$.

\end{lem}

\Proof

(i) ${\mathfrak S}: On \to On$ evidently also has a recursive definition:
\begin{quote}
${\mathfrak S}(\alpha + 1) = {\mathfrak S}(\alpha) + \alpha$\ \ \ and\\  
${\mathfrak S}(\lambda) = $ Sup$\{{\mathfrak S}(\alpha): \alpha < \lambda\}$ for $\lambda$ limit.\end{quote}
\ldots from which it is clear that ${\mathfrak S}$ is a normal function. 

\medskip

(ii)
Use the division algorithm for normal functions to show that there is a 
$\beta$ s.t ${\mathfrak S}(\beta)\leq\omega_\alpha<{\mathfrak S}(\beta+1)$.  
If ${\mathfrak S}(\beta) < \omega_\alpha$ then we have 
$\omega_\alpha \leq {\mathfrak S}(\beta+1)={\mathfrak S}(\beta)+\beta$ 
which is impossible, since ${\mathfrak S}(\beta)$ and $\beta$ both have 
cardinality below $\aleph_\alpha$.\endproof

\medskip

We want to show that $(\aleph_\alpha)^2 = \aleph_\alpha$. $\aleph_\alpha$ is 
defined as the cardinal $\{\beta:\beta<\omega_\alpha\}$, which means that 
the canonical set of size $(\aleph_\alpha)^2$ is the cartesian product  
$\{\beta: \beta < \omega_\alpha\} \times \{\beta: \beta < \omega_\alpha\}$.  
We partition this last set into three pieces: \begin{quote}

(i) the [graph of] the identity relation restricted to $\{\beta:\beta<\alpha\}$, and

(ii), (iii) the two triangles above-and-to-the-left, and
below-and-to-the-right of the diagonal.  \end{quote}

To be slightly more formal about it, we partition the cartesian product 
$\{\beta: \beta <\alpha\} \times \{\beta: \beta < \alpha\}$ into the 
three pieces $\{\tuple{\beta,\gamma}: \beta < \gamma < \alpha\}$, 
$\{\tuple{\beta,\gamma}: \beta = \gamma < \alpha\}$ and 
$\{\tuple{\beta,\gamma}: \gamma < \beta < \alpha\}$.

It is clear that the third piece is of order type 
${\mathfrak S}(\alpha)$ in the lexicographic order.

The idea is to show that these three pieces all have cardinality
$\aleph_\alpha$.  That's obvious for the second piece, the identity
relation.  Also there is an obvious bijection between the first and
third piece (``flip your ordered pairs'') so it will suffice to prove
that the third piece (``the bottom-right triangle'') has cardinality
$\aleph_\alpha$.

\smallskip

Now we can prove

$$(\forall \alpha)(\aleph_\alpha = (\aleph_\alpha)^2).$$

\Proof

By induction on $\alpha$.  The fact that it holds for $\alpha = 0$ you
learnt in 1a.


Assume true for all alephs $< \aleph_\alpha$. By lemma \ref{lem:S},
$\omega_\alpha$ is a value of $\mathfrak S$; we want to show that it
is actually a fixed point.  Now $\omega_\alpha$ is an initial ordinal,
which is to say that for any $\beta < \omega_\alpha$, the cardinal
$|\{\gamma:\gamma<\beta\}|$ is less than $\aleph_\alpha$, and (by
induction hypthesis) is equal to its own square.  Suppose
$\omega_\alpha$ were ${\mathfrak S}(\beta)$ for some $\beta
<\omega_\alpha$.  This would entail that the size of the cartesian
product $\{\gamma: \gamma < \beta\} \times \{\gamma: \gamma < \beta\}$
is at least $\aleph_\alpha$, contradicting the induction. So
$\omega_\alpha$ is a fixed point of $\mathfrak S$.  This means that
the lower-right triangle of the cartesian product $\{\gamma: \gamma <
\omega_\alpha\} \times \{\gamma: \gamma < \omega_\alpha\}$---which can
be wellordered to length ${\mathfrak S}(\omega_\alpha)
=\omega_\alpha$---is of cardinality $\aleph_\alpha$.  It's clearly
naturally isomorphic to the upper-left triangle (as remarked earlier)
so the cartesian product is now a union of three sets each of size
$\aleph_\alpha$, giving $(\aleph_\alpha)^2 = \aleph_\alpha +
\aleph_\alpha + \aleph_\alpha = \aleph_\alpha$ as desired.

\endproof


\smallskip

Now we prove Right $\to$ Left.

\smallskip

\begin{quote}
{\bf If $\alpha = \alpha^2$ for all infinite cardinals, then AC follows.}\end{quote}

\Proof  Let $\alpha$ be an arbitrary infinite cardinal, and suppose $\beta^2 = \beta$ for all infinite cardinals $\beta$.  Then we have
\begin{tabbing}
$\alpha + \aleph(\alpha)$\ \= = \=$(\alpha + \aleph(\alpha))^2$\\
                         \> = \> $\alpha^2 + 2\cdot\alpha\cdot \aleph(\alpha) + (\aleph(\alpha))^2$\\
                         \> = \> $\alpha + 2\cdot\alpha\cdot \aleph(\alpha) + \aleph(\alpha)$\\
                         \> = \> $\alpha + \alpha\cdot \aleph(\alpha) + \aleph(\alpha)$\\
                         \> = \> $\alpha(1 + \aleph(\alpha)) + \aleph(\alpha)$\\
                         \> = \> $(\alpha \cdot \aleph(\alpha)) + \aleph(\alpha)$\\
                         \> = \> $(\alpha + 1) \cdot \aleph(\alpha)$\\
                         \> = \> $\alpha \cdot \aleph(\alpha)$\\
\end{tabbing}

Now we use Bernstein's lemma (which says that $\gamma + \delta = \alpha \cdot \beta\ \to\ \alpha \leq^* \gamma \vee \beta \leq \delta$) to infer
$$\aleph(\alpha) \leq \alpha \vee \alpha \leq^* \aleph(\alpha)$$
The first disjunct is of course impossible---by definition of
$\aleph(\alpha)$---so we infer the second, which tells us that any set
of size $\alpha$ is a surjective image of a wellordered set.  But any
such surjective image can be wellordered, and this gives us our result. 
\endproof




\section{Question 2}
  What is a boolean algebra? What is a filter? An ultrafilter?  Prove
  that in every boolean algebra every element belongs to some
  ultrafilter.  Deduce that every boolean algebra is isomorphic to
  some subalgebra of a power set algebra.

\subsection*{Answer to Question 2}

A Boolean algebra is a distributive complemented lattice. 
It also obeys the de Morgan Laws: $\overline{a \vee b} = \overline{a} \wedge \overline{b}$ and $\overline{a \wedge b} = \overline{a} \vee \overline{b}$.

A filter in a Boolean algebra is an upward-closed subset closed also
under $\wedge$.  An ultrafilter is a $\subseteq$-maximal filter that
is {\sl proper}: not containing the $0$ of the algebra.

In any boolean algebra ${\cal B}$, and for any two elements $b,b'\in{\cal B}$ 
with $b \not\leq b'$, the set of proper filters $F$ with $b\in F$ but 
$b' \not\in F$ is a chain-complete poset, and has a maximal element by 
Zorn's lemma.  This ensures that the map $h$ sending each element 
$b \in {\cal B}$ to the set of ultrafilters containing it is injective, 
and respects $\wedge$ and $\vee$.  Furthermore, every ultrafilter $\U$ 
must contain $b$ or $\overline{b}$, $\forall b \in {\cal B}$.  This 
ensures that $h$ respects complementation.

Thus $h$ is an injective boolean homomorphism from ${\cal B}$ into the
power set of the set of all ultrafilters $\subset {\cal B}$.

[where do we use distributivity?]

 
\section{Question 3}


Which of the following assertions about ordinals are true?  Justify your answers

\begin{tabbing}
(i)\ \ \ \ \= $\alpha < \beta$\ \ \ \ \    \= \ $\to \gamma\alpha < \gamma \beta$;\\

(ii)       \> $\alpha < \beta$    \> \ $\to \alpha\gamma < \beta\gamma$;\\

(iii)      \> $\alpha^2 = \beta^2$  \> $\to \alpha = \beta$;\\

(iv)       \> $\alpha\beta = \beta\alpha$ \> $\to \alpha^2\beta^2 = \beta^2\alpha^2$?
\end{tabbing}

\subsection*{Answer to Question 3}

(i) is true (at least if $\gamma > 0$) and (ii) is false.  We
illustrate the falsehood of (ii) by setting $\alpha = 1$, $\beta = 2$
and $\gamma = \omega$.

\smallskip

(iii) is true.  (We prove the contrapositive). Suppose $\alpha <
\beta$.  Then $\alpha^2 < \alpha\beta$ by (i).  But $\alpha^2 =
\beta^2$ so we can substitute to get $\beta^2 < \alpha\beta$.  Also we
have $\alpha\beta \leq \beta^2$ (because $\alpha \leq \beta$) giving
$\beta^2 < \beta^2$.

\smallskip

(iv) is true. Suppose $\alpha\beta = \beta\alpha$.  Multiply both
sides by $\alpha$ on the left and $\beta$ on the right to obtain
$\alpha^2\beta^2 = \underline{\alpha\beta}\alpha\beta$.  Now permute
the underlined bits to obtain

$$\alpha^2\beta^2 = \beta\alpha\underline{\alpha\beta}$$  and permute again to obtain

$$\alpha^2\beta^2 = \beta\underline{\alpha\beta}\alpha$$ and yet  {\sl again} to obtain

$$\alpha^2\beta^2 = \beta^2\alpha^2.$$


But actually they ask about the converse: If $\alpha^2\beta^2 = \beta^2\alpha^2$ must we have $\alpha\beta = \beta\alpha$.   Well, suppose $\alpha\beta < \beta\alpha$.

Then $\alpha\beta$ copies of $\alpha$ is less than $ \beta\alpha$ copies of $\alpha$ giving
$$\alpha^2\beta < \alpha\beta\alpha$$

But $\alpha\beta < \beta\alpha$ so $\alpha$ copies of $\alpha\beta$ $\leq$ $\alpha$ copies of $\beta\alpha$ giving
$$\alpha^2\beta < \alpha\beta\alpha \leq \beta\alpha^2$$
in other words
$$\alpha^2\beta < \beta\alpha^2$$
Look at the RHS of this inequality. $\alpha^2\beta < \beta\alpha^2$ so
$\alpha^2\beta$ copies of $\beta$ $\leq$  $\beta\alpha^2$ copies of $\beta$ giving $$\alpha^2\beta^2 < \beta^2\alpha^2$$



\section{Question 4}


 Show that, for a first-order theory $T$, the class of models of $T$
 is closed under substructure if and only if $T$ is a universal
 theory.

\subsection*{Answer to Question 4}

  The {\bf diagram} $D_{\M}$ of a structure $\M$ is the theory
  obtained by expanding $\M$ by giving names to every $m \in M$, and
  collecting all true atomic assertions about them.

We need a lemma:

\begin{lem}

  For any consistent theory $T$ and any model $\M$ of $T_\forall$ (the
  set of universal consequences of $T$) the theory $T \cup D_\M$, is
  consistent.
\end{lem}

\Proof

Let $\M$ be a model of $T_\forall$, with carrier set $M$.  Add to
${\cal L}(T)$ names for every member of $M$.  Add to $T$ all the
(quantifier-free) assertions about the new constants that $\M$
believes to be true.  This theory is $T \cup D_\M$.  We want this
theory to be consistent.  How might it not be?  Well, if it isn't,
there must be an inconsistency to be deduced from a conjunction $\psi$
of finitely many of the new axioms. This rogue $\psi$ mentions
finitely many of the new constants.  We have a proof of $\neg \psi$
from $T$.  $T$ knows nothing about these new constants, so clearly we
must have a UG proof of $(\forall \vec x)\neg\psi$.  But this would
contradict the fact that $\M$ satisfies every universal consequence of
$T$. \endproof

We can now prove:

\begin{thm}\label{universalsubstructure}\mbox{\negthinspace}\\
  $T$ is universal iff every substructure of a model of $T$ is a model
  of $T$.\end{thm}

\Proof

L $\to$ R is easy.  We prove only the hard direction.

\smallskip

Suppose that $T$ is a theory such that every substructure of a
model of $T$ is also a model of $T$.  Let $\M$ be an arbitrary model
of $T_\forall$.  We will show that it must be a model of $T$.  We know
already from the foregoing that the theory $T \cup D_\M$ is consistent,
and so it must have a model---$\M^*$, say. $\M^*$ is a model of $T$,
and $\M$ is a submodel of $\M^*$ and therefore (by assumption on $T$)
a model of $T$---as desired.

But all we knew about $\M$ was that it was a model of the universal
consequences of $T$.  So any $\M$ that was a model of the universal
consequences of $T$ is a model of $T$.  So $T$ is axiomatised by its
universal consequences. \endproof


\end{document}



The implication the other way 



$\alpha\beta = \beta\cdot\alpha \to (\exists \gamma)(\omega^\gamma \leq \alpha, \beta < \omega^{\gamma + 1})$





$\alpha^2\beta^2 = \beta^2\alpha^2\ \to\ alpha\beta = \beta\alpha$



\begin{question}
Let ${\cal X}$ be an arbitrary set.  Consider the set ${\cal E}$ of
equivalence relations on $X$ ordered by $E_1 \leq E_2$ if $E_2
\subseteq E_1$ as sets of ordered pairs. Show that $\tuple{{\cal X},
  \leq}$ is a complete poset.  Is it a distributive lattice?

\medskip

Explain briefly why $V_\alpha$ is transitive for all $\alpha$.  Fix an
ordinal $\alpha$ and consider the poset of equivalence relations on
$V_\alpha$ partially ordered as above.  If $R$ is an equivalence
relation on $V_\alpha$ so is [the restriction to $V_\alpha$ of] the
relation $R^+$ defined by $R^+(x,y)$ iff$_{df}$ $(\forall x'\in
x)(\exists y' \in y)(R(x',y') \wedge (\forall y' \in y)(\exists x'\in
x)(R(x',y')$.  Fixed points for + are {\sl contractions}.  If $R$ is a
contraction show how to turn the quotient over $R$ into a structure
for the language of set theory.
\end{question}

\begin{question}
\end{question}


\commentblock{
\begin{answer}

(a) 

By contraposition.  Suppose there is no $T$-valuation omitting $\Sigma$.  
Then every formula in $\Sigma$ is a theorem of $T$, so there is an 
expression $\phi$ (namely `$\top$') such that $T \vdash \phi \to \sigma$ 
for every $\sigma \in \Sigma$ but $T \not\vdash \neg \phi$. But of course 
$T \vdash \top$.  Contraposing, we infer that if $T \vdash \neg \phi$ for 
every $\phi$ such that $T \vdash \phi \to \sigma$ for every $\sigma \in \Sigma$
then there is a $T$-valuation omitting $\Sigma$. 

(b)

We will show that: whenever $T \cup \{\neg A_1, \ldots \neg A_i\}$ is
consistent, where $A_n \in \Sigma_n$ for each $n \leq i$, then we can
find $A_{i+1} \in \Sigma_{i+1}$ such that $T \cup \{\neg A_1, \ldots
\neg A_i, \neg A_{i+1}\}$ is consistent.

Suppose not, then $\displaystyle{T \vdash (\bigwedge_{1 \leq j \leq i} \neg A_j) \to
A_{i+1}}$ for every $A_{i+1} \in \Sigma_{i+1}$.  But, by assumption,
$T$ locally omits $\Sigma_{i+1}$, so we would have 
$\displaystyle{T \vdash \neg \bigwedge_{1 \leq j \leq i} \neg A_j}$ contradicting 
the assumption that $T \cup \{\neg A_1, \ldots \neg A_i\}$ is 
consistent.

Now, as long as there is an enumeration of the formul{\ae} in ${\cal
  L}(T)$, we can run an iterative process where at each stage we pick
for $A_{i+1}$ the first formula in $\Sigma_{i+1}$ such that $T \cup
\{\neg A_1, \ldots \neg A_i, \neg A_{i+1}\}$ is consistent.  This
gives us a theory $T \cup \{\neg A_i: i \in \Nn\}$ which is consistent
by compactness.  Any model of $T \cup \{\neg A_i: i \in \Nn\}$ is a
model of $T$ that omits each $\Sigma_i$.

\smallskip

The first part is the {\sl Omitting Types Theorem} and the second part
is the {\sl Extended omitting Types Theorem}. They come to life when
you consider the analogues for predicate logic.  The standard model of
arithmetic is one that omits the $1$-type $\{a \not= 0, a \not=S(0), a \not=S(S(0)) \ldots\}$

\end{answer}
}





